
\section{The laypersons introduction}

Mathematics is one of the longest, richest and best preserved traditions humanity has ever created. In all probability, mathematics started by humans (and animals) realizing that things can be counted --- that collections of things can be said to have a certain numerical size. This developed to the simplest theory of numbers: \emph{arithmetic}. For millennia counting and arithmetic was used to create new knowledge and new technology, everything from understanding how the seasons change, undestanding lunar cycles, solar cycles and astronomy, to agriculture, crop cycles and animal populations. 

Another part of mathematics came about later, when humans discovered that things in nature can be well described by certain diagrammatic abstractions of their shapes. This formed the field of \emph{geometry}, initially used to describe landmasses, hence its name. These two fields of mathematics was essentially all there was for thousands of years, and in some very general way, these are still all there is to mathematics as a whole. 

These two subjects continued in tradition, all the way through antiquity and the premodern eras. The students at the first universities, in particular those at the philosophical faculties, studied the \emph{seven liberal arts}. The words ``arts'' here has a different meaning than in modern society, where the original meaning is closer to that of a skill. These seven skills are what was deemed necessary in antiquity in order to be \emph{free} --- a person worthy of attending public debates, defending themselves in court, participating in juries, serving in the military. The first three, named the \emph{trivium}, consisted of: grammar, rhetoric and logic. The remaining four, called the \emph{quadrivium}, consisted of music, arithmetic, geometry and astronomy. Here we again see the appearance of our two fields of mathematics --- arithmetic and geometry. These are still somewhat kept separate and studied in different ways, and by different tools. 

In more modern times, knowledge and education has received several sorely needed revolutions. Research --- and thus knowledge --- is now an incredibly rigorous process; education a well oiled more standardized machine. The two fields of mathematics have expanded to immense sizes, and now contain hundreds upon hundreds of subfields. One of the most interesting development --- in my humble completely unbiased opinion --- was the development of bridges, connections and similarities between the two fields of mathematics. Numbers, now meaning not only the numbers used for counting, but also concepts like real and complex numbers, gave way to coordinate systems, functions and analysis, and things that locally looked like coordinate systems became manifolds, which are incredibly geometric --- or shape-like in nature. The study of the structure of systems of numbers turned into \emph{algebra}, the study of the structure of shapes formed \emph{topology}, and the interactions between these two fields became \emph{algebraic topology}. 

An example of such an interaction is the following. Imagine you have two circles and want to understand how many continuous functions there are between them. Up to a bit of rotation and scaling, there is essentially only one thing you can do: you can wind the first circle around the other. This can be done in either a clockwise or counter-clockwise fashion, and with as many turns as one could wish for. If I first loop five times around clockwise, and then two times around counter-clockwise, then we have effectively looped around only three times clockwise. This means that we can ``add'' or ``subtract'' these windings around the circle together. In particular, this has exactly the same structure as the whole numbers $\{\ldots,-2, -1, 0, 1, 2,\ldots\}$! This study of all continuous maps between the circle and some space $X$ is called the \emph{fundamental group} of $X$, and is always a ``number system'' where one can add and subtract. This then, forms a bridge between shapes and number systems. 

The act of building connections between the two fields of mathematics is also what this thesis is about. It is about continuing the long and deep tradition of understanding the interplay of these fields --- about furthering the development and understanding in one of their modern subfields. These subfields are \emph{homological algebra} and \emph{stable homotopy theory}. The former a subfield of algebra, that tries to study the structure of the systems of all different systems of numbers; the latter a subfield of topology, that tries to study the structure of systems of all different systems of shapes. This is not very precise, but the precise concepts are technical in nature and will be introduced in the mathematical introduction of the thesis. Both of the above concepts are formulated through the language of \emph{$\infty$-categories}, which can be thought of as a way to study collections of things, the relations between them, the relations between the relations, and so on ad infinitum. 

We can then make a very general description of what the contents of this thesis is about: it is about studying three different bridges between ``shapes'' and ``numbers''. These three bridges are each located in their own research paper, which form the main content of this thesis. Let us very briefly try to explain what each of these bridges are: 

{\hyperref[ch1]{The first paper}} is the most direct bridge, where we directly compare a specific system of shapes and a specific system of numbers, and prove that they are in fact equivalent --- they have exactly the same structure. The system of shapes we study is in some sense a very ``fundamental'' system, as it arises as the smallest pieces --- the atoms --- of perhaps the most important system we have in stable homotopy theory. 

{\hyperref[ch2]{The second paper}} has a bit more of an indirect bridge, where we take a concept from algebra and try to study an analogous concept in topology. Doing this we are able to recover and generalize some already known results in topology, now seen from a completely new angle. 

{\hyperref[ch3]{The third paper}} is again more direct, where we have a direct comparison between certain substructures of shapes, to certain substructures of numbers. We prove that there is a one-to-one correspondence between these collections of substructures, providing new insight into the substructures, and illuminating previously known such relationships.
  