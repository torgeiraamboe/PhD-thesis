
\section{Central ideas}
\label{ch1:sec:Central-ideas}

As a backdrop for this entire thesis lies the ubiquitous concept of \emph{$\infty$-categories}, as developed by Joyal, Lurie and others --- the canonical references being \cite{joyal_02}, \cite{lurie_09} and \cite{Lurie_HA}. We will assume familiarity with $\infty$-categories and their associated standard constructions, and use them all willy-nilly throught the rest of the thesis. 

Most $\infty$-categories considered will be \emph{presentable}, in the sense of \cite[Chapter 5]{lurie_09}. The $(\infty, 2)$-category of presentable $\infty$-categories and colimit-preserving functors, $\PrL$, has a symmetric monoidal structure via the Lurie tensor product $\otimes^L$, and we will say a presentable $\infty$-category is \emph{presentably symmetric monoidal} if it is a commutative monoid in $\PrL$. Any such category $\C$ has a symmetric monoidal structure, with the property that the tensor product in $\C$ preserves colimits separately in each variable. The unit for the Lurie tensor product on $\PrL$ is the category of \emph{spaces}, denoted $\Spaces$, which is an $\infty$-categorical version of the classical category of topological spaces. 

We will also assume knowledge about \emph{stable} $\infty$-categories, which are an $\infty$-categorical enhancement of triangulated categories. The $(\infty, 2)$-category of presentable stable $\infty$-categories and exact colimit preserving functors, $\PrLs$, inherits a symmetric monoidal structure from $\PrL$. An $\infty$-category $\C$ is a \emph{presentably symmetric monoidal stable $\infty$-category} if it is a commutative monoid in $\PrLs$. This means that it is presentably symmetric monoidal, and the tensor product preserves the stable structure. 

The unit for the Lurie tensor product on $\PrLs$ is the category of \emph{spectra}, denoted $\Sp$. Given any presentable $\infty$-category, one can form its \emph{stabilization}, given by formally inverting the desuspension functor $\Omega$. The category of spectra can then be defined as the stabilization of the category of spaces. 

The categories of spaces, spectra, and many other interesting categories, satisfy some even nicer conditions than merely being presentable: they have an explicit collection of generators, which satisfy some ``smallness'' condition. 

\begin{definition}
    \label{ch1:def:compact-object}

\end{definition}

The full subcategory of compact objects will be denoted $\Co$. 

\begin{definition}
    \label{ch1:def:compactly-generated-category}

\end{definition}

The compact generator for the category of spaces, $\Spaces$, are the \emph{finite spaces}, which correspond to the classical finite CW-complexes. The compact generators for the category of spectra, $\Sp$, are the \emph{finite spectra}. A spectrum is finite if it is the desuspension of a suspension spectrum $\Sigma^{-n}\Sigma^\infty K$ for some number $n$, where $K$ is a finite space. 

In the presence of symmetric monoidal structures we have another ``smallness'' condition, slightly different from being compact. 

\begin{definition}
    \label{ch1:def:dualizable-object}
    
\end{definition}

In a certain sense, being compact is about being small with respect to colimits, while being dualizable is about being small with respect to the monoidal structure. In very well-behaved categories, these two notion of smallness coincide. 

\begin{definition}
    \label{ch1:rigidly-generated-category}
\end{definition}

An example is again our favorite stable $\infty$-category $\Sp$. Every compact object is dualizable, and conversely, every dualizable object is compact. Hence, $\Sp$ is a rigidly generated symmetric monoidal stable $\infty$-category. 


\subsection{Localizing subcategories and ideals}
\label{ch1:ssec:localizing-subcategories-and-ideals}

If we were to assign this thesis a single protagonist, it would be the idea of a localizing subcategory. It will heavily feature in all the different parts of the thesis: 
\begin{enumerate}
    \item In \cref{ch:2} we study how a specific localizing subcategory, appearing in chromatic homotopy theory, interacts with a specific homological functor.
    \item In \cref{ch:3} we study how, in certain situations, the category of comodules over a coalgebra in a stable $\infty$-category forms a localizing subcategory. 
    \item In \cref{ch:4} we classify certain localizing subcategories along nicely behaved $t$-structures on stable $\infty$-categories. 
\end{enumerate}

Given a presentable stable $\infty$-category $\C$, one should think about a localizing subcategory as being a collection of objects in $\C$, that themselves form a nice presentable stable $\infty$-category, compatible with $\C$. In other words, they are the ``structure preserving subcategories'', in a certain precise way. 

\begin{definition}
    \label{ch1:def:localizing-subcategory}
    If $\C$ is a presentable stable $\infty$-category, then a full subcategory $\L\subseteq \C$ is \emph{localizing} if it is closed under desuspensions, colimits and retracts. 
\end{definition}

This means that $\L$ is itself a presentable stable $\infty$-category, and that computing colimits in $\L$ is equivalent to computing colimits in $\C$. 

\begin{definition}
    Let $\C$ be a presentable stable $\infty$-category. Given a collection of objects $\K\subseteq \C$ we denote by $\Loc(\K)$ the smallest localizing subcategory of $\C$ containing $\K$. We will often call it the localizing subcategory \emph{generated} by $\K$. 
\end{definition}

\begin{remark}
    \label{ch1:rm:compactly-generated-localizing-subcategory}
    If the collection $\K \subseteq \C$ consists of only compact objects, in the sense of \cref{ch1:def:compact-object}, then the localizing subcategory $\Loc(\K)$ is said to be a \emph{compactly generated} localizing subcategory. 
\end{remark}

A presentable stable $\infty$-category $\C$ is compactly generated --- as in \cref{ch1:def:compactly-generated-category} --- if and only if the smallest localizing subcategory containing the collection of all compact objects $\Co$ is the entire $\infty$-category $\C$. In other words, there is an equivalence  
\[\C\simeq \Loc(\Co)\]
of presentable stable $\infty$-categories. 




If our presentable stable $\infty$-category is also symmetric monoidal, then we we want a version of localizing subcategories that preserve the monoidal structure. If one thinks of a presentably symmetric monoidal stable $\infty$-category as a categorified version of a ring, then the natural such sub-structure should model that of an ideal in a ring. 

\begin{definition}
    \label{ch1:def:localizing-ideal}
    If $\C$ is a presentably symmetric monoidal stable $\infty$-category, then a full subcategory $\L\subseteq \C$ is a \emph{localizing $\otimes$-ideal} if it is a localizing subcategory, and for any $L\in \L$ and $C\in \C$, we have $L\otimes C\in \L$. 
\end{definition}

The definition of an ideal here is completely analogous to the classical setting of discrete rings. 

\begin{definition}
    Let $\C$ be a presentably symmetric monoidal stable $\infty$-category. Given a collection of objects $\K\subseteq \C$ we denote by $\Loc^\otimes(\K)$ the smallest localizing $\otimes$-ideal of $\C$ containing $\K$. We will, as before, often refer to this as the localizing $\otimes$-ideal \emph{generated} by $\K$. 
\end{definition}

Any ideal $I$ in a discrete ring $R$ is a non-unital subring of $R$. This is also the case for a localizing $\otimes$-ideal $\L\subseteq \C$, which becomes a non-unital presentably symmetric monoidal stable $\infty$-category. However, in some good cases $\L$ is actually unital, but the unit will naturally have to be different than the unit for the monoidal structure on $\C$, which we denote by $\1_\C$. The localizing ideals we study in both \cref{ch:2} and \cref{ch:3} will have this property. In particular, as we will see in the next section, anylocalizing $\otimes$-ideal which is compactly generated in the sense of \cref{ch1:rm:compactly-generated-localizing-subcategory} will have this property.


\subsection{Local duality}
\label{ch1:ssec:local-duality}

The theory of abstract local duality, proved in \cite{hovey-palmiery-strickland_97} and generalized to the $\infty$-categorical setting in \cite{barthel-heard-valenzuela_2018}, is one of the central ideas of this thesis that will show up several times. 

\begin{definition}
    \label{ch1:def:local-duality-context}
    A pair $(\C, \K)$, where $\C$ is a presentably symmetric monoidal stable $\infty$-category compactly generated by dualizable objects, and $\K\subseteq \Co$ is a subset of compact objects, is called a \emph{local duality context}.
\end{definition}

Any choice of local duality context allows us to assign three new categories, which together decomposes the category $\C$. 

\begin{construction}
    Let $(\C, \K)$ be a local duality context. We define $\C^{\K-tors}$ to be the localizing tensor ideal generated by $\K$, denoted $\Loc^\otimes_\C(\K)$. Further we define $\C^{\K-loc}$ to be the left orthogonal complement $(\C^{\K-tors})^\perp$, i.e., the full subcategory consisting of objects $C\in \C$ such that $\Hom_\C(T,C)\simeq 0$ for all $T\in \C^{\K-tors}$. Similarly, define $\C^{\K-comp}$ to be the left-orthogonal complement of $\C^{\K-loc}$, i.e. $\C^{\K-comp}= (\C^{\K-loc})^\perp$. These full subcategories are respectively called the $\K$-torsion, $\K$-local and $\K$-complete objects in $\C$. We have inclusions into $\C$, denoted $i_{\K-tors}$, $i_{\K-loc}$ and $i_{\K-comp}$ respectively. 
    
    By the adjoint functor theorem, \cite[5.5.2.9]{lurie_09}, the inclusions $i_{\K-loc}$ and $i_{\K-comp}$ have left adjoints $L_\K$ and $\Lambda_\K$ respectively, while $i_{\K-tors}$ and $i_{\K-loc}$ have right adjoints $\Gamma_\K$ and $V_\K$ respectively. These are then, by definition, localizations and colocalizations. Since the torsion, local and complete objects are ideals, these localizations and colocalizations are compatible with the symmetric monoidal structure of $\C$, in the sense of \cite[2.2.1.7]{Lurie_HA}. In particular, by \cite[2.2.1.9]{Lurie_HA} we get unique induced symmetric monoidal structures such that $L_\K$, $\Lambda_\K$, $\Gamma_\K$ and $V_\K$ are symmetric monoidal functors. 

    For any $X\in \C$, these functors assemble into two cofiber sequences:
    $$\Gamma_\K X \longrightarrow X \longrightarrow L_\K X \quad \text{and}\quad V_\K X \longrightarrow X \longrightarrow \Lambda_\K X.$$
    Note also that these functors only depend on the localizing subcategory $\C^{\K-tors}$, not on the particular choice of generators $\K$. Thus, when the set $\K$ is clear from the context, we often omit it as a subscript when writing the functors. 
\end{construction}

The following theorem is a slightly restricted version of the abstract local duality theorem of \cite[3.3.5]{hovey-palmiery-strickland_97} and \cite[2.21]{barthel-heard-valenzuela_2018}.  

\begin{theorem}
    \label{ch1:thm:local-duality}
    \index{Local duality}
    Let $(\C, \K)$ be a local duality context. Then
    \begin{enumerate}
        \item the functors $\Gamma$ and $L$ are \emph{smashing}, meaning that there are natural equivalences $\Gamma X\simeq X\otimes \Gamma \1$ and $LX\simeq X\otimes L\1$,
        \item the functors $\Lambda$ and $V$ are \emph{cosmashing}, meaning there are natural equivalences $\Lambda X\simeq \iHom(\Gamma \1,X)$ and $VX\simeq \iHom(L\1, X)$, and 
        \item the functors $\Gamma\colon \C^{\K-comp}\longrightarrow \C^{\K-tors}$ and $\Lambda\colon \C^{\K-tors}\longrightarrow \C^{\K-comp}$ are mutually inverse symmetric monoidal equivalences of categories,
    \end{enumerate}
    This can be summarized by the following diagram of adjoints
    \begin{center}
        \begin{tikzcd}
                & {\C^{\K-loc}} \\
                & {\C} \\
                {\C^{\K-tors}} && {\C^{\K-comp}}
                \arrow["L", xshift=-4pt, from=2-2, to=1-2]
                \arrow[from=1-2, to=2-2]
                \arrow["V", xshift=4pt, from=2-2, to=1-2, swap]

                \arrow["\Lambda", yshift=2pt, xshift=2pt, from=2-2, to=3-3]
                \arrow[yshift=-2pt, xshift=0pt, from=3-3, to=2-2]

                \arrow["\Gamma", yshift=-2pt, xshift=0pt, from=2-2, to=3-1]
                \arrow[yshift=2pt, xshift=-2pt, from=3-1, to=2-2]
                
                \arrow[bend left=35, dashed, from=3-1, to=1-2]
                \arrow[bend left=35, dashed, from=1-2, to=3-3]

                \arrow["\simeq"', swap, from=3-1, to=3-3]
        \end{tikzcd}    
    \end{center}
\end{theorem}

\begin{remark}
    \label{ch1:rm:monoidal-structure-in-local-duality}
    \cref{ch1:thm:local-duality} implies, in particular, that the symmetric monoidal structure induced by the localization $L$ and the colocalization $\Gamma$ is just the symmetric monoidal structure on $\C$ restricted to the full subcategories. This is not the case for $\C^{\K-comp}$, where the symmetric monoidal structure is given by $\Lambda(-\otimes_\C-)$. The functor $V$ also induces a symmetric monoidal structure on $\C^{\K-loc}$, but this coincides with the one induced by $L$, due to their associated endofunctors on $\C$ defining an adjoint symmetric monoidal monad-comonad pair. Note that we will not need or focus on the functor $V$, hence it will usually be omitted from the local duality diagrams for the rest of the thesis. 
\end{remark}

\begin{example}
    The object $\Z_{(p)}/p$ is compact in $\Der(\Z)_{(p)}$, hence $(D(\Z)_{(p)}, \Z_{(p)}/p)$ forms a local duality context. The category of local objects, $\Der(\Z)_{(p)}^{\K-loc}$, has objects in which $p$ is invertible. But, as all other primes are already invertible, all of these are necessarily rational, giving $\Der(\Z)_{(p)}^{\K-loc} \simeq \Der(\Q)$. The category $\Der(\Z)_{(p)}^{\K-tors}$ is equivalent to the category of derived $p$-torsion objects in $\Der(\Z)_{(p)}$. Dually, the category $\Der(\Z)_{(p)}^{\K-comp}$ is equivalent to the derived $p$-complete objects in $\Der(\Z)_{(p)}$, which gives a local duality diagram 
    \begin{center}
        \begin{tikzcd}
            & {\Der(\Q)} \\
            & {\Der(\Z_{(p)})} \\
            {\Der(\Z_{(p)})^{p-tors}} && {\Der(\Z_{(p)})^{p-comp}}
            \arrow["L_p", xshift=-2pt, from=2-2, to=1-2]
            \arrow[xshift=2pt, from=1-2, to=2-2]

            \arrow["\Delta_p", yshift=2pt, xshift=2pt, from=2-2, to=3-3]
            \arrow[yshift=-2pt, xshift=0pt, from=3-3, to=2-2]

            \arrow["\Gamma^p", yshift=-2pt, xshift=0pt, from=2-2, to=3-1]
            \arrow[yshift=2pt, xshift=-2pt, from=3-1, to=2-2]

            \arrow[bend left=35, dashed, from=3-1, to=1-2]
            \arrow[bend left=35, dashed, from=1-2, to=3-3]

            \arrow["\simeq"', swap, from=3-1, to=3-3]
        \end{tikzcd}    
    \end{center}
\end{example}












\subsection{Chromatic homotopy theory}
\label{ch1:ssec:chromatic-homotopy-theory}

The following introduction to chromatic homotopy theory is inspred by \cite{barthel-beaudry_19}. 

\subsubsection{Fracture squares and field objects}
\label{sssec:fracture-squares}

In light of Waldhausen's viewpoint of stable homotopy theory as an enhancement of algebra, usually called \emph{brave new algebra}, one should view the category of spectra $\sp$ as a homotopical enrichment of the derived category of abelian groups $\Der(\Z)$. We know that abelian groups can be studied one prime at the time, which corresponds to studying $D(\Z)_{(p)}$, the $p$-local derived category. In \cite{bousfield_1979_localization}, Bousfield developed a general machinery for studying localizations on $\sp$, by inverting maps that are equivalences with respect to some spectrum $F$. The corresponding localization dunctor is denoted $L_F$. We can then create $p$-localization on $\sp$, by Bousfield localizing at the $p$-local Moore spectrum $M\Z_{(p)}$. On homotopy groups this has the effect of $p$-localizing, i.e., inverting all primes except for $p$. The category of $p$-local spectra, denoted $\sp_{(p)}$, should then be thought of as a homotopical enrichment of $D(\Z)_{(p)}$. 

% \begin{construction}
%     \label{const:bousfield-localization}
%     Let $F$ be a spectrum and $f\colon X\longrightarrow Y$ a map of spectra. We say $f$ is an {\defn $F$-equivalence}, if $f\otimes F$ is an equivalence. If the unique map $0\longrightarrow X$ is an $F$-equivalence, we say $X$ is {\defn $F$-acyclic}. The collection of $F$-acyclic spectra form a localizing ideal (\cref{def:localizing-ideal}), hence the fully faithful inclusion of its left orthogonal complement (\cref{def:left-orthogonal-complement}), denoted {\defn $\sp_F$}, has a left adjoint {\defn $L_F$} as in \cref{ex:localization-from-localizing-subcategory}. This is called the Bousfield localization at $F$, of sometimes just the {\defn $F$-localization}. 
% \end{construction}

\begin{remark}
    Both $L_{(p)}\colon D(\Z)\longrightarrow D(\Z)_{(p)}$ and $L_{(p)}\colon \sp\longrightarrow \sp_{(p)}$ are smashing localizations. 
\end{remark}

%Via the lens of tensor-triangulated geometry, one could think of $\sp$ as the category of quasi-coherent sheaves of tensor-triangulated categories over the Balmer spectrum $\mathrm{Spc}(\sp)$, and similarily for $D(\Z)$. On a spectrum $X$, $p$-localization is given by restricting the corresponding sheaf to the subspectrum lying under the closed point corresponding to $p$. Similarily, for $D(\Z)$: its balmer spectrum is homeomorphic to $\mathrm{Spec}(\Z)$, and localization is restriction to the closed(?) set containing only $p$ and the generic point.


The study of $D(\Z)_{(p)}$ can be further reduced to the study of its ``atomic pieces'', which are the minimal localizing subcategories. 

\begin{definition}
    \label{ch1:def:minimal-localizing-subcategory}
    A localizing subcategory $\L\subseteq \C$ is said to be \emph{minimal} if any proper localizing subcategory $\L'\subset \L$ is $(0)$.  
\end{definition}

\begin{remark}
    If $\L$ is a minimal localizing subcategory, then any non-zero object $K\in \L$ generates $\L$ as $\Loc_\C(K)\simeq \L$.
\end{remark}

The study of minimal localizing subcategories is tightlu connected to local duality, as in \cref{ch1:ssec:local-duality}. By \cite[2.26]{barthel-heard-valenzuela_2018}, we get from any local duality diagram a fracture square, which for the local duality context $(D(\Z)_{(p)}, \Z_{(p)}/p)$ above gives the classical arithmetic fracture square
\begin{center}
    \begin{tikzcd}
        \Z_{(p)} \arrow[r] \arrow[d] & \Z_p \arrow[d]  \\
        \Q \arrow[r]           & \Q\otimes \Z_p
    \end{tikzcd}
\end{center}
which decomposes $\Z_{(p)}$ into a rational part and a $p$-complete part. This also extends to a general chain complex $A\in D(\Z)_{(p)}$, where we have a homotopy pullback square 
\begin{center}
    \begin{tikzcd}
        A \arrow[r] \arrow[d] & A^\wedge_p \arrow[d]  \\
        \Q\otimes A \arrow[r] & \Q\otimes_\Z A^\wedge_p
    \end{tikzcd}    
\end{center}
where $(-)_p^\wedge$ denotes derived $p$-completion as in \crefme. We can then wonder whether these also give our minimal localizing subcategories, which is indeed the case. 

\begin{proposition}
    Let $\L$ be a minimal localizing subcategory of $D(\Z)_{(p)}$. Then either $\L \simeq D(\Q)$ or $\L$ is the category of derived $p$-complete objects, $\L\simeq D(\Z)^\wedge_p$.
\end{proposition}

Now, if $\sp_{(p)}$ is supposed to be a homotopical enrichment, we should expect there to be an analogy of this decomposition for $p$-local spectra, which is indeed the case. The first to study such squares in topology was Sullivan in his 1970 MIT notes, where he constructed the analogous square for nilpotent spaces, see \cite[3.20]{sullivan_05}. This was later lifted up to spectra by Bousfield in \cite[2.9]{bousfield_1979_localization}, and takes the following form. 

If $\S_{(p)}$ denotes the $p$-local sphere spectrum, we have a spectral artithmetic fracture square 

\begin{center}
    \begin{tikzcd}
        \S_{(p)} \arrow[r] \arrow[d] & \S_p^\wedge \arrow[d]  \\
        H\Q \arrow[r]           & H\Q\otimes \S_p^\wedge
    \end{tikzcd}
\end{center}

where $\S_p^\wedge$ denotes the $p$-complete sphere. This also extends to any object $X\in \sp_{(p)}$, just like for $A\in D(\Z)_{(p)}$. 

We can then ask the same natural question as we did above: do these give all the minimal localizing subcategories of $\sp_{(p)}$? Recall that this was indeed the case before, but now, this is no longer true. In fact, we now have an infinite sequence of minimal localizing subcategories, indexed by a natural number $n$, interpolating between the rational spectra $\sp_{\Q}$ and the $p$-complete spectra $\sp_p^\wedge$. \footnote{In fact even more is true: By \cite{burklund-hahn-levy-schlank_23}, there are at least two such infinite sequences. We can make sure that there is a single such sequence if we translate over to tensor-triangulated ideals of compact objects, but for the above exposition, we have chosen to push these details under a huge telescope-shaped rug.}

We can identify these ``intermediary'' subcategories by an analysis of field objects. For $D(\Z)_{(p)}$ there are exactly two field objects associated to $\Z_{(p)}$, namely $\Q$ and $\F_p$. For $Sp_{(p)}$ we have a field object for any number $n\in \N\cup \{\infty\}$, usually denoted $K(n)$, or $K_p(n)$ if we want to remember the prime. As we have $K(0)=H\Q$ and $K(\infty) = H\F_p$, this sequence of field objects really forms an interpolation between the two field objects coming from algebra. 

\begin{notation}
    The object $K_p(n)$ is called the \emph{height $n$ Morava K-theory}. Its associated minimal localizing subcategory is the category of $K(n)$-local spectra, denoted $\SpK$.  
\end{notation}

These field objects $K_p(n)$ were constructed by Morava in the early 70's, and the categories $\SpK$ have been under intense study ever since. We do not cover precise constructions here and instead refer the interested reader to \cite{hovey-strickland_99}. 

\begin{proposition}
    \label{prop:properties-of-K(n)}
    Let $p$ be a prime and $n$ a natural number. The height $n$ Morava K-theory spectrum $K_p(n)$ is a complex oriented $\E_1$-ring spectrum with coefficients 
    $$K_p(n)_*:=\pi_* K_p(n) \simeq \F_p[v_n^{\pm}],$$ 
    with $|v_n|=2p^n-2$, whose associated formal group is the height $n$ Honda formal group. Furthermore, for any two spectra $X, Y\in \sp$, there is a Künneth isomorphism 
    $$K_p(n)_*(X\times Y)\simeq K_p(n)_*X\otimes_{K_p(n)_*} K_p(n)_*Y.$$
\end{proposition}

\begin{remark}
    While the $\E_1$-ring structure on $K_p(n)$ can be shown to be essentially unique, it does admit uncountably many $\E_1$-$MU$-algebra structures -- see \cite{angeltveit_2011}. 
\end{remark}

So, how are these new field objects related to the fracture squares above? If the $\SpK$'s form minimal localizing subcategories, then we should, by the previous discussion, expect there to be an infinite sequence of pullback squares converging to $\S_{(p)}$. This is indeed the case. 

Let $L_n := L_{K_p(0)\vee \cdots \vee K_p(n)}$. By Ravenel's smash product theorem, see \cite[7.5.6]{ravenel_92}, the functor $L_n\colon \sp_{(p)}\longrightarrow \sp_{(p)}$ is a smashing localization (\crefme), hence the relevant fracture squares for the two bottom cases $n=0$ and $n=1$ are given by
\begin{center}
    \begin{tikzcd}
        L_1\S \arrow[r] \arrow[d] & L_{K(1)}\S \arrow[d]  &L_2\S \arrow[r] \arrow[d] & L_{K(2)}\S \arrow[d]\\
        H\Q \arrow[r] & H\Q\otimes L_{K(1)}\S &L_1\S \arrow[r] &L_1\S \otimes L_{K(2)}\S
    \end{tikzcd}
\end{center}
making the general square have the form
\begin{center}
    \begin{tikzcd}
        L_n\S \arrow[r] \arrow[d] & L_{K_p(n)}\S \arrow[d]  \\
        L_{n-1}\S \arrow[r] & L_{n-1}\S\otimes L_{K_p(n)}\S 
    \end{tikzcd}    
\end{center}
This is called the \emph{chromatic fracture square}, see for example \cite[4.3]{hovey_95}. The spectra $L_n\S$ assemble into a tower 
$$\cdots \longrightarrow L_3\S \longrightarrow L_2\S \longrightarrow L_1\S \longrightarrow L_0 \S = L_\Q\S$$
called the chromatic filtration, and by the chromatic convergence theorem of Hopkins-Ravenel, see \cite[7.5.7]{ravenel_92}, we can recover $\S_{(p)}$ as the limit of this diagram. 


% \begin{remark}
%     \label{rm:chromatic-square-from-duality}
%     Reducing to the subcategory of $\sp_{(p)}$ containing the $L_n$-local spectra, we should then expect there to be a local duality diagram categorifying the chromatic fracture square. This is precisely the goal of \cref{ssec:monochromatic-duality}, but first, we need to understand this $L_n$-local category. 
% \end{remark}



%%%%%%%%%%%%%%%%%%%%%%%%%%%%%%%%%%%%%%%%%%%%%%%%%%%%%%%%%%%%%%%%%%%%%%%
%%%%%%%%%%%%%%%%%%%%%%%%%%%%%%%%%%%%%%%%%%%%%%%%%%%%%%%%%%%%%%%%%%%%%%%




\subsubsection{Morava \texorpdfstring{$E$}{E}-theories}
\label{ch1:sssec:morava-E-theories}

In the previous section, we obtained a localization functor $L_n$, which collected the information coming from height $0$ up to, and including, height $n$. This localization is good for many purposes, but when we later want to tie the homotopy theory to algebra, we need another approach. In particular, we want a spectrum $E$ such that localizing at $E$ is the same as using $L_n$, but with some additional better properties. There are several approaches to obtaining such a spectrum $E$, and the goal of this short section is to give a brief overview of the ones we will need later. We will assume general knowledge about formal groups -- all needed background can be found in \cite[Appendix 2]{ravenel_86}. 

\begin{remark}
    Let $p$ be a prime and $k$ be a perfect field of characteristic $p$. Lubin and Tate proved in \cite{lubin-tate_66} that for any formal group law $F$ of height $n$ over $k$, there is a universal deformation $\bar{F}$ over the \emph{Lubin-Tate ring} $E(k, F)=\W(k)[\![ u_1, \ldots, u_{n-1}]\!]$ of formal power series over the Witt vectors of $k$. Using the algebraic geometry of formal groups, Morava interpreted this universal deformation as a normal bundle over a formal neighborhood of the height $n$ Honda formal group law, leading to a spectrum $E^{Mor}_n$. 
\end{remark} 

Using the theory of manifolds with singularities developed by Baas-Sullivan (see \cite{baas_73a} and \cite{baas_73b}), Johnson and Wilson constructed in \cite{johnson-wilson_75} an alternative spectrum exhibiting the same information as Morava's spectrum. Using Landweber's exact functor theorem, we can obtain a simpler description. 

\begin{definition}
    Let $p$ be a prime, $n$ a natural number and $E(n)_* := \Z_{(p)}[v_1, \ldots, v_{n-1}, v_n^{\pm}]$. The ideal $(p, v_1, \ldots, v_{n-1})$ is a regular invariant ideal, meaning in particular that $E(n)_*$ is Landweber exact. In particular, there is a spectrum $E(n)$, called the height $n$ \emph{Johnson-Wilson theory}, with coefficients $E(n)_*$. 
\end{definition}

\begin{remark}
    \label{rm:K-as-quotient-of-E}
    The construction of $E(n)$ has the added benefit that quotienting by the maximal ideal $I_n = (p, v_1, \ldots, v_{n-1})$ gives $E(n)_*/I_n \cong \F_p[v_n^{\pm}] = K_p(n)_*$. This can also be suitably interpreted as a quotient of spectra. 
\end{remark}

\begin{definition}
    An $\E_1$-ring spectrum $R$ is said to be concentrated in degrees divisible by $q$ if $\pi_k R \cong 0$ for all $k \not = 0 \mod q$. 
\end{definition}

\begin{proposition}
    \label{prop:Johnson-Wilson-properties}
    Let $p$ be a prime and $n$ a natural number. Height $n$ Johnson-Wilson theory $E(n)$ is a complex oriented, Landweber exact, $\E_1$-ring spectrum concentrated in degrees divisible by $2p-2$. 
\end{proposition}

Later, using a $2$-periodic analogue of the universal deformation theory of Lubin and Tate, Hopkins and Miller constructed a $2$-periodic $\E_1$-version of Morava's spectrum, which was later enhanced to an $\E_\infty$-ring spectrum $E_n$ via Goerss--Hopkins theory, see \cite{goerss-hopkins_04} or \cite{pstragowski_vankoughnett_2022} for a modern treatment. In essence, Hopkins--Miller constructed a functor from pairs $(k, F)$ of a perfect field $k$ of characteristic $p$, together with a choice of height $n$ formal group law $F$, to even periodic ring spectra. For a specific choice of $(k, F)$, we can summarize the properties as follows.  

\begin{proposition}
    Let $p$ be a prime, $k$ a perfect field of characteristic $p$, and $F$ a formal group law of height $n$ over $k$. The spectrum $E(k,F)$ is a $2$-periodic, complex oriented, Landweber exact, $\E_\infty$-ring spectrum, such that $\pi_0 E(k,F)=\W(k)[\![ u_1, \ldots, u_{n-1}]\!]$ and the associated formal group law is the universal deformation of $F$. 
\end{proposition}


%Let $FGL$ denote the category of pairs $(k, F)$ for $k$ a perfect field of characteristic $p$ and $F$ a formal group of height $n$ over $k$. Morphisms in the category are pairs $(f,\phi)\colon (k, F)\to (k', F')$ where $f\colon k'\to k$ is a ring homomorphism and $\phi\colon F\to f^* F'$ is an isomorphism.  

%\begin{theorem}[{\cite[2.1]{rezk_98}}]
%    There is a functor $E(-,-)\colon FGL \longrightarrow Alg(\sp)$
%\end{theorem}

\begin{definition}
    For the specific choice $(k,F) = (\F_{p^n}, H_n)$ we simply write $E(\F_{p^n}, H_n) = E_n$, and call it the height $n$ \emph{Morava $E$-theory}. 
\end{definition}

\begin{remark}
    One can also study maps of ring spectra $E_n \longrightarrow K_n$ such that the induced map on homotopy groups is given by taking the quotient by the maximal ideal, just as in \cref{rm:K-as-quotient-of-E}. Such spectra $K_n$ are $2$-periodic versions of Morava $K$-theory and have been studied, for example, in \cite{hopkins-lurie_17} and \cite{barthel-pstragowski_2021}. 
\end{remark}

\begin{remark}
    One nice benefit with $E_n$ over $E(n)$ is that the former is $K(n)$-local, making its chromatic behavior even more interesting. In fact, the unit map $L_{K_p(n)}\S \longrightarrow E_n$ is a pro-Galois extension in the sense of \cite{rognes_08}, where the Galois group is the extended Morava stabilizer group $\G_n$, see \cite{devinatz-hopkins_2004}. We can, however, fix this by instead using a completed version $\widehat{E}(n)$, often called \emph{completed Johnson-Wilson theory}. It has most of the same properties as that of $E(n)$, except that it is $K_p(n)$-local and its coefficients are $p$-adic and $I_n$-complete: $\widehat{E}(n)_* \simeq \Z_p[v_1, \cdots, v_{n-1}, v_n^{\pm}]^\wedge_{I_n}$. 
\end{remark}

\begin{remark}
    An $\E_\infty$-version of Morava's original spectrum $E_n^{Mor}$ can be recovered from $E_n$ by taking the homotopy fixed points with respect to the Galois action $\mathrm{Gal}(\F_{p^n}/\F_p)\cong \Z/n$. Another alternative is to use $E_n^{h\F_p^\times}$. This spectrum is concentrated in degrees divisible by $2p-2$, hence serves as a nice $\E_\infty$-version of the $\E_1$-ring spectrum $E(n)$. This is the model of $E$ used, for example, in Barkan's monoidal algebraicity theory, see \cite{barkan_2023}. 
\end{remark}

We have now introduced several versions of $E$-theory, all in light of trying to understand the localization functor $L_n$. Hence, we round off this section by stating that the Bousfield localizations at any of the above $E$-theories are equivalent. 

\begin{proposition}[{\cite[1.12]{hovey_95}}]
    \label{prop:all-E-local-cats-are-equivalent}
    Let $p$ be a prime and $n$ a natural number. Then there are symmetric monoidal equivalences of stable $\infty$-categories 
    $$\sp_n \simeq \sp_{E(n)} \simeq \sp_{E(k,F)}\simeq \sp_{E_n} \simeq \sp_{\widehat{E}(n)}\simeq \sp_{E_n^{h\F_p^\times}}.$$
    In fact, if $E$ is any Landweber exact $v_n$-periodic spectrum, then $\sp_E$ is equivalent to the above categories. 
\end{proposition}

\begin{notation}
    We will use the common notation $\sp\np$ for any of the above categories. 
\end{notation}

\begin{remark}
    Note that even though the different models for $\sp\np$ are equivalent, some of them have non-equivalent associated module categories. For example, $\Mod_{E_n}\not \simeq \Mod_{E(n)}$, as the ring spectra $E_n$ and $E(n)$ have different periodicity -- the former is $2$-periodic while the latter is $(2p^n-2)$-periodic. Whenever such a distinction is relevant, we will make this explicit. 
\end{remark}

























\subsection{Hopf algebroids and their comodules}
\label{ch1:ssec:hopf-algebroids-and-their-comodules}



\begin{definition}
    \label{def:hopf-algebroid}
    A (graded) \emph{Hopf algebroid} is a cogroupoid object $(A, \Psi)$ in the category of graded commutative rings. 
\end{definition}

The use of Hopf algebroids in situations related to homotopy theory was studied by Ravenel in \cite[A.1]{ravenel_86} and later in more detail by Hovey in \cite{hovey_04}. 

\begin{remark}
    In the literature outside of topology, the assumptions of being commutative and graded are usually not present. But, as all our examples will be of this kind, we keep in line with the topological tradition. 
\end{remark}

\begin{definition}
    \label{def:comodule-over-hopf-algebroid}
    Let $(A, \Psi)$ be a Hopf algebroid. A \emph{$\Psi$-comodule} is an $A$-module $M$ together with a coassociative and counital map $\psi\colon M\longrightarrow M\otimes_A \Psi$. The category of comodules over $(A, \Psi)$ is denoted \emph{$\Comod_\Psi$}. 
\end{definition}

\begin{example}
    \label{ex:modules-as-discrete-Hopf-algebroids}
    For any commutative graded ring $A$, the pair $(A, A)$ is a called a \emph{discrete Hopf algebroid}. The category of comodules over this Hopf algebroid is the normal abelian category $\Mod_A$ of modules over $A$. 
\end{example}

\begin{remark}
    \label{rm:presenting-stacks}
    In algebraic geometry, Hopf algebroids are usually formulated dually as groupoid objects in affine schemes. The left and right unit maps $A\rightrightarrows \Psi$ induces a presentation of stacks $\mathrm{Spec}(\Psi)\rightrightarrows \mathrm{Spec}(A)$, and the category $\Comod_\Psi$ is equivalent to the category of quasi-coherent sheaves on the presented stack, see \cite[Thm 8]{naumann_07}. 
\end{remark}

\begin{construction}
    Given an Adams Hopf algebroid $(A, \Psi)$, we can define a discretization map $\epsilon\colon (A, \Psi)\longrightarrow (A, A)$, which is given by the identity on $A$ and the counit on $\Psi$. By \cite[A1.2.1]{ravenel_86} and \cite[4.6]{barthel-heard-valenzuela_2018} it induces a faithful exact forgetful functor $\epsilon_* \colon \Comod_\Psi \longrightarrow \Mod_A$ with a right adjoint $\epsilon^*$ given by $\epsilon^*(M)\simeq \Psi\otimes_A M$. A comodule in the essential image of $\epsilon^*$ is called an \emph{extended comodule}. 
\end{construction}

\begin{definition}
    \label{def:adams-hopf-algebroid}
    We say a Hopf algebroid $(A, \Psi)$ is of \emph{Adams type} if $\Psi$ is a filtered colimit $\colim_k \Psi_k \simeq \Psi$ of dualizable comodules $\Psi_k$.
\end{definition}

\begin{proposition}[{\cite[1.3.1, 1.4.1]{hovey_04}}]
    \label{prop:comod-is-sm-grothendieck}
    Let $(A,\Psi)$ be an Adams Hopf algebroid. Then, the category $\Comod_\Psi$ is a Grothendieck abelian category generated by the dualizable comodules. There is a symmetric monoidal product $-\otimes_\Psi -$, which on the underlying modules is the normal tensor product of $A$-modules. It has a right adjoint $\iHom_\Psi(-,-)$, making $\Comod_\Psi$ a closed symmetric monoidal category. 
\end{proposition}

As in \cref{ssec:local-duality}, we have certain objects that are especially important --- the compact objects and the dualizable objects. In Grothendieck abelian categories it is, in addition, important to understand the injective objects. This will also become important later in \cref{sec:exotic-algebraic-models}, as we will use injective objects to approximate other objects and to build certain spectral sequences.  

\begin{proposition}
    \label{rm:dualizable/compact-comodules}
    Let $(A, \Psi)$ be an Adams Hopf algebroid. A $\Psi$-comodule $M$ is dualizable if and only if its underlying $A$-module $\epsilon_* M$ is dualizable, i.e., it is finitely generated and projective. Similarly, a $\Psi$-comodule is compact if and only if its underlying $A$-module is compact, which coincides with being finitely presented. 
\end{proposition}
\begin{proof}
    The first claim is \cite[1.3.4]{hovey_04} and the second is \cite[1.4.2]{hovey_04}. 
\end{proof}

\begin{remark}
    \label{rm:dualizables-compact-generators}
    As colimits in $\Comod_\Psi$ are exact and are computed in $\Mod_A$, all the dualizable comodules are compact. Hence, the full subcategory of dualizable comodules is a set of compact generators for $\Comod_\Psi$. 
\end{remark}

\begin{proposition}[{\cite[2.1]{hovey-strickland_2005b}}]
    \label{rm:injective-comodules}
    Let $(A, \Psi)$ be an Adams Hopf alebroid. If $I$ is an injective object in $\Comod_\Psi$, then there is an injective $A$-module $Q$, such that $I$ is a retract of the extended comodule $\Psi\otimes_A Q$. 
\end{proposition}

\begin{remark}
    Note that as $\Comod_{\Psi}$ is Grothendieck abelian, it has enough injective objects. This allows us to construct injective resolutions and thus $\Ext$-groups, which we will see later, greatly help in computing information in stable homotopy theory. For example, the pair $(\F_2, \mathcal{A}_*)$ where $\mathcal{A}_*$ is the dual Steenrod algebra is a Hopf algebroid, and the groups $\Ext^s_{\mathcal{A}_*}(\F_2, \F_2)$ are used in the Adams spectral sequence to approximate homotopy groups of spheres, see \cite{adams_58}. 
\end{remark}

Given an Adams Hopf algebroid $(A, \Psi)$, we also have an associated derived category. By \cite[2.1.2, 2.1.3]{hovey_04} the category of chain complexes of $\Psi$-comodules, $\Ch_\Psi$, has a cofibrantly generated stable symmetric monoidal model structure. In \cite{barnes-roitzheim_2011} this model structure was modified slightly to more easily compare it to the periodic derived category, which we will consider more closely in \cref{ch:2}. The homotopy category associated to this model structure is the usual unbounded derived category $D(\Comod_\Psi)$ associated to the Grothendieck abelian category $\Comod_\Psi$. 
 

\begin{notation}
    We will use $D(\Psi)$ as our notation for the underlying symmetric monoidal stable $\infty$-category associated with the above model structure. The monoidal unit is $A$, treated as a chain complex centered in degree $0$.
\end{notation}

\begin{remark}
    We warn the reader that some authors use the notation $D(\Psi)$ to reffer to the above-mentioned periodic derived category of $(A, \Psi)$. This is the case, for example, in \cite{pstragowski_2021}. 
\end{remark}

We also get an induced discretization adjunction on the level of derived categories. 

\begin{proposition}
    Let $(A,\psi)$ be an Adams Hopf algebroid. Then the discretization adjunction $(\epsilon_*\dashv \epsilon^*)\colon \Comod_\Psi\longrightarrow \Mod_A$ induces an adjunction $(\epsilon_*\dashv \epsilon^*)\colon D(\Psi)\longrightarrow D(A).$
\end{proposition}
\begin{proof}
    This follows from the fact that $\Psi$ is flat over $A$, which implies that both $\epsilon_*$ and $\epsilon^*$ on the abelian categories are exact. 
\end{proof}



%%%%%%%%%%%%%%%%%%%%%%%%%%%%%%%%%%%%%%%%%%%%%%%%%%%%%%%%%%%%%%%%%%%%%%%
%%%%%%%%%%%%%%%%%%%%%%%%%%%%%%%%%%%%%%%%%%%%%%%%%%%%%%%%%%%%%%%%%%%%%%%



\subsection{Torsion and completion for comodules}
\label{ssec:torsion-and-completion-for-comodules}

There are two approaches to studying torsion and completion in $D(\Psi)$ -- one ``internal'' and one ``external''. The internal approach uses the classical theory of torsion objects in abelian categories, while the external uses local duality, as in \cref{thm:local-duality}. These two approaches are luckily equivalent in the situations we are interested in. 

We first review the abelian situation: the internal approach. We follow \cite{barthel-heard-valenzuela_2018} and \cite{barthel-heard-valenzuela_2020} in notation and results. 

\begin{definition}
    \label{def:I-power-torsion-module}
    Let $A$ be a commutative ring and $I\subseteq R$ a finitely generated ideal. The $I$-power torsion of an $A$-module $M$ is defined as
    $$T_I^A M = \{x\in M \mid I^k x = 0 \text{ for some } k\in \N\}.$$
    We say a module $M$ is \emph{$I$-torsion} if the natural map $T_I^A M\longrightarrow M$ is an isomorphism. 
\end{definition}

\begin{definition}
    Let $A$ be a commutative ring and $I\subseteq R$ a finitely generated ideal. The $I$-adic completion of an $A$-module $M$ is defined as
    $$C_I^A M = \lim_k A/I^k\otimes_A M.$$
    We say a module $M$ is \emph{$I$-adically complete} if the natural map $M\longrightarrow C_I^A M$ is an isomorphism. 
\end{definition}

\begin{remark}
    \label{rm:I-complete-vs-I-adically-complete}
    The resulting category of $I$-adically complete modules is not very well-behaved. The $I$-adic completion functor is often neither left nor right exact, and the category is often not abelian. To fix these issues, Greenlees and May introduced the notion of $L$-complete modules in \cite{greenlees-may_92}, using instead the zeroth left derived functor $L=\mathbb{L}_0 C_I^A$. Thus, it is also sometimes referred to as derived completion. One then defines \emph{$I$-complete} modules, also called $L$-complete or derived complete, to be those $R$-modules such that the natural map $M\longrightarrow L M$ is an equivalence. 
\end{remark}

\begin{notation}
    We denote the full subcategory consisting of $I$-power torsion $A$-modules by $\Mod_A^{I-tors}$ and the full subcategory of $I$-complete $A$-modules by $\Mod_A^{I-comp}$. 
\end{notation}

\begin{remark}
    The category $\Mod_A^{I-tors}$ is a Grothendieck abelian category. On the other hand, $\Mod_A^{I-comp}$ is abelian, but not Grothendieck in general. It is, however, the abelian category of contramodules over the $I$-adic completion of $A$, see \cite{positselski_2022_contramodules}. \citeme
\end{remark}

The inclusion of the full subcategory $\Mod_A^{I-tors}\hookrightarrow \Mod_A$ has a right adjoint, which coincides with the $I$-power torsion $T_I^A(-)$. This gives the $I$-power torsion another description as the colimit 
$$T_I^A M \cong \colim_k \iHom_A (A/I^k, M).$$

We want to extend the construction of $I$-torsion and $L$-complete modules to general Adams Hopf algebroids $(A,\Psi)$. For this, we need to choose sufficiently nice ideals that interact nicely with the additional comodule structure. 

\begin{definition}
    Let $(A, \Psi)$ be an Adams Hopf algebroid, and $I$ an ideal in $A$. We say $I$ is an {\defn invariant ideal} if, for any comodule $M$, the comodule $IM$ is a subcomodule of $M$. If $I$ is finitely generated by $(x_1, \ldots, x_r)$ and $x_i$ is non-zero-divisor in $R/(x_1, \ldots, x_{i-1})$ for each $i=1, \ldots, r$, then we say $I$ is {\defn regular}. 
\end{definition}

\begin{definition}
    \label{def:I-power-torsion-comodule}
    Let $(A,\Psi)$ be an Adams Hopf algebroid and $I\subseteq A$ a regular invariant ideal. The $I$-power torsion of a comodule $M$ is defined as 
    $$T_I^\Psi M = \{x\in M \mid I^kx = 0 \text{ for some } k\in \N\}.$$
    We say a comodule $M$ is {\defn $I$-torsion} if the natural map $T_I^\Psi M\longrightarrow M$ is an equivalence. 
\end{definition}

\begin{remark}
    \label{rm:torsion-comodules-grothendieck-monoidal}
    By \cite[5.10]{barthel-heard-valenzuela_2018} the full subcategory of $I$-torsion comodules, which we denote {\defn $\Comod_\Psi^{I-tors}$}, is a Grothendieck abelian category. It also inherits a symmetric monoidal structure from $\Comod_\Psi$. This also makes $\Mod_A^{I-tors}$ Grothendieck abelian and symmetric monoidal by \cref{ex:modules-as-discrete-Hopf-algebroids}. 
\end{remark}

\begin{remark}
    \label{rm:complete-comodules-not-abelian}
    Unfortunately, the corresponding versions of $I$-adically complete and $L$-complete comodules do not form abelian categories in general, as we can have problems with the comodule structure on certain cokernels.
\end{remark}

As for the case of modules, the inclusion $\Comod_\Psi^{I-tors}\hookrightarrow \Comod_\Psi$ has a right adjoint that corresponds to the $I$-power torsion construction $T_I^\Psi$. This, by \cite[5.5]{barthel-heard-valenzuela_2018} also has the alternative description
$$T_I^\Psi M \cong \colim_k \iHom_\Psi (A/I^k, M).$$

The construction of $I$-power torsion in $\Mod_A$ and $\Comod_\Psi$ are completely analogous, so one can wonder whether they agree on the underlying modules. This turns out to be the case. 

\begin{lemma}[{\cite[5.7]{barthel-heard-valenzuela_2018}}]
    \label{lm:torsion-comodule-iff-torsion-module}
    For any $\Psi$-comodule $M$ there is an isomorphism of $A$-modules $\epsilon_* T^\Psi_I M \cong T^A_I \epsilon_* M.$
    Furthermore, if an $A$-module $N$ is $I$-power torsion, then the extended comodule $\Psi\otimes_A N$ is $I$-power torsion. In particular, a $\Psi$-comodule $M$ is $I$-power torsion if and only if the underlying $A$-module is $I$-power torsion. 
\end{lemma}

As mentioned above, we will later make use of injectives in $\Comod_\Psi^{I-tors}$. Hence, we relate some facts about these. 

\begin{lemma}
    \label{lm:injectives-in-torsion-comodules}
    Let $(A, \Psi)$ be an Adams Hopf algebroid and $I$ a regular invariant ideal.
    \begin{enumerate}
        \item If $J$ is an injective in $\Comod_\Psi$ then $T_I^\Psi J$ is an injective in $\Comod_\Psi^{I-tors}$.
        \item There are enough injectives in $\Comod_\Psi^{I-tors}$.
        \item Any injective $J'$ in $\Comod_\Psi^{I-tors}$ is a retract of an object of the form $T_I^\Psi J$ for an injective $\Psi$-comodule $J$.
    \end{enumerate} 
\end{lemma}
\begin{proof}
    The first point is \cite[2.1.4]{brodmann-sharp_1998}, while the second is a consequence of $\Comod_\Psi^{I-tors}$ being Grothendieck abelian, as mentioned in \cref{rm:torsion-comodules-grothendieck-monoidal}. The third point is stated in the proof of \cite[3.16]{barthel-heard-valenzuela_2020}. 
\end{proof}

\begin{remark}
    \label{rm:injectives-in-torsion-modules}
    Choosing a discrete Hopf algebroid $(A,A)$, \cref{lm:injectives-in-torsion-comodules} implies that injectives in $\Mod_A^{I-tors}$ are retracts of $T_I^A(Q)$ for some injective $A$-module $Q$ and that $T_I^A$ preserves injectives. As noted in \cref{rm:injective-comodules}, an injective object in $\Comod_\Psi$ is a retract of an extended comodule of the form $\Psi\otimes_A Q$ for an injective $A$-module $Q$. This means that all injectives $J$ in $\Comod_\Psi^{I-tors}$ are retracts of $T_I^\Psi(\Psi\otimes_A Q)$ where $Q$ is an injective $A$-module. 
\end{remark}

\begin{remark}
    \label{rm:dualizable/compact-torsion-comodule}
    As colimits in $\Comod_\Psi^{I-tors}$ are computed in $\Comod_\Psi$, we have, similar to \cref{rm:dualizable/compact-comodules}, that an $I$-power torsion $\Psi$-comodule $M$ is dualizable (resp. compact) if and only if its underlying $A$-module is finitely generated and projective (resp. finitely presented). 
\end{remark}

\begin{lemma}
    \label{lm:torsion-comodules-generated-by-compacts}
    Let $(A,\Psi)$ be an Adams Hopf algebroid, where $A$ is noetherian and $I\subseteq A$ a regular invariant ideal. Then $\Comod_\Psi^{I-tors}$ is generated under filtered colimits by the compact $I$-power torsion comodules. 
\end{lemma}
\begin{proof}
    By \cite[3.4]{barthel-heard-valenzuela_2020} $\Comod_\Psi^{I-tors}$ is generated by the set 
    $$\mathrm{Tors}_\Psi^{fp}:=\{G\otimes A/I^k \mid G \in \Comod_\Psi^{fp}, k\geq 1\},$$
    where $\Comod_\Psi^{fp}$ is the full subcategory of dualizable $\Psi$-comodules. Since $I$ is finitely generated and regular, $A/I^k$ is finitely presented as an $A$-module, hence it is compact in $\Comod_\Psi^{I-tors}$ by \cref{rm:dualizable/compact-comodules} and \cref{rm:dualizable/compact-torsion-comodule}. As $A$ is noetherian, being finitely generated and finitely presented coincide. The tensor product of finitely generated modules is finitely generated, hence any element in $\mathrm{Tors}_\Psi^{fp}$ is compact. 
\end{proof}

\begin{remark}
    The assumption that the ring $A$ is noetherian can most likely be removed, but it makes no difference to the results in this paper.  
\end{remark}


\begin{notation}
    Since $\Comod_\Psi^{I-tors}$ is Grothendieck abelian we have an associated derived stable $\infty$-category $D(\Comod_\Psi^{I-tors})$ which we denote simply by $D(\Psi^{I-tors})$.
\end{notation}





We now move to the external approach, using local duality as in \cref{ch1:ssec:local-duality}. 

\begin{construction}
    \label{const:local-duality-hopf-algebroid}
    Let $(A, \Psi)$ be an Adams Hopf algebroid and $I\subseteq A$ a regular invariant ideal. Then $A/I$, treated as a complex concentrated in degree zero, is by \cite[5.13]{barthel-heard-valenzuela_2018} a compact object in $D(\Psi)$. Thus, $(D(\Psi), A/I)$ is a local duality context, and we can consider the corresponding local duality diagram
    \begin{equation*}
        \begin{tikzcd}
            & {D(\Psi)^{I-loc}} \\
            & {D(\Psi)} \\
            {D(\Psi)^{I-tors}} && {D(\Psi)^{I-comp}}
            \arrow["L_I^\Psi", xshift=-2pt, from=2-2, to=1-2]
            \arrow[xshift=2pt, from=1-2, to=2-2]
            \arrow["\Delta_I^\Psi", yshift=2pt, xshift=2pt, from=2-2, to=3-3]
            \arrow[yshift=-2pt, xshift=-1pt, from=3-3, to=2-2]
            \arrow["\Gamma^\Psi_I", yshift=-2pt, xshift=2pt, from=2-2, to=3-1]
            \arrow[yshift=2pt, xshift=-1pt, from=3-1, to=2-2]
            \arrow[bend left=35, dashed, from=3-1, to=1-2]
            \arrow[bend left=35, dashed, from=1-2, to=3-3]
            \arrow["\simeq"', swap, from=3-1, to=3-3]
        \end{tikzcd}    
    \end{equation*}
    where we have used the superscript $I$ instead of $A/I$ for simplicity. This gives, in particular, a definition of $I$-torsion objects in $D(\Psi)$ as $D(\Psi)^{I-tors}$. 
\end{construction}

Our goal was to give two constructions and prove that they were equal in the cases we were interested in. 

\begin{lemma}[{\cite[3.7(2)]{barthel-heard-valenzuela_2020}}]
    \label{lm:derived-torsion-if-homology-torsion}
    Let $(A,\Psi)$ be an Adams Hopf algebroid and $I\subseteq A$ a regular invariant ideal. There is an equivalence of categories 
    $$D(\Psi)^{I-tors}\simeq D(\Psi^{I-tors}).$$ 
    Furthermore, an object $M\in D(\Psi)$ is $I$-torsion if and only if the homology groups $H_* M$ are $I$-power torsion $\Psi$-comodules.
\end{lemma}

\begin{remark}
    \label{rm:right-completed-derived-category}
    One can wonder whether the same is true for the $I$-complete derived category, but this is unfortunately not true as $\Comod_\Psi^{I-comp}$ is not abelian. A partial result can, however, be recovered for discrete Hopf algebroids $(A, A)$. 
\end{remark}

We follow \cite{barthel-heard-valenzuela_2020} in the following construction. 

\begin{construction}
    \label{const:completed-derived-category}
    Recall that $\Mod_A^{I-comp}$ denotes the category of $L$-complete $A$-modules for $I\subseteq A$ a regular ideal. By \cite[2.11]{barthel-heard-valenzuela_2020} the category has enough projectives, hence by \cite[1.3.2]{Lurie_HA} we can associate to it the right bounded category $D^-(\Mod_A^{I-comp})$. This has a by \cite[1.3.2.19, 1.3.3.16]{Lurie_HA} a left complete $t$-structure with heart equivalent to $\Mod_A^{I-comp}$. We can then form its right completion, which we denote $\overline{D}(\Mod_A^{I-comp})$, and call the completed derived category of $\Mod_A^{I-comp}$. 
\end{construction}

This is what allows us the partial version of \cref{lm:derived-torsion-if-homology-torsion} in the case of $I$-completion. 

\begin{proposition}[{\cite[3.7(1)]{barthel-heard-valenzuela_2020}}]
    \label{prop:pulling-out-completion}
    Let $A$ be a commutative ring and $I\subseteq A$ a regular ideal. Then, there is an equivalence 
    $$D(\Mod_A)^{I-comp}\simeq \overline{D}(\Mod_A^{I-comp}),$$
    where the former category is the full subcategory of $A/I$-complete objects in $D(\Mod_A)$ while the latter is the completed derived category of $\Mod_A^{I-comp}$. 
\end{proposition}





































