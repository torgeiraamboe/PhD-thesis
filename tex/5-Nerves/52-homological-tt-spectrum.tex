
\section{The homological Balmer spectrum}
\label{sec:homological-Balmer-spectrum}

In this section we briefly recall the theory of modules over a $tt$-category, as well as introduce their stable enhancements. We then use this to introduce the new kid in class, the \emph{homological Balmer spectrum}. 

\begin{definition}
    A \emph{big $tt$-category} is a presentably symmetric monoidal stable $\infty$-category $\C$ that is rigidly compactly generated. Its compact-dualizable objects $\Co$ form a \emph{rigid small $tt$-category}, in other words a small symmetric monoidal stable $\infty$-category where every object is dualizable. 
\end{definition}

For the rest of the paper $\C$ will denote a big $tt$-category in the above sense.

\begin{definition}
    A full subcategory $P\subseteq \Co$ is said to be \emph{thick} if it is closed under retracts, cofiber sequences and finite colimits. It is a \emph{thick $\otimes$-ideal} if $t\otimes c\in P$ for all $p\in P$ and $c\in \Co$. It is further said to be \emph{prime} if $p\otimes q \in P$ implies either $p\in P$ or $q\in P$. 
\end{definition}

The \emph{Balmer spectrum} of $\C$, also called the $tt$-spectrum, is the set of prime thick $\otimes$-ideals in $\Co$. We can equip it with a topology, defined by having a basis of closed sets determined by the \emph{support} of compact objects: 
\[\Supp(c)= \{P\in \Spc(\Co)\mid c\not\in P\}.\]
By \cite[2.9]{balmer_2005} this topology makes $\Spc(\Co)$ a $T_0$-space, meaning that all points are topologically distinguishable. In fact, $\Spc(\Co)$ is a \emph{spectral space}. 

\begin{remark}
    \label{rm:homotopy-category-spectrum}
    Classically $\C$ is not an $\infty$-category, so one might wonder whether this setup is different from the classical one. For any big $tt$-category $\C$, its homotopy category $h\C$, given by truncating all of its mapping spaces, is a big tensor-triangulated category in the classical sense. The inclusion $h\Co\to \Co$ and projection $\Co \to h\Co$ gives a one-to-one equivalence on prime $\otimes$-ideals, meaning there is a homeomorphism
    \[\Spc(\Co)\simeq \Spc(h\Co).\]
    In particular, there is really no need to worry about the distinction between $\C$ and $h\C$, unless one wants to study tautological examples of categories that do not omit $\infty$-categorical enhancements. The author is not aware of any big tensor-triangulated category where this is the case. In any case, all of the results in this paper still hold when using $h\C$ instead of $\C$, hence any reader not familiar with $\infty$-categories can safely ignore the difference between $\C$ and $h\C$. 
\end{remark}


\subsection{Modules and homological primes}

For a more comprehensive introduction to the Freyd envelope and the related module category, see \cite[Appendix A.]{balmer_krause_stevenson_2020}. We review mostly the parts we will need for our proofs. 

\begin{definition}
    The category of $\Co$-modules, denoted $\FE$, is the category of additive functors $\C^{\omega, \mathrm{op}} \to \Ab$. We will reffer to this as the category of \emph{integral presheaves} on $\Co$. 
\end{definition}

\begin{remark}
    The above notation for this category is not standard in the $tt$-geometry literature. It is usually written $\Mod\!-\!\Co$, or simply $\A$. We have chosen the above notation to induce a sense of connection to its stable enhancement, which we will introduce shortly. 
\end{remark}

The category $\FE$ is a Grothendieck abelian category, and admits a symmetric monoidal structure via Day-convolution such that $-\otimes -$ preserves colimits separately in each variable. The formation of representable functors gives a symmetric monoidal functor $y\: \C \to \FE$, defined by $y(C) = \Hom_\C(-, C)_{\mid \Co}$, which we call the discrete Yoneda embedding. This is by \cite[2.4]{krause_2000} the universal homology theory on $\C$, in the sense that any other homology theory factors through it. 

The compact objects in $\FE$ are the finitely presented presheaves, i.e., those functors $M$ that is the cokernel of a map $y(C)\to y(D)$. The full subcategory of finitely presented integral presheaves, denoted $\FEo$, is precisely the \emph{Freyd envelope} of $\C$, as studied for example in \cite[Chapter 5]{neeman_2014}. This is also an abelian category, which inherits a symmetric monoidal structure from $\FE$ preserving finite colimits separately in each variable. Any object $M\in \FE$ is a filtered colimit of finitely presented representable functors $y(c_\alpha)$ for $c_\alpha \in \Co$, and any finitely presented object is a finite colimit of finitely presented representables --- or equivalently, a quotient of a direct sum of these. The discrete Yoneda embedding $y$ restricts to a symmetric monoidal functor on compact objects, giving a commutative diagram of symmetric monoidal functors

\begin{center}
    \begin{tikzcd}
        \C \arrow[r, "y"]                  & \FE                  \\
        \Co \arrow[r, "y"] \arrow[u, hook] & \FEo \arrow[u, hook]
    \end{tikzcd}
\end{center}

\begin{remark}
    \label{rm:recovering-classical-using-h}
    We note that this setup is slightly different from the classical story, as here $\C$ is an $\infty$-category. This means, for example, that $y$ is neither conservative or fully faithful. We can, however, recover the classical situation as follows. As $\Ab$ is a $1$-category, any presheaf of abelian groups on $\C$ factors through the homotopy category $h\C$, as in \cref{rm:homotopy-category-spectrum}. In fact, the projection $\C\to h\C$ gives an equivalence 
    \[\FE \simeq \mathrm{P}_\Sigma^\Ab(h\Co),\]
    see \cite[2.34]{patchkoria-pstragowski_2021}. Whenever $\C$ is a big $tt$-category, the homotopy category $h\C$ is a big tensor-triangulated category in the classical sense, meaning that on the level of homotopy categories the discrete Yoneda embedding gives a fully faithful conservative functor 
    \[y\: h\Co\to \FEo\]
    via the Yoneda lemma. 
\end{remark}

% \begin{lemma}[{\citeme, see \cite[2.45]{patchkoria-pstragowski_2021}}]
%     The Freyd envelope $\FEo$ is a Frobenius category, and the representable functors $y(c)$ are both projective and injective, and generates $\FEo$ under finite limits. 
% \end{lemma}

% \begin{remark}
%     All of the representable functors are in particular $\otimes$-flat, and the Freyd envelope $\FEo$ is also, almost by definition, generated by the representables under finite colimits. 
% \end{remark}

\begin{definition}
    A \emph{homological prime} in $\C$ is a maximal thick $\otimes$-ideal in $\FEo$. The \emph{homological spectrum} of $\C$, denoted $\Spch(\Co)$, is the set of homological primes in $\C$. 
\end{definition}

\begin{remark}
    By \cref{rm:recovering-classical-using-h} this coincides with the standard use of the terminology, and the use of $\infty$-categories gives no extra information. In particular, there is an isomorphism $\Spch(\Co)\simeq \Spch(h\Co)$. 
\end{remark}

As mentioned in the introduction, Balmer proves in \cite[3.3]{balmer_2019} that pulling back a homological prime along the discrete Yoneda embedding $y$, gives a set-theoretical map $\phi\: \Spch(\Co)\to \Spc(\Co)$. The initial topology on $\Spch(\Co)$ for this map is given by declaring that there is a basis for the closed sets of $\Spch(\Co)$ given by $\phi^{-1}(\Supp(c))$ for all $c\in \Co$. For a given compact $c$, this closed set is called the \emph{homological support} of $c$. 
\[\Supph(c) := \phi^{-1}(\Supp(c)) = \{B\in \Spch(\Co)\mid y(c)\not\in B\}.\]
Balmer shows in \cite{balmer_2020} that this is in fact a support theory for $\Co$, which, by the universal property of the Balmer spectrum --- see \cite[3.2]{balmer_2005} --- gives a continuous map 
\[\Spch(\Co) \to \Spc(\Co),\]
coinciding exactly with the comparison map $\phi$. 

By \cref{rm:homotopy-category-spectrum} and \cref{rm:recovering-classical-using-h} there is no distinction between the classical setup and the $\infty$-categorical setup presented here.












\subsection{The stable Freyd envelope}

Before introducing stable enhancements we recall some facts about prestable ones, as set up in \cite{patchkoria-pstragowski_2021} by Patchkoria and Pstr\a{}gowski. Their setup is much more general than what we will need, but we cover only the theory in our particular case. 

\begin{definition}
    The category of prestable $\Co$-modules, denoted $\pFE$, is the category of space-valued additive functors $\C^{\omega, \mathrm{op}} \to \S$. We will reffer to this as the category of \emph{spatial presheaves} on $\Co$. 
\end{definition}

This is by \cite[C.1.5.9, C.1.5.10]{lurie_SAG} a complete Grothendieck prestable $\infty$-category. It is also a presentably symmetric monoidal $\infty$-category via Day convolution. The heart (in the prestable sense) of $\pFE$ is the full subcategory of discrete objects, which is precicely the category of integral presheaves $\FE$. There is a space-valued Yoneda embedding $\yo\: \C \to \pFE$ (read: yo), given by sending an object $C$ to the mapping space functor $\Map(-,C)_{\mid \Co}$. 

Postcomposing $\yo$ with the map 
\[\pi_0 = \tau\leqz \: \pFE \to \FE\]
gives us precicely the discrete Yoneda embedding $y$. In other words, there is a factorization of $y$ as 
\[\C \overset{\yo}\to \pFE \overset{\pi_0}\to \FE.\]
If one uses the homotopy category $h\C$ instead of $\C$, as is perhaps more standard in the $tt$-geometry literature, we have $\pi_k \yo c \simeq 0$ for all $k\neq 0$, hence the connection is perhaps even clearer in this setting. In the language of Patchkoria--Pstr\a{}gowski $\yo$ is a prestable enhancement of $y$. 

\begin{definition}
    \label{def:prestable-enhancement}
    A pair $(F, \E)$ consisting of a Grothendieck prestable category $\E$ and a functor $\C \to \E$ preserving filtered colimits and finite limits, is said to be a \emph{prestable enhancement} of $y$ if $\pi_0 \circ F \simeq y$. 
\end{definition}

\begin{definition}
    We define the \emph{prestable Freyd envelope} of $\C$ to be the full subcategory of compact spatial presheaves, $\sFEo$.
\end{definition}

\begin{remark}
    Our notion of the prestable Freyd envelope coincides with Patchkoria and Pstr\a{}gowski's \emph{perfect} prestable Freyd envelope of $\Co$. This means, that $\pFEo$ is the smallest subcategory of $\pFE$ closed under finite colimits that contains the representable presheaves $\nu(c)$. 
\end{remark}

\begin{remark}
    \label{rm:pFE-is-rigidly-compactly-generated}
    The prestable Freyd envelope generates $\pFE$ under filtered colimits, as noted in the proof of \cite[6.47]{patchkoria-pstragowski_2021}. The space-valued Yoneda embedding is, almost by definition, symmetric monoidal, and does preserve compacts. Hence, as $\C$ is assumed to be rigidly compactly generated, the objects $\nu(c)$ are also compact-dualizable. In particular, the unit $\yo \1$ is a compact-dualizable. Since the compact-dualizable objects $\yo c$ generate $\pFE$, and the unit is compact, $\pFE$ is a rigidly compactly generated presentably symmetric monoidal Grothendieck prestable $\infty$-category.  
\end{remark}

\begin{remark}
    One of the central features of the space-valued Yoneda embedding $\yo$ is that it satisfies a similar universal property as the discrete Yoneda embedding $y$: it is the universal prestable enhancement of $y$, see \cite[6.40. 6.47]{patchkoria-pstragowski_2021}. 
\end{remark}

For the non-$\infty$-categorical readers we have included the following result, which actually makes some of the story simpler by using the $tt$-category $h\C$. 

\begin{lemma}
    \label{lm:on-homotopy-categories-pi0-enough}
    A spatial presheaf $X\in \mathrm{P}_\Sigma^\S(h\Co)$ is compact if and only if $\pi_0 X$ is compact in $\FE$, i.e., $\pi_0 X$ is finitely presentable. 
\end{lemma}
\begin{proof}
    This is \cite[4.37]{patchkoria-pstragowski_2021}. 
\end{proof}

Let us now finally turn our attention to the main player of this paper: the stable enhancement of the discrete Yoneda embedding. By a \emph{stable enhancement} of $y$ we will mean a functor $F\: \C\to \D$, such that $\D$ has a $t$-structure and $\tau\geqz \circ F$ is a prestable enhancement in the sense of \cref{def:prestable-enhancement}. 

There are two natural ways of obtaining such a stable enhancement. We will define them both, and see that they are in a sense very compatible. The base for both of them is the stabilization of $\pFE$, which by \cite[2.13]{pstragowski_2022} is the category of presheaves of spectra on $\Co$. 

\begin{definition}
    The category of stable $\Co$-modules, denoted $\sFE$, is the category of additive functors $\C^{\omega, \mathrm{op}} \to \Sp$. We will reffer to this as the category of \emph{spherical presheaves} on $\Co$. 
\end{definition}

\begin{remark}
    The category of spherical presheaves on $\C$ has by \cite[2.16]{pstragowski_2022} a natural $t$-structure, which is right complete and compatible with filtered colimits and the monoidal structure. Its heart is given by $\FE$. This also follows from \cite[C.1.5.10]{lurie_SAG}, as $\sFE$ is the stabilization of $\pFE$, which is a Grothendieck prestable category. In the language of \cref{ch:4}, $\sFE$ is a $t$-stable category. 
\end{remark}

This means that there is an adjunction 
\begin{center}
    \begin{tikzcd}
        \pFE \arrow[r, yshift=2pt, "\Sigma^\infty"] & \sFE \arrow[l, yshift=-2pt, "\Omega^\infty"]
    \end{tikzcd}
\end{center}
where $\Sigma^\infty$ is symmetric monoidal and fully faithful, with essential image $\sFE\geqz$. Under this equivalence, $\Omega^\infty$ is the connected cover $\tau\geqz\: \sFE \to \sFE\geqz$, which is a lax monoidal functor. 

\begin{remark}
    The heart-valued homotopy groups $\pi_k$ associated to the above-mentioned $t$-structure, are computed level-wise, in the sense that $(\pi^\heart_k M)(c)\simeq \pi_k (M(c))$, where the latter $\pi_k$ denotes the homotopy groups in spectra. 
\end{remark}

The first stable enhancement to the discrete Yoneda embedding $y$ is given by the restricted \emph{spectral Yoneda embedding}, which we denote by $Y$. It is defined as $Y(c) = \map(-,c)_{\mid \Co}$, where $\map$ denotes the mapping spectrum in $\C$. It is a fully faithful symmetric monoidal functor that preserves all limits and colimits. In particular, it is a $tt$-functor. 

The second stable enhancement is given by the infinite suspension object of the prestable enhancement $\yo$ of $y$, i.e., the composition
\[\C \overset{\yo}\to \pFE \overset{\Sigma^\infty}\to \sFE.\]
Essentially it is obtained from $\yo$ by recognizing that an additive presheaf of spaces is a presheaf of infinite loop spaces, which we can deloop to a presheaf of connective spectra, see \cite[2.20]{pstragowski_2022}. This is often denoted by $\nu$, and called the \emph{synthetic analog}, due to its original appearance for synthetic spectra in \cite{pstragowski_2022}. As both $\yo$ and $\Sigma^\infty$ are symmetric monoidal, also $\nu$ is. However, it is not an exact functor, hence not a $tt$-functor. This is because $\nu (c)$ is connective, and the fiber of a map between connective objects need not be connective in $\sFE$. 

\begin{remark}
    \label{rm:nu-is-connected-cover-of-Y}
    Due to the equivalence $\pFE \simeq \sFE\geqz$, we can identify $\nu$ as the connected cover of $Y$, as the connective spectrum underlying the infinite loop space $\yo(c)(d)$ is $\tau\geqz Y(c)(d)$. Hence, there is an equivalence $\tau\geqz Y(c)\simeq \nu(c)$ for all $c\in \Co$. This means that also $\nu$ is a stable enhancement of $y$. We can summarize this in the following commutative diagram:
    \begin{center}
        \begin{tikzcd}
            \C \arrow[rd, "y"'] \arrow[r, "\yo"] & \pFE \arrow[d, "\pi_0"] \arrow[r, "\Sigma^\infty"] & \sFE \arrow[ld, "\pi_0^\heart"] \\
                                                 & \FE                                                &                                
        \end{tikzcd}
    \end{center}
\end{remark}

The synthetic analogs of compact objects $c\in \C$ also form a collection of generators for $\sFE$, just as in \cref{rm:pFE-is-rigidly-compactly-generated}. 

\begin{lemma}
    The category $\sFE$ is rigidly compactly generated, with generators given by $\nu (c)$ for $c\in \Co$. 
\end{lemma}
\begin{proof}
    The fact that the $\nu(c)$'s are compact and generate $\sFE$ under filtered colimits is completely analogous to the story in synthetic spectra, see \cite[4.14]{pstragowski_2022}. As $\C$ is rigidly compactly generated, and $\nu$ is symmetric monoidal on compacts, the objects $\nu(c)$ are also dualizable. As the unit $\nu(\1)$ is compact, this implies that $\sFE$ is rigidly compactly generated. 
\end{proof}

\begin{remark}
    This means that the compact objects $\sFEo$ is generated by the collection of $\nu(c)$ under finite colimits, similar again to how $\pFE$ is generated by the perfect prestable Freyd envelope, see \cref{rm:pFE-is-rigidly-compactly-generated}. 
\end{remark}








Having defined the stable replacement of $\FE$, we can now define the new spectrum of interest. The name and notation is simply a combination of the two things this object combines: the Balmer spectrum of $\C$ and the homological spectrum of $\C$. 

\begin{definition}
    \label{def:homological-Balmer-spectrum}
    The \emph{homological Balmer spectrum} of $\C$, which we denote by $\SpcH(\Co)$, is defined to be the Balmer spectrum of its stable Freyd envelope: 
    \[\SpcH(\Co):= \Spc(\sFEo).\] 
\end{definition}

As $\SpcH(\Co)$ is a Balmer spectrum we can equip it with a topology with a basis of closed sets given by the support of compact objects: 
\[\Supp(m) := \{P\in \SpcH(\Co)\mid m\not\in P\}.\]
The two stable enhancements of $y$ give us two maps of Balmer spectra. 

\begin{proposition}
    \label{prop:spectral-yoneda-surjective}
    The spectral Yoneda embedding $Y\: \C\to \sFE$ induces a surjective continuous map $f\: \SpcH(\Co)\to \Spc(\Co)$. 
\end{proposition}
\begin{proof}
    Any symmetric monoidal exact functor between rigidly compactly generated categories induces a continuous map on Balmer spectra. Surjectivity follows from \cite[1.4]{barthel_castellana_heard_sanders_2024}, as $Y$ is conservative. 
\end{proof}

Even though, as mentioned above, $\nu$ is not an exact functor, it also induces a map on Balmer spectra. 

\begin{lemma}
    \label{lm:synthetic-analog-pulls-back-primes}
    Let $P$ be a prime thick $\otimes$-ideal in $\sFEo$, then $\nu^{-1}P$ is a prime thick $\otimes$-ideal in $\Co$. 
\end{lemma}
\begin{proof}
    As $\nu$ is symmetric monoidal and preserves compacts, we only need to show that the inverse image of a cofiber sequence is again a cofiber sequence. This holds as $\nu (x)\rightarrow \nu (y)\rightarrow \nu (z)$ is a cofiber sequence if and only if $x\rightarrow y\rightarrow z$ is a cofiber sequence, and the map $y\rightarrow z$ is a $y$-epimorphism.  
\end{proof}

\begin{remark}
    As $\nu$ is a conservative functor, \cref{lm:synthetic-analog-pulls-back-primes} gives a surjective map $f\: \SpcH(\Co)\to \Spc(\Co)$ from the Homological Balmer spectrum to the Balmer spectrum of $\C$. We have $f^{-1}(\Supp(c) = \Supp(\nu (c)))$, which is a closed set in $\SpcH(\Co)$, meaning that $f$ is continuous. 
\end{remark}