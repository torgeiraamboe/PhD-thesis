
\section{Central ideas}
\label{ch0:sec:Central-ideas}

As a backdrop for this entire thesis lies the ubiquitous concept of \emph{$\infty$-categories}, as developed by Joyal, Lurie and others---the canonical references being \cite{joyal_02}, \cite{lurie_09} and \cite{Lurie_HA}. We will assume familiarity with $\infty$-categories and their associated standard constructions, and use them all willy-nilly throughout the rest of the thesis. We will sometimes omit the prefix; the distinction between $\infty$-categories and classical categories should hopefully be clear from context. Almost all of the $\infty$-categories considered will be \emph{presentable}, in the sense of \cite[Chapter 5]{lurie_09}. 

\begin{example}
    \index{Space}
    The archetypal example is the category of \emph{spaces}, denoted $\Spaces$. This is an $\infty$-categorical version of the classical category of topological spaces. 
\end{example}

The $(\infty, 2)$-category of presentable $\infty$-categories, where the morphisms are the colimit-preserving functors, is denoted $\PrL$. It has a symmetric monoidal structure via the Lurie tensor product $\otimes^L$, allowing us to make the following definition, which will be one of the main objects of study for this thesis. 

\begin{definition}
    \index{Presentable $\infty$-category!Symmetric monoidal}
    \index{Commutative algebra}
    A presentable $\infty$-category $\C$ is \emph{presentably symmetric monoidal} if it is a commutative algebra in $\PrL$ under the Lurie tensor product. 
\end{definition}

Any such category $\C$ has a symmetric monoidal structure, with the property that the tensor product in $\C$ preserves colimits separately in each variable. The symmetric monoidal product will usually be denoted $-\otimes_\C -$, and its associated unit by $\1_\C$. 

We will also assume knowledge about \emph{stable} $\infty$-categories\index{Presentable $\infty$-category!Stable}, which are $\infty$-categorical enhancements of triangulated categories. One can, for example, construct stable $\infty$-categories via \emph{stabilization}\index{Stabilization}. The stabilization of a pointed $\infty$-category $\mathcal{E}$ is defined by formally inverting the desuspension functor $\Omega$:
\[\Sp(\mathcal{E}) = \colim (\cdots\overset{\Omega}\to\mathcal{E}\overset{\Omega}\to\mathcal{E}\overset{\Omega}\to\mathcal{E}).\] 
The $(\infty, 2)$-category of presentable stable $\infty$-categories and exact colimit-preserving functors, which we denote by $\PrLs$, inherits a symmetric monoidal structure from $\PrL$. An $\infty$-category $\C$ is a \emph{presentably symmetric monoidal stable $\infty$-category} if it is a commutative monoid in $\PrLs$. This means that it is presentably symmetric monoidal, and the tensor product preserves the stable structure. 

\begin{example}
    \index{Spectra}
    \index{Sphere spectrum}
    The stabilization of the $\infty$-category of spaces is the $\infty$-category of \emph{spectra}, denoted $\Sp$. It is the unit for the Lurie tensor product on $\PrLs$, and it is the initial presentably symmetric monoidal stable $\infty$-category. The unit of the symmetric monoidal structure is the \emph{sphere spectrum} $\S$, which is the suspension spectrum of $S^0$.  
\end{example}

\begin{example}
    Another example is the derived $\infty$-category of abelian groups, $\Der(\Z)$. This is an $\infty$-categorical version of the classical triangulated derived category of $\Z$. It is a presentably symmetric monoidal stable $\infty$-category, and the unit is the integers $\Z$, treated as a chain complex concentrated in degree $0$. 
\end{example}

The derived $\infty$-category of $\Z$, as well as the $\infty$-categories of spaces, spectra, and many other interesting $\infty$-categories, satisfy some even nicer conditions than merely being presentable: they have an explicit collection of generators, which satisfy some ``smallness'' condition, defined as follows. 

\begin{definition}
    \label{ch0:def:compact-object}
    \index{Compact object}
    An object $X\in \C$ is said to be \emph{compact} if the functor $\Hom_\C(X,-)$ commutes with filtered colimits. The full subcategory of compact objects in $\C$ will be denoted $\Co$.
\end{definition}

\begin{example}
    \index{Spectra!Finite}
    The compact objects in $\Spaces$ are the \emph{finite spaces}, which correspond to the classical finite CW-complexes. The compact generators in $\Sp$ are the \emph{finite spectra}, where a spectrum is finite if it is the desuspension of a suspension spectrum $\Sigma^{-n}\Sigma^\infty K$ for some number $n$, where $K$ is a finite space. The compact objects in $\Der(\Z)$ are the \emph{perfect complexes}, which are the bounded complexes of finitely generated projective abelian groups. 
\end{example}

\begin{definition}
    \label{ch0:def:compactly-generated-category}
    \index{Compactly generated}
    A presentable $\infty$-category $\C$ is \emph{compactly generated} if $\Co$ generates $\C$ under filtered colimits. 
\end{definition}

\begin{remark}
    All three of the categories $\Spaces$, $\Sp$ and $\Der(\Z)$ are compactly generated by their respective collection of compact objects. In fact, they are all compactly generated by a single object, being $S^0$, $\S$ and $\Z$ respectively. 
\end{remark}

In the presence of symmetric monoidal structures we have another ``smallness'' condition, slightly different from being compact. As $\C$ is assumed to be presentable, we know that the symmetric monoidal structure preserves colimits separately in each variable. Hence, by the adjoint functor theorem, \cite[5.5.2.9]{lurie_09}, the functor $X\otimes (-)$ has a right adjoint, denoted $\iHom_\C(X,-)$. This equips $\C$ with an \emph{internal hom}\index{Internal hom} 
\[\iHom_\C(-,-)\: \C\op\times \C\to \C.\] 
We can then construct a duality duality functor 
\[(-)^\vee:=\iHom_\C(-,\1_\C)\: \C\op \to \C,\] 
which we will call the \emph{linear dual}\index{Linear dual}, or sometimes just \emph{the dual}. The unit map on $X^\vee$ induces by adjunction a map $ev_X\: X\otimes X^\vee \to \1_\C$, called the evaluation map. For any $Y\in \C$ this gives a map 
\[ev_X \otimes \id_Y\: X\otimes X^\vee \otimes Y\to Y.\]

\begin{definition}
    \label{ch0:def:dualizable-object}
    \index{Dualizable object}
    An object $X\in \C$ is \emph{dualizable} if for any other object $Y$, the map $X^\vee\otimes Y\to \iHom(X,Y)$, adjoint to the map $ev_X\otimes \id_Y$, is an equivalence. The full subcategory of dualizable objects in $\C$ will be denoted $\C\dual$. 
\end{definition}

In a certain sense, being compact is about being small with respect to colimits, while being dualizable is about being small with respect to the monoidal structure. In very well-behaved categories, these two notions of smallness coincide. 

\begin{definition}
    \label{ch0:rigidly-generated-category}
    \index{Compactly generated!Rigid}
    A presentably symmetric monoidal stable $\infty$-category $\C$ is \emph{rigidly compactly generated} if it is compactly generated and $\Co\simeq \C\dual$. 
\end{definition}

\begin{example}
    A good example is again our favorite stable $\infty$-category $\Sp$. Every compact object---being the finite spectra---is dualizable, and conversely, every dualizable object is compact. These also generate $\Sp$, hence it is a rigidly compactly generated symmetric monoidal stable $\infty$-category. Another example is the derived category $\Der(\Z)$, which is rigidly compactly generated by the perfect complexes. 
\end{example}

\begin{remark}
    \label{ch0:rm:compacts-equal-dualizable}
    As shown in \cite[2.1.3]{hovey-palmiery-strickland_97}, a presentably symmetric monoidal $\infty$-category $\C$ is rigidly compactly generated if $\C$ is compactly generated by dualizable objects, and the unit $\1_\C$ is compact. 
\end{remark}







\subsection{Localizing subcategories and ideals}
\label{ch0:ssec:localizing-subcategories-and-ideals}

If we were to assign this thesis a single protagonist, it would be the idea of a localizing subcategory. It will heavily feature in all the different parts of the thesis: 
\begin{enumerate}
    \item In \hyperref[ch1]{Paper I} we study how a specific localizing subcategory, appearing in chromatic homotopy theory, interacts with a specific homological functor.
    \item In \hyperref[ch2]{Paper II} we study how, in certain situations, the category of comodules over a coalgebra in a stable $\infty$-category forms a localizing subcategory. 
    \item In \hyperref[ch3]{Paper III} we prove a classification for certain localizing subcategories along nicely behaved $t$-structures on stable $\infty$-categories. 
\end{enumerate}

Given a presentable stable $\infty$-category $\C$, one should think about a localizing subcategory as being a collection of objects in $\C$, that themselves form a nice presentable stable $\infty$-category, compatible with $\C$. In other words, they are the ``structure preserving subcategories''.

\begin{definition}
    \label{ch0:def:localizing-subcategory}
    \index{Localizing subcategory}
    If $\C$ is a presentable stable $\infty$-category, then a full subcategory $\L\subseteq \C$ is \emph{localizing} if it is closed under retracts, desuspensions and colimits. 
\end{definition}

This means that $\L$ is itself a presentable stable $\infty$-category, and that computing colimits in $\L$ is equivalent to computing colimits in $\C$. 

\begin{definition}
    Let $\C$ be a presentable stable $\infty$-category. Given a collection of objects $\K\subseteq \C$ we denote by $\Loc_\C(\K)$ the smallest localizing subcategory of $\C$ containing $\K$. We will often call it the localizing subcategory \emph{generated} by $\K$. If the category $\C$ is understood, we will sometimes just write $\Loc(\K)$ for simplicity.
\end{definition}

Localizing subcategories are interconnected with the idea of ``primeness''. We give some examples for $\Der(\Z)$. 

\begin{example}
    \label{ch0:ex:p-local-ab}
    Given a chain complex of abelian groups $A$, one can form its $p$-localization $A_{(p)}$ by inverting the map $q\: A\to A$ for all other primes $q\neq p$. The derived category of all $p$-local abelian groups, denoted $\Der(\Z)^{p-\mathrm{loc}}$, is a localizing subcategory of $\Der(\Z)$. In fact, it is equivalent to $\Der(\Z_{(p)})$, the derived category of the $p$-local integers, hence also generated by $\Z_{(p)}$, treated as a complex in degree $0$. 
\end{example}

\begin{example}
    \label{ch0:ex:localizing-away-from-p}
    For a prime $p$ we can also invert only the map $p\: A\to A$, which is called localizing \emph{away} from $p$. The associated derived category denoted $\Der(\Z[\frac{1}{p}])$, is a localizing subcategory of $\Der(\Z)$. 
\end{example}

\begin{example}
    \label{ch0:ex:rational-ab}
    Inverting all primes $p$ gives the \emph{rationalization} of the complex $A$, often denoted $A_\Q$. The derived category of rational abelian groups is equivalent to the derived category of $\Q$, and is also a localizing subcategory of $\Der(\Z)$. This is also generated by the object $\Q$, treated as a chain complex in degree $0$. 
\end{example}

% \begin{example}
%     \label{ch0:ex:p-torsion-ab}
%     The $p$-power torsion of an abelian group $A$ is defined as $T_p A := \{a\in A \mid p^k a = 0 \text{ for some }k > 0\}$. There is a natural map $T_p A\to A$, and $A$ is said to be $p$-power torsion if this map is an isomorphism. The full subcategory of complexes in $\Der(\Z)$, whose homology is $p$-power torsion, denoted $\Der_{H}^p(\Z)$, is a localizing subcategory.  
% \end{example}

\begin{remark}
    \label{ch0:rm:compactly-generated-localizing-subcategory}
    \index{Localizing subcategory!Compactly generated}
    If the collection $\K \subseteq \C$ consists of only compact objects, in the sense of \cref{ch0:def:compact-object}, then the localizing subcategory $\Loc_\C(\K)$ is said to be a \emph{compactly generated} localizing subcategory. 
\end{remark}

\begin{example}
    Even though neither $\Z_{(p)}$ nor $\Q$ are perfect complexes in $\Der(\Z)$, their associated localizing subcategories are in fact compactly generated. 
\end{example}

\begin{remark}
    A more rigorous way to state that a presentable stable $\infty$-category $\C$ is compactly generated---as defined in \cref{ch0:def:compactly-generated-category}---is to say that it is so if and only if the smallest localizing subcategory containing the collection of all compact objects $\Co$ is the entire $\infty$-category $\C$. In other words, there is an equivalence $\C\simeq \Loc_\C(\Co)$ of presentable stable $\infty$-categories. 
\end{remark}

If our presentable stable $\infty$-category is also symmetric monoidal, then we want a version of localizing subcategories that preserve the monoidal structure. If one thinks of a presentably symmetric monoidal stable $\infty$-category as a categorified version of a ring, then the natural such sub-structure should model that of an ideal in a ring. 

\begin{definition}
    \label{ch0:def:localizing-ideal}
    \index{Localizing subcategory!$\otimes$-ideal}
    If $\C$ is a presentably symmetric monoidal stable $\infty$-category, then a full subcategory $\L\subseteq \C$ is a \emph{localizing $\otimes$-ideal} if it is a localizing subcategory, and for any $L\in \L$ and $X\in \C$, we have $L\otimes X\in \L$. 
\end{definition}

The definition of an ideal here is completely analogous to the classical setting of discrete rings. 

\begin{definition}
    \index{Localizing subcategory!Compactly generated $\otimes$-ideal}
    Let $\C$ be a presentably symmetric monoidal stable $\infty$-category. Given a collection of objects $\K\subseteq \C$ we denote by $\Loc_\C^\otimes(\K)$ the smallest localizing $\otimes$-ideal of $\C$ containing $\K$. We will, as before, often refer to this as the localizing $\otimes$-ideal \emph{generated} by $\K$. If $\K$ consists of compact objects we will say that $\Loc^\otimes_\C(\K)$ is a \emph{compactly generated localizing $\otimes$-ideal}. 
\end{definition}

\begin{remark}
    Any ideal $I$ in a discrete ring $R$ is a non-unital subring of $R$. This is also the case for a localizing $\otimes$-ideal $\L\subseteq \C$, which becomes a non-unital presentably symmetric monoidal stable $\infty$-category. However, in some good cases $\L$ is actually unital, but the unit $\1_\L$ will naturally have to be different than the unit $\1_\C$, otherwise we would have $\L=\C$. The localizing ideals we study in \hyperref[ch1]{the first paper}, as well as some of the ones in \hyperref[ch2]{the second}, will have this property. In particular, as we will see in the next section, any localizing $\otimes$-ideal which is compactly generated in the sense of \cref{ch0:rm:compactly-generated-localizing-subcategory} will have this property.
\end{remark}

\begin{example}
    The examples we saw earlier, \cref{ch0:ex:p-local-ab}, \cref{ch0:ex:localizing-away-from-p} and \cref{ch0:ex:rational-ab}, are all localizing $\otimes$-ideals. We will see in the next section that they, or some slight variation of these categories, are compactly generated localizing $\otimes$-ideals, which by the above comment mean they are all presentably symmetric monoidal stable $\infty$-categories themselves. 
\end{example}





\subsection{Local duality}
\label{ch0:ssec:local-duality}

The theory of abstract local duality---proved in \cite{hovey-palmiery-strickland_97} and generalized to the $\infty$-categorical setting in \cite{barthel-heard-valenzuela_2018}---is one of the central ideas of this thesis that we will repeatedly encounter and use. 

\subsubsection{Localizations}

To understand local duality, and also its use of localizing subcategories, we look at certain functors called localizations. In spirit, these are functors that invert a class of maps. 

\begin{definition}
    \label{ch0:def:localization}
    \index{Localization}
    Let $\C$ and $\D$ be presentable stable $\infty$-categories. A \emph{localization} is a colimit preserving functor $f\colon \C\longrightarrow \D$ such that the right adjoint $i$ is fully faithful.  
\end{definition}

There are several ways to construct localizations, but one method particularly important for us will be via localizing subcategories. 

\begin{definition}
    \label{ch0:def:right-orthogonal-complement}
    \index{Right orthogonal complement}
    Let $\L\subseteq \C$ be a full subcategory. The \emph{right orthogonal complement} of $\L$, is the full subcategory $\L^\perp$ consisting of objects $X\in \C$ such that $\Hom_\C(L,X)\simeq 0$ for all $L\in \L$.  
\end{definition}

\begin{example}
    \label{ch0:ex:localization-from-localizing-subcategory}
    Let $\L\subseteq \C$ be a localizing subcategory. The inclusion of the complement $\L^\perp \hookrightarrow \C$ is fully faithful and has a left adjoint $f\colon \C\longrightarrow \L^\perp$. Hence, $f$ is a localization, and its kernel is precisely $\L$. 
\end{example}

\begin{example}
    \index{p-completion}
    \label{ch0:ex:derived-p-completion}
    Let $\C=\Der(\Z)$ and $\L = \Der(\Z[\frac{1}{p}])$ as in \cref{ch0:ex:localizing-away-from-p}. The right orthogonal complement is the category of derived $p$-complete abelian groups, $\Der(\Z)^{p-\mathrm{comp}}$. The associated localization $\Lambda_p$ is the total left derived functor of the $p$-adic completion functor $C^p$, defined by sending an abelian group $A$ to the colimit $C^p(A)=A^\wedge_p := \colim_k A/p^k$. In other words, we have a localization 
    \[\Lambda_p \simeq \bbL C^p \: \Der(\Z)\to \Der(\Z)^{p-\mathrm{comp}}.\]
\end{example}

Throughout the thesis we will mostly focus on localizations of presentably symmetric monoidal stable $\infty$-categories, hence we also want to make sure that the localizations of interest preserve the monoidal structure. This is done as follows. 

\begin{definition}
    \label{ch0:def:L-equivalence}
    Let $\C$ and $\D$ be two presentably symmetric monoidal stable $\infty$-categories and $f\colon \C\longrightarrow \D$ a functor. A map $\phi$ in $\C$ is called an \emph{$f$-equivalence} if $f(\phi)$ is an equivalence. The functor $f$ is said to be \emph{$\otimes$-compatible} if the $f$-equivalences are stable under tensor product: in the sense that for any $f$-equivalence $X\longrightarrow Y$ and object $Z\in \C$, the induced map $X\otimes Z\longrightarrow Y\otimes Z$ is again an $f$-equivalence. 
\end{definition}

\begin{definition}
    \label{ch0:def:monoidal-localization}
    \index{Localization!Monoidal}
    Let $\C$ and $\D$ be presentably symmetric monoidal stable $\infty$-categories. A \emph{monoidal localization} is a $\otimes$-compatible functor $f\colon \C\longrightarrow \D$ with a fully faithful right adjoint $i$. 
\end{definition}

\begin{remark}
    \label{ch0:rm:localizations-tensor-compatible}
    For the rest of this thesis we will assume that, whenever $\C$ is a presentably symmetric monoidal, that any localization of $\C$ is $\otimes$-compatible. We will usually omit the prefix ``monoidal'' from localizations. It is really only in \hyperref[ch3]{the third paper} that we will see non-monoidal localizations, hence the distinction between them should hopefully be clear from the context.
\end{remark}

\begin{remark}
    Let $f\colon \C\longrightarrow \D$ be a localization. The composition of $f$ with the fully faithful right adjoint $i$ is denoted $L$. The functor $i$ gives an equivalence between $\D$ and a full subcategory of $\C$, denoted $\C_L$. By \cite[5.2.7.4]{lurie_09} there is an equivalence between localizations $f\colon \C\longrightarrow \D$ and functors $L\colon \C\longrightarrow \C_L$ ($L$ viewed as a functor to its essential image) that are left adjoint to the inclusion. We will usually use this perspective, using the functor $L$ rather than $f$. 
\end{remark}

\begin{definition}
    \index{Localization!Local object}
    Given a localization $L\colon \C\longrightarrow \C_L$, any object $X\in \C$ admits a map $X\longrightarrow LX$ coming from the unit of the adjunction, called its \emph{$L$-localization}. The object $X$ is said to be \emph{$L$-local} if this is an $L$-equivalence. By definition the category of $L$-local objects is $\C_L$. 
\end{definition}

\begin{proposition}[{\cite[1.3.4.3]{Lurie_HA}}]
    If $L\colon \C\longrightarrow \C_L$ is a localization, then the category of local objects $\C_L$ is equivalent to the full subcategory of $\C$ obtained by inverting the collection of $L$-equivalences $W_L$. In other words, there is an equivalence of symmetric monoidal stable $\infty$-categories $\C_L\simeq \C[W_L^{-1}]$.
\end{proposition}

\begin{remark}
    \label{ch0:rm:monoidal-localization}
    Let $L\colon \C\longrightarrow \C_L$ be a localization on a presentably symmetric monoidal stable $\infty$-category $\C$. The symmetric monoidal structure on $\C$ induces a unique symmetric monoidal structure on $\C_L$, defined by $L(-\otimes_\C -)$, making $L$ into a symmetric monoidal functor. This follows from \cite[2.2.1.9]{Lurie_HA} by our standing assumption that all localizations on symmetric monoidal categories are $\otimes$-compatible, see \cref{ch0:rm:localizations-tensor-compatible}. 
\end{remark}

\begin{remark}
    If $L\: \C\to \C_L$ is a monoidal localization, then the kernel $\Ker L$ is a localizing $\otimes$-ideal of $\C$. 
\end{remark}

\begin{remark}
    \index{Colocalization}
    Similarly to localizations, we can define \emph{colocalizations} as functors $g\colon \C\longrightarrow \D$ admitting a fully faithful left adjoint $i$. The composition $i\circ g$ is denoted $\Gamma$. The adjoint gives an equivalence between $\D$ and a full subcategory $\C^\Gamma$ of $\C$, and the datum of a colocalization is equivalent to the datum of a functor $\Gamma\colon \C\longrightarrow \C^\Gamma$ that is right adjoint to the inclusion. Dually to localizations, we get for any $X\in \C$ a colocalization map $\Gamma X\to X$, and we say $X$ is \emph{$\Gamma$-colocal}\index{Colocalization!Colocal object} if this is an equivalence. 
\end{remark}

Let $\C$ be a presentably symmetric monoidal stable $\infty$-category. For any localization $L\colon \C\longrightarrow \C_L$, the image of the unit $L\1_\C$ is a ring object, and any $L$-local object $X$ admits the structure of an $L\1_\C$ module via the map of functors $L\1_\C \otimes L(-)\longrightarrow L(-)$. 

\begin{definition}
    \label{ch0:def:smashing-localization}
    \index{Localization!Smashing}
    We say a localization $L$ is \emph{smashing} if the $L\1_\C$-module map above is an equivalence. 
\end{definition}

\begin{example}
    \label{ch0:ex:p-localization-ab-smashing}
    \index{p-localization}
    The $p$-localization $L\:\Der(\Z)\to \Der(\Z_{(p)})$ of \cref{ch0:ex:p-local-ab}, given by inverting all primes $q\neq p$, is a smashing localization. By definition it is then given by $L(-) \simeq \Z_{(p)}\otimes_\Z (-)$. 
\end{example}

\begin{remark}
    \label{ch0:rm:smashing-then-modules-over-unit}
    \index{Commutative algebra!Idempotent}
    If $L$ is a smashing localization, then we have by definition that being $L$-local is equivalent to being a module over $L\1_\C$, giving a symmetric monoidal equivalence $\C_L \simeq \Mod_{L\1_\C}(\C)$. When $L$ is smashing the commutative algebra $L\1_\C$ is idempotent, and one can show that there is a one-to-one correspondence between idempotent commutative algebras and smashing localizations. 
\end{remark}

\begin{definition}
    \label{ch0:rm:smashing-colocalization}
    \index{Colocalization!Smashing}
    In a similar fashion, for any colocalization $\Gamma$ there is a comparison map $\Gamma (-)\to \Gamma \1_\C \otimes \Gamma(-)$, and $\Gamma$ is said to be \emph{smashing} if it is an equivalence. 
\end{definition}

\begin{remark}
    \label{ch0:rm:L1-module-adjoint-map}
    By the tensor-internal hom adjunction, the $L\1_\C$-module structure on an $L$-local object $X$ is equivalent to a map $L(-)\to \iHom(L\1_\C,-)$. However, for colocalizations this is not the case, in the sense that a $\Gamma\1_\C$-comodule structure on a $\Gamma$-colocal object $X$ may not be equivalent to a map into $\iHom(\Gamma \1_\C, X)$. This leads to ``two different notions of modules'' in this setting, a setup which we study in detail in the \hyperref[ch2]{second paper}.
\end{remark}

\begin{remark}
    Any monoidal localization $L$ equips $\C_L$ with a symmetric monoidal structure, as seen in \cref{ch0:rm:monoidal-localization}. If $L$ is a smashing localization, then the induced tensor product is the same as in the category $\C$. The same applies to smashing colocalizations. 
\end{remark}





\subsubsection{The local duality theorem}

We are now ready to present the setup for local duality, which is a natural duality theory for compactly generated localizing $\otimes$-ideals. 

\begin{definition}
    \label{ch0:def:local-duality-context}
    \index{Local duality!Context}
    A \emph{local duality context} is a pair $(\C, \K)$, consisting of a presentably symmetric monoidal stable $\infty$-category $\C$, that is compactly generated by dualizable objects, and a subset $\K\subseteq \Co$. 
\end{definition}

Any choice of local duality context allows us to assign to it three new categories, which together decomposes the category $\C$. 

\begin{construction}
    \label{ch0:const:local-duality-categories}
    \index{Local duality!Torsion objects}
    \index{Local duality!Local objects}
    \index{Local duality!Complete objects}
    Let $(\C, \K)$ be a local duality context. We define the category of \emph{$\K$-torsion objects} in $\C$ to be the localizing $\otimes$-ideal generated by $\K$, and denote it by $\C\Ktors:= \Loc_\C^\otimes(\K)$. Further we define the category of \emph{$\K$-local objects} in $\C$ to be the right orthogonal complement---see \cref{ch0:def:right-orthogonal-complement}---of $\C\Ktors$. In other words $\C\Kloc := (\C\Ktors)^\perp$. Finally we define the category of \emph{$\K$-complete objects} in $\C$ to be the right orthogonal complement of $\C\Kloc$, i.e., $\C\Kcomp= (\C\Kloc)^\perp$. 
    
    These three categories have fully faithful inclusions into $\C$, denoted $i_{\K-\mathrm{tors}}$, $i_{\K-\mathrm{loc}}$ and $i_{\K-\mathrm{comp}}$ respectively. By the adjoint functor theorem, \cite[5.5.2.9]{lurie_09}, the inclusions $i_{\K-\mathrm{loc}}$ and $i_{\K-\mathrm{comp}}$ have left adjoints $L_\K$ and $\Lambda_\K$ respectively, while $i_{\K-\mathrm{tors}}$ and $i_{\K-\mathrm{loc}}$ have right adjoints $\Gamma_\K$ and $V_\K$ respectively. These are then, by definition, localizations and colocalizations. 
    
    The torsion, local and complete objects all form $\otimes$-ideals, meaning that the localizations and colocalizations above are compatible with the symmetric monoidal structure of $\C$, in the sense of \cref{ch0:def:L-equivalence}. In particular, by \cref{ch0:rm:monoidal-localization} the categories inherit unique induced symmetric monoidal structures such that $L_\K$, $\Lambda_\K$, $\Gamma_\K$ and $V_\K$ are symmetric monoidal functors. 

    For any object $X\in \C$, these functors assemble into two cofiber sequences:
    \[\Gamma_\K X \longrightarrow X \longrightarrow L_\K X \quad \text{and}\quad V_\K X \longrightarrow X \longrightarrow \Lambda_\K X.\]
    Note also that these functors only depend on the localizing subcategory $\C\Ktors$, not on the particular choice of generators $\K$. Thus, when the set $\K$ is clear from the context, we sometimes omit it as a subscript when writing the functors. 
\end{construction}

\begin{remark}
    \label{ch0:rm:tors-loc-comp-compactly-generated}
    By definition $\C\Ktors$ is compactly generated, and by \cite[2.17]{barthel-heard-valenzuela_2018} both $\C\Kloc$ and $\C\Kcomp$ are as well. 
\end{remark}

The following theorem is a slightly simplified version of the abstract local duality theorem of \cite[3.3.5]{hovey-palmiery-strickland_97} and \cite[2.21]{barthel-heard-valenzuela_2018}.  

\begin{theorem}
    \label{ch0:thm:local-duality}
    \index{Local duality!Theorem}
    If $(\C, \K)$ is a local duality context, then
    \begin{enumerate}
        \item $\Gamma$ is a smashing colocalization and $L$ is a smashing localization;
        \item there are equivalences of functors 
        \[\Lambda \simeq \iHom(\Gamma \1,-) \quad\text{and} \quad V \simeq \iHom(L\1, -);\] 
        \item the functors 
        \[\Gamma\colon \C\Kcomp\longrightarrow \C\Ktors \quad \text{and}\quad \Lambda\colon \C\Ktors\longrightarrow \C\Kcomp\] 
        are mutually inverse equivalences of symmetric monoidal stable $\infty$-categories.
    \end{enumerate}
    This can be summarized by the following diagram of adjoints\index{Local duality!Diagram}
    \begin{center}
        \begin{tikzcd}
                & {\C\Kloc} \\
                & {\C} \\
                {\C\Ktors} && {\C\Kcomp}
                \arrow["L", xshift=-4pt, from=2-2, to=1-2]
                \arrow[from=1-2, to=2-2]
                \arrow["V", xshift=4pt, from=2-2, to=1-2, swap]

                \arrow["\Lambda", yshift=2pt, xshift=2pt, from=2-2, to=3-3]
                \arrow[yshift=-2pt, xshift=0pt, from=3-3, to=2-2]

                \arrow["\Gamma", yshift=-2pt, xshift=0pt, from=2-2, to=3-1]
                \arrow[yshift=2pt, xshift=-2pt, from=3-1, to=2-2]
                
                \arrow[bend left=35, dashed, from=3-1, to=1-2]
                \arrow[bend left=35, dashed, from=1-2, to=3-3]

                \arrow["\simeq"', swap, from=3-1, to=3-3]
        \end{tikzcd}    
    \end{center}
\end{theorem}

\begin{remark}
    \label{ch0:rm:monoidal-structure-in-local-duality}
    \cref{ch0:thm:local-duality} implies that the unique symmetric monoidal structure induced by the localization $L$ and the colocalization $\Gamma$ is simply the symmetric monoidal structure on $\C$ restricted to the full subcategories. This is not the case for $\C\Kcomp$, where the symmetric monoidal structure is given by $\Lambda(-\otimes_\C-)$. We will not need or focus on the functor $V$ in this thesis, hence it will usually be omitted from the local duality diagrams. 
\end{remark}

\begin{remark}
    When the subset $\K$ consists of a single element $K$, we will sometimes use $K$ as the superscript rather than $\K = \{K\}$, as in the following example.   
\end{remark}

\begin{example}
    The object $\Z_{(p)}/p \cong \F_p$ is compact in the derived category of $p$-local abelian groups, $\Der(\Z_{(p)})$. This means that $(\Der(\Z_{(p)}), \F_p)$ forms a local duality context. The category of local objects, $\Der(\Z_{(p)})^{\F_p-\mathrm{loc}}$, has objects in which $p$ is invertible. But, as all other primes are already invertible, all of these are necessarily rational, giving $\Der(\Z_{(p)})^{\F_p-\mathrm{loc}} \simeq \Der(\Q)$. The category $\Der(\Z_{(p)})^{\F_p-\mathrm{tors}}$ is equivalent to the category of derived $p$-torsion objects in $\Der(\Z_{(p)})$. Dually, the category $\Der(\Z_{(p)})^{\F_p-\mathrm{comp}}$ is equivalent to the derived $p$-complete objects in $\Der(\Z)_{(p)}$---see \cref{ch0:ex:derived-p-completion}. We can summarize this example with the following local duality diagram 
    \begin{center}
        \begin{tikzcd}
            & {\Der(\Q)} \\
            & {\Der(\Z_{(p)})} \\
            {\Der(\Z_{(p)})^{p-\mathrm{tors}}} && {\Der(\Z_{(p)})^{p-\mathrm{comp}}}
            \arrow["L_p", xshift=-2pt, from=2-2, to=1-2]
            \arrow[xshift=2pt, from=1-2, to=2-2]

            \arrow["\Lambda_p", yshift=2pt, xshift=2pt, from=2-2, to=3-3]
            \arrow[yshift=-2pt, xshift=0pt, from=3-3, to=2-2]

            \arrow["\Gamma_p", yshift=-2pt, xshift=0pt, from=2-2, to=3-1]
            \arrow[yshift=2pt, xshift=-2pt, from=3-1, to=2-2]

            \arrow[bend left=35, dashed, from=3-1, to=1-2]
            \arrow[bend left=35, dashed, from=1-2, to=3-3]

            \arrow["\simeq"', swap, from=3-1, to=3-3]
        \end{tikzcd}    
    \end{center}
\end{example}






































































