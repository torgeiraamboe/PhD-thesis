

\subsection{Hopf algebroids and their comodules}
\label{ch0:ssec:hopf-algebroids-and-their-comodules}

As mentioned before, this thesis is about understanding certain relationship between stable homotopy theory and homological algebra. One of these relationships for $\Sp$, perhaps the most well known one, comes from the notion of \emph{homology}. 

\begin{definition}
    Let $R$ be a ring spectrum. Associated to $R$ we have an \emph{$R$-homology} functor 
    \[\R_* \: \Sp \to \Ab\]
    defined by $R_*(-):= \pi_*(R\otimes -)$. 
\end{definition}

As $R\otimes X$ is an $R$-module for any spectrum $X$, the $R$-homology functor actually lands in the category of $R_*$-modules, where $R_*:= =R_*(\S) = \pi_* R$. In fact, even more is true: the image of a spectrum under $R_*$ has certain cooperations, making it more structured than just an $R_*$-module. This section is precisely about understanding this extra structure. 

\begin{construction}
    Denote $R_*R:= R_*(R)$, which we for now assume are both commutative (graded) rings. From the unit map $\S\longrightarrow R$, the multiplication map $\mu\colon R\otimes R\longrightarrow R$ and the twist map $\tau\colon R\otimes R\longrightarrow R\otimes R$ we get maps on $R_*$-homology
    \begin{enumerate}
        \item $\eta_L\colon R_*\longrightarrow R_*R$, from the identification $R\otimes \S\simeq R$
        \item $\eta_R\colon R_*\longrightarrow R_*R$, from the identification $\S\otimes R\simeq R$
        \item $\epsilon\colon R_*R\longrightarrow R$, from $\mu$
        \item $c\colon R_*R\longrightarrow R_*R$, from $\tau$
        \item $R_*(R\otimes R)\longrightarrow R_*R$, from $\mu$
    \end{enumerate}
    We have a comparison map $R_*R\otimes_{R_*}R_*R\longrightarrow R_*(R\otimes R)$, which is an isomorphism in nice cases -- for example, if $R_*R$ is a flat module over $R_*$. If this is the case we can extend the map $R_*R\longrightarrow R_*(R\otimes R)$ through the above isomorphism to get a coassociative comultiplication
    \[\Delta\colon R_*R\longrightarrow R_*R\otimes_{R_*}R_*R\] 
    as well as a multiplication map 
    \[\nabla\colon R_*R\otimes_{R_*}R_*R\longrightarrow R_*R\]
    from the fifth map in the above list. The relations on ring spectra also induce relations on the pair $(R_*, R_*R)$, like coassociativity, counitality, and the antipode relation.   
\end{construction}

\begin{remark}
    \label{ch0:rm:fields-give-hopf-algebras}
    If $R_*$ is a field object, for example, $K_p(n)$ or $H\F_p$, then the operations described above, together with the associated relations, make $(R_*, R_*R)$ into a Hopf algebra. In particular, the left and right unit maps are equal: $\eta_L=\eta_R$. 
\end{remark}

\begin{definition}
    \label{ch0:def:flat-and-adams-type-ring}
    \index{Ring spectrum!Adams type}
    A ring spectrum $R$ is called \emph{flat} if $R_*R$ is a flat module over $R_*$. We say $R$ is of \emph{Adams type} if it can be written as a filtered colimit $R\simeq \colim_\alpha R_\alpha$, where each $R_\alpha$ is a finite spectrum such that $R_*R_\alpha$ is a finitely generated projective $R_*$-module and the natural map 
    \[R^*R_\alpha \longrightarrow \Hom_{R_*}(R_*R_\alpha, R_*)\] 
    is an isomorphism.
\end{definition}

In particular, all Adams type ring spectra are flat, as the filtered colimit $R\simeq \colim_\alpha R_\alpha$ gives a filtered colimit 
\[R_*R\cong \colim_\alpha R_*R_\alpha\] 
of projective objects. 

Most of the following examples were given by Adams in \cite[III.13.4]{adams_74}, except for $K_p(n)$, which was not discovered yet. 

\begin{example}[{\cite[1.4.7, 1.4.9]{hovey_04}}]
    \label{ch0:ex:adams-type-ring-spectra}
    The ring spectra $\MU$, $MSp$, $\KO$, $H\F_p$, $K_p(n)$, $E_p(n)$, $E\np$ are all of Adams type. We also have the following class of examples: if $R$ is Adams type, then any Landweber exact $R$-algebra is also Adams type. 
\end{example}

As we mentioned in \cref{ch0:rm:fields-give-hopf-algebras}, the pair $(R_*, R_*R)$ was a Hopf algebra whenever $R_*$ was a field. The following definition is generalization of a Hopf algebra, which will allow us to claim a similar statement for general ring spectra $R$. 

\begin{definition}
    \label{ch0:def:hopf-algebroid}
    \index{Hopf algebroid}
    A (graded) \emph{Hopf algebroid} is a cogroupoid object $(A, \Psi)$ in the category of graded commutative rings. 
\end{definition}

The use of Hopf algebroids in situations related to homotopy theory was studied by Ravenel in \cite[A.1]{ravenel_86} and later in more detail by Hovey in \cite{hovey_04}. 


\begin{remark}
    In the literature outside of topology, the assumptions of being commutative and graded are usually not present. But, as all our examples will be of this kind, we keep in line with the topological tradition. 
\end{remark}

\begin{definition}
    \label{ch0:def:adams-hopf-algebroid}
    \index{Hopf algebroid!Adams type}
    We say a Hopf algebroid $(A, \Psi)$ is of \emph{Adams type} if $\Psi$ is a filtered colimit $\colim_k \Psi_k \simeq \Psi$ of dualizable comodules $\Psi_k$.
\end{definition}


Recall the definition of a flat, and an Adams type ring spectrum in \cref{ch0:def:flat-and-adams-type-ring}. The following proposition is standard---see, for example, \cite[1.4.6]{hovey_04}. 

\begin{proposition}
    \label{ch0:prop:hopf-algebroid-from-spectra}
    \index{Hopf algebroid!Ring spectrum}
    Let $R$ be a flat ring spectrum such that $R_*R$ is a commutative ring. Then, the pair $(R_*, R_*R)$ is a Hopf algebroid. If $R$ is Adams, then $(R_*, R_*R)$ is an Adams Hopf algebroid. 
\end{proposition}

Given a Hopf algebroid $(A,\Psi)$ we can always talk about modules over the ring $A$, but the added extra structure of the cooperations mentioned earlier comes from adding the structure coming from $\Psi$. 

\begin{definition}
    \label{ch0:def:comodule-over-hopf-algebroid}
    \index{Hopf algebroid!Comodule}
    Let $(A, \Psi)$ be a Hopf algebroid. A \emph{$\Psi$-comodule} is an $A$-module $M$ together with a coassociative and counital map $\psi\colon M\longrightarrow M\otimes_A \Psi$. The category of comodules over $(A, \Psi)$ is denoted \emph{$\Comod_\Psi$}. 
\end{definition}

\begin{remark}
    \label{ch0:rm:presenting-stacks}
    In algebraic geometry, Hopf algebroids are usually formulated dually as groupoid objects in affine schemes. The left and right unit maps $A\rightrightarrows \Psi$ induces a presentation of stacks $\mathrm{Spec}(\Psi)\rightrightarrows \mathrm{Spec}(A)$, and the category $\Comod_\Psi$ is equivalent to the category of quasi-coherent sheaves on the presented stack, see \cite[Thm 8]{naumann_07}. 
\end{remark}

\begin{proposition}[{\cite[1.3.1, 1.4.1]{hovey_04}}]
    \label{ch0:prop:comod-is-sm-grothendieck}
    \index{Grothendieck abelian}
    Let $(A,\Psi)$ be an Adams Hopf algebroid. Then, the category $\Comod_\Psi$ is a Grothendieck abelian category generated by the dualizable comodules. There is a symmetric monoidal product $-\otimes_\Psi -$, which on the underlying modules is the normal tensor product of $A$-modules. It has a right adjoint $\iHom_\Psi(-,-)$, making $\Comod_\Psi$ a closed symmetric monoidal category. 
\end{proposition}

\begin{example}
    \label{ch0:ex:modules-as-discrete-Hopf-algebroids}
    \index{Hopf algebroid!Discrete}
    For any commutative graded ring $A$, the pair $(A, A)$ is a called a \emph{discrete Hopf algebroid}. The category of comodules over this Hopf algebroid is the normal abelian category $\Mod_A$ of modules over $A$. 
\end{example}

\begin{construction}
    \index{Hopf algebroid!Extended comodule}
    Given an Adams Hopf algebroid $(A, \Psi)$, we can define a discretization map $\epsilon\colon (A, \Psi)\longrightarrow (A, A)$, which is given by the identity on $A$ and the counit on $\Psi$. By \cite[A1.2.1]{ravenel_86} and \cite[4.6]{barthel-heard-valenzuela_2018} it induces a faithful exact forgetful functor $\epsilon_* \colon \Comod_\Psi \longrightarrow \Mod_A$ with a right adjoint $\epsilon^*$ given by $\epsilon^*(M)\simeq \Psi\otimes_A M$. A comodule in the essential image of $\epsilon^*$ is called an \emph{extended comodule}.
\end{construction}

Let us now finally present the class of examples that are of particular interest for this thesis. 

\begin{example}
    If $R$ is an Adams type ring spectrum, then the $R$-homology functor $R_*(-)$ takes values in the Grothendieck abelian category $\Comod_{R_*R}$. In particular, given any spectrum $X$, then $R_*X$ has a coaction 
    \[R_*X \to R_*X \otimes_{R_*} R_*R,\]
    which is both coassociative and counital.
\end{example}

\begin{remark}
    We don't need the Adams type condition in order for $R_*X$ to be a comodule, but in this case, $\Comod{R_*R}$ is not a Grothendieck abelian category.
\end{remark}

\begin{remark}
    In \cref{ch0:ex:adams-type-ring-spectra} we saw that the complex cobordism spectrum $\MU$ was an Adams type ring spectrum. In light of \cref{ch0:rm:presenting-stacks} we can then state the celebrated result of Quillen, see \cite{quillen_1969}, relating stable homotopy theory to formal groups:
    \[\mathrm{Spec}(\MU_*\MU)\rightrightarrows \mathrm{Spec}(\MU*)\rightarrow \M_{\mathrm{fg}},\]
    where $\M_{fg}$ is the moduli stack of formal groups. Note that we are brushing some technical details under the rug here, but this is correct at least in spirit. 
\end{remark}

In \cref{ch0:sssec:morava-E-theories} we saw serveral versions of $E$-theory, and by \cref{ch0:prop:all-E-local-cats-are-equivalent} we know that all their corresponding $E$-local categories are equivalent. The same occurs for the categories of comodules associated to their respective Adams Hopf algebroid.

\begin{proposition}[{\cite[4.2]{hovey-strickland_2005a}}]
    Let $p$ be a prime and $n$ a positive natural number. In this situation the categories of comodules over the Hopf algebroids associated to $E\np$, $E_p(n)$ and $A = E\np^{h\F_p^\times}$ are all equivalent: 
    \[\Comod_{E{\np}_*\E\np}\simeq \Comod_{E_p(n)_*E_p(n)}\simeq \Comod_{A_*A}.\]
    Furthermore, given any Landweber exact $v_n$-periodic ring spectrum $E$, then its associated category of comodules is equivalent to the ones above. 
\end{proposition}

\begin{notation}
    When the chromatic height $n$ and the prime $p$ is understood, we will use the common notation $\ComodE$ for any of the above categories. 
\end{notation}

As in \cref{ch0:ssec:local-duality}, we have certain objects that are especially important---the compact objects and the dualizable objects. In Grothendieck abelian categories it is, in addition, important to understand the injective objects. This will also become important later in \cref{ch1}, as we will use injective objects to approximate other objects and to build certain spectral sequences.  

\begin{proposition}
    \label{ch0:rm:dualizable/compact-comodules}
    \index{Hopf algebroid!Dualizable comodule}
    \index{Hopf algebroid!Compact comodule}
    Let $(A, \Psi)$ be an Adams Hopf algebroid. A $\Psi$-comodule $M$ is dualizable if and only if its underlying $A$-module $\epsilon_* M$ is dualizable, i.e., it is finitely generated and projective. Similarly, a $\Psi$-comodule is compact if and only if its underlying $A$-module is compact, which coincides with being finitely presented. 
\end{proposition}
\begin{proof}
    The first claim is \cite[1.3.4]{hovey_04} and the second is \cite[1.4.2]{hovey_04}. 
\end{proof}

\begin{remark}
    \label{ch0:rm:dualizables-compact-generators}
    As colimits in $\Comod_\Psi$ are exact and are computed in $\Mod_A$, all the dualizable comodules are compact. Hence, the full subcategory of dualizable comodules is a set of compact generators for $\Comod_\Psi$. 
\end{remark}

\begin{proposition}[{\cite[2.1]{hovey-strickland_2005b}}]
    \label{ch0:rm:injective-comodules}
    Let $(A, \Psi)$ be an Adams Hopf algebroid. If $I$ is an injective object in $\Comod_\Psi$, then there is an injective $A$-module $Q$, such that $I$ is a retract of the extended comodule $\Psi\otimes_A Q$. 
\end{proposition}

\begin{remark}
    Note that as $\Comod_{\Psi}$ is Grothendieck abelian, it has enough injective objects. This allows us to construct injective resolutions and thus $\Ext$-groups, which we will see later, greatly help in computing information in stable homotopy theory. For example, the pair $(\F_2, \mathcal{A}_*)$ where $\mathcal{A}_*$ is the dual Steenrod algebra is a Hopf algebroid, and the groups $\Ext^s_{\mathcal{A}_*}(\F_2, \F_2)$ are used in the Adams spectral sequence to approximate homotopy groups of spheres, see \cite{adams_58}. 
\end{remark}

Given an Adams Hopf algebroid $(A, \Psi)$, we also have an associated derived category. By \cite[2.1.2, 2.1.3]{hovey_04} the category of chain complexes of $\Psi$-comodules, $\Ch_\Psi$, has a cofibrantly generated stable symmetric monoidal model structure. In \cite{barnes-roitzheim_2011} this model structure was modified slightly to more easily compare it to the periodic derived category, which we will consider more closely in \cref{ch1}. The homotopy category associated to this model structure is the usual unbounded derived category $\Der(\Comod_\Psi)$ associated to the Grothendieck abelian category $\Comod_\Psi$. 

\begin{notation}
    \index{Hopf algebroid!Derived category}
    We will use $\Der(\Psi)$ as our notation for the underlying symmetric monoidal stable $\infty$-category associated with the above model structure. The monoidal unit is $A$, treated as a chain complex centered in degree $0$.
\end{notation}

\begin{remark}
    We warn the reader that some authors use the notation $\Der(\Psi)$ to reffer to the above-mentioned periodic derived category of $(A, \Psi)$. This is the case, for example, in \cite{pstragowski_2021}. 
\end{remark}

We also get an induced discretization adjunction on the level of derived categories. 

\begin{proposition}
    \index{Hopf algebroid!Discretization}
    Let $(A,\psi)$ be an Adams Hopf algebroid. Then the discretization adjunction $(\epsilon_*\dashv \epsilon^*)\colon \Comod_\Psi\longrightarrow \Mod_A$ induces an adjunction $(\epsilon_*\dashv \epsilon^*)\colon \Der(\Psi)\longrightarrow \Der(A).$
\end{proposition}
\begin{proof}
    This follows from the fact that $\Psi$ is flat over $A$, which implies that both $\epsilon_*$ and $\epsilon^*$ on the abelian categories are exact. 
\end{proof}




\subsubsection{Torsion and completion for comodules}
\label{ch0:sssec:torsion-and-completion-for-comodules}

As we have stated, the notion of localizing subcategories will be important in this thesis; not only in the situation of stable $\infty$-categories, but also in the Grothendieck abelian situration. In the following section we review the construction of a particular kind of localizing subcategory of the Grothendieck abelian category $\Comod_\Psi$ associated to the Hopf algebroid $(A,\Psi)$. These are the categories of comodules that are torsion with respect to some nicely behaved ideal $I\subseteq A$. 

This will give us two approaches to studying torsion and completion in $\Der(\Psi)$---one ``internal'' and one ``external''. The internal approach uses the classical theory of torsion objects in abelian categories, while the external uses local duality, as in \cref{ch0:thm:local-duality}. These two approaches are luckily equivalent in the situations we are interested in. 

We first review the abelian situation: the internal approach. We follow \cite{barthel-heard-valenzuela_2018} and \cite{barthel-heard-valenzuela_2020} in notation and results. 

\begin{definition}
    \label{def:I-power-torsion-module}
    \index{$I$-power torsion!Module}
    Let $A$ be a commutative ring and $I\subseteq R$ a finitely generated ideal. The $I$-power torsion of an $A$-module $M$ is defined as
    \[T_I^A M = \{x\in M \mid I^k x = 0 \text{ for some } k\in \N\}.\]
    We say a module $M$ is \emph{$I$-power torsion} if the natural comparison map $T_I^A M\longrightarrow M$ is an isomorphism. 
\end{definition}

\begin{remark}
    The category $\Mod_R\Itors$ is a localizing subcategry of $\Mod_R$, in the sense that it is a full subcategory closed under quotients, subobjects, extensions and arbitrary coproducts. 
\end{remark}

\begin{remark}
    Of particular importance for \cref{ch1} is the category $\modt$, where $E$ is the Johnson--Wilson spectrum $E_p(n)$, and $I_n$ is the Landweber ideal 
    \[I_n = (p, v_1, \ldots, v_{n-1})\subseteq \pi_* E_p(n).\]
\end{remark}

\begin{definition}
    \index{$I$-adically complete module}
    Let $A$ be a commutative ring and $I\subseteq R$ a finitely generated ideal. The $I$-adic completion of an $A$-module $M$ is defined as
    \[C_I^A M = \lim_k A/I^k\otimes_A M.\]
    We say a module $M$ is \emph{$I$-adically complete} if the natural map $M\longrightarrow C_I^A M$ is an isomorphism. 
\end{definition}

\begin{remark}
    \label{ch0:rm:I-complete-vs-I-adically-complete}
    \index{$I$-complete module}
    \index{$L$-complete module}
    The resulting category of $I$-adically complete modules is not very well-behaved. The $I$-adic completion functor is often neither left nor right exact, and the category is often not abelian. To fix these issues, Greenlees and May introduced the notion of $L$-complete modules in \cite{greenlees-may_92}, using instead the zeroth left derived functor $L=\mathbb{L}_0 C_I^A$. Thus, it is also sometimes referred to as derived completion. One then defines \emph{$I$-complete} modules, also called $L$-complete or derived complete, to be those $R$-modules such that the natural map $M\longrightarrow L M$ is an equivalence. 
\end{remark}

\begin{notation}
    We denote the full subcategory consisting of $I$-power torsion $A$-modules by $\Mod_A\Itors$ and the full subcategory of $I$-complete $A$-modules by $\Mod_A\Icomp$. 
\end{notation}

\begin{remark}
    The category $\Mod_A\Itors$ is a Grothendieck abelian category. On the other hand, $\Mod_A\Icomp$ is abelian, but not Grothendieck in general. It is, however, a locally presentable abelian category with enough projectives, which is kind of ``dual'' to being Grothendieck abelian. There is also an equivalence between $\Mod_A\Icomp$ and the abelian category of contramodules over the $I$-adic completion of $A$, see \cite[Section 2.2]{positselski_2022_contramodules}. We will see more of the duality perspective between torsion and completion from this perspective in \cref{ch2}. 
\end{remark}

The inclusion of the full subcategory $\Mod_A\Itors\hookrightarrow \Mod_A$ has a right adjoint, which coincides with the $I$-power torsion $T_I^A(-)$. This gives the $I$-power torsion another description as the colimit 
\[T_I^A M \cong \colim_k \iHom_A (A/I^k, M).\]

We want to extend the construction of $I$-torsion and $L$-complete modules to general Adams Hopf algebroids $(A,\Psi)$. For this, we need to choose sufficiently nice ideals that interact nicely with the additional comodule structure. 

\begin{definition}
    \index{Invariant ideal}
    \index{Invariant ideal!Regular}
    Let $(A, \Psi)$ be an Adams Hopf algebroid, and $I$ an ideal in $A$. We say $I$ is an \emph{invariant ideal} if, for any comodule $M$, the comodule $IM$ is a subcomodule of $M$. If $I$ is finitely generated by $(x_1, \ldots, x_r)$ and $x_i$ is non-zero-divisor in $R/(x_1, \ldots, x_{i-1})$ for each $i=1, \ldots, r$, then we say $I$ is \emph{regular}. 
\end{definition}

\begin{definition}
    \label{ch0:def:I-power-torsion-comodule}
    \index{$I$-power torsion!Comodule}
    Let $(A,\Psi)$ be an Adams Hopf algebroid and $I\subseteq A$ a regular invariant ideal. The $I$-power torsion of a comodule $M$ is defined as 
    \[T_I^\Psi M = \{x\in M \mid I^kx = 0 \text{ for some } k\in \N\}.\]
    We say a comodule $M$ is \emph{$I$-power torsion} if the natural map $T_I^\Psi M\longrightarrow M$ is an equivalence. The full subcategory of $I$-power torsion comodules is denoted $\Comod_\Psi\Itors$. 
\end{definition}

\begin{remark}
    \label{ch0:rm:torsion-comodules-grothendieck-monoidal}
    By \cite[5.10]{barthel-heard-valenzuela_2018} the full subcategory of $I$-power torsion comodules is a Grothendieck abelian category. It also inherits a symmetric monoidal structure from $\Comod_\Psi$. This also makes $\Mod_A^{I-tors}$ Grothendieck abelian and symmetric monoidal by \cref{ch0:ex:modules-as-discrete-Hopf-algebroids}. 
\end{remark}

\begin{remark}
    In particular, $\Comod_\Psi\Itors$ is a localizing subcategory of $\Comod_\Psi$. 
\end{remark}

\begin{example}
    Of particular importance for \cref{ch1} and \cref{ch3:addendum} is the category $\Comodt$, where $E$ is some  version of Morava $E$-theory, and $I_n$ is again the Landweber ideal. This example will follow us through the whole thesis, as it is the ``abelian version'' of the monochromatic category $\M\np$---we will use a significant amount of time and effort to study their relationship. 
\end{example}

\begin{notation}
    Since $\Comod_\Psi\Itors$ is Grothendieck abelian we have an associated derived stable $\infty$-category $\Der(\Comod_\Psi\Itors)$ which we denote simply by $\Der(\Psi\Itors)$.
\end{notation}

\begin{remark}
    \label{ch0:rm:complete-comodules-not-abelian}
    Unfortunately, the corresponding versions of $I$-adically complete and $L$-complete comodules do not form abelian categories in general, as we can have problems with the comodule structure on certain cokernels.
\end{remark}

As for the case of modules, the inclusion $\Comod_\Psi\Itors\hookrightarrow \Comod_\Psi$ has a right adjoint that corresponds to the $I$-power torsion construction $T_I^\Psi$. This, by \cite[5.5]{barthel-heard-valenzuela_2018} also has the alternative description
\[T_I^\Psi M \cong \colim_k \iHom_\Psi (A/I^k, M).\]

As we have now seen, the construction of $I$-power torsion in $\Mod_A$ and $\Comod_\Psi$ are completely analogous. Hence, one can wonder whether they agree on the underlying modules. This turns out to be the case. 

\begin{lemma}[{\cite[5.7]{barthel-heard-valenzuela_2018}}]
    \label{ch0:lm:torsion-comodule-iff-torsion-module}
    For any $\Psi$-comodule $M$ there is an isomorphism of $A$-modules $\epsilon_* T^\Psi_I M \cong T^A_I \epsilon_* M.$
    Furthermore, if an $A$-module $N$ is $I$-power torsion, then the extended comodule $\Psi\otimes_A N$ is $I$-power torsion. In particular, a $\Psi$-comodule $M$ is $I$-power torsion if and only if the underlying $A$-module is $I$-power torsion. 
\end{lemma}

As mentioned above, we will later make use of injective objects in $\Comod_\Psi\Itors$. Hence, we relate some facts about these. 

\begin{lemma}
    \label{ch0:lm:injectives-in-torsion-comodules}
    Let $(A, \Psi)$ be an Adams Hopf algebroid and $I$ a regular invariant ideal.
    \begin{enumerate}
        \item If $J$ is an injective in $\Comod_\Psi$ then $T_I^\Psi J$ is an injective in $\Comod_\Psi\Itors$.
        \item There are enough injectives in $\Comod_\Psi\Itors$.
        \item Any injective $J'$ in $\Comod_\Psi\Itors$ is a retract of an object of the form $T_I^\Psi J$ for an injective $\Psi$-comodule $J$.
    \end{enumerate} 
\end{lemma}
\begin{proof}
    The first point is \cite[2.1.4]{brodmann-sharp_1998}, while the second is a consequence of $\Comod_\Psi\Itors$ being Grothendieck abelian, as mentioned in \cref{ch0:rm:torsion-comodules-grothendieck-monoidal}. The third point is stated in the proof of \cite[3.16]{barthel-heard-valenzuela_2020}. 
\end{proof}

\begin{remark}
    \label{ch0:rm:injectives-in-torsion-modules}
    \index{Hopf algebroid!Discrete}
    Choosing a discrete Hopf algebroid $(A,A)$, \cref{ch0:lm:injectives-in-torsion-comodules} implies that injectives in $\Mod_A\Itors$ are retracts of $T_I^A(Q)$ for some injective $A$-module $Q$ and that $T_I^A$ preserves injectives. As noted in \cref{ch0:rm:injective-comodules}, an injective object in $\Comod_\Psi$ is a retract of an extended comodule of the form $\Psi\otimes_A Q$ for an injective $A$-module $Q$. This means that all injectives $J$ in $\Comod_\Psi\Itors$ are retracts of $T_I^\Psi(\Psi\otimes_A Q)$ where $Q$ is an injective $A$-module. 
\end{remark}

\begin{remark}
    \label{ch0:rm:dualizable/compact-torsion-comodule}
    \index{Hopf algebroid!Dualizable comodule}
    \index{Hopf algebroid!Compact comodule}
    As colimits in $\Comod_\Psi\Itors$ are computed in $\Comod_\Psi$, we have, similar to \cref{ch0:rm:dualizable/compact-comodules}, that an $I$-power torsion $\Psi$-comodule $M$ is dualizable (resp. compact) if and only if its underlying $A$-module is finitely generated and projective (resp. finitely presented). 
\end{remark}

% \begin{lemma}
%     \label{ch0:lm:torsion-comodules-generated-by-compacts}
%     Let $(A,\Psi)$ be an Adams Hopf algebroid, where $A$ is noetherian and $I\subseteq A$ a regular invariant ideal. Then $\Comod_\Psi^{I-tors}$ is generated under filtered colimits by the compact $I$-power torsion comodules. 
% \end{lemma}
% \begin{proof}
%     By \cite[3.4]{barthel-heard-valenzuela_2020} $\Comod_\Psi^{I-tors}$ is generated by the set 
%     $$\mathrm{Tors}_\Psi^{fp}:=\{G\otimes A/I^k \mid G \in \Comod_\Psi^{fp}, k\geq 1\},$$
%     where $\Comod_\Psi^{fp}$ is the full subcategory of dualizable $\Psi$-comodules. Since $I$ is finitely generated and regular, $A/I^k$ is finitely presented as an $A$-module, hence it is compact in $\Comod_\Psi^{I-tors}$ by \cref{ch0:rm:dualizable/compact-comodules} and \cref{ch0:rm:dualizable/compact-torsion-comodule}. As $A$ is noetherian, being finitely generated and finitely presented coincide. The tensor product of finitely generated modules is finitely generated, hence any element in $\mathrm{Tors}_\Psi^{fp}$ is compact. 
% \end{proof}

% \begin{remark}
%     We are under the impression that the assumption that the ring $A$ is noetherian can most likely be removed, but for the results in this thesis we will not need that added level of generality.  
% \end{remark}


We now move to the external approach, using local duality as in \cref{ch0:ssec:local-duality}. 

\begin{construction}
    \label{ch0:const:local-duality-hopf-algebroid}
    \index{Local duality!Diagram}
    Let $(A, \Psi)$ be an Adams Hopf algebroid and $I\subseteq A$ a regular invariant ideal. Then $A/I$, treated as a complex concentrated in degree zero, is by \cite[5.13]{barthel-heard-valenzuela_2018} a compact object in $\Der(\Psi)$. Thus, $(\Der(\Psi), \{A/I\})$ is a local duality context, and we can consider the corresponding local duality diagram
    \begin{equation*}
        \begin{tikzcd}
            & {\Der(\Psi)\Iloc} \\
            & {\Der(\Psi)} \\
            {\Der(\Psi)\Itors} && {\Der(\Psi)\Icomp}
            \arrow["L_I^\Psi", xshift=-2pt, from=2-2, to=1-2]
            \arrow[xshift=2pt, from=1-2, to=2-2]
            \arrow["\Delta_I^\Psi", yshift=2pt, xshift=2pt, from=2-2, to=3-3]
            \arrow[yshift=-2pt, xshift=-1pt, from=3-3, to=2-2]
            \arrow["\Gamma^\Psi_I", yshift=-2pt, xshift=2pt, from=2-2, to=3-1]
            \arrow[yshift=2pt, xshift=-1pt, from=3-1, to=2-2]
            \arrow[bend left=35, dashed, from=3-1, to=1-2]
            \arrow[bend left=35, dashed, from=1-2, to=3-3]
            \arrow["\simeq"', swap, from=3-1, to=3-3]
        \end{tikzcd}    
    \end{equation*}
    where we have used the superscript $I$ instead of $A/I$ for simplicity. This gives, in particular, a definition of $I$-torsion objects in $\Der(\Psi)$ as $\Der(\Psi)\Itors$. 
\end{construction}

Our goal was to give two constructions and prove that they were equal in the cases we were interested in. 

\begin{lemma}[{\cite[3.7(2)]{barthel-heard-valenzuela_2020}}]
    \label{ch0:lm:derived-torsion-if-homology-torsion}
    Let $(A,\Psi)$ be an Adams Hopf algebroid and $I\subseteq A$ a regular invariant ideal. There is an equivalence of categories 
    \[\Der(\Psi)\Itors\simeq \Der(\Psi\Itors).\] 
    Furthermore, an object $M\in \Der(\Psi)$ is $I$-torsion if and only if the homology groups $H_* M$ are $I$-power torsion $\Psi$-comodules.
\end{lemma}

One can wonder whether the same is true for the $I$-complete derived category, but this is unfortunately not true, as $\Comod_\Psi\Icomp$ is not abelian. A partial result can, however, be recovered for discrete Hopf algebroids $(A, A)$. We follow \cite{barthel-heard-valenzuela_2020} in the following construction. 

\begin{construction}
    \label{ch0:const:completed-derived-category}
    Recall that $\Mod_A\Icomp$ denotes the category of $L$-complete $A$-modules for $I\subseteq A$ a regular ideal. By \cite[2.11]{barthel-heard-valenzuela_2020} the category has enough projectives, hence by \cite[1.3.2]{Lurie_HA} we can associate to it the right bounded category $\Der^-(\Mod_A\Icomp)$. This has a by \cite[1.3.2.19, 1.3.3.16]{Lurie_HA} a left complete $t$-structure with heart equivalent to $\Mod_A\Icomp$. We can then form its right completion, which we denote $\overline{\Der}(\Mod_A\Icomp)$, and call the completed derived category of $\Mod_A\Icomp$. 
\end{construction}

This is what allows us the partial version of \cref{ch0:lm:derived-torsion-if-homology-torsion} in the case of $I$-completion. 

\begin{proposition}[{\cite[3.7(1)]{barthel-heard-valenzuela_2020}}]
    \label{ch0:prop:pulling-out-completion}
    Let $A$ be a commutative ring and $I\subseteq A$ a regular ideal. Then, there is an equivalence 
    \[\Der(\Mod_A)\Icomp\simeq \overline{\Der}(\Mod_A\Icomp),\]
    where the former category is the full subcategory of $A/I$-complete objects in $\Der(\Mod_A)$ while the latter is the completed derived category of $\Mod_A\Icomp$. 
\end{proposition}











