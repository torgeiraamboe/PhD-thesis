

\subsection{Hopf algebroids and their comodules}
\label{ch0:ssec:hopf-algebroids-and-their-comodules}

As mentioned before, this thesis is about understanding certain relationships between stable homotopy theory and homological algebra. One of these relationships---perhaps the most well known one---comes from the notion of \emph{homology}. Homology is a generalization of the Betti numbers, and ``connectedness numbers'' mentioned in \cref{ch0:layperson}. 

\begin{definition}
    \index{Homology theory}
    Let $R$ be a ring spectrum. Associated to $R$ we a functor
    \[R_* \: \Sp \to \Ab\]
    defined by $R_*(X):= \pi_*(R\otimes X)$, called the \emph{$R$-homology}. 
\end{definition}

\begin{example}
    Letting $R = H\F_p$, then $H\F_p$-homology coincides with standard cellular homology with $\F_p$-coefficients. 
\end{example}

\begin{example}
    If $R=\KU$, then $\KU$-homology coincides with complex topological $K$-theory. 
\end{example}

As $R\otimes X$ is an $R$-module for any spectrum $X$, the $R$-homology functor actually lands in the category of $R_*$-modules, where 
\[R_*:= R_*(\S) = \pi_* R.\] 
In fact, even more is true: the image of a spectrum $X$ under $R_*(-)$ has certain cooperations, coming from the relationship between $R_*$ and $R_*R:= R_*(R)$, making it more structured than just an $R_*$-module. This section is precisely about understanding this extra structure. 

\begin{definition}
    \label{ch0:def:flat-and-adams-type-ring}
    \index{Spectra!Adams type}
    A ring spectrum $R$ is called \emph{flat} if $R_*R$ is a flat module over $R_*$. We say $R$ is of \emph{Adams type} if it can be written as a filtered colimit $R\simeq \colim_\alpha R_\alpha$, where each $R_\alpha$ is a finite spectrum such that $R_*R_\alpha$ is a finitely generated projective $R_*$-module and the natural map 
    \[R^*R_\alpha \longrightarrow \Hom_{R_*}(R_*R_\alpha, R_*)\] 
    is an isomorphism.
\end{definition}

\begin{remark}
    In particular, all Adams type ring spectra are flat, as the presentation $R\simeq \colim_\alpha R_\alpha$ gives a presentation
    \[R_*R\cong \colim_\alpha R_*R_\alpha\] 
    of projective objects. 
\end{remark}

Most of the following examples were given by Adams in \cite[III.13.4]{adams_74}---except for the Morava theories.

\begin{example}[{\cite[1.4.7, 1.4.9]{hovey_04}}]
    \label{ch0:ex:adams-type-ring-spectra}
    The ring spectra $\MU$, $\mathrm{MSp}$, $\KU$, $\KO$, $H\F_p$, $K_p(n)$, $E_p(n)$, $E\np$ are all of Adams type. We also have the following class of examples: if $R$ is Adams type, then any Landweber exact $R$-algebra is also Adams type. 
\end{example}

Given a flat ring spectrum $R$, we can now make the following constructions. 

\begin{construction}
    \label{ch0:const:hopf-algebroid-maps-from-spectra}
    From the unit map $\S\longrightarrow R$, the ring spectrum multiplication map $\mu\: R\otimes R\longrightarrow R$ and the twist map 
    \[\tau\: R\otimes R\longrightarrow R\otimes R,\] 
    we get maps on $R_*$-homology
    \begin{enumerate}
        \item $\eta_L\: R_*\longrightarrow R_*R$, from the identification $R\otimes \S\simeq R$
        \item $\eta_R\: R_*\longrightarrow R_*R$, from the identification $\S\otimes R\simeq R$
        \item $\epsilon\: R_*R\longrightarrow R$, from $\mu$
        \item $c\: R_*R\longrightarrow R_*R$, from $\tau$
        \item $R_*(R\otimes R)\longrightarrow R_*R$, from $\mu$
    \end{enumerate}
    There is a comparison map
    \[R_*R\otimes_{R_*}R_*R\longrightarrow R_*(R\otimes R)\simeq \pi_*(R\otimes R\otimes R),\]
    which is an isomorphism precisely when $R$ is flat. There is also a map 
    \[R\otimes R \to R\otimes \S \otimes R \to R\otimes R\otimes R\]
    induced by the unit map, which on homotopy groups gives 
    \[\Delta\: R_*R \to \pi_*(R\otimes R\otimes R)\overset{\cong}\longleftarrow R_*R\otimes_{R_*}R_*R.\]
    The left and right unit maps $\eta_L$ and $\eta_R$ make $R_*R$ a bimodule over $R_*$, hence this map is a \emph{comultiplication} on $R_*R$. As $R\otimes R$ is again a ring spectrum, we also get an induced multiplication map  
    \[\nabla \: R_*R\otimes_{R_*}R_*R\longrightarrow R_*R\]
    The relations on the maps of ring spectra also induce relations on the pair $(R_*, R_*R)$, like coassociativity, counitality, and the antipode relation.   
\end{construction}

\begin{remark}
    \label{ch0:rm:fields-give-hopf-algebras}
    \index{Morava $K$-theory}
    If $R$ is a field object, for example, $K_p(n)$ or $H\F_p$, then the operations described above, together with the associated relations, make $(R_*, R_*R)$ into a Hopf algebra. In particular, the left and right unit maps are equal: $\eta_L=\eta_R$. In the case of $H\F_p$, the Hopf algebra $H\F_{p*}H\F_p$ is the dual Steenrod algebra $\mathcal{A}^*_p$, studied for example in \cite{milnor_1958}. 
\end{remark}

Not all flat ring spectra are field objects, so we cannot expect to obtain a Hopf algebra in general. However, we do obtain a similar, more general structure, encapsulated by the concept of Hopf algebroids. The interested reader can read \cite[Appendix A.1]{ravenel_86} for a more comprehensive treatment. 

\begin{definition}
    \label{ch0:def:hopf-algebroid}
    \index{Hopf algebroid}
    A (graded) \emph{Hopf algebroid} is a cogroupoid object $(A, \Psi)$ in the category of graded commutative rings. More precisely it consists of two graded commutative rings $A$ and $\Psi$, together with maps 
    \begin{enumerate}
        \item $\eta_L\: A\longrightarrow \Psi$, 
        \item $\eta_R\: A\longrightarrow \Psi$,
        \item $\epsilon\: \Psi\longrightarrow A$,
        \item $c\: \Psi\longrightarrow \Psi$,
        \item $\Delta\: \Psi\longrightarrow \Psi\otimes_A\Psi$
    \end{enumerate}
    satisfying natural unitality, coassociativity, twist and antipode relations. 
\end{definition}

\begin{remark}
    Any Hopf algebra corepresents a functor from commutative rings into groups. Hopf algebroids instead corepresent functors from commutative rings into groupoids, hence the name. 
\end{remark}

% By now the use of Hopf algebroids in situations related to homotopy theory has a long tradition. Good introductions can be found in \cite[A.1]{ravenel_86} and \cite{hovey_04}. 

We can also mimic the notion of being Adams type in the setting of Hopf algebroids, and then naturally relate these to ring spectra of Adams type. 

\begin{definition}
    \label{ch0:def:adams-hopf-algebroid}
    \index{Hopf algebroid!Adams type}
    We say a Hopf algebroid $(A, \Psi)$ is of \emph{Adams type} if $\Psi$ is a filtered colimit of dualizable comodules $\Psi_j$. 
\end{definition}

The key feature, which by now hopefully is expected, is that the maps in \cref{ch0:const:hopf-algebroid-maps-from-spectra} make the pair $(R_*, R_*R)$ into a Hopf algebroid. 

\begin{proposition}[{\cite[1.4.6]{hovey_04}}]
    \label{ch0:prop:hopf-algebroid-from-spectra}
    If $R$ is a flat ring spectrum, then the pair $(R_*, R_*R)$ is a Hopf algebroid. If $R$ is Adams type, then $(R_*, R_*R)$ is an Adams Hopf algebroid. 
\end{proposition}

Given a Hopf algebroid $(A,\Psi)$ we can always talk about modules over the ring $A$. The added extra structure of cooperations, as mentioned earlier, now comes from adding in the relationship to $\Psi$. 

\begin{definition}
    \label{ch0:def:comodule-over-hopf-algebroid}
    \index{Hopf algebroid!Comodule}
    Let $(A, \Psi)$ be a Hopf algebroid. A \emph{comodule} over $(A, \Psi)$, sometimes referred to as a $\Psi$-comodule, is an $A$-module $M$ together with a coassociative and counital map 
    \[\psi\: M\longrightarrow M\otimes_A \Psi.\] 
    The category of comodules over $(A, \Psi)$ is denoted \emph{$\Comod_\Psi$}. 
\end{definition}

\begin{example}
    \label{ch0:ex:modules-as-discrete-Hopf-algebroids}
    \index{Hopf algebroid!Discrete}
    For any commutative graded ring $A$, the pair $(A, A)$ is called a \emph{discrete Hopf algebroid}. The category of comodules over this Hopf algebroid is the normal Grothendieck  abelian category $\Mod_A$ of modules over $A$. The prefix ``discrete'' again comes from the functor perspective. A discrete Hopf algebroid corepresents a functor into discrete groupoids, which are the groupoids with only identity morphisms. 
\end{example}

\begin{construction}
    \label{ch0:const:discretization-adjunction}
    \index{Hopf algebroid!Extended comodule}
    Given an Adams Hopf algebroid $(A, \Psi)$, we can define a discretization map $(A, \Psi)\longrightarrow (A, A)$, which is given by the identity on $A$ and the counit $\epsilon$ on $\Psi$. By abuse of notation we also denote the discretization map by $\epsilon$. From \cite[A1.2.1]{ravenel_86} and \cite[4.6]{barthel-heard-valenzuela_2018} we know that $\epsilon$ induces a faithful exact forgetful functor $\epsilon_* \: \Comod_\Psi \longrightarrow \Mod_A$ with a right adjoint $\epsilon^*$ given by $\epsilon^*(M)\simeq \Psi\otimes_A M$. A comodule in the essential image of $\epsilon^*$ is called an \emph{extended comodule}, or sometimes a \emph{cofree comodule}. Given a $\Psi$-comodule $N$, the $A$-module $\epsilon_* N$ is called the \emph{underlying module} of $N$.  
\end{construction}

It is not true that the category $\Comod_\Psi$ is Grothendieck abelian in general. But, it will be the case in the examples we are interested in. 

\begin{proposition}[{\cite[1.3.1, 1.4.1]{hovey_04}}]
    \label{ch0:prop:comod-is-sm-grothendieck}
    \index{Grothendieck abelian}
    If $(A,\Psi)$ is an Adams Hopf algebroid, then the category $\Comod_\Psi$ is a Grothendieck abelian category generated by the dualizable comodules. Furthermore, there is a symmetric monoidal product $-\otimes_\Psi -$, which on the underlying modules is the normal tensor product of $A$-modules. It has a right adjoint $\iHom_\Psi(-,-)$, making $\Comod_\Psi$ a closed symmetric monoidal category. 
\end{proposition}

\begin{example}
    If $R$ is an Adams type ring spectrum, then \cref{ch0:prop:comod-is-sm-grothendieck} and \cref{ch0:prop:hopf-algebroid-from-spectra} implies that $\Comod_{R_*R}$ is Grothendieck abelian and generated by the dualizable objects. Given any spectrum $X$, then its $R_*$-homology $R_*X$ has a coaction 
    \[R_*X \to R_*X \otimes_{R_*} R_*R,\]
    which is both coassociative and counital, meaning that the homology functor $R_*(-)$ takes values in the more structured category $\Comod_{R_*R}$. 
\end{example}

In \cref{ch0:sssec:morava-E-theories} we saw several versions of $E$-theory, and by \cref{ch0:prop:all-E-local-cats-are-equivalent} we know that all their corresponding $E$-local categories are equivalent. The same occurs for the categories of comodules associated to their respective Adams Hopf algebroid.

\begin{proposition}
    If $p$ is a prime and $n$ a non-negative integer, then the categories of comodules over the Hopf algebroids associated to the spectra $E\np$, $E_p(n)$ and $A = E\np^{h\F_p^\times}$, are all equivalent as symmetric monoidal categories: 
    \[\Comod_{E{\np}_*E\np}\simeq \Comod_{E_p(n)_*E_p(n)}\simeq \Comod_{A_*A}.\]
    Furthermore, given any Landweber exact $v_n$-periodic $\BP$-algebra $E$, then its associated category of comodules is equivalent to the ones above. 
\end{proposition}
\begin{proof}
    The equivalence is given by induction along a weak equivalence of Hopf algebroid, see \cite[4.2, 6.5]{hovey-strickland_2005a}; these are always symmetric monoidal. 
\end{proof}

\begin{notation}
    When the chromatic height $n$ and the prime $p$ is understood, we will use the common notation $\ComodE$ for any of the above categories. 
\end{notation}

\begin{example}
    \label{ch0:ex:E-homology-functor}
    \index{Homology theory!Conservative}
    As we will use this functor a lot, we hammer in the fact that the above Morava $E$-theories are of Adams type. This means that the $E$-homology functor naturally lands in the symmetric monoidal Grothendieck abelian category $\ComodE$. We will mostly focus on the homology theory
    \[E_*\: \Sp\np \to \ComodE,\]
    where we have restricting the domain to $\Sp\np$. This is because it is a \emph{conservative} functor, meaning that it detects equivalences. 
\end{example}

\begin{remark}
    \label{ch0:rm:presenting-stacks}
    In algebraic geometry, Hopf algebroids are usually formulated dually as groupoid objects in affine schemes. The left and right unit maps $A\rightrightarrows \Psi$ induces a presentation of stacks $\mathrm{Spec}(\Psi)\rightrightarrows \mathrm{Spec}(A)$, and the category $\Comod_\Psi$ is equivalent to the category of quasi-coherent sheaves on the presented stack, see \cite[Thm 8]{naumann_07}. 
\end{remark}

\begin{remark}
    In \cref{ch0:ex:adams-type-ring-spectra} we saw that the complex cobordism spectrum $\MU$ was an Adams type ring spectrum. In light of \cref{ch0:rm:presenting-stacks} we can then state the celebrated result of Quillen, see \cite{quillen_1969}, relating stable homotopy theory to formal groups:
    \[\mathrm{Spec}(\MU_*\MU)\rightrightarrows \mathrm{Spec}(\MU_*)\rightarrow \EuScript{M}_{\mathrm{fg}},\]
    where $\EuScript{M}_{\mathrm{fg}}$ is the moduli stack of formal groups. Note that we are brushing some technical details under the rug here, but this is correct at least in spirit. Under these ideas, the category $\ComodE$ corresponds to quasi-coherent sheaves on the open substack $\EuScript{M}_{\mathrm{fg}}^{\leq n} \subset \EuScript{M}_{\mathrm{fg}}$ consisting of formal groups of height less than or equal to $n$. 
\end{remark}

In a presentably symmetric monoidal stable $\infty$-category we had certain objects of special importance---the compact objects and the dualizable objects. These types of objects are also important in the setting of Grothendieck abelian categories. In addition it is important to understand the injective objects. These will become important later in \hyperref[ch1]{Paper I}, as we will use injective objects to approximate other objects and to build certain spectral sequences.  

\begin{proposition}
    \label{ch0:rm:dualizable/compact-comodules}
    \index{Hopf algebroid!Dualizable comodule}
    \index{Hopf algebroid!Compact comodule}
    Let $(A, \Psi)$ be an Adams Hopf algebroid. A $\Psi$-comodule $M$ is dualizable if and only if its underlying $A$-module $\epsilon_* M$ is dualizable, i.e., it is finitely generated and projective. Similarly, a $\Psi$-comodule is compact if and only if its underlying $A$-module is compact, which coincides with being finitely presented. 
\end{proposition}
\begin{proof}
    The first claim is \cite[1.3.4]{hovey_04} and the second is \cite[1.4.2]{hovey_04}. 
\end{proof}

\begin{remark}
    \label{ch0:rm:dualizables-compact-generators}
    As colimits in $\Comod_\Psi$ are exact and are computed in $\Mod_A$, all the dualizable comodules are compact. Hence, the full subcategory of dualizable comodules is a set of compact generators for $\Comod_\Psi$. 
\end{remark}

Let us turn our attention to the injective objects. 

\begin{proposition}[{\cite[2.1]{hovey-strickland_2005b}}]
    \label{ch0:rm:injective-comodules}
    Let $(A, \Psi)$ be an Adams Hopf algebroid. If $I$ is an injective object in $\Comod_\Psi$, then there is an injective $A$-module $Q$, such that $I$ is a retract of the extended comodule $\Psi\otimes_A Q$. 
\end{proposition}

\begin{remark}
    Note that as $\Comod_{\Psi}$ is Grothendieck abelian, it has enough injective objects. This allows us to construct injective resolutions and thus $\Ext$-groups, which is a highly effective computational technique in stable homotopy theory. For example, the pair $(\F_2, \mathcal{A}_2^*)$ where the latter is the dual Steenrod algebra, is a Hopf algebroid. The groups $\Ext^s_{\mathcal{A}_*}(\F_2, \F_2)$ are used in the Adams spectral sequence to approximate homotopy groups of spheres, see \cite{adams_58}. 
\end{remark}

Given an Adams Hopf algebroid $(A, \Psi)$, we also have an associated derived category. By \cite[2.1.2, 2.1.3]{hovey_04} the category of chain complexes of $\Psi$-comodules, $\Ch_\Psi$, has a cofibrantly generated stable symmetric monoidal model structure---see also \cite{barnes-roitzheim_2011} for a slightly different construction. The homotopy category associated to this model structure is the usual unbounded derived category $\Der(\Comod_\Psi)$ associated to the Grothendieck abelian category $\Comod_\Psi$. 

\begin{notation}
    \index{Hopf algebroid!Derived category}
    We will use $\Der(\Psi)$ as our notation for the underlying symmetric monoidal stable $\infty$-category associated with the above model structure. The monoidal unit is $A$, treated as a chain complex in degree $0$.
\end{notation}

\begin{remark}
    We warn the reader that some authors use the notation $\Der(\Psi)$ to refer to a periodic version of derived category of $(A, \Psi)$, especially in the case of $\Psi = E_*E$. This is the case, for example, in \cite{pstragowski_2021}. We will use this periodic category in \hyperref[ch1]{Paper I}, but will keep the two derived categories notationally distinct by adding a superscript: $\Dper(\Psi)$. 
\end{remark}

The discretization adjunction of \cref{ch0:const:discretization-adjunction} also induces an adjunction on the level of derived categories, allowing us to compare the derived category of $(A, \Psi)$ to the derived category of $A$. 

\begin{proposition}
    \index{Hopf algebroid!Discretization}
    Let $(A,\Psi)$ be an Adams Hopf algebroid. In this situation the discretization adjunction 
    \begin{center}
    \begin{tikzcd}
        \Comod_\Psi \arrow[r, yshift = 2pt, "\epsilon_*"] & \Mod_A \arrow[l, yshift=-2pt, "\epsilon^*"]
    \end{tikzcd}
    \end{center}
    induces an adjunction
    \begin{center}
    \begin{tikzcd}
        \Der(\Psi) \arrow[r, yshift = 2pt, "\epsilon_*"] & \Der(A) \arrow[l, yshift=-2pt, "\epsilon^*"].
    \end{tikzcd}
    \end{center}
\end{proposition}
\begin{proof}
    This follows from the fact that $\Psi$ is flat over $A$, which implies that both $\epsilon_*$ and $\epsilon^*$ on the abelian categories are exact. 
\end{proof}




\subsubsection{Torsion and completion for comodules}
\label{ch0:sssec:torsion-and-completion-for-comodules}

As we have stated, the notion of localizing subcategories will be important in this thesis; not only in the situation of stable $\infty$-categories, but also in the Grothendieck abelian situation. In the following section we review the construction of a particular kind of localizing subcategory of the Grothendieck abelian category $\Comod_\Psi$ associated to the Hopf algebroid $(A,\Psi)$. These are the categories of comodules that are torsion with respect to some nicely behaved ideal $I\subseteq A$. 

We will consider two approaches to torsion in $\Der(\Psi)$: one ``internal'' and one ``external''. The internal approach uses the classical theory of torsion objects in abelian categories, while the external approach uses local duality, as in \cref{ch0:thm:local-duality}. These two will luckily be equivalent in the situations we are interested in. 

We first review the abelian situation---the internal approach. We follow \cite{barthel-heard-valenzuela_2018} and \cite{barthel-heard-valenzuela_2020} in notation and results. 

\begin{definition}
    \label{def:I-power-torsion-module}
    \index{I-power torsion!Module}
    Let $A$ be a commutative ring and $I\subseteq R$ a finitely generated ideal. The $I$-power torsion of an $A$-module $M$ is defined as
    \[T_I^A M = \{x\in M \mid I^k x = 0 \text{ for some } k\in \N\}.\]
    We say a module $M$ is \emph{$I$-power torsion} if the natural comparison map $T_I^A M\longrightarrow M$ is an isomorphism. We denote the full subcategory consisting of $I$-power torsion $A$-modules by $\Mod_A\Itors$. 
\end{definition}

\begin{example}
    Of particular importance for \hyperref[ch1]{the first paper} is the category $\modt$, where $E$ is the Johnson--Wilson spectrum $E_p(n)$, and $I_n$ is the Landweber ideal 
    \[I_n = (p, v_1, \ldots, v_{n-1})\subseteq \pi_* E_p(n).\]
\end{example}

\begin{definition}
    \index{I-adically complete module}
    Let $A$ be a commutative ring and $I\subseteq R$ a finitely generated ideal. The $I$-adic completion of an $A$-module $M$ is defined as
    \[C_I^A M = \lim_k A/I^k\otimes_A M.\]
    We say a module $M$ is \emph{$I$-adically complete} if the natural map 
    \[M\longrightarrow C_I^A M\]
    is an isomorphism. 
\end{definition}

\begin{remark}
    \label{ch0:rm:I-complete-vs-I-adically-complete}
    \index{I-complete module}
    \index{L-complete module}
    The resulting category of $I$-adically complete modules is not very well-behaved. The $I$-adic completion functor is usually neither left nor right exact; the resulting category is often not abelian. To fix these issues, Greenlees and May introduced the notion of $L$-complete modules in \cite{greenlees-may_92}, using instead the zeroth left derived functor $L=\mathbb{L}_0 C_I^A$. One then defines \emph{$I$-complete} modules, also called $L$-complete modules, to be those $R$-modules such that the natural map $M\longrightarrow L M$ is an equivalence. The full subcategory of $I$-complete $A$-modules is denoted by $\Mod_A\Icomp$
\end{remark}

\begin{remark}
    \index{Grothendieck abelian}
    In nice cases the abelian category $\Mod_A\Itors$ is even Grothendieck. The category $\Mod_A\Icomp$ is abelian, but not Grothendieck in general. It is, however, a locally presentable abelian category with enough projective objects, which is kind of ``dual'' to being Grothendieck abelian. 
\end{remark}

\begin{remark}
    The inclusion of the full subcategory 
    \[\Mod_A\Itors\hookrightarrow \Mod_A\] 
    has a right adjoint, which coincides with the $I$-power torsion $T_I^A(-)$. This gives the $I$-power torsion another description as the colimit 
    \[T_I^A M \cong \colim_k \iHom_A (A/I^k, M).\]
    Furthermore, it means that the category $\Mod_R\Itors$ is a localizing subcategory of $\Mod_R$, in the sense that it is a full subcategory closed under quotients, subobjects, extensions and arbitrary coproducts. 
\end{remark}

We want to extend the construction of $I$-torsion and $L$-complete modules to general Adams Hopf algebroids $(A,\Psi)$. For this, we need to choose sufficiently nice ideals that interact nicely with the additional comodule structure. 

\begin{definition}
    \index{Invariant ideal}
    \index{Invariant ideal!Regular}
    Let $(A, \Psi)$ be an Adams Hopf algebroid, and $I$ an ideal in $A$. We say $I$ is an \emph{invariant ideal} if, for any comodule $M$, the comodule $IM$ is a subcomodule of $M$. If $I$ is finitely generated by $(x_1, \ldots, x_r)$ and $x_i$ is non-zero-divisor in $R/(x_1, \ldots, x_{i-1})$ for each $i=1, \ldots, r$, then we say $I$ is \emph{regular}. 
\end{definition}

\begin{definition}
    \label{ch0:def:I-power-torsion-comodule}
    \index{I-power torsion!Comodule}
    Let $(A,\Psi)$ be an Adams Hopf algebroid and $I\subseteq A$ a regular invariant ideal. The $I$-power torsion of a comodule $M$ is defined as 
    \[T_I^\Psi M = \{x\in M \mid I^kx = 0 \text{ for some } k\in \N\}.\]
    We say a comodule $M$ is \emph{$I$-power torsion} if the natural map 
    \[T_I^\Psi M\longrightarrow M\] 
    is an equivalence. The full subcategory of $I$-power torsion comodules is denoted $\Comod_\Psi\Itors$. 
\end{definition}

\begin{remark}
    \label{ch0:rm:torsion-comodules-grothendieck-monoidal}
    By \cite[5.10]{barthel-heard-valenzuela_2018} the full subcategory of $I$-power torsion comodules is a Grothendieck abelian category. It also inherits a symmetric monoidal structure from $\Comod_\Psi$.
\end{remark}

\begin{example}
    Of particular importance for \hyperref[ch1]{the first paper} and \cref{ch3:addendum} is the category $\Comodt$, where $E$ is some version of Morava $E$-theory, and $I_n$ is again the Landweber ideal. This example will follow us through the whole thesis, as it is the ``abelian version'' of the monochromatic category $\M\np$---we will use a significant amount of time and effort to study their relationship. 
\end{example}

\begin{notation}
    Since $\Comod_\Psi\Itors$ is a Grothendieck abelian category, we have an associated derived $\infty$-category $\Der(\Comod_\Psi\Itors)$. We will usually simplify the notation and denote it by $\Der(\Psi\Itors)$.
\end{notation}

\begin{remark}
    \label{ch0:rm:complete-comodules-not-abelian}
    Unfortunately, the corresponding versions of $I$-adically complete and $L$-complete comodules do not in general form abelian categories, as we can have problems with the comodule structure on certain cokernels. However, there are some ways to improve the situation, see for example \cite{baker_2009}. 
\end{remark}

\begin{remark}
    As for modules, the inclusion 
    \[\Comod_\Psi\Itors\hookrightarrow \Comod_\Psi\] 
    has a right adjoint that corresponds to the $I$-power torsion construction $T_I^\Psi$. This, by \cite[5.5]{barthel-heard-valenzuela_2018} also has the alternative description
    \[T_I^\Psi M \cong \colim_k \iHom_\Psi (A/I^k, M).\]
    This again means that the category $\Comod_\Psi\Itors$ is a localizing subcategory of $\Comod_\Psi$. 
\end{remark}

Comparing side by side, the constructions of $I$-power torsion in $\Mod_A$ and $\Comod_\Psi$ are completely analogous. Hence, one can wonder whether they agree on the underlying modules. This turns out to be the case. 

\begin{lemma}[{\cite[5.7]{barthel-heard-valenzuela_2018}}]
    \label{ch0:lm:torsion-comodule-iff-torsion-module}
    For any $\Psi$-comodule $M$ there is an isomorphism of $A$-modules 
    \[\epsilon_* T^\Psi_I M \cong T^A_I \epsilon_* M.\]
    Furthermore, if an $A$-module $N$ is $I$-power torsion, then the extended comodule $\Psi\otimes_A N$ is $I$-power torsion. In particular, a $\Psi$-comodule $M$ is $I$-power torsion if and only if the underlying $A$-module is $I$-power torsion. 
\end{lemma}

As mentioned above, we will later make use of injective objects in $\Comod_\Psi\Itors$. Hence, we recall some facts about these. 

\begin{lemma}
    \label{ch0:lm:injectives-in-torsion-comodules}
    Let $(A, \Psi)$ be an Adams Hopf algebroid and $I$ a regular invariant ideal.
    \begin{enumerate}
        \item If $J$ is an injective object in $\Comod_\Psi$, then $T_I^\Psi J$ is an injective object in $\Comod_\Psi\Itors$.
        \item There are enough injective objects in $\Comod_\Psi\Itors$.
        \item Any injective $J'$ in $\Comod_\Psi\Itors$ is a retract of an object of the form $T_I^\Psi J$ for an injective $\Psi$-comodule $J$.
    \end{enumerate} 
\end{lemma}
\begin{proof}
    The first point is \cite[2.1.4]{brodmann-sharp_1998}. The second follows from the category $\Comod_\Psi\Itors$ being Grothendieck abelian, as mentioned in \cref{ch0:rm:torsion-comodules-grothendieck-monoidal}. The third point is stated in the proof of \cite[3.16]{barthel-heard-valenzuela_2020}. 
\end{proof}

\begin{remark}
    \label{ch0:rm:injectives-in-torsion-modules}
    \index{Hopf algebroid!Discrete}
    By choosing a discrete Hopf algebroid $(A,A)$, \cref{ch0:lm:injectives-in-torsion-comodules} implies that injective objects in $\Mod_A\Itors$ are retracts of $T_I^A(Q)$ for some injective $A$-module $Q$ and that $T_I^A$ preserves injective objects. As noted in \cref{ch0:rm:injective-comodules}, an injective object in $\Comod_\Psi$ is a retract of an extended comodule of the form $\Psi\otimes_A Q$ for an injective $A$-module $Q$. This means that any injective object $J$ in $\Comod_\Psi\Itors$ is a retract of $T_I^\Psi(\Psi\otimes_A Q)$ where $Q$ is an injective $A$-module. 
\end{remark}

\begin{remark}
    \label{ch0:rm:dualizable/compact-torsion-comodule}
    \index{Hopf algebroid!Dualizable comodule}
    \index{Hopf algebroid!Compact comodule}
    As colimits in the category $\Comod_\Psi\Itors$ are computed in $\Comod_\Psi$, we have, similar to \cref{ch0:rm:dualizable/compact-comodules}, that an $I$-power torsion $\Psi$-comodule $M$ is dualizable if and only if its underlying $A$-module is finitely generated and projective. Similarly, it is compact if and only if the underlying $A$-module is finitely presented. 
\end{remark}

% \begin{lemma}
%     \label{ch0:lm:torsion-comodules-generated-by-compacts}
%     Let $(A,\Psi)$ be an Adams Hopf algebroid, where $A$ is noetherian and $I\subseteq A$ a regular invariant ideal. Then $\Comod_\Psi^{I-tors}$ is generated under filtered colimits by the compact $I$-power torsion comodules. 
% \end{lemma}
% \begin{proof}
%     By \cite[3.4]{barthel-heard-valenzuela_2020} $\Comod_\Psi^{I-tors}$ is generated by the set 
%     $$\mathrm{Tors}_\Psi^{fp}:=\{G\otimes A/I^k \mid G \in \Comod_\Psi^{fp}, k\geq 1\},$$
%     where $\Comod_\Psi^{fp}$ is the full subcategory of dualizable $\Psi$-comodules. Since $I$ is finitely generated and regular, $A/I^k$ is finitely presented as an $A$-module, hence it is compact in $\Comod_\Psi^{I-tors}$ by \cref{ch0:rm:dualizable/compact-comodules} and \cref{ch0:rm:dualizable/compact-torsion-comodule}. As $A$ is noetherian, being finitely generated and finitely presented coincide. The tensor product of finitely generated modules is finitely generated, hence any element in $\mathrm{Tors}_\Psi^{fp}$ is compact. 
% \end{proof}

% \begin{remark}
%     We are under the impression that the assumption that the ring $A$ is noetherian can most likely be removed, but for the results in this thesis we will not need that added level of generality.  
% \end{remark}


We now move to the external approach, using local duality as in \cref{ch0:ssec:local-duality}. 

\begin{construction}
    \label{ch0:const:local-duality-hopf-algebroid}
    \index{Local duality!Diagram}
    If $(A, \Psi)$ is an Adams Hopf algebroid and $I\subseteq A$ a regular invariant ideal, then $A/I$, treated as a complex concentrated in degree zero, is by \cite[5.13]{barthel-heard-valenzuela_2018} a compact object in $\Der(\Psi)$. Thus, $(\Der(\Psi), A/I)$ is a local duality context, and we can consider the corresponding local duality diagram
    \begin{equation*}
        \begin{tikzcd}
            & {\Der(\Psi)\Iloc} \\
            & {\Der(\Psi)} \\
            {\Der(\Psi)\Itors} && {\Der(\Psi)\Icomp}
            \arrow["L_I^\Psi", xshift=-2pt, from=2-2, to=1-2]
            \arrow[xshift=2pt, from=1-2, to=2-2]
            \arrow["\Delta_I^\Psi", yshift=2pt, xshift=2pt, from=2-2, to=3-3]
            \arrow[yshift=-2pt, xshift=-1pt, from=3-3, to=2-2]
            \arrow["\Gamma^\Psi_I", yshift=-2pt, xshift=2pt, from=2-2, to=3-1]
            \arrow[yshift=2pt, xshift=-1pt, from=3-1, to=2-2]
            \arrow[bend left=35, dashed, from=3-1, to=1-2]
            \arrow[bend left=35, dashed, from=1-2, to=3-3]
            \arrow["\simeq"', swap, from=3-1, to=3-3]
        \end{tikzcd}    
    \end{equation*}
    where we have used the superscript $I$ instead of $A/I$ for simplicity, and to invoke the connection to $I$-power torsion and $L$-complete objects. This gives, in particular, an alternative notion of $I$-torsion objects in $\Der(\Psi)$ as $\Der(\Psi)\Itors$. 
\end{construction}

Our goal was to give two constructions and prove that they were equal in the cases we were interested in. 

\begin{lemma}[{\cite[3.7(2)]{barthel-heard-valenzuela_2020}}]
    \label{ch0:lm:derived-torsion-if-homology-torsion}
    \index{Invariant ideal!Regular}
    Let $(A,\Psi)$ be an Adams Hopf algebroid and $I\subseteq A$ a regular invariant ideal. There is an equivalence of categories 
    \[\Der(\Psi)\Itors\simeq \Der(\Psi\Itors).\] 
    Furthermore, an object $M\in \Der(\Psi)$ is $I$-torsion if and only if the homology groups $H_* M$ are $I$-power torsion $\Psi$-comodules.
\end{lemma}

One can wonder whether the same is true for the $I$-complete derived category, but this is unfortunately not true, as $\Comod_\Psi\Icomp$ is not abelian. A partial result can, however, be recovered for discrete Hopf algebroids $(A, A)$. We follow \cite{barthel-heard-valenzuela_2020} in the following construction. 

\begin{construction}
    \label{ch0:const:completed-derived-category}
    \index{I-complete module}
    \index{L-complete module}
    Recall that $\Mod_A\Icomp$ denotes the category of $I$-complete $A$-modules for $I\subseteq A$ a regular ideal. By \cite[2.11]{barthel-heard-valenzuela_2020} the category has enough projective objects, hence by \cite[1.3.2]{Lurie_HA} we can associate to it the right bounded derived category $\Der^-(\Mod_A\Icomp)$. This has a by \cite[1.3.2.19, 1.3.3.16]{Lurie_HA} a left complete $t$-structure with heart equivalent to $\Mod_A\Icomp$. We can then form its right completion, which we denote $\overline{\Der}(\Mod_A\Icomp)$, and call the completed derived category of $\Mod_A\Icomp$. 
\end{construction}

This is what allows for a partial version of \cref{ch0:lm:derived-torsion-if-homology-torsion} in the present case. 

\begin{proposition}[{\cite[3.7(1)]{barthel-heard-valenzuela_2020}}]
    \label{ch0:prop:pulling-out-completion}
    Let $A$ be a commutative ring and $I\subseteq A$ a regular ideal. Then, there is an equivalence 
    \[\Der(\Mod_A)\Icomp\simeq \overline{\Der}(\Mod_A\Icomp),\]
    where the former category is the full subcategory of $A/I$-complete objects in $\Der(\Mod_A)$ while the latter is the completed derived category of $\Mod_A\Icomp$. 
\end{proposition}











