
\section{The laypersons introduction}
\label{ch0:layperson}

Mathematics is one of the longest, richest and best preserved traditions humanity has ever created. It is generally accepted that mathematics has its origins in the realization that things can be counted and enumerated---that collections of things can be said to have a certain numerical size. This realization developed to the simplest theory of numbers: \emph{arithmetic}. The process of counting and doing arithmetic was used to create new knowledge and new technology: seasonal changes; lunar cycles; solar cycles; astronomy; agriculture; crop cycles; animal populations. The list could go on. Another part of mathematics came about later, when humans needed to more precisely describe the shape of land they owned---forming the field of \emph{geometry}. In fact, one of the earliest uses of the word \emph{mathematics} was by the Pythagoreans as a common name for arithmetic and geometry (\cite[1.1]{history}).

These two parts of mathematics was essentially all there was for thousands of years; in some very general way, these are still all there is to mathematics as a whole. Arithmetic and geometry continued in their own traditions all the way through antiquity and the premodern eras. This is seen, for example, in the \emph{seven liberal arts}, which acted as the educational base for students at the first universities (\cite{universities})---based on ideas dating back to Plato (\cite{plato}) and Aristotle (\cite{aristotle}). The first three, named the \emph{trivium}, consisted of grammar, rhetoric and logic. The remaining four, called the \emph{quadrivium}, consisted of music, arithmetic, geometry and astronomy.

Throughout the ages, research and the search for knowledge has undergone several revolutions; research is by now an incredibly rigorous process. The two fields of mathematics have expanded to immense sizes, and now contain hundreds upon hundreds of subfields. One of the most interesting developments---in my humble unbiased opinion---was the development of bridges, connections and similarities between geometry and arithmetic. For example, numbers---now meaning not only the numbers used for counting, but also concepts like real and complex numbers---were used by Descartes (\cite{descartes}) to make coordinate systems, where one could more easily study functions and analysis via equations. Using these coordinate systems one can build manifolds, which are geometric objects generalizing surfaces to arbitrary dimensions. 

The study of number systems eventually became the mathematical field of \emph{algebra}, while the study of shapes formed the mathematical field of \emph{topology}. It was soon discovered that using methods and techniques from algebra could greatly affect topology, leading to the field of \emph{algebraic topology}. This arguably started with Riemann studying ``connectedness numbers'' for certain spaces (\cite{riemann_1857}), an idea that was further developed by Betti (\cite{betti_1870}) and Poincaré (\cite{poincare_1895}). These numbers describe how many $n$-dimensional holes a space has. It can be difficult to imagine what holes of different dimension are, so perhaps the following comparison is useful: a circle has a hole in the middle, but so does a sphere, only that this hole is somehow of a larger dimension. The first hole is considered to be $1$-dimensional, as the hole is bound by a $1$-dimensional line; the second is considered to be $2$-dimensional, as it is bound by a $2$-dimensional surface. 

These numbers allows us to distinguish different abstract mathematical spaces. For example, if we are given two spaces $X$ and $Y$, and we can show that $X$ has a hole in dimension $13$ while $Y$ does not, then the spaces cannot be the same. The existence of holes is something that we can often compute even though we cannot visually describe the space, for example due to its large dimension. Assigning these special numbers to a space gives us a concrete computable property of it, and builds a ``bridge'' between spaces and numbers. In certain situations knowing exactly how many holes a space has of each dimension allows us to uniquely determine which space it is, so these numbers can sometimes work as a replacement for the space itself. 

The act of building connections between the two fields of mathematics is also what this thesis is about; it is about continuing the long and deep tradition of understanding the interplay of these fields. Specifically we focus on the interactions between two modern modern subfields: \emph{homological algebra} and \emph{stable homotopy theory}. The former is a subfield of algebra, where one studies the structure of the ``system of all different systems of numbers''. The latter is a subfield of topology, where one studies the structure of the ``system of all different systems of shapes''. 

We can then make a very general description of what the contents of this thesis is about: it is about studying three different bridges between stable homotopy theory and homological algebra---between shapes and numbers. These three bridges are each located in their own research paper, which form the main content of this thesis. Let us very briefly try to explain what each of these bridges are: 

The most concrete bridge comes from {\hyperref[ch1]{the first paper}}. The system of shapes at hand is in some sense a very ``fundamental'' system, as it arises as the smallest constituents---the atoms, if you will---of perhaps the most important system we have in stable homotopy theory. We directly compare this atomic system of shapes to a specific system of numbers, and prove that they are in fact equivalent; they have exactly the same structure. 

In {\hyperref[ch2]{the second paper}} the bridge is more indirect. We take a feature of certain number systems, and study an analogous concept for certain shapes. Doing this we are able to recover and generalize some already known results in topology, now seen from a completely new angle. For example, we are able to obtain two new descriptions of the atomic system we studied in the first paper. 

The bridge in {\hyperref[ch3]{the third paper}} is again quite direct, where we have a comparison between certain substructures of shapes, and certain substructures of numbers. We prove that there is a one-to-one correspondence between these collections of substructures, which provides new insight into the system of shapes that we started with. It also gives us a deeper understanding of the equivalent systems from the first paper. 