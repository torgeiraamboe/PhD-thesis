

\section*{Abstract}
\addcontentsline{toc}{section}{Abstract}

Chromatic homotopy theory aims to study the category of spectra, $\Sp$, by splitting it into an infinite family of colors---like shining white light through a prism. Each individual color is specified by a wavelength, or periodicty, given by $2p^n-2$, where $p$ is a prime number and $n$ a non-negative number called the chromatic height. The corresponding piece of $\Sp$ is controlled by a spectrum $\Kpn$, called the height $n$ Morava $K$-theory. The overarching goal of monochromatic homotopy theory, and this thesis, is to understand this piece, called the category of $\Kpn$-local spectra, $\Sp_\Kpn$. 

At chromatic height $0$ we have $K_p(0) = H\Q$, giving us rational stable homotopy theory---a completely algebraic theory. It is known that the analogous statement fails at positive heights: $\Sp_\Kpn$ is not algebraic. However, our first paper uses Patchkoria--Pstr\a{}gowski's version of Franke's algebraicity theorem to prove that $\Sp_\Kpn$ is \emph{exotically} algebraic when $p$ is much larger than $n$. More precisely we prove that $\Sp_\Kpn$ is exotically equivalent to $\Fr\np\Inc$: the Franke category of derived $I_n$-complete comodules over $E_*E$. 

In the second paper we prove a version of Positselski's comodule-contramodule correspondence for coidempotent cocommutative coalgebras in a presentable $\infty$-category $\C$. When $\C$ is stable, we show that this recovers local duality in the sense of Hovey--Palmieri--Strickland, allowing us to describe the category of $\Kpn$-local spectra as contramodules over the monochromatization of the sphere spectrum. 

Let $\C$ be a stable $\infty$-category with a $t$-structure $(\C\geqz, \C\leqz)$. In the third paper we classify which localizing subcategories of $\C$ that are completely determined by the Grothendieck abelian heart $\C\geqz \cap \C\leqz = \C^\heart$. This generalizes previous work by Takahashi, and extends a similar classification due to Lurie. It also allows us to construct a deformation between $\Sp_\Kpn$ and $\Fr\np\Inc$, and to prove a symmetric monoidal version of the exotic equivalence from the first paper. 




% This thesis consists of three papers---all focused on understanding certain features of localizing subcategories; all motivated by understanding features of chromatic homotopy theory. 

% Our first paper uses Patchkoria--Pstr\a{}gowski's version of Franke's algebraicity theorem to prove that monochromatic homotopy theory is completely algebraic when the prime $p$ is large compared to the chromatic height $n$. In particular, we prove that the category of spectra, localized the Morava $K$-theory spectrum $K_p(n)$, is equivalent to the derived $I_n$-complete objects in Franke's category of periodic comodules over $E_*E$---the Hopf algebroid associated to height $n$ Morava $E$-theory. 

% In the second paper we introduce contramodules over a cocommutative coalgebra in a presentably symmetric monoidal $\infty$-category. When the coalgebra $C$ is coidempotent, we prove that there is a symmetric monoidal duality between comodules and contramodules over $C$, which we call Positselski duality. Furthermore, when the ambient category is stable and compactly generated by dualizable objects, this duality recovers local duality in the sense of Hovey--Palmieri--Strickland, allowing us to describe the category of $K_p(n)$-local spectra as contramodules over the monochromatization of the sphere spectrum. 

% The third paper studies how certain localizing subcategories compatible with a given $t$-structure on a stable $\infty$-category $\C$, can be classified by using the associated Grothendieck prestable $\infty$-category $\C\geqz$ and the associated Grothendieck abelian heart $\C^\heart$. In particular, we prove that there is a one-to-one correspondence between $t$-structure compatible localizing subcategories in $\C$, and prestable localizing subcategories of $\C\geqz$. This allows us to extend a result of Lurie to the stable setting, and prove a classification result for these localizing subcategories. 


\newpage 
\section*{Sammendrag}
\addcontentsline{toc}{section}{Sammendrag}

Kromatisk homotopiteori forsøker å studere kategorien av spektra, $\Sp$, ved å dele den opp i et uendelig antall farger---som å skinne hvitt lys gjennom et prisme. Hver individuelle farge har en bølgelengde, eller periodisitet, gitt som $2p^n-2$, der $p$ er et primtall og $n$ er et ikke-negativt heltall kalt den kromatiske høyden. Den korresponderende delen av $\Sp$ kontrolleres av et spektrum $\Kpn$ kalt Morava $K$-teori av høyde $n$. Det overordnede målet med monokromatisk homotopiteori, samt denne abhandlingen, er å forstå denne delen, som kalles kategorien av $\Kpn$-lokale spektra, $\Sp_\Kpn$. 

Når den kromatiske høyden er $0$ har vi $K_p(0) = H\Q$, som gir oss rasjonal stabil homotopiteori---en fullstending algebraisk teori. Det er kjent at den tilsvarende påstanden ikke er sann ved positive høyder: $\Sp_\Kpn$ er ikke algebraisk. I avhandlingens første viser vi likevell at $\Sp_\Kpn$ er \emph{eksotisk} algebraisk når $p$ er mye større enn $n$,ved å anvende Patchkoria--Pstr\a{}gowskis versjon av Frankes algebraisitetsteorem. Mer presist viser vi at $\Sp_\Kpn$ er eksotisk ekvivalent til $\Fr\np\Inc$: Frankes kategori av derivert $I_n$-komplette komoduler over $E_*E$. 

I den andre artikkelen beviser vi en versjon av Positselskis komodul-kontramodul-korrespondanse for koidempotente kokommutative koalgebraer i en presenterbar $\infty$-kategori $\C$. Når $\C$ er stabil viser vi at dette gjennskaper Hovey--Palmieri--Stricklands lokal dualitet, som lar oss beskrive kategorien av $\Kpn$-lokale spektra som kontramoduler over monokromatiseringen av sfærespektumet. 

La $\C$ være en stabil $\infty$-kategori med en $t$-struktur $(\C\geqz, \C\leqz)$. I den tredje artikkelen klassifiserer vi hvilke lokaliserende underkategorier av $\C$ som er fullstendig bestemt av det Grothendieck-abelske hjertet $\C\geqz \cap \C\leqz = \C^\heart$. Dette generaliserer tidligere resultater av Takahashi, og utvider en lignende klassifisering av Lurie. Det lar også konstruere en deformasjon mellom $\Sp_\Kpn$ og $\Fr\np\Inc$, og vise en symmetrisk monoidal versjon av den eksotiske eksivalensen fra den første artikkelen. 

% Denne avhandlingen består av tre artikler---alle fokusert på å forstå visse egenskaper av lokaliserende underkategorier; alle motivert av å forstå egenskaper iboende kromatisk homotopiteori. 

% Vår første artikkel bruker Patchkoria--Pstr\a{}gowskis versjon av Frankes algebraisitetsteorem til å bevise at monokromatisk homotopiteori er fullstendig algebraisk når primtallet $p$ er mye større enn den kromatiske høyden $n$. Mer presist viser vi at kategorien av spectra lokalisert ved Morava $K$-teorispektrumet $K_p(n)$, er ekvivalent til $I_n$-komplette objekter i Frankes kategori av periodiske komodules over $E_*E$---Hopf algebroiden assosiert til Morava $E$-teori med høyde $n$. 

% I den andre artikkelen introduserer vi kontramoduler over en kokommutativ koalgebra i en presenterbar symmetrisk monoidal $\infty$-kategori. Når koalgebraen $C$ er koidempotent viser vi at det er en symmetrisk monoidal dualitet mellom komoduler og kontramoduler over $C$, som vi kaller Positselski-dualitet. Videre viser vi at dersom bakgrunnskategorien er stabil og kompakt-generert av dualiserbare objekter, så gjennskaper dette Hovey--Palmieri--Stricklands teori om lokal dualitet, som lar oss beskrive kategorien av $K_p(n)$-lokale spektra som kontramoduler over monokromatiseringen av sfærespektumet. 

% Den tredje artikkelen studerer hvordan lokaliserende underkategorier som er kompatible med en gitt $t$-struktur på en stabil $\infty$-kategori $\C$, kan klassifiseres via den tilhørende Grothendieck-prestabile $\infty$-kategorien $\C\geqz$ og det Grothendieck-abelske hjertet $\C^\heart$. Vi viser at det er en en-til-en korrespondanse mellom disse $t$-struktur-kompatible lokaliserende underkategoriene av $\C$, og prestabile lokaliserende underkategorier av $\C\geqz$. Dette lar oss utvide et resultat av Lurie til den stabile settingen, og bevise et klassifiserings resultat for disse lokaliserende underkategoriene. 


