

\section*{Abstract}
\addcontentsline{toc}{section}{Abstract}

This thesis consists of three papers---all focused on understanding certain features of localizing subcategories; all motivated by understanding features of chromatic homotopy theory. 

Our first paper uses Patchkoria--Pstr\a{}gowski's version of Franke's algebraicity theorem to prove that monochromatic homotopy theory is completely algebraic when the prime $p$ is large compared to the chromatic height $n$. In particular, we prove that the category of spectra, localized the Morava $K$-theory spectrum $K_p(n)$, is equivalent to the derived $I_n$-complete objects in Franke's category of periodic comodules over $E_*E$---the Hopf algebroid associated to height $n$ Morava $E$-theory. 

In the second paper we introduce contramodules over a cocommutative coalgebra in a presentably symmetric monoidal $\infty$-category. When the coalgebra $C$ is coidempotent, we prove that there is a symmetric monoidal duality between comodules and contramodules over $C$, which we call Positselski duality. Furthermore, when the ambient category is stable and compactly generated by dualizable objects, this duality recovers local duality in the sense of Hovey--Palmieri--Strickland, allowing us to describe the category of $K_p(n)$-local spectra as contramodules over the monochromatization of the sphere spectrum. 

The third paper studies how certain localizing subcategories compatible with a given $t$-structure on a stable $\infty$-category $\C$, can be classified by using the associated Grothendieck prestable $\infty$-category $\C\geqz$ and the associated Grothendieck abelian heart $\C^\heart$. In particular, we prove that there is a one-to-one correspondence between $t$-structure compatible localizing subcategories in $\C$, and prestable localizing subcategories of $\C\geqz$. This allows us to extend a result of Lurie to the stable setting, and prove a classification result for these localizing subcategories. 


\newpage 
\section*{Sammendrag}
\addcontentsline{toc}{section}{Sammendrag}

Denne avhandlingen består av tre artikler---alle fokusert på å forstå visse egenskaper av lokaliserende underkategorier; alle motivert av å forstå egenskaper iboende kromatisk homotopiteori. 

Vår første artikkel bruker Patchkoria--Pstr\a{}gowskis versjon av Frankes algebraisitetsteorem til å bevise at monokromatisk homotopiteori er fullstendig algebraisk når primtallet $p$ er mye større enn den kromatiske høyden $n$. Mer presist viser vi at kategorien av spectra lokalisert ved Morava $K$-teorispektrumet $K_p(n)$, er ekvivalent til $I_n$-komplette objekter i Frankes kategori av periodiske komodules over $E_*E$---Hopf algebroiden assosiert til Morava $E$-teori med høyde $n$. 

I den andre artikkelen introduserer vi kontramoduler over en kokommutativ koalgebra i en presenterbar symmetrisk monoidal $\infty$-kategori. Når koalgebraen $C$ er koidempotent viser vi at det er en symmetrisk monoidal dualitet mellom komoduler og kontramoduler over $C$, som vi kaller Positselski-dualitet. Videre viser vi at dersom bakgrunnskategorien er stabil og kompakt-generert av dualiserbare objekter, så gjennskaper dette Hovey--Palmieri--Stricklands teori om lokal dualitet, som lar oss beskrive kategorien av $K_p(n)$-lokale spektra som kontramoduler over monokromatiseringen av sfærespektumet. 

Den tredje artikkelen studerer hvordan lokaliserende underkategorier som er kompatible med en gitt $t$-struktur på en stabil $\infty$-kategori $\C$, kan klassifiseres via den tilhørende Grothendieck-prestabile $\infty$-kategorien $\C\geqz$ og det Grothendieck-abelske hjertet $\C^\heart$. Vi viser at det er en en-til-en korrespondanse mellom disse $t$-struktur-kompatible lokaliserende underkategoriene av $\C$, og prestabile lokaliserende underkategorier av $\C\geqz$. Dette lar oss utvide et resultat av Lurie til den stabile settingen, og bevise et klassifiserings resultat for disse lokaliserende underkategoriene. 


