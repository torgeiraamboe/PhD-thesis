
\section*{Information}
\addcontentsline{toc}{section}{Information}

This thesis is structured into four chapters, mainly consisting of three research papers where the candidate is the only author.

\hyperref[ch0]{The zero'th} chapter consists of mathematical preliminaries, as well as a short summary of each paper. There is also an introduction for the layperson, mostly aimed at family and friends, for situating the topics of this thesis amidst the broad world of mathematics, and to give a sense of what its contents is about. \hyperref[ch1]{The first chapter} presents the paper \emph{Algebraicity in monochromatic homotopy theory} (\cite{aambo_2024_algebraicity}). \hyperref[ch2]{The second chapter} presents the paper \emph{Positselski duality in stable $\infty$-categories} (\cite{aambo_2024_positselski}), as well as an addendum on contramodules over topological rings. Lastly, \hyperref[ch3]{the third chapter} presents the paper \emph{Classification of localizing subcategories along $t$-structures} (\cite{aambo_2024_localizing}), together with an addendum on monochromatic synthetic spectra, linking the third paper back to the contents of the first. The introduction is loosely based on the background material from the original version of the first paper, which was removed from the later versions. 

Before each of the papers there is a title page and a drawing. The title pages were made with an old typewriter, and then color-inverted to easy spot the transitions between papers when flipping through the thesis. The drawings have two functions: enumerate the papers and give an abstract visualization for what the paper is about. A description of the drawing can be found on each subsequent page. Their style was inspired by the iconic album art of Joy Division's debut album \emph{Unknown Pleasures}, \cite{joy-division_79}, made by Peter Saville.  Following each paper there is a poem---in the form of a limerick---summarizing the main result from the associated paper. It also serves as a separator between the paper and the addendums. 

The acknowledgements, the bibliography, as well an an index, can be found at the end of the thesis. 

All of the content in the thesis, written and graphical, is made by the author; no artificial intelligence has been used, in any shape or form. 