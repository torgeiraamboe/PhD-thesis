
\section*{Information}
\addcontentsline{toc}{section}{Information}

The contents of this thesis consist mainly of material from the papers \cite{aambo_2024_algebraicity}, \cite{aambo_2024_positselski} and \cite{aambo_2024_localizing}, where the candidate is the only author. In addition there are some added remarks, some further results not yet presented in any papers, some more historical background, as well as more in-depth introductions to the central ideas of the thesis. 

This material is structured into four chapters. \hyperref[ch0]{Chapter 0} consists of mathematical preliminaries and background, as well as a short summary of each paper. There is also an introduction for the layperson, mostly aimed at family and friends, for situating the topics of this thesis amidst the broad world of mathematics, and to give a vague sense of what its contents is about.  

The three remaining chapters each consists of one of the above-mentioned papers. \hyperref[ch1]{The first chapter} presents the paper \emph{Algebraicity in monochromatic homotopy theory} (\cite{aambo_2024_algebraicity}). \hyperref[ch2]{The second chapter} presents the paper \emph{Positselski duality in stable $\infty$-categories} (\cite{aambo_2024_positselski}) as well as an addendum on contramodules over topological rings. Lastly, \hyperref[ch3]{the third chapter} presents the paper \emph{Classification of localizing subcategories along $t$-structures} (\cite{aambo_2024_localizing}), together with an addendum on monochromatic synthetic spectra, linking the third paper back to the contents of the first one, attempting to create a certain sense of cohesion and circularity. 

Before each of the papers there is a titlepage, a poem and a drawing, each representing the contents of the paper. These are all made by the author. The poems are in the form of limericks, and each describe the main result from the associated paper. The drawings have two functions: enumerate the papers and give a visual feel for what the paper is about. A description of the drawing can be found on each subsequent page. The style of the drawings is inspired by the iconic album art of Joy Division's debut album \emph{Unknown Pleasures}, \cite{joy-division_79}, made by Peter Saville. 
