
\section{Addendum: Subcategories of synthetic spectra}
\label{ch3:addendum}

One of the main motivations for \cref{ch3:thm:main} was to understand localizing subcategories of Pstr\a{}gowski's category of $E$-based synthetic spectra---the main reference is \cite{pstragowski_2022}. This section is not part of the paper \cite{aambo_2024_localizing}, but is added to flesh out this example further, and to relate this paper to \cref{ch1}. In particular, our goal is to establish a proof of \cref{ch1:conj:monochromatic-monoidally-algebraic}, giving a symmetric monoidal version of \cref{ch1:thm:main-spectra}. 

We will focus only on the case of synthetic spectra based on height $n$ Morava $E$-theory $E=E\np$ in this section, even though the results we cover will also work in more general setups. We will \emph{not} use the standard category of synthetic spectra, but instead a ``local'' variant, that for familiar readers lie somewhere between $E$-based synthetic spectra and its hypercompletion. 

\begin{definition}
    A finite spectrum $P\in \Sp^\omega$ is said to \emph{$E$-finite projective} if its $E$-homology is finitely generated and projective as an $E_{*}$-module. The full subcategory of $E$-finite projective spectra is denoted $\Sp\fp$. 
\end{definition}

\begin{remark}
    By \cref{ch0:rm:dualizable/compact-comodules}, a finite spectrum $P$ is $E$-finite projective if and only if $E_{*}P$ is a dualizable comodule. 
\end{remark}

We can equip the category $\Sp\fp$ with a Grothendieck topology, defined by covers being single maps $P\to P'$ such that the induced map on $E$-homology is an epimorphism. This makes $\Sp\fp$ an excellent $\infty$-site---see \cite[Section 2.3]{pstragowski_2022} for details. 

\begin{definition}
    \index{Synthetic spectra}
    An \emph{synthetic spectrum} is an additive sheaf $X\: \Sp^{\mathrm{fp}, \mathrm{op}}\to \Sp$. The category of synthetic spectra will be denoted $\SynE := \mathrm{P}_\Sigma(\Sp\np\fp;\Sp)$. 
\end{definition}

Localizing all of the finite projective spectra at $E$ gives us another excellent $\infty$-site, which we denote by $\Sp\np\fp$. To be clear, objects in $\Sp\np\fp$ are of the form $L\np P$, for $P\in \Sp\fp$. 

\begin{definition}
    \index{Synthetic spectra!$E$-local}
    An $E$-local synthetic spectrum is an additive sheaf $X\: \Sp\np^{\mathrm{fp}, \mathrm{op}}\to \Sp$. The category of $E$-local synthetic spectra will be denoted $\LSynE := \mathrm{P}_\Sigma(\Sp\np\fp;\Sp)$. 
\end{definition}

\begin{remark}
    As mentioned, the category of $E$-local synthetic spectra is slightly different from the categories already appearing in the literature. It should be thought of as living somewhere between normal synthetic spectra and hypercomplete synthetic spectra. This category is used to avoid the bad properties of the hypercompletion functor, which for example, is not smashing, meaning compact generation for hypercomplete synthetic spectra is a bit tricky. 
\end{remark}

\begin{remark}
    A slightly more high-brow perspective on $E$-local synthetic spectra is the following. One reason normal synthetic spectra based on $\MU$ work so well is that the sphere $\S$ is $\MU$-nilpotent complete, meaning that the Adams--Novikov spectral sequence converges to $\pi_*\S$. For $E$ this is not the case: the sphere is not $E$-nilpotent complete. But, as $E$-localization is smashing, the $E$-nilpotent completion of the sphere is precisely the $E$-local sphere $L\np \S$. Hence one can think about $\LSynE$ as the natural category of formal ``conditionally convergent'' $E$-Adams spectral sequences, similarly to how $\Syn_\MU$ is the category of formal ``conditionally convergent'' Adams--Novikov spectral sequences. 
\end{remark}

Most of the theory for synthetic spectra work straight out of the box also for $\LSynE$, and their proofs are completely analogous. We mention only the most important ones.  

There is a map of excellent $\infty$-sites $f\: \Sp\fp\to \Sp\np\fp$ given by $E$-localizaton, which satisfies the covering lifting property of \cite[A.12]{pstragowski_2022}. Hence, the associated adjunction 
\[f^*:\SynE \rightleftarrows \LSynE:f_*\]
has $f_* \: \LSynE \to \SynE$ a colimit preserving $t$-exact functor by \cite[2.22, 2.23]{pstragowski_2022}, given by precomposing with $f$. As $E$-localization does not alter the $E_*$-homology of a spectrum, the $t$-structure on $\LSynE$ is also $\ComodE$, and the functor $f_*$ extends the identity on the level of hearts. 

Similarily to \cite[4.4, 4.38]{pstragowski_2022}, there is a lax monoidal fully faithful functor $\nu\:\Sp\np \to \LSynE$ called the synthetic analog. The category $\LSynE$ is compactly generated by the objects $\nu L\np P$ for $P\in \Sp\fp$, which are also dualizable, just as in \cite[4.14]{pstragowski_2022}. The unit is also compact, which implies that $\LSynE$ is rigidly compactly generated. 

The functor $f^*$ is the unique colimit preserving functor such that $f^* \nu P \simeq \nu f P$, and as noted above, $f_*$ is given by precomposition. As both $f^*$ and $f_*$ preserve colimits, we can easily check that $f_*$ is fully faithful, via checking that the counit map $f^*f_* \nu P \to \nu P$ is an equivalence. By the description of $f^*$ above, this is obvious, as $E$-localization is an idempotent functor. It follows then, that $\LSynE$ is a smashing localization of $\SynE$---the fact that $f^*$ is compatible with the monoidal structure follows from the fact that it preserves colimits, and $\nu$ is symmetric monoidal on $\Sp\fp$. 

% Hypercompletion = nuE localization 

\begin{remark}
    The $\nu E$-localization functor on normal synthetic spectra is not smashing---a fact following from $\eta$ not being nilpotent in the homotopy groups of spheres. This means that our category of $E$-local synthetic spectra is not equivalent to hypercomplete $E$-based synthetic spectra, as $\nu E$-localization is equivalent to hypercompletion by \cite[5.4]{pstragowski_2022}. 
\end{remark}

The functor $\nu$ induces a deformation parameter $\tau$ on any synthetic spectrum $X$, see \cite[Section 4.3]{pstragowski_2022}, making $\LSynE$ act as a one-parameter deformation between $\Sp\np$ and $\Stable\EE$, as in the following result. 

\begin{theorem}
    \label{ch3:add:thm:deformation-properties-of-LSynE}
    Inverting the deformation parameter $\tau$ gives an equivalence 
    \[\LSynE[\tau^{-1}]\simeq \Sp\np\]
    of symmetric monoidal stable $\infty$-categories. Furthermore, killing $\tau$, via tensoring with its cofiber, gives an equivalence 
    \[\Mod_{C\tau}(\LSynE)\simeq \Stable\EE\]
    of symmetric monoidal stable $\infty$-categories. 
\end{theorem}
\begin{proof}
    The proof of the first equivalence is completely analogous to the setting of normal syntehtic spectra, see \cite[4.37, 4.40]{pstragowski_2022}, except that the associated spectral Yoneda embedding used to give the equivalence starts in $E$-local spectra, i.e., $Y\: \Sp\np \to \LSynE[\tau^{-1}]$. 

    For the second we note that $E$-localization does not alter the $E_*$-homology. Via \cite[2.22, 4.43]{pstragowski_2022} this gives an adjunction 
    \[\LSynE \rightleftarrows \StableE,\]
    which by following the rest of \cite[Section 4.5]{pstragowski_2022} is an equivalence. 
\end{proof}

In a certain sense, this makes $\LSynE$ a categorification of the \emph{conservative} adapted homology theory $E_{n*}\: \Sp\np\to \ComodE$, instead of normal synthetic spectra $\SynE$ being a categorification of $E_* \: \Sp\to \ComodE$. The main result for this addendum is to construct a categorification of the restricted adapted homology theory $E_{*}\: \Mn \to \Comodt$ that we studied in \cref{ch1}, with associated natural deformation properties as above. 

% Given any $E_n$-finite projective $P$ we can localize it at $E_n$ to obtain a compact $E_n$-local spectrum $L_n P$. As $E_n$-homology is invariant under $E_n$-localization, this still has finitely generated and projective $E_n$-homology. We will denote by $\Sp_n\fp$ the full subcategory of $\Sp_n^\omega$ consisting of $E_n$-localizations of $E_n$-finite projectives. This still has a Grothendieck topology given by single $E_n$-epimorphisms, which, as $E_n$-localization is smashing, still is an excellent $\infty$-site. 

% \begin{definition}
%     An $E_n$-local synthetic spectrum is an additive sheaf $X\: \Sp_n^{\mathrm{fp}, \mathrm{op}}\to \Sp$. The category of $E_n$-local synthetic spectra will be denoted $\LSynE := \mathrm{P}_\Sigma(\Sp_n\fp;\Sp)$. 
% \end{definition}

% All of the natural constructions for $\SynE$ also go through for $\LSynE$. We will not go through these here, but refer to \cite{pstragowski_2022}. We do, however, summarize the main properties that we will use for th rest of this addendum. 


\begin{conjecture}
    The connected part of the $t$-structure on $\LSynE$ is equivalent to Patchkoria--Pstr\a{}gowski's separated perfect derived category $\Der^\omega(\Sp\np)$, see \cite[6.49]{patchkoria-pstragowski_2021}. We know this to be true at large primes $p-1>n$, due to the argument in \cite[6.57]{patchkoria-pstragowski_2021}, together with the fact that the category $\ComodE$ has finite cohomological dimension in this range, implying that both the categories are already hypercomplete.
\end{conjecture}




\subsection{Monochromatic synthetic spectra}

In \cite{hovey-strickland_2005a} Hovey and Strickland construct a partial classification of the lcoalizing subcategories of $\Comod_{\BP_*\BP}$---partial in the sense of requiring the containment of compact objects. This result was then used by Barthel--Heard to make a complete classification of the localizing subcategories of $\ComodE$, which we now recall. 

For any $0\leq k\leq n$ we have an ideal $I_k \subseteq \pi_* E_n$, called the Landweber ideals of $E$. More precisely these are given by $I_k = (p, v_1, v_2, \ldots, v_{k-1})$. These ideals are finitely generated regular invariant ideals, hence \cref{ch0:sssec:torsion-and-completion-for-comodules} gives us for any such $k$ a localizing subcategory $\ComodE^{I_k-\mathrm{tors}}\subseteq \ComodE$, called the category of $I_k$-power torsion comodules. \index{$I$-power torsion!Comodules}

\begin{theorem}[{\cite[2.21]{barthel-heard_2018}}]
    \label{ch3:add:thm:classification-of-abelian-localizing}
    If $\T$ is a localizing subcategory in $\ComodE$, then there is an integer $0\leq k\leq n$ such that $\T \simeq \ComodE^{I_k-\mathrm{tors}}$. 
\end{theorem}

\begin{remark}
    \label{ch3:add:rm:chain-of-localizing-subcategories}
    By the above result we do, in fact, get a chain of localizing subcategories
    \[\T_0 \subseteq \T_1\subseteq \cdots \subseteq \T_n \]
    in $\ComodE$, corresponding each to one of the generators in the Landweber ideal $I_n=(p, v_1, v_2, \ldots, v_{n-1})\subseteq \pi_* E$. Hence this result also classifies the localizing subcategories in the torsion categories $\ComodE^{I_k-\mathrm{tors}}$ themselves.  
\end{remark}

For simplicity we will focus only on the case when $k=n$, giving us the category $\Comodt$. We make the following definition. 

\begin{definition}
    \index{Synthetic spectra!Monochromatic}
    \index{Monochromatic!Synthetic spectra}
    \index{Localizing subcategory!$\pi$-exact}
    The unique $\pi$-exact lift of the localizing subcategory $\Comodt$ is denoted $\MSynE$. We call it the category of height $n$ \emph{monochromatic synthetic spectra}. 
\end{definition}

The justification for this name also in the synthetic setting is due to the following result. 

\begin{lemma}
    \label{ch3:add:lm:mono-iff-syn-mono}
    If $X\in \Sp\np$ is an $E$-local spectrum, then we have $\nu X \in \MSynE$ if and only if $X\in \M\np$. 
\end{lemma}
\begin{proof}
    By definition we have 
    \[\nu X \in \MSynE \iff \pi_k^\heart \nu X \in \Comodt \text{ for all } k\in \Z.\]
    By \cite[4.21, 4.22]{pstragowski_2022} there is an isomorphism of $E_*E$-comodules $\pi_k^\heart \nu X \simeq E_{*} X[-k]$, meaning that the $E_{*}$-homology of $X$ is $I_n$-torsion. By \cref{ch1:lm:monochromatic-iff-torsion-comodules} this is the case if and only if $X\in \M\np$, finishing the proof. 
\end{proof}










\subsection{Compact generation}

By our result \cref{ch1:lm:torsion-comodules-generated-by-compacts} from \cref{ch1} we know that the category $\Comodt$ is compactly generated, with an explicit set of compact generators given by 
\[\TorsE := \{G\otimes E_*/I_n^k \mid G\in \ComodE\fp, k\geq 1\}.\]
For the moment we want to avoid any near-lying problematic telescopes in $E$-local synthetic spectra, and prove that the unique lift of $\Comodt$---which was our definition of $\MSynE$---is also a compactly generated localizing subcategory. This will require a more refined analysis compared to just lifting the localizing subcategories via \cref{ch3:thm:main}. 

Let us start with the prestable case. The functor 
\[\pi_0 \: \LSyn_{E, \geq 0} \to \ComodE\]
preserves compact objects, so instead of lifting $\Comodt$ to the whole category $\LSyn_{E, \geq 0}$ we can instead lift the compact objects in $\Comodt$, which we know is $\Thick(\TorsE)$. This process will not give us a prestable localizing subcategory, but a ``small'' version thereof. 

\begin{definition}
    \index{Thick subcategory!Prestable}
    A full subcategory $\T\geqz \subseteq \LSyn_{E, \geq 0}^\omega$ is \emph{thick} if it is closed under finite coproducts, cofiber sequences and subobjects. 
\end{definition}

\begin{remark}
    This is exactly the definition of a localizing subcategory of a prestable $\infty$-category, just with finite coproducts rather than all coproducts. Just as prestable localizing subcategories are a mix of stable localizing subcategories and abelian localizing subcategories, a prestable thick subcategory is supposed to be a mix of a stable thick subcategory and a Serre subcategory. 
\end{remark}

We can further sharpen the analogy between Serre subcategories and thick subcategories as follows. 

\begin{lemma}
    If $\T^\heart$ is a Serre subcategory of $\ComodE^\omega$, then the full subcategory $\T\geqz\subseteq \LSyn_{E, \geq 0}^\omega$ such that $t\in \T\geqz$ if and only if $\pi^\heart_k t \in \T^\heart$ for all $k\geq 0$, is a thick subcategory of $\LSyn_{E, \geq 0}^\omega$. 
\end{lemma}
\begin{proof}
    The proof is identical to \cite[C.5.2.7]{lurie_SAG}, just with finite coproducts rather than all coproducts. 
\end{proof}

\begin{definition}
    \index{Lift!Prestable}
    The category $\T\geqz$ associated to a Serre subcategoru $\T$ is called the \emph{prestable lift} of $\T$. 
\end{definition}

\begin{remark}
    Similarly to Lurie's classification of abelian localizing subcategories, see \cref{ch3:thm:Lurie-correspondence}, one gets a one-to-one correspondence between separating thick subcategories and Serre subcategories. Hence, the prestable lift is unique, as it is separating by definition. 
\end{remark}

After lifting to the prestable category, the next step---as before---is to stabilize. In the situation of small categories we find ourselves in, this is not done via spectrum objects $\Sp(-)$, but instead via the Spanier--Whitehead construction. 

\begin{definition}
    \index{Spanier--Whitehead category}
    Let $\mathcal{E}$ be a pointed category with finite limits. The \emph{Spanier--Whitehead category} of $\mathcal{E}$, denoted $\SW(\mathcal{E})$, is defined to be the colimit of the diagram 
    \begin{center}
        \begin{tikzcd}
            \mathcal{E} \arrow[r, "\Sigma"] & \mathcal{E} \arrow[r, "\Sigma"] & \mathcal{E} \arrow[r, "\Sigma"] & \cdots
        \end{tikzcd}
    \end{center}
    where $\Sigma$ is the suspension functor, given by the cofiber of the map $0\rightarrow x$ for $x\in \mathcal{E}$. 
\end{definition}

The first thing we need is to compare prestable and stable compact objects. 

\begin{theorem}
    \label{ch3:add:thm:prestable-freyd-stabilizes-to-stable-Freyd}
    The inclusion $\LSyn_{E, \geq 0}^\omega \to \LSynE^\omega$ induces an equivalence $\SW(\LSyn_{E, \geq 0}^\omega)\simeq \LSynE^\omega$ of symmetric monoidal stable $\infty$-categories. 
\end{theorem}
\begin{proof}
    The category $\LSyn_{E, \geq 0}^\omega$ is a small prestable category closed under finite limits, hence it is the connected part of a $t$-structure on the Spanier--Whitehead category $\SW(\LSyn_{E, \geq 0}^\omega)$, see \cite[C.1.1, C.1.2]{lurie_SAG}. 

    By \cite[C.1.1.6]{lurie_SAG} there is a commutative diagram
    \begin{center}
        \begin{tikzcd}
            \Cat^{\mathrm{rex}}_\infty \arrow[r, "\SW(-)"] \arrow[d, "\Ind(-)"'] & \Cat^{\mathrm{rex}}_\infty \arrow[d, "\Ind(-)"] \\
            \PrL \arrow[r, "\Sp(-)"']                                                    & \PrL                                           
        \end{tikzcd}
    \end{center}
    meaning that there is an equivalence
    \[\Ind(\SW(\LSyn_{E, \geq 0}^\omega))\simeq \Sp(\Ind(\LSyn_{E, \geq 0}^\omega)).\]
    As all functors are symmetric monoidal, the equivalence is also symmetric monoidal. The category $\Ind(\LSyn_{E, \geq 0}^\omega)$ is $\LSyn_{E, \geq 0}$---as it is compactly generated---which we know stabilizes to $\LSynE$. This category we know has a collection of compact generators, $\LSynE^\omega$, which is a small stable $\infty$-category, giving an equivalence $\Ind(\LSynE^\omega)\simeq \LSynE$ by definition. As the functor $\Ind$ is an equivalence between small stable $\infty$-categories and compactly generated $\infty$-categories, we get our wanted equivalence $\SW(\LSyn_{E, \geq 0}^\omega)\simeq \LSynE^\omega$. 
\end{proof}

We can now lift all the way from the abelian level to the stable level. 

\begin{definition}
    \index{Lift!Stable}
    Given a Serre subcategory $\T^\heart$, we define its \emph{stable lift} $\T$ to be the Spanier--Whitehead category of its prestable lift $\T:= \SW(\T\geqz)$. 
\end{definition}

\begin{remark}
    Intuitively one should think about this as the ``small'' version of the construction from \cref{ch3}, where one lifts an abelian localizing subcategory through the $t$-structure by first lifting to the prestable category and then stabilizing. The Spanier--Whitehead construction is the natural version of stabilization for small categories, as is made clear by the commutative diagram in the above proof. 
\end{remark}

We have now defined our lift, and it remains to prove that it has the expected properties: it should in particular be a thick subcategory---in the stable sense. We want to thank Thomas Blom for the below argument, which was much simpler than what we were able to cook up ourselves. 

\begin{lemma}
    \label{ch3:add:lm:stable-lift-is-thick}
    Let $\T^\heart\subseteq \ComodE^\omega$ be a Serre subcategory. The stable lift $\T$ is a thick subcategory of $\LSynE^\omega$. 
\end{lemma}
\begin{proof}
    We have a fully faithful inclusion $\T\geqz\hookrightarrow\LSyn_{E, \geq 0}^\omega$, which gives a fully faithful inclusion 
    \[\T = \SW(\T\geqz)\hookrightarrow \SW(\LSyn_{E, \geq 0}^\omega)\simeq \LSynE^\omega\]
    by \cref{ch3:add:thm:prestable-freyd-stabilizes-to-stable-Freyd}. As $\T$ is a stable $\infty$-category by definition, we need only to check that it is closed under finite colimits in $\LSynE^\omega$. 
    
    Given a finite colimit in $\T = \SW(\T\geqz)$, it has to factor through $\T\geqz$ at some finite stage in the diagram 
    \begin{center}
        \begin{tikzcd}
            \T\geqz \arrow[r, "\Sigma"] & \T\geqz \arrow[r, "\Sigma"] & \T\geqz \arrow[r, "\Sigma"] & \cdots
        \end{tikzcd}
    \end{center}
    As $\T\geqz$ is closed under finite colimits in $\LSyn_{E, \geq 0}^\omega$, together with the fact that
    \begin{center}
        \begin{tikzcd}
            \T\geqz \arrow[r, "\Sigma"] \arrow[d] & \T\geqz \arrow[d] \\
            \LSyn_{E, \geq 0}^\omega \arrow[r, "\Sigma"']                  & \LSyn_{E, \geq 0}^\omega                  
        \end{tikzcd}
    \end{center}
    commutes, and lastly that all of the maps $\T\geqz\to \SW(\T\geqz)\simeq \T$ and 
    \[\LSyn_{E, \geq 0}^\omega \to \SW(\LSyn_{E, \geq 0}^\omega)\simeq \LSynE^\omega\] 
    preserve finite colimits---see \cite[C.1.1.5]{lurie_SAG}---this implies that also the fully faithful inclusion $\T \subseteq \LSynE^\omega$ preserves finite colimits, finishing the proof. 
\end{proof}

We now claim that the unique lift $\MSynE$ of the compactly generated localizing subcategory $\Comodt$ is a compactly generate localizing subcategory. In the following result we let the stable lift of $\Thick(\TorsE)$ be denoted by $\T$. 

\begin{theorem}
    There is an equivalence of localizing subcategories $\MSynE \simeq \Loc_{\LSynE}(\T)$. In other words, $\MSynE$ is compactly generated. 
\end{theorem}
\begin{proof}
    As $\T$ is a thick stable subcategory by \cref{ch3:add:lm:stable-lift-is-thick}, we have equivalences 
    \[\Loc_{\LSynE}(\T)\simeq \Ind(\T)\simeq \Ind(\SW(\T\geqz)),\]
    where the first is by \cite[2.15]{barthel-heard-valenzuela_2018}, and the second is by the definition of the stable lift. By \cite[C.1.1.6]{lurie_SAG} we again have an equivalence
    \[\Ind(\SW(\T\geqz)) \simeq \Sp(\Ind(\T\geqz)),\]
    hence it remains only to show that $\Ind(\T\geqz)$ is the unique separating localizing subcategory of $\LSyn_{E, \geq 0}$ with heart $\Comodt$, which by definition is the category $\MSyn_{E, \geq 0}$. As $\T\geqz$ is a separating thick prestable subcategory, $\T^\heart$ generates $\Comodt$ under filtered colimits and $\pi_n$ commutes with filtered colimits, the category $\Ind(\T\geqz)$ is a separating localizing subcategory with heart $\Comodt$. Hence, it has to be equivalent to $\MSyn_{E, \geq 0}$ by \cite[C.5.2.5, C.5.2.6]{lurie_SAG}, finishing the proof. 
\end{proof}

% By \cref{ch3:thm:main} we know that there is a unique $\pi$-exact lift of any localizing subcategory of $\ComodE$. 

% \begin{lemma}
%     \label{ch3:add:lm:lift-of-B-works-with-localizing-lift}
%     For any Serre subcategory $\T^\heart\subseteq \ComodE^\omega$, we have $\L\cap \LSynE^\omega = \T$. 
% \end{lemma}
% \begin{proof}
%     As $\T^\heart \subseteq \Loc(\T^\heart)$ we also have $\T \subseteq \L$ by \cref{ch3:lm:pi-stable-are-the-biggest}, as the latter is $\pi$-stable. This gives the first of the inclusions:
%     \[\T = \T \cap \LSynE^\omega\subseteq \L \cap \LSynE^\omega.\]
    
%     Let $l$ be an object in $(\L\cap \LSynE^\omega)\geqz$. This means that $l\in \L\geqz$ and $l \in \LSyn_{E, \geq 0}^\omega$. Hence, $\pi_k l \in \Loc(\T^\heart)\cap \ComodE^\omega \simeq \T^\heart$ for all $k\geq 0$, which by definition implies $l \in \LSyn_{E, \geq 0}^\omega$, giving 
%     \[(\L\cap \LSynE^\omega)\geqz \subseteq \T\geqz.\]
%     This gives the other inclusion upon taking Spanier--Whitehead categories. 
% \end{proof}


% For the uniqueness of $\T$ we will need the following lemma, stating that the lift of a compactly generated abelian localizing subcategory is a compactly generated stable localizing subcategory. 

% \begin{lemma}
%     \label{ch3:add:lm:loc-of-lift-is-localizing-lift}
%     There is an equivalence of localizing subcategories $\Loc(\T) = \L$. 
% \end{lemma}
% \begin{proof}
%     By \cref{ch3:add:lm:lift-of-B-works-with-localizing-lift} these have the same compact objects, hence $\Loc(\T)\subseteq \L$. If we can prove that $\Loc(\T)$ is a $\pi$-stable localizing subcategory with heart $\Loc(\T^\heart)$, then we are done by the uniqueness of the lift $\L$. 

%     Now, $\T$ is $\pi$-stable, hence we have $\pi_k t \in \Loc(\T^\heart)$ if and only if $t\in \Loc(\T)$ for all compact $t$. We know that $\Loc(\T)$ is generated by $\T$ under filtered colimits, which means, as $\pi_0$ preserves filtered colimits and $\Loc(\T^\heart)$ is closed under these, that also $\pi_k t\in \Loc(\T)$ if and only if $t\in \Loc(\T)$ for all (not necessarily compact) objects $t$. It also follows from this that 
%     \[\Loc(\T)^\heart = \Loc(\T^\heart),\]
%     hence we get $\Loc(\T) = \L$ by uniqueness of the lift. 
% \end{proof}

% We can now finally prove that the lift $\T$ is unique. 

% \begin{theorem}
%     \label{ch3:add:thm:uniqueness-of-lift}
%     Given a Serre subcategory $\T\subseteq \ComodE^\omega$, the lift $\T$ is unique. 
% \end{theorem}
% \begin{proof}
%     Let $\T'$ be another stable lift of $\T^\heart$, in other words it is a $\pi$-stable thick subcategory with heart $\T^\heart$. By the same arguments as in \cref{ch3:add:lm:loc-of-lift-is-localizing-lift}, we get two $\pi$-stable localizing subcategories $\Loc(\T)$ and $\Loc(\T')$, which necessarily must have the same heart $\Loc(\T^\heart)$. By uniqueness of the lift $\L$ we must then have $\Loc(\T) = \L = \Loc(\T')$. By \cref{ch3:add:lm:lift-of-B-works-with-localizing-lift} we conclude that 
%     \[\T = \Loc(\T)\cap \LSynE^\omega = \Loc(\T')\cap \LSynE^\omega = \T',\]
%     finishing the proof. 
% \end{proof}

% As a consequence we get that the category of monochromatic synthetic spectra $\MSynE$ is compactly generated, as it is equivalent to the category $\Loc(\T)$ associated to the stable lift $\T$ of the Serre subcategory of compact objects in $\Comodt$. 

% \begin{corollary}
%     \index{Compactly generated}
%     The category of monochromatic synthetic spectra $\MSynE$ is compactly generated. 
% \end{corollary}

% We can also show that $\Mn\LSynE$ is a $\otimes$-ideal, which will mean that we can apply the local duality results from \cref{ch1:app:barr-beck} to compute the deformation properties of $\Mn\LSynE$. To spoil the result, it will be precicely what it should. 

% First note that the maximal localizing subcategory $\Comodt$ is in fact a $\otimes$-ideal. Furthermore, the full subcategory of compact objects $\ComodE^{\omega, I_n-tors}$ is a maximal Serre $\otimes$-ideal of $\ComodE^\omega$. We will use this to our advantage. 

% \begin{lemma}
%     The unique separating prestable thick subcategory $\T\geqz$, lifting $\ComodE^{\omega, I_n-tors}$, is a prestable maximal thick $\otimes$-ideal. 
% \end{lemma}
% \begin{proof}
%     The maximality of $\T\geqz$ is immediate from the fact that it is separating, and that $\T^\heart \simeq \ComodE^{\omega, I_n-tors}$ is maximal. 

%     Assume there are objects $t\in \T\geqz$ and $x \in \LSyn_{E, \geq 0}^\omega$ such that $t \otimes x \not\in \T\geqz$. As $\T\geqz$ is maximal this means that the subset $\T\geqz \cup \{t \otimes x\}$ generates $\LSyn_{E, \geq 0}^\omega$ under finite colimits, i.e., 
%     \[M:= \mathrm{Thick}(\T\geqz \cup \{t \otimes x\}) \simeq \LSyn_{E, \geq 0}^\omega.\]
%     But, the discrete objects in $M$ can only generate $\ComodE^{\omega, I_n-tors}$, as it is a $\otimes$-ideal, and $\pi_0$ is symmetric monoidal by \cite[A.12]{antieau_nikolaus_2020}, leading to a contradiction. Hence, no such two objects can exist, finishing the proof. 
% \end{proof}

% We can also quite easily see that this implies that the Spanier--Whitehead category $\T := \SW(\T\geqz)$ is a thick $\otimes$-ideal of $\LSynE^\omega$. We can do this by categorifying the property of being a $\otimes$-ideal. 

% \begin{lemma}
%     \label{ch3:add:lm:categorical-ideal-property}
%     Let $\E$ be a small symmetric monoidal $\infty$-category, and $\L\subseteq \E$ a presentable full subcategory. Precomposing with the inclusion we get a functor 
%     \[\mu\: \L\times \E \to \E\times\E \to \E.\]
%     The subcategory $\L$ is an $\otimes$-ideal if and only if $\mu$ factors through the inclusion $\L\hookrightarrow \E$. 
% \end{lemma}
% \begin{proof}
%     This it literally just a reformulation of the definition of an ideal, as the functor $\mu\: \L\times \E \to \E$ sends $(L,E) \mapsto L\otimes E$. 
% \end{proof}

% This will imply the ideal property for the Spanier--Whitehead category. 

% \begin{lemma}
%     \label{ch3:add:lm:stabilizing-ideal-gives-ideal}
%     If $\T\geqz$ is a thick $\otimes$-ideal in $\LSyn_{E, \geq 0}^\omega$, then its Spanier--Whitehead stabilization $\T$ is a thick $\otimes$-ideal of $\LSynE^\omega$. 
% \end{lemma}
% \begin{proof}
%     We have already proven the thick subcategory property, hence it remains only to show that it is an ideal. The Spanier--Whitehead functor $\SW(-)\: \Cat_\infty^{\mathrm{rex}}\to \Cat_\infty^{\mathrm{rex}}$ is symmetric monoidal with respect to the cartesian product, hence it sends the composite 
%     \[\T\geqz\times \LSyn_{E, \geq 0}^\omega \to \LSyn_{E, \geq 0}^\omega\times \LSyn_{E, \geq 0}^\omega \to \LSyn_{E, \geq 0}^\omega\]
%     to 
%     \[\T\times \LSynE^\omega \to \LSynE^\omega\times \LSynE^\omega\to \LSynE^\omega.\]
%     As the symmetric monoidal structures on $\LSyn_{E, \geq 0}^\omega$ and $\T\geqz$ was induced from $\LSynE^\omega$, we do recover the original symmetric monoidal structure in this process. The former functor factors through $\T\geqz$ by \cref{ch3:add:lm:categorical-ideal-property}, hence the latter factors through $\T$. This means that it is an ideal, again by \cref{ch3:add:lm:categorical-ideal-property}. 
% \end{proof}

% \begin{remark}
%     This means that there is a collection of compact objects $\K \subseteq \Mn\LSynE$ such that $\Mn\LSynE \simeq \Loc^\otimes (\K)$. By \cref{ch0:thm:local-duality} this means that the associated colocalization functor
%     \[\Gamma_n\: \LSynE \to \Mn\LSynE\]
%     is a smashing colocalization. 
% \end{remark}

It is certainly nice to know that $\MSynE$ is compactly generated, but for computing its deformation properties---like in \cref{ch3:add:thm:deformation-properties-of-LSynE}---we need to know more explicitly what the compact generators are. The category of monochromatic spectra $\M\np$ is compactly generated by the $E$-localization of any finite type $n$ spectrum $F(n)$, see \cref{ch0:def:type-n-spectrum} and \cref{ch0:prop:torsion-is-monochromatic}\index{Type $n$ spectrum}. Hence, a natural guess for the compact generators of monochromatic synthetic spectra would be to lift these to the synthetic setting. 

\begin{construction}
    By \cite[4.23]{pstragowski_2022} we can lift the fiber sequence $L\np\S\overset{p}\to L\np\S\to L\np\S/p$ to a fiber sequence 
    \[\nu L\np\S \overset{\widetilde{p}}\to \nu L\np\S\to \nu (L\np\S/p)\] 
    in synthetic spectra $\LSynE$, as it induces a short exact sequence 
    \[0\to E_{*} \overset{\cdot p}\to E_{*}\to E_{*}/p\to 0\]
    on $E_{*}$-homology. In particular, $\nu (L\np\S/p)\simeq (\nu L\np\S)/\widetilde{p}$. Similar ideas were used by Burklund to prove the existence of $\E_1$ structures on Moore spectra in \cite{burklund_2022}. 
    
    Now, as $\nu (L\np\S/p)$ is a finite number of cones away from the synthetic sphere, it is a compact object in $\LSynE$. We can iterate this construction to lift generalized Moore spectra into the synthetic setting. These are then compact synthetic objects that behave similarily to finite type $n$ spectra. 
\end{construction}

\begin{lemma}
    There is a finite type $n$ spectrum $F(n)$, whose $E$-local synthetic analog $\nu L\np F(n)$ is compact. 
\end{lemma}
\begin{proof}
    Let $I_n$ be the Landweber ideal $(p,v_1, v_2, \ldots, v_{n-1})$. By \cite[4.14]{hovey-strickland_99} there is a finite type $n$ generalized Moore spectrum $L\np\S/J$ for $J=(p^{i_0}, v_1^{i_1}, \ldots, v_{n-1}^{i_{n-1}})$ constructed by iterated fiber sequences. These fiber sequences all induce short exact sequences on $E_{*}$-homology, hence we can lift them to fiber sequences in synthetic $\LSynE$ by \cite[4.23]{pstragowski_2022}. In particular we have $\nu (L\np\S/J) \simeq (\nu L\np\S)/\widetilde{J}$ for $\widetilde{J} = (\widetilde{p}^{i_0}, \widetilde{v}_1^{i_1}, \ldots, \widetilde{v}_{n-1}^{i_{n-1}})$. Since we used a finite number of shifts and fiber sequences, $\nu (L\np\S/J)$ is a compact object in $\LSynE$. 
\end{proof}

\begin{definition}
    \index{Type $n$ spectrum!Synthetic}
    An $E$-local synthetic spectrum $X$ is said to be of \emph{synthetic type} $n$, if it is compact, and $X\simeq \nu L\np F(n)$ for a finite type $n$ spectrum $F(n)$. 
\end{definition}

Our goal is to show that such a spectrum $\nu L\np F(n)$ indeed generates $\MSynE$ as a localizing subcategory. 

\begin{lemma}
    \label{ch3:add:monochromatic-synthetic-is-gen-by-type-n}
    There localizing subcategory $\MSynE$ is compactly generated by a synthetic type $n$ spectrym. In other words, there is an equivalence 
    \[\Loc_{\LSynE}(\nu L\np F(n))\simeq \MSynE\]
    of stable $\infty$-categories. 
\end{lemma}
\begin{proof}
    As the spectrum $L\np F(n)$ is monochromatic, we have by \cref{ch3:add:lm:mono-iff-syn-mono} that $\nu L\np F(n)$ lies in $\MSynE$. In particular we have 
    \[\Loc_{\LSynE}(\nu L\np F(n))\subseteq \MSynE.\]
    By \cite[2.2]{neeman_1992} the compact objects in $\MSynE$ are precisely those compact objects in $\LSynE$ that lie in $\MSynE$. As we have shown that $\MSynE$ is compactly generated, we must then have 
    \[\MSynE \simeq \Loc_{\LSynE}(\{\nu M\np P\}),\]
    for $P \in \Sp\fp$ all the $E$-finite projective spectra. This is because $\nu L\np P$ compactly generate $\LSynE$, and these lie in $\MSynE$ precisely when $L\np P\in \M\np$, again by \cref{ch3:add:lm:mono-iff-syn-mono}. 

    Now, as $\nu$ preserves filtered colimits we have 
    \[\nu \left(\Loc_{\Sp\np}(L\np F(n))\right)\subseteq \Loc_{\LSynE}(\nu L\np F(n)).\]
    As $L\np F(n)$ generates $\M\np$ under filtered colimits, this implies that 
    \[\Loc_{\LSynE}(\{\nu M\np P\})\subseteq \Loc_{\LSynE}\left(\{\nu M\}\right) \subseteq \Loc_{\LSynE}(\nu L\np F(n)),\]
    where $\Loc_{\LSynE}\left(\{\nu M\}\right)$ denotes the localizing subcategory generated by the synthetic analogs of all monochromatic spectra $M\in \M\np$. As we have shown that $\Loc_{\LSynE}(\{\nu M\np P\})\simeq \MSynE$, this finishes the proof. 
\end{proof}



\subsection{Deformation properties}

We now show that $\MSynE$ has the desired deformation properties. First we need to show that it is in fact a localizing $\otimes$-ideal, and not just a localizing subcategory. 

\begin{lemma}
    The synthetic type $n$ spectrum $\nu L\np F(n)$ generates the category $\MSynE$ as a localizing $\otimes$-ideal. In other words, there is an equivalence $\MSynE \simeq \Loc_{\LSynE}^\otimes(\nu L\np F(n))$ of symmetric monoidal stable $\infty$-categories. 
\end{lemma}
\begin{proof}
    By \cref{ch3:add:monochromatic-synthetic-is-gen-by-type-n} it is enough to show that $\MSynE$ is closed under tensoring with objects of $\LSynE$. As the tensor product commutes with colimits separately in each variable, and the category $\MSynE$ is closed under colimits, it is enough to check this on generators of $\LSynE$ and $\MSynE$, namely $\nu L\np P$ and $\nu L\np F(n)$ respectively. By \cite[4.24]{pstragowski_2022} we have an equivalence 
    \[\nu L\np P \otimes \nu L\np F(n) \simeq \nu(L\np P \otimes L\np F(n)).\]
    As $\M\np$ is a localizing ideal of $\Sp\np$, the spectrum $L\np P \otimes L\np F(n)$ is monochromatic. This means that its synthetic analog is a synthetic monochromatic spectrum by \cref{ch3:add:lm:mono-iff-syn-mono}. 
\end{proof}

The computations for the generic and special fibers of the deformation parameter $\tau$ now follow quite easily from the results in \cref{ch1:app:barr-beck}. Let us start with the special fibre. 

Recall the definition of the derived $I_n$-torsion stable category of comodules in \cite[2.4]{barthel-heard-valenzuela_2020} as 
\[\StableE\Int := \Loc_{\StableE}^\otimes(E_{*}/I_n)\] 
as a localizing subcategory of $\Stable\EE$. We furthermore recall the definition of the stable category of $I_n$-power torsion comodules, see \cite[3.5]{barthel-heard-valenzuela_2020}, as 
\[\Stable(\Comodt) := \Ind(\Thick^\otimes(\TorsE),\]
where $\TorsE$ denotes the collection of compact generators for $\Comodt$, see \cref{ch1:lm:torsion-comodules-generated-by-compacts}. 

\begin{theorem}
    \label{ch3:add:thm:monochromatic-synthetic-special-fiber}
    There is an equivalence
    \[\Mod_{C\tau}(\MSynE) \simeq \Stable(\Comodt)\]
    of symmetric monoidal stable $\infty$-categories. 
\end{theorem}
\begin{proof}
    By \cref{ch3:add:thm:deformation-properties-of-LSynE} there is a monoidal Barr--Beck adjunction 
    \[\LSynE \rightleftarrows \Stable\EE.\]
    As $\nu X \otimes C\tau \simeq E_{*} X$, we get a local duality adjunction 
    \[(\LSynE, \nu L\np F(n))\rightleftarrows (\Stable\EE, E_{*}F(n)).\]
    By \cref{ch1:thm:modular-bb-torsion} there is an induced monoidal Barr--Beck adjunction 
    \[\Loc_{\LSynE}^\otimes(\nu L\np F(n))\rightleftarrows \Loc_{\StableE}^\otimes(E_{*}F(n)).\]
    The left hand side is equivalent to $\MSynE$ by \cref{ch3:add:monochromatic-synthetic-is-gen-by-type-n}, so we need only to identify the right. The localizing ideal is only dependent on the radical of the ideal $J$ used to construct the type $n$ synthetic spectrum $F(n)$, and the radical of $J$ is equivalent to the radical of $I_n$, meaning that the right hand side can be identified with the localizing ideal generated by $E_{*}/I_n$, which gives 
    \[\Loc_{\StableE}^\otimes(E_{*}/I_n) \simeq \Stable\EE\Int\simeq \Stable(\Comodt),\]
    where the last equivalence is due to \cite[3.17]{barthel-heard-valenzuela_2020}. Hence, as the above adjunction is Barr--Beck, we get 
    \[\Mod_{C\tau}(\MSynE) \simeq \Stable(\Comodt),\]
    finishing the proof. 
\end{proof}

The computation of the generic fibre is very similar to \cref{ch3:add:thm:monochromatic-synthetic-special-fiber}, but we include it for completion. 

\begin{theorem}
    Inverting the deformation parameter $\tau$ gives an equivalence $\MSynE[\tau^{-1}] \simeq \M\np$ of symmetric monoidal stable $\infty$-categories. 
\end{theorem}
\begin{proof}
    It follows from \cref{ch3:add:thm:deformation-properties-of-LSynE} that there is a local duality adjunction 
    \[(\LSynE, \nu L\np F(n)) \rightleftarrows (\Sp\np, \tau^{-1}\nu L\np F(n))\]
    which by \cref{ch1:thm:modular-bb-torsion} induces a Barr--Beck adjunction on the respective localizing $\otimes$-ideals. By \cite[4.40]{pstragowski_2022} there is an equivalence $\tau^{-1}\nu X \simeq X$, hence we get a monoidal Barr--Beck adjunction 
    \[\Loc_{\LSynE}^\otimes(\nu L\np F(n))\rightleftarrows \Loc_{\Sp\np}^\otimes(L\np F(n)).\]
    The former is again $\MSynE$, and the latter is $\M\np$. As the adjunction is Barr--Beck we get 
    \[\Mod_{\tau^{-1}\nu \M\np\S}(\MSynE)\simeq \MSynE[\tau^{-1}]\simeq \M\np,\]
    just as wanted. 
\end{proof}

This proves in essence that $\MSynE$ is the correct deformation underlying the adapted homology theory 
\[E_*\: \M\np \to \Comodt,\] 
showing that the deformation theory plays well with the classification result for $\pi$-exact localizing subcategories in \cref{ch3:thm:main}. 

\subsection{Monoidal algebraicity}

It is now natural to wonder wether this solves \cref{ch1:conj:monochromatic-monoidally-algebraic}---which states that at large enough primes the equivalence of $k$-categories $h_k\M\np \simeq h_k\Fr\np\Int$ from \cref{ch1:thm:main-spectra} is symmetric monoidal---as the category $\MSynE$ seems to satisfy all the needed criteria to apply Barkan's monoidal algebraicity results of \cite{barkan_2023}. However, the category $\Comodt$ does not have enough flat objects, a requirement needed to use Barkan's result directly. Luckilly, this be avoided in this specific case. 

To the best of our knowledge the only part of \cite[Theorem H]{barkan_2023} that requires the existence of flats, is to fix \cite[Warning 3..30]{barkan_2023}, which states that the categories $\Mod_{C\tau}(\C\geqz)$ and $\Der(\C^\heart)\geqz$---where $\C$ is some ``deformation''---might be equivalent, but not monoidally so. In the case $\C= \MSynE$ we can directly prove that these categories are equivalent as symmetric monoidal stable $\infty$-categories, hence avoiding the above issue. 

First, we need to pass to large enough primes. By \cite[4.11]{barthel-heard_2018} there is for large enough primes a symmetric monoidal equivalence
\[\StableE \simeq \Der(E_*E),\]
which with more modern techniques has a very simple proof. 

\begin{lemma}
    \label{ch3:add:cor:hypercomplete-at-large-primes}
    If $p-1>n$, then the $t$-exact localization functor
    \[\StableE \to \Der(E_*E),\]
    is a symmetric monoidal equivalence of stable $\infty$-categories.  
\end{lemma}
\begin{proof}
    The former category is given by spherical sheaves on the site $\ComodE\fp$, see \cite[3.7]{pstragowski_2022}, and by the proof of \cite[4.54]{pstragowski_2022} the localization above is the hypercompletion functor. By \cite[2.5]{pstragowski_2021} the category $\ComodE$ has finite cohomological dimension $n^2+n$ whenever $p-1>n$, which by \cite[2.10]{clausen-mathew_2021} implies that all sheaves of spectra on $\ComodE\fp$ are already hypercomplete, finishing the claim. 
\end{proof}

This now induces a symmetric monoidal equivalence on the respective localizing ideals. 

\begin{corollary}
    \label{ch3:add:cor:torsion-hypercomplete-at-large-primes}
    If $p-1> n$, then the $t$-exact hypercompletion functor
    \[\Stable(\Comodt) \to \Der(E_*E\Int),\]
    is a symmetric monoidal equivalence of stable $\infty$-categories.  
\end{corollary}
\begin{proof}
    The symmetric monoidal equivalence is immediate from \cref{ch3:add:cor:hypercomplete-at-large-primes}. The $t$-exactness also follows from this equivalence, as $\Der(E_*E\Int)$ is a $\pi$-stable localizing subcategory by \cite[3.7(2)]{barthel-heard-valenzuela_2020}, hence also compatible with the $t$-structure. 
\end{proof}

\begin{remark}
    It is possible to interpret $\Stable(\Comodt)$ as a category of sheaves itself, and run the same argument as for $\StableE$. This gives an even stronger claim, as $\Comodt$ has finite cohomological dimension $n^2+2$ whenever $p-1\nmid n$---see \cref{ch1:lm:cohomological-dimension-torsion-comodules}. This makes the situation very reminiscent of \cite[4.7]{barthel-heard_2018}, where Barthel--Heard prove an analogous statement for the Hopf algebroid $(E_*, K_*E)$. 
\end{remark}

\begin{theorem}
    \label{ch3:add:thm:monoidal-equivalence-of-ctau-modules}
    If $p-1>n$, then there is an equivalence 
    \[\Mod_{C\tau}(\MSyn_{E, \geq 0}) \simeq \Der(\Comodt)\geqz\]
    of symmetric monoidal stable $\infty$-categories. 
\end{theorem}
\begin{proof}
    This follows from \cref{ch3:add:thm:monochromatic-synthetic-special-fiber} and \cref{ch3:add:cor:torsion-hypercomplete-at-large-primes}. 
\end{proof}

We can then conclude with a symmetric monoidal algebraicity result for $\M\np$. 

\begin{theorem}
    \label{ch3:add:thm:monochromatic-monoidally-algebraic}
    If $p$ is a prime, and $n, k$ positive integers such that $2p-2>n^2+(k+3)n+k-1$, then the equivalence of $k$-categories
    \[h_k \M\np\simeq h_k \Fr\np\Int\]
    is symmetric monoidal. 
\end{theorem}
\begin{proof}
    This now follows from \cite[Theorem H]{barkan_2023} by incorporating \cref{ch3:add:thm:monoidal-equivalence-of-ctau-modules} into its proof. 
\end{proof}

% \begin{remark}
%     As we mentioned in \cref{ch1:add:shaul-comment-monoidal}, we already know this to be true by a localizing ideal argument, as the symmetric monoidality of the equivalence $h_k \Sp\np\simeq h_k \Fr\np$ allows us to identify the image of the unit. 
% \end{remark}

As the equivalences $\M\np \simeq \Sp_\Kpn$ and $\Fr\np\Int \simeq \Fr\np\Inc$ coming from local duality are symmetric monoidal, this also gives us a symmetric monoidal algebraicity result for $\Kpn$-local spectra. 

\begin{corollary}
    If $p$ is a prime, and $n, k$ positive integers such that $2p-2>n^2+(k+3)n+k-1$, then the equivalence of $k$-categories
    \[h_k \Sp_\Kpn\simeq h_k \Fr\np\Inc\]
    is symmetric monoidal. 
\end{corollary}
