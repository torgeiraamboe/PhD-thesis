

\section{The correspondences}

The goal of this section is to prove our two main results. We start with the classification of weak localizing subcategories, before proving the non-weak case. The former does not need any of the connections to prestable categories, hence can also be viewed as a self contained argument. The latter, however, relies on Lurie's correspondence between certain prestable localizing subcategories of $\C\geqz$ and localizing subcategories of $\C^\heart$. 


\subsection{Classification of weak localizing subcategories}
\label{ch3:ssec:classificartion-weak-localizing}

The goal of this section is to prove \cref{ch3:thm:A}, and the following lemma is the first step for obtaining the wanted correspondence. 

\begin{lemma}
    \label{ch3:lm:t-stable-then-weak-localizing-heart}
    \index{Localizing subcategory!t-stable}
    \index{Localizing subcategory!Abelian weak}
    Let $\C$ be a $t$-stable category. If $\L$ is a $t$-stable localizing subcategory, then $\L^\heart$ is a weak localizing subcategory of $\C^\heart$. 
\end{lemma}
\begin{proof}
    As $\L$ is $t$-stable we know that the fully faithful inclusion $\L\to \C$ is $t$-exact. By \cite[2.19]{antieau-gepner-heller_2019} the induced functor $\L^\heart\to\C^\heart$ is exact and fully faithful, and $\L^\heart$ is closed under extensions. In particular, $\L^\heart$ is an abelian subcategory closed under extensions, so it remains only to show that $\L^\heart$ is closed under coproducts.

    As $\L^\heart \subseteq \L$ we can include a coproduct of objects in $\L^\heart$ into $\L$. The inclusion and $\pi_n^\heart$ preserves coproducts for all $n$. Hence, as $\L$ is localizing it is closed under coproducts, implying that also $\L^\heart$ is. 
\end{proof}

This means that the heart construction $\C \longmapsto \C^\heart$ determines a map
\[\tcompatible\overset{(-)^\heart}\to \weaklocalizing\] 
for any $t$-stable category $\C$. 

This map is in general not injective, meaning we have to restrict our domain. As described in the introduction, we will use the localizing subcategories where objects can be identified by their heart-valued homotopy groups. The precise definition is as follows. 

\begin{definition}
    \label{ch3:def:pi-stable-localizing-subcategory}
    \index{Localizing subcategory!$\pi$-stable}
    Let $\C$ be a stable category with a $t$-structure. A stable localizing subcategory $\L$ is said to be \emph{$\pi$-stable} if $X\in \L$ if and only if $\pi_n^\heart X\in \L^\heart$ for all $n$. 
\end{definition}

\begin{remark}
    The terminology is motivated by, and generalizes, Takahashi's definition of $H$-stable subcategories of the unbounded derived category of a commutative noetherian ring, see \cite[2.11]{takahashi_2009}. These are subcategories of derived categories where one can detect containment by checking on homology. Letting $\C=\Der(R)$ for a Noetherian commutative ring $R$ considered with the natural $t$-structure, then we have $\pi_n^\heart = H_n$, meaning that being $\pi$-stable is equivalent to being $H$-stable. Note, however, that the homological algebra literature often uses cohomological indexing, while we follow Lurie's convention of using the homological one. 
\end{remark}

\begin{remark}
    The above definition is also equivalent to Zhang--Cai's generalization of Takahashi's $H$-stable subcategories, see \cite{zhang-cai_2017}. Note that the authors of loc. cit. do not consider the subcategories themselves to have $t$-structures, but rather just includes the image of $\pi_k^\heart$ back into the stable category. 
\end{remark}

\begin{example}
    \label{ch3:ex:abelian-torsion}
    \index{I-power torsion!Module}
    Let $R$ be a commutative noetherian ring and $I\subseteq R$ a finitely generated regular ideal. Then the full subcategory of $I$-power torsion modules, $\Mod_R^{I-tors}\subseteq \Mod_R$ is an abelian weak localizing subcategory. It is in particular a Grothendieck abelian category, hence has a derived category $\Der(\Mod_R^{I-tors})$. We can also form the derived $I$-torsion category $\Der(R)^{I-tors}$, which is the localizing subcategory generated by $A/I$. The categories $\Der(R)^{I-tors}$ and $\Der(\Mod_R^{I-tors})$ are both $\pi$-stable localizing subcategories of $\Der(R)$ with heart $\Mod_R^{I-tors}$---see \cite{greenlees-may_92} or \cite{barthel-heard-valenzuela_2018} for more details. These categories are equivalent, seemingly implying that having the same heart is enough for the stable categories to be equivalent as well. This also generalizes to other similar situations, see for example \cite[3.15, 3.17]{barthel-heard-valenzuela_2020} or \cref{ch1:thm:pulling-out-torsion}. Such equivalences were one of the main inspirations for this paper, where the author wanted an easier way of checking similar statements, which led to the main result \cref{ch3:thm:A}. 
\end{example}

\begin{proposition}
    \label{ch3:prop:pi-stable-then-t-stable}
    Let $\L$ be a localizing subcategory of $\C$. If $\L$ is $\pi$-stable, then $\L$ is $t$-stable. 
\end{proposition}
\begin{proof}
    Let $X\in \L$. We need to show that $\tau\geqz X\in \L$ and $\tau\leqz X\in \L$. The proofs are similar, hence we only cover the former. 
    
    We have $\pi_n^\heart \tau\geqz X \simeq \pi_n^\heart X$ for all $n\geq 0$ and $\pi_n^\heart \tau\geqz X\simeq 0$ for all $n<0$. This means that $\pi_n^\heart \tau\geqz X \in \L^\heart$ for all $n$, which implies $\tau\geqz X \in \L$ by the assumption that $\L$ was $\pi$-stable. 
\end{proof}

\begin{remark}
    \label{ch3:rm:recurring-2}
    In light of \cref{ch3:prop:pi-stable-then-t-stable} we can continue our recurring remark (see \cref{ch3:rm:recurring-1}) about the dependencies of the different definitions. We now have implications
    \begin{center}
        \begin{tikzcd}
            & \pi\text{-stable} \arrow[d, Rightarrow] \\
            t\text{-exact} \arrow[r, Rightarrow] & t\text{-stable}                    
        \end{tikzcd}
    \end{center}
    for any localizing subcategory $\L$ of a $t$-stable category $\C$. 
\end{remark}

\begin{remark}
    \label{ch3:rm:pi-stable-implies-t-stable}
    If $\L$ is a $\pi$-stable localizing subcategory then \cref{ch3:lm:localizing-inherits-completeness-and-colimits} implies that $\L$ is itself a $t$-stable category. This is rather convenient, as it allows us to treat nested pairs of $\pi$-stable localizing subcategories $\L_2 \subseteq \L_1 \subseteq \C$ either as both being subcategories of $\C$, or as $\L_2$ being a $\pi$-stable localizing subcategory of $\L_1$. 
\end{remark}

% Now, we want to relate stable localizing subcategories and prestable localizing subcategories to certain subcategories in the Grothendieck abelian heart. Hence we need to know what the corresponding notion of ``abelian localizing subcategory'' is. In fact, we need two such notions, because we will have two slightly different classification results for stable and prestable localizing subcategories. 

% \begin{definition}
%     \label{ch3:def:weak-serre-subcategory}
%     Let $\A$ be a Grothendieck abelian category. A full subcategory $\T$ is said to be a \emph{weak Serre subcategory} if it satisfies the following criteria:
%     \begin{enumerate}
%         \item $\T$ is closed under coproducts, and
%         \item for any short exact sequence $0\rightarrow A\rightarrow B\rightarrow C\rightarrow 0$ in $\A$, if two of the objects $A, B, C$ are in $\T$, then also the last one is. 
%     \end{enumerate}
% \end{definition}

% \begin{remark}
%     A weak Serre subcategory $\T$ is an abelian subcategory closed under extensions and coproducts. In particular, the inclusion functor $\T\to \A$ is an exact fully faithful functor of abelian categories. 
% \end{remark}

% \begin{remark}
%     Weak Serre subcategories are usually not required to contain arbitrary coproducts, but as we will require this property for all of our subcategories, we lump it into the definition. The second item $(2)$ in the definition is usually stated as follows: For any exact sequence
%     \[A_1\rightarrow A_2\rightarrow A_3 \rightarrow A_4 \rightarrow A_5\]
%     in $\A$, if $A_1, A_2, A_4, A_5 \in \T$ then also $A_3$ is in $\T$. By splitting this exact sequence into short exact sequences, one can indeed see that this is equivalent to condition $(2)$ in \cref{ch3:def:weak-serre-subcategory}. 
% \end{remark}

% The following lemma is the first step for obtaining the correspondence in \cref{ch3:thm:A}. 

% \begin{lemma}
%     \label{ch3:lm:t-stable-then-weak-localizing-heart}
%     Let $\C$ be a $t$-stable category. If $\L$ is a $\pi$-stable localizing subcategory, then $\L^\heart$ is a weak localizing subcategory of $\C^\heart$. 
% \end{lemma}
% \begin{proof}
%     As $\L$ is $\pi$-stable, it is by \cref{ch3:prop:pi-stable-then-t-stable} a $t$-stable subcategory, meaning in particular that the fully faithful inclusion $\L\to \C$ is $t$-exact. By \cite[2.19]{antieau-gepner-heller_2019} the induced functor $\L^\heart\to\C^\heart$ is exact and fully faithful, and $\L^\heart$ is closed under extensions. In particular, $\L^\heart$ is an abelian subcategory closed under extensions, so it remains only to show that $\L^\heart$ is closed under coproducts.

%     As $\L^\heart \subseteq \L$ we can include a coproduct of objects in $\L^\heart$ into $\L$. The inclusion and $\pi_n^\heart$ preserves coproducts for all $n$. Hence, as $\L$ is localizing it is closed under coproducts, implying that also $\L^\heart$ is. 
% \end{proof}

% Let $X_\alpha$ be a collection of objects in $\L$. In particular, $\pi_n^\heart X_\alpha \in \L^\heart$ for all $n$. As $\pi_n^\heart$ commutes with coproducts we have $\underset{\alpha}\coprod \pi_n^\heart X_\alpha \cong \pi_n^\heart (\underset{\alpha}\coprod X_\alpha)$. As $\L$ is localizing it is closed under coproducts. Hence $X\simeq \underset{\alpha}\coprod X_\alpha \in \L$, which implies $\pi_n^\heart X \in \L^\heart$. 

% Let $0\rightarrow A\rightarrow B\rightarrow C\rightarrow 0$ be an exact sequence in $\C^\heart$. By \cite[2.9]{antieau-gepner-heller_2019} $A\rightarrow B\rightarrow C$ is a cofiber sequence in $\C$. Assume that $A, B\in \L^\heart$. We have $A, B \in \L^\heart \subseteq \L$, and as $\L$ is closed under cofibers, also $C\in \L$. This implies that $C\in \L^\heart = \C^\heart \cap \L$. Similarly, if $B, C\in \L^\heart$, then $A\in \L$ as it is closed under fibers. Hence $A\in \C^\heart \cap \L = \L^\heart$.  


\cref{ch3:prop:pi-stable-then-t-stable} implies that the heart construction $\L\longmapsto \L^\heart$ gives a map 
\[\pistable\overset{(-)^\heart}\to \weaklocalizing\] 
as the heart of any $t$-stable localizing subcategory $\L\subseteq \C$ is an abelian weak localizing subcategory $\L^\heart \subseteq \C^\heart$ by \cref{ch3:lm:t-stable-then-weak-localizing-heart}. The claim of \cref{ch3:thm:A} is that this map is a bijection. 

It turns out that the $\pi$-stable localizing subcategories are the largest localizing subcategories with a given heart. This is the stable analog of \cite[C.5.2.5]{lurie_SAG} for prestable categories. 

\begin{lemma}
    \label{ch3:lm:pi-stable-are-the-biggest}
    Let $\C$ be a $t$-stable category. Given two $t$-stable localizing subcategories $\L_0$ and $\L_1$, where $\L_1$ is $\pi$-stable, then $\L_0 \subseteq \L_1$ if and only if $\L_0^\heart \subseteq \L_1^\heart$. 
\end{lemma}
\begin{proof}
    First, notice that as both categories are $t$-stable the truncation functors and the homotopy groups functors $\pi_k^\heart$ are the same, see \cref{ch3:rm:t-stable-truncation-homotopy-the-same-functors}. 
    
    Assume $\L_0^\heart \subseteq \L_1^\heart$ and $X\in \L_0$. Then $\pi_k^\heart X\in \L_0^\heart \subseteq \L_1^\heart$ for all $k$. This implies that $X\in \L_1$ by the assumption that it is $\pi$-stable. 

    For the converse, assume $\L_0\subseteq \L_1$. As the truncation functors are the same in $\L_0$ and $\L_1$ we have that $\L_0$ is a $t$-stable localizing subcategory of the $t$-stable category $\L_1$, see \cref{ch3:rm:pi-stable-implies-t-stable}. In particular, $\L_0^\heart = \L_1^\heart \cap \L_0$, hence we have $\L_0^\heart \subseteq \L_1^\heart$. 
\end{proof}

This immediately implies the injectivity of our proposed one-to-one correspondence. 

\begin{corollary}
    \label{ch3:cor:pi-stable-heart-injective}
    For any $t$-stable category $\C$, the map
    \[\pistable\overset{(-)^\heart}\to \weaklocalizing\] 
    is injective. 
\end{corollary}
\begin{proof}
    Let $\L_0$ and $\L_1$ be $\pi$-stable localizing subcategories such that $\L_0^\heart \simeq \L_1^\heart$ as subcategories of $\C^\heart$. In particular, they are contained in each other, hence \cref{ch3:lm:pi-stable-are-the-biggest} implies that $\L_0 \subseteq \L_1$ and $\L_1 \subseteq \L_0$ as they are both $\pi$-stable. 
\end{proof}

It remains to show that the map is also surjective. 

\begin{theorem}[{\cref{ch3:thm:A}}]
    \label{ch3:thm:premain}
    \index{Localizing subcategory!Abelian weak}
    \index{Localizing subcategory!$\pi$-stable}
    Let $\C$ be a $t$-stable category. In this situation, the map 
    \[\pistable\overset{(-)^\heart}\to \weaklocalizing\] 
    is a bijection. 
\end{theorem}
\begin{proof}
    We know by \cref{ch3:cor:pi-stable-heart-injective} that the map is injective, hence it remains to prove surjectivity. To do this we follow the proof of \cite[C.5.2.7]{lurie_SAG}, adapted to the stable setting. 
    
    Let $\A$ be a weak localizing subcategory of $\C^\heart$. Define $\L\subseteq \C$ to be the full subcategory spanned by objects $X$ such that $\pi_n^\heart X\in \A$. We prove that it is a stable localizing subcategory; it will obviously be $\pi$-stable by definition. In particular we prove that it is closed under cofiber sequences, desuspension and colimits. 

    Let $A\rightarrow B\rightarrow C$ be a cofiber sequence in $\C$. We need to show that if two of the objects $A, B, C$ lie in $\L$, then also the last one does. The long exact sequence of heart-valued homotopy groups has the form 
    \[\cdots\rightarrow \pi^{\heart}_{n}A \rightarrow \pi^{\heart}_{n}B \rightarrow \pi^{\heart}_{n}C \rightarrow \pi^{\heart}_{n+1}A\rightarrow \pi^{\heart}_{n+1}B \rightarrow \cdots \]
    Assuming that $A, B$ are in $\L$ we get by the definition of $\L$ that the four objects $\pi^{\heart}_{n}A, \pi^{\heart}_{n}B, \pi^{\heart}_{n+1}A, \pi^{\heart}_{n+1}B$ are in $\A$. Hence, as $\A$ is a weak Serre subcategory we have $\pi^\heart_n C\in \A$. This works for all $n$, hence we must have $C\in \L$ as well. The proof is identical in the case that $A, C$ or $B, C$ lie in $\L$. 
    
    The full subcategory $\L$ is also closed under desuspension, as we have an isomorphism $\pi_n^\heart (\Sigma^{-1} X)\cong \pi_{n+1}^\heart(X)$ by the long exact sequence in heart-valued homotopy groups. Hence $\L$ is a full stable subcategory of $\C$. In particular this means it is closed under finite colimits. Now, as $\pi_n^\heart$ preserves coproducts, and $\A$ is closed under coproducts, we also get that $\L$ is closed under coproducts. This implies that $\L$ is closed under colimits, which finishes the proof. 
\end{proof}

\begin{remark}
    It is somewhat unfortunate that the terminology does not align perfectly in these two situations---meaning that we had to add a prefix ``weak'' for the abelian case. As both are inspired by the existence of localization functors, they are the natural terminology in their respective settings, and we should perhaps not expect everything to always agree perfectly. In \cref{ch3:thm:B} we will use the abelian localizing subcategories, and then again be left with a choice of a different prefix for the stable version. 
\end{remark}

\cref{ch3:thm:A} recovers, and generalizes, a theorem by Takahashi for commutative noetherian rings. Note that Takahashi does not refer to the abelian subcategories as weak localizing, but as thick subcategories closed under coproducts. 

\begin{corollary}[{\cite{takahashi_2009}}]
    \label{ch3:cor:takahashi-weak-localizing}
    If $R$ is a commutative noetherian ring, then there is a bijection between the set of $H$-stable localizing subcategories of $\Der(R)$ and the set of weak localizing subcategories in $\Mod_R$. 
\end{corollary}

In light of \cref{ch3:thm:A} we can now generalize Takahashi's result to a noetherian scheme $X$. 

\begin{corollary}
    \label{ch3:cor:noetherian-scheme-weak-localizing}
    If $X$ noetherian scheme, then there is a bijection between the set of stable localizing subcategories of $\Der_{qc}(X)$ closed under homology, and the set of weak localizing subcategories in $\QCoh(X)$. 
\end{corollary}

A theorem of Krause---see \cite[3.1]{krause_2008}---shows that the weak localizing subcategories of $\Mod_R$ are bijection with certain subsets of $\Spec R$, which Krause calls the \emph{coherent subsets}. By the above corollary it is natural to conjecture that this also holds for noetherian schemes. 

\begin{conjecture}
    \label{ch3:conj:coherent-noetherian-scheme}
    If $X$ is a noetherian scheme, then there is a bijection between the collection of coherent subsets of $X$ and the collection of weak localizing subcategories of $\QCoh(X)$. 
\end{conjecture}

\begin{remark}
    A hint towards the truth of this conjecture comes from a theorem by Gabriel (\cite[VI.2.4(b)]{gabriel_1962}), where he shows that the above proposed bijection restricts to a bijection between specialization closed subsets of $X$ and localizing subcategories of $\QCoh(X)$. 
\end{remark}

Now, we want to mention that we also obtain a classification of weak Serre subcategories of $\C$. This is done by recognizing that the proofs of \cref{ch3:lm:t-stable-then-weak-localizing-heart}, \cref{ch3:cor:pi-stable-heart-injective} and \cref{ch3:thm:A} also hold without the assumption about coproducts. The proofs treats coproducts as a separate part, hence just omitting it from the proofs gives the following result. 

\begin{proposition}
    \label{ch3:prop:classification-weak-serre}
    Let $\C$ be a $t$-stable category. In this situation, the map 
    \[\pistablethick\overset{(-)^\heart}\to \weakserre\] 
    is a bijection. 
\end{proposition}

This recovers the following classification of weak Serre subcategories in the case where the $t$-structure on $\C$ is bounded, due to Zhang--Cai.

\begin{corollary}[{ \cite{zhang-cai_2017}}]
    \label{ch3:cor:classification-weak-serre-bounded}
    \index{Serre subcategory!Weak}
    Let $\C$ be a triangulated category with a bounded $t$-structure. In this situation there is a bijection between $\pi$-stable subcategories of $\C$ and weak Serre subcategories of $\C^\heart$.  
\end{corollary}

We can summarize the contents of this section with half of the diagram from the introduction. 

\begin{center}
    \begin{tikzcd}
        \tcompatible 
        \arrow[rrd, "(-)^\heart"]
        &&\\
        \pistable 
        \arrow[rr, "\simeq"] 
        \arrow[u, hook, "\subseteq", swap] 
        && 
        \weaklocalizing                          
        \end{tikzcd}
\end{center}

\subsection*{Digression on Grothendieck homology theories}

There is a slight generalization of the surjectivity result above, which we decided to include here for future reference. The generalization comes from realizing that there are other functors that have similar properties to the heart valued homotopy group functor $\pi_*^\heart \: \C\to \C^\heart$. 

Let $\C$ be a presentable stable $\infty$-category and $\A$ be a graded Grothendieck abelian category---meaning it comes equipped with an autoequivalence $[1]\: \A\to \A$, which we think of as a grading shift functor. 

\begin{definition}
    \index{Homology theory!Grothendieck}
    A functor $H\: \C\to \A$ is called a \emph{Grothendieck homology theory} if it satisfies the following properties:
    \begin{enumerate}
        \item It is additive.
        \item It sends cofiber sequences $X\rightarrow Y \rightarrow Z$ to exact sequences $HX\rightarrow HY\rightarrow HZ$.
        \item It is a graded functor, i.e. $H(\Sigma X) \cong (HX)[1]$.
        \item It preserves coproducts. 
    \end{enumerate}
\end{definition}

\begin{remark}
    The first two criteria defines $H$ to be what is usually called a homological functor. Adding the third criteria makes $H$ a homology theory, and the last is what makes it Grothendieck. 
\end{remark}

The main example of these come from the category of spectra, $\Sp$, where the associated homology theory to any spectrum is a Grothendieck homology theory.

\begin{example}
    Let $\C=\Sp$ and $R$ be a graded commutative ring. The Eilenberg--MacLane spectrum $HR$ is a commutative ring spectrum, and the associated homology theory 
    \[HR_* := [\S, HR\otimes (-)]_*\: \Sp \to \Mod_R\] 
    is a Grothendieck homology theory. This homology theory is equivalent to singular homology with $R$ coefficients. 
\end{example}

The above example holds more generally as well. 

\begin{example}
    If $\C$ is monoidal and the unit $\1$ is compact, then for any $H\in \C$ the associated functor
    \begin{align*}
        H_* \: \C &\to \Ab\gr \\
        X &\longmapsto [\1, H\otimes X]_*
    \end{align*}
    is a Grothendieck homology theory. 
\end{example}

\begin{proposition}
    Let $H\: \C\to \A$ be a Grothendieck homology theory and $\T$ a weak localizing subcategory of $\A$. In this situation, the full subcategory $\L\subseteq \C$ consisting of objects $X$ such that $HX\in \T$, is a localizing subcategory of $\C$. 
\end{proposition}
\begin{proof}
    This holds by using the same surjectivity argument from \cref{ch3:thm:premain}, just exchanging $\pi_n^\heart(-)$ with $H(-)[n]$. 
\end{proof}

This gives a commutative diagram
\begin{center}
    \begin{tikzcd}
        \C \arrow[r, "H"]           & \A           \\
        \L \arrow[r, "H"] \arrow[u] & \T \arrow[u]
    \end{tikzcd}    
\end{center}
where both of the fully faithful vertical functors have right adjoints. Note that the adjoint diagram might not commute. 

\begin{remark}
    In addition to being a localizing subcategory, we have by definition that we can check containment of $\L$ on the associated Grothendieck abelian category $\T$. This means that $\L$ also has a certain $\pi$-stability property, which one might call being $H$-stable, generalizing both \cref{ch3:def:pi-stable-localizing-subcategory} and Takahashi's notion of $H$-stability. 
\end{remark}





\subsection{Classification of localizing subcategories}
\label{ch3:ssec:classificartion-localizing}

The goal of this section is to prove \cref{ch3:thm:B}, and that it interacts well with both Lurie's classification via prestable categories, and \cref{ch3:thm:A}. As in \cref{ch3:ssec:classificartion-weak-localizing} we start by proving that the wanted map of sets exists.

\begin{lemma}
    \label{ch3:lm:pi-exact-then-abelian-localizing-heart}
    \index{Localizing subcategory!t-exact}
    \index{Localizing subcategory!Abelian}
    Let $\C$ be a $t$-stable category. If $\L$ is a $t$-exact localizing subcategory, then $\L^\heart$ is an abelian localizing subcategory of $\C^\heart$. 
\end{lemma}
\begin{proof}
    The $t$-exact localization $L\:\C\to \D$ and its right adjoint $i$ induces an adjunction 
    \begin{center}
        \begin{tikzcd}
            \C^\heart \arrow[r, "L^\heart", yshift=2pt] & \D^\heart \arrow[l, yshift=-2pt]
        \end{tikzcd}
    \end{center}
    on the corresponding hearts. As $L$ was $t$-exact, the functor $L^\heart$ is exact. In particular, the heart $\L^\heart$ is the kernel of $L^\heart$, which by \cref{ch3:prop:abelian-localizing-iff-kernel-of-localization} means that $\L^\heart$ is an abelian localizing subcategory of $\C^\heart$. 
\end{proof}

This means that we have a map 
\[\texact\overset{(-)^\heart}\to \ablocalizing\] 


Just as for the non-$t$-exact case, this map is not injective in general, meaning we have to restrict to a type of subcategory with more structure. 

% The following definition is in the homological algebra literature often called \emph{localizing subcategory}. But, as the term localizing is getting overloaded, we follow \cite{hovey-strickland_2005a} in using the terminology \emph{hereditary torsion theory} instead. This is usually something slightly different, but the two notions -- localizing and hereditary torsion theory -- agree for Grothendieck abelian categories. 

% \begin{definition}
%     Let $\A$ be a Grothendieck abelian category. A full subcategory $\T$ is said to be a hereditary torsion theory if it closed under quotients, sub-objects, extensions and coproducts. 
% \end{definition}

% \begin{remark}
%     All hereditary torsion theories are thick subcategories, but the converse is not true. A hereditary torsion theory $\T\subseteq \A$ can equivalently be described as full subcategories closed under coproducts where given a short exact sequence $0\rightarrow A\rightarrow B\rightarrow C\rightarrow 0$ in $\A$ we have $B\in \T$ if and only if $A, C\in \T$. 
% \end{remark}


\begin{definition}
    \label{ch3:def:pi-exact-localizing-subcategory}
    \index{Localizing subcategory!$\pi$-exact}
    A localizing subcategory $\L$ of a $t$-stable category $\C$ is said to be a $\pi$-exact localizing subcategory if 
    \begin{enumerate}
        \item it is $\pi$-stable, and
        \item it is the kernel of a $t$-exact localization. 
    \end{enumerate}
\end{definition}

\begin{remark}
    \label{ch3:rm:recurring-3}
    We continue our recurring remark about the dependencies of the different kinds of localizing subcategories introduced in the paper, see \cref{ch3:rm:recurring-1} and \cref{ch3:rm:recurring-2}. We now have implications
    \begin{center}
        \begin{tikzcd}
            \pi\text{-exact} 
            \arrow[d, Rightarrow] 
            \arrow[r, Rightarrow] 
            & 
            \pi\text{-stable} 
            \arrow[d, Rightarrow] 
            \\
            t\text{-exact} 
            \arrow[r, Rightarrow]                         
            & 
            t\text{-stable}                    
        \end{tikzcd}
    \end{center}
    for any localizing subcategory $\L$ of a $t$-stable category $\C$. 
\end{remark}

\begin{remark}
    The above remark also shows how the classification results are related. By \cref{ch3:thm:A} we know that $\pi$-stable corresponds to abelian weak localizing subcategories, and by \cref{ch3:cor:t-exact-corresponds-to-prestable-localizing} we know that $t$-exact corresponds to prestable localizing subcategories. By Lurie's classification, see \cref{ch3:thm:Lurie-correspondence}, we should expect the combination of the two to yield a correspondence between $\pi$-exact localizing subcategories and abelian localizing subcategories. 
\end{remark}

As $\pi$-exact localizing subcategories are by definition $t$-exact, we immediately get that the map $(-)^\heart$ restricts to a map 
\[\piexact\overset{(-)^\heart}\to \ablocalizing\]
The claim of \cref{ch3:thm:B} is that this map is a bijection.  

The $\pi$-exact localizing subcategories are the stable analogs of Lurie's notion of separating prestable localizing subcategories, defined as follows.

\begin{definition}
    \index{Localizing subcategory!Separating}
    Let $\C\geqz$ be a Grothendieck prestable category. A prestable localizing subcategory $\L\geqz \subseteq \C\geqz$ is said to be \emph{separating}, if for every $X\in \C\geqz$ such that $\pi_n^\heart X\in \L^\heart$ for all $n$, then $X\in \L\geqz$. 
\end{definition}

What we mean by saying that these are the stable analogs, is that the bijection
\[\texact \overset{(-)\geqz}\to \prlocalizing\]
from \cref{ch3:cor:t-exact-corresponds-to-prestable-localizing} restricts to a bijection between $\pi$-exact stable localizing subcategories and separating prestable localizing subcategories. We prove this in two steps. 

\begin{lemma}
    \label{ch3:lm:pi-exact-then-separating}
    Let $\C$ be a $t$-stable category. If $\L$ is a $\pi$-exact localizing subcategory of $\C$, then $\L\geqz$ is a separating localizing subcategory of $\C\geqz$. 
\end{lemma}
\begin{proof}
    By \cref{ch3:cor:t-exact-corresponds-to-prestable-localizing} we know that $\L\geqz$ is a prestable localizing subcategory of $\C\geqz$, so it remains to check that it is separating. Assume $X\in \C\geqz$ and $\pi_n^\heart X \in \L^\heart$ for all $n\geq 0$. Treating $X$ as an object in $\C$ via the inclusion $\C\geqz \hookrightarrow \C$ we have $\pi_i^\heart X = 0$ for all $i<0$. Hence,by the assumption that $\L$ is $\pi$-stable, we must have $X\in \L$. This means that $X \in \C\geqz\cap \L = \L\geqz$, which finishes the proof. 
\end{proof}

\begin{lemma}
    \label{ch3:lm:separating-then-pi-exact}
    If $\L\geqz$ is a separating prestable localizing subcategory of $\C\geqz$, then $\Sp(\L\geqz)$ is a $\pi$-exact localizing subcategory of $\C$. 
\end{lemma}
\begin{proof}
    We know by \cref{ch3:cor:t-exact-corresponds-to-prestable-localizing} that $\Sp(\L\geqz)$ is a $t$-exact localizing subcategory of $\C$, so it remains to show that it is $\pi$-stable. 

    For the sake of a contradiction, assume that there is some $X\in \C$ with $\pi_n^\heart X\in \L^\heart$ for all $n$, but $X\not\in \L$. Using a suspension argument, it is enough to assume that $X$ is not coconnective. As the corresponding localization functor $L\:\C\to \D$ is $t$-exact we get $L\tau\geqz X \simeq \tau\geqz LX$, which is by assumption non-zero, as $X$ was not in $\L$. This means, however, that there is an object $Y=\tau\geqz X$ in $\C\geqz$ with $\pi_n^\heart Y \in \L^\heart$ but $Y$ not in $\L\geqz$, which contradicts $\L\geqz$ begin separating.  
\end{proof}



% \begin{remark}
%     In \cref{ch3:ex:abelian-torsion} we explained how the subcategory of $I$-power torsion $R$-modules for some commutative noetherian ring $R$ with an ideal $I\subseteq R$ is an abelian weak localizing subcategory, and that the two lifts $\Der(\Mod_R^{I-tors})$ and $\Der(R)^{I-tors}$ are equivalent localizing subcategories of $\Der(R)$. The category $\Mod_R^{I-tors}$ is in fact usually an abelian localizing subcategory, not just a weak one. With reference to \cref{ch3:lm:t-stable-then-weak-localizing-heart} this should mean that the lifts to the stable setting should have more properties than just being $\pi$-stable localizing subcategories. This was the main inspiration for \cref{ch3:thm:B}. 
% \end{remark}





% Before we prove our second main result we recall Lurie's correspondence between prestable localizing subcategories and hereditary torsion theories. 

We are now ready to prove \cref{ch3:thm:B}. As for \cref{ch3:thm:A} we prove that the map $(-)^\heart$ is both injective and surjective, starting with the former. 

\begin{lemma}
    \label{ch3:lm:pi-exact-are-the-biggest}
    Let $\C$ be a $t$-stable category. Given two $t$-exact localizing subcategories $\L_0$ and $\L_1$, where $\L_1$ is $\pi$-exact, then we have $\L_0 \subseteq \L_1$ if and only if $\L_0^\heart \subseteq \L_1^\heart$. 
\end{lemma}
\begin{proof}
    This immediately follows from the non-$t$-exact case from \cref{ch3:lm:pi-stable-are-the-biggest}, as $\L_1$ is $\pi$-stable and $\L_0$ is $t$-stable. 
\end{proof}

As before, this implies that the wanted map is injective. 

\begin{corollary}
    \label{ch3:cor:pi-exact-heart-injective}
    For any $t$-stable category $\C$, the map
    \[\piexact \overset{(-)^\heart}\to \ablocalizing\]
    is injective. 
\end{corollary}

It remains then to show that the map is also surjective. In order to do this we invoke Lurie's correspondence. The author originally wanted to have a proof not relying on the prestable case. But, we currently do not know how to directly lift an abelian subcategory to a kernel of a $t$-exact functor, without passing through the bijection from \cref{ch3:cor:t-exact-corresponds-to-prestable-localizing}. There is a more direct approach in certain contexts---for example if the $t$-structure is bounded, see \cite[2.20]{antieau-gepner-heller_2019}, or the inclusion $\L\subseteq \C$ preserves compacts, see \cite[2.7]{hennion-porta-vezzosi_2016}---but as far as the author is aware, there is no general way to know when the localization determined by a localizing subcategory $\L$ is $t$-exact. 

\begin{theorem}[{\cite[C.5.2.7]{lurie_SAG}}]
    \label{ch3:thm:Lurie-correspondence}
    \index{Prestable $\infty$-category!Grothendieck}
    For any Grothendieck prestable category $\C\geqz$, there is a bijection 
    \[\separating \to \ablocalizing\]
    given by $\L\geqz\longmapsto \L^\heart$. 
\end{theorem}

Using this, together with \cref{ch3:lm:separating-then-pi-exact} we finally get our wanted one-to-one correspondence. 

\begin{theorem}[\cref{ch3:thm:B}]
    \label{ch3:thm:main}
    \index{Presentable $\infty$-category!t-stable}
    Let $\C$ be a $t$-stable category. There is a bijective map
    \[\piexact\overset{(-)^\heart}\to \ablocalizing\]
    given by $\L\longmapsto \L^\heart$. 
\end{theorem}
\begin{proof}
    The map is injective by \cref{ch3:cor:pi-exact-heart-injective}, so it remains only to show surjectivity. Let $\A\subseteq \C^\heart$ be an abelian localizing subcategory. By \cref{ch3:thm:Lurie-correspondence} there is a unique separating prestable localizing subcategory $\L\geqz \subseteq \C\geqz$ such that $\L^\heart \simeq \A$. By \cref{ch3:lm:separating-then-pi-exact} the spectrum objects in this category, $\Sp(\L\geqz)$ is a $\pi$-exact stable localizing subcategory of $\C$ with heart $\A$. Hence, the map is also surjective. 
\end{proof}

% Wrong attempt?:
% We first show that the map is injective. Let $\L$ and $\T$ be $\pi$-stable localizing subcategories of $\C$ such that $\L^\heart \simeq \T^\heart$. By \cref{ch3:lm:strong-t-stable-then-hereditary} the hearts are both hereditary torsion theories in $\C^\heart$, and as $(-)^\heart$ factors through the set of separating localizing subcategories by \cref{ch3:lm:pi-stable-then-separating}, we have an equivalence $\L\geqz\simeq \T\geqz$ by \cref{ch3:thm:Lurie-correspondence}. By \cref{ch3:lm:if-prestable-equiv-then-stable-equiv} this means that $\L\simeq \T$, as they are both stable categories with right complete $t$-structures by \cref{ch3:lm:localizing-inherits-completeness-and-colimits}. Hence the map $(-)^\heart$ is injective. 

From this we again obtain some natural corollaries. The first one is a partial converse to \cite[2.13]{takahashi_2009}.

\begin{corollary}
    \label{ch3:cor:smashing-pi-exact}
    \index{Localization!Smashing}
    Let $R$ be a commutative noetherian ring and equip $\Der(R)$ with its natural $t$-structure. In this situation there is a bijection between the collection of smashing localizing subcategories and the collection of $\pi$-exact localizing subcategories in $\Der(R)$. 
\end{corollary}
\begin{proof}
    A theorem of Gabriel, see \cite[VI.2.4(b)]{gabriel_1962}, shows that there is a bijection between the collection of localizing subcategories of $\Mod_R$ and specialization closed subsets of $\Spec R$. Further, Neeman shows in \cite[3.3]{neeman_bokstedt_1992} that there is a bijection between specialization closed subsets of $\Spec R$ and smashing localizing subcategories of $\Der(R)$. The result then follows from these, together with \cref{ch3:thm:main}. 
\end{proof}

We can also obtain an extension of \cref{ch3:cor:smashing-pi-exact} to noetherian schemes $X$. Recall that we denote the abelian category of quasi-coherent sheaves on $X$ by $\QCoh(X)$, and its associated derived category of quasi-coherent $\mathcal{O}_X$-modules by $\Der_{qc}(X)$. 

\begin{lemma}
    For any noetherian scheme $X$, there are bijections
    \[\DXsmashing\simeq \subsetsX \simeq \QCohlocalizing.\]
\end{lemma}
\begin{proof}
    \index{Localizing subcategory!$\otimes$-ideal}
    The latter bijection is again due to Gabriel---\cite[VI.2.4(b)]{gabriel_1962}. By \cite[4.13]{tarrio-lopez-salorio_2004} the telescope conjecture holds for noetherian schemes. In particular, this means that there is a bijection between subsets of $X$ and localizing $\otimes$-ideals in $\Der_{qc}(X)$, see \cite[8.13]{stevenson_2013}, which restricts to a bijection
    \[\DXsmashing\simeq \subsetsX,\]
    giving the first bijection. 
\end{proof}

Utilizing this, together with \cref{ch3:thm:main}, we obtain the following generalization. 

\begin{corollary}
    Let $X$ be a noetherian scheme and equip $\Der_{qc}(X)$ with its natural $t$-structure. In this situation, there is a bijection
    \[\DXsmashing \simeq \DXlocalizing.\]
\end{corollary}

Using \cref{ch3:cor:noetherian-scheme-weak-localizing} we then get a partial extension of the two bottom rows in the main result of \cite{takahashi_2009} to the case of noetherian schemes.  

\begin{center}
    \adjustbox{scale=0.9,center}{
    \begin{tikzcd}
        \DXstable 
        \arrow[r, lightgray, "\simeq"]
        \arrow[rr, "\simeq", bend left = 11]                 
        & 
        \coherentX 
        \arrow[r, lightgray, "\simeq"]                
        & 
        \QCohweak                       \\
        \DXsmashing 
        \arrow[u, hook, "\subseteq", swap] 
        \arrow[r, "\simeq"] 
        & 
        \subsetsX 
        \arrow[u, hook, lightgray, "\subseteq", swap] 
        \arrow[r, "\simeq"] 
        & 
        \QCohlocalizingthree 
        \arrow[u, hook, "\subseteq", swap]
    \end{tikzcd}
}
\end{center}

Here the grey color indicates the conjectured relationship from \cref{ch3:conj:coherent-noetherian-scheme}. 

We can also use the proof of the telescope conjecture for certain algebraic stacks, due to Hall--Rydh (\cite{hall-rydh_2017}), to extend the above corollary even further. We leave the details of this to the interested reader. 

By work of Kanda we can almost extend this to the locally noetherian setting. In particular, for $X$ a locally noetherian scheme, Kanda proves in \cite[1.4]{kanda_2015} that there is a bijection between localizing subcategories of $\QCoh(X)$ and specialization closed subsets of $X$. However, as the telescope conjecture is---to the best of our knowledge---currently unresolved for locally noetherian schemes, we do not get a bijection to smashing localizing subcategories. The best we can obtain is then the following corollary. 

\begin{corollary}
    For $X$ a locally noetherian scheme, there are bijections
    \[\DXlocalizing\simeq \subsetsX \simeq \QCohlocalizing\]
\end{corollary}

\begin{remark}
    It would be very interesting to have a more direct proof for the fact that $\pi$-exact localizing subcategories of $\Der(R)$ and $\Der_{qc}(X)$ correspond to smashing localizations. Having a direct proof would allow for a new proof of the telescope conjecture for commutative noetherian rings and noetherian schemes, and could shed some new light on the currently unsolved telescope conjecture for locally noetherian schemes. 
\end{remark}

\begin{remark}
    We also want to highlight other work of Kanda, where he shows that localizing subcategories of a locally noetherian Grothendieck abelian category $\A$ are classified by the \emph{atom spectrum} of $\A$, see \cite[5.5]{kanda_classifying_2012}. It would be interesting to see if these atomic methods could provide new insight also into the stable $\infty$-category $\C$. 
\end{remark}





% -------------------------- Surjectivity 1 -----------------------------

% For surjectivity we follow the proof of \cite[C.5.2.7]{lurie_SAG}, adapted to the stable setting. Let $\A$ be a hereditary torsion theory in $\C^\heart$. Define $\L\subseteq \C$ to be the full subcategory spanned by objects $X$ such that $\pi_n^\heart X\in \A$. We prove that it is a stable localizing subcategory---it will obviously be $t$-stable by definition. By \cref{ch3:rm:prestable-localizing-in-stable-then-stable-localizing} it is enough to show that it is a prestable localizing subcategory of $\C$, i.e., that it is closed under coproducts, cofiber sequences and sub-objects. 

% As $\pi_n^\heart$ commutes with coproducts, and any hereditary torsion theory is closed under these, also $\L$ must be closed under coproducts. 

% Let $X\rightarrow Y\rightarrow Z$ be a cofiber sequence in $\C$. We show that if two out of the three objects $X, Y, Z$ are in $\L$, then so is the last. The proofs for the different choices of two objetcs are very similar, hence we only spell out the details for $X, Y\in \L$. For any $n$ we get an exact sequence $\pi_n^\heart X\rightarrow \pi_n^\heart Y \rightarrow \pi_n^\heart Z$. As $X, Y\in \L$ we have $\pi_n^\heart X$ and $\pi_n^\heart Y$ in $\A$. Since $\A$ is a hereditary torsion theory, it is closed under quotients, hence also $\pi_n^\heart Z \in \A$ for all $n$. By definition this implies that $Z\in \L$. As the other two proofs are completely analogous, we conclude that $\L$ is closed under cofiber sequences. 

% Now, suppose we are given $X\in \L$ and a subobject $X'$, i.e., a cofiber sequence $X'\rightarrow X\rightarrow X''$ where $X''\in \C^\heart$. By the long exact sequence in heart-valued homotopy groups, the maps $\pi_n^\heart X'\to \pi_n^\heart X$ are monomorphisms for all $n\neq -1$. 

% Not completely sure how to resolve this for n = -1. 

% ---------------------------- Surj 2 -----------------------------

% Let now $\A\subseteq \C^\heart$ be a hereditary torsion theory. By \cref{ch3:thm:Lurie-correspondence} there is a unique separating Grothendieck prestable localizing subcategory $\T \subseteq \C\geqz$ such that $\L^\heart \simeq \A$. We claim that there is a $t$-stable localizing ideal $\L\subseteq \C$ such that $\L\geqz\simeq \T$. Define $\L$ to be the full subcategory of $\Sp(\T)$ such that $\pi_n^\heart X \in \A$ for all $n$. We claim that this category fits the requirements. 

% The category $\T$ is the full subcategory of the Grothendieck prestable category $\C\geqz$ consisting of objects $X$ such that $\pi_0^\heart X \in \A$ for all $n\geq 0$. By \cref{ch3:prop:stabilizing-localizing-is-localizing} we know that $\Sp(\T)$ is a stable localizing subcategory. 

To summarize this section, we construct the bottom part of the diagram from the introduction. By \cref{ch3:lm:pi-exact-then-separating} the bijection from \cref{ch3:thm:B} factors through the bijection of \cref{ch3:thm:Lurie-correspondence}. In particular, we get bijections 
\begin{center}
    \adjustbox{scale=0.95,center}{
    \begin{tikzcd}
        \piexact \arrow[r, "(-)\geqz", yshift=2pt] & \separating \arrow[r, "(-)\leqz"] \arrow[l, "\Sp(-)", yshift=-2pt] & \ablocalizing
    \end{tikzcd}
}
\end{center}
such that the composite map from the left to the right is the map $(-)^\heart$ from \cref{ch3:thm:B}. This finally gives the wanted diagram. 

\begin{center}
    \adjustbox{scale=0.95,center}{
    \begin{tikzcd}
        \piexact 
        \arrow[r, "\simeq"] 
        \arrow[d, hook, "\subseteq"]
        & 
        \separating 
        \arrow[r, "\simeq"]
        \arrow[d, hook, "\subseteq"]
        & 
        \ablocalizing
        \\
        \texact 
        \arrow[r, "(-)\geqz"]
        & 
        \prlocalizing 
        \arrow[ru, "(-)\leqz", swap] 
        &                                
    \end{tikzcd}
    }
\end{center}





\subsection{Comparing stable categories with the same heart}

We round off the paper by proving some easy corollaries of \cref{ch3:thm:A} and \cref{ch3:thm:B} for stable categories with $t$-structures with the same heart. The first immediate corollary is the following. 

\begin{corollary}
    \index{Grothendieck abelian}
    Let $\A$ be any Grothendieck abelian category. For any two $t$-stable categories $\C$ and $\D$ with $\C^\heart \simeq \A\simeq \D^\heart$ there are one-to-one correspondences
    \[\pistable\to\pistableD\]
    and 
    \[\piexact\to \piexactD.\]
\end{corollary}

The above correspondence might not be induced by a functor between $\C$ and $\D$, but is just an abstract isomorphism. However, in the case when there is a functor, the $\pi$-stable localizing subcategories are also functorially related. We can set this up as follows. 

\begin{lemma}
    \index{t-exact functor}
    \label{ch3:lm:restricted-functor-on-pi-localizing}
    Let $\C$ and $\D$ be $t$-stable categories with $\A^\heart\subseteq \C^\heart$ and $\T^\heart\subseteq \D^\heart$ abelian weak localizing subcategories of the respective hearts. If there is a $t$-exact functor $F\:\C\to \D$ such that the functor on hearts $F^\heart\: \C^\heart\to \D^\heart$ restricts to a functor $F^\heart_{\vert \A^\heart}\: \A^\heart \to \T^\heart$, then the functor $F$ restricts to the unique $\pi$-stable localizing subcategories $F_{\vert \A}\:\A\to \T$. 
\end{lemma}
\begin{proof}
    As $F$ is $t$-exact we have $F(\pi_{\C,n}^\heart X)\simeq \pi_{\D, n}^\heart F(X)$. By assumption we know that $F(\pi_{\C,n}^{\heart}X) \simeq F^\heart(\pi_{\C,n}^\heart X) \in \T^\heart$, hence any $Y$ in the image of $F$ has $\pi_{\D,n}^\heart Y \in \T^\heart$ for any $n$. Since $\T$ is $\pi$-stable this implies that $Y\in \T$, proving the claim.
\end{proof}

Let $F\:\C\to\D$ be a $t$-exact functor of $t$-stable categories such that the induced functor $F^\heart \: \C^\heart\overset{\simeq}\to \D^\heart$ is an equivalence. Assume further that $\A$ is an abelian weak localizing subcategory of $\C^\heart$, and that $F^\heart$ restricts to a functor $F^\heart_{\vert \A}\:\A\to\A$. By \cref{ch3:lm:restricted-functor-on-pi-localizing} we get restricted functors $F_{\vert \A_\C}\: \A_\C \to \A_\D$, where $\A_\C$ and $\A_\D$ respectively denote the unique $\pi$-stable localizing subcategories of $\C$ and $\D$ obtained via \cref{ch3:thm:A}.

\begin{corollary}
    If $F$ is an equivalence, then every such restricted functor $F_{\vert \A_\C}$ is an equivalence. 
\end{corollary}

One interesting feature of the $\infty$-categorical framework is the existence of realization functors in reasonable generalities. If $\C$ is a $t$-stable category, then a realization functor for $\C$ is a functor $R\:\Der(\C^{\heart})\to \C$, extending the inclusion of the heart. In particular, $R$ restricts to the identity on $\Der(\C^\heart)^\heart\simeq \C^\heart$. These realization functors are rarely equivalences, even rarely full or faithful, but we can still apply \cref{ch3:lm:restricted-functor-on-pi-localizing} to functorially relate the $\pi$-stable localizing subcategories. Note that as $R$ restricts to the identity in hearts, we do not even need to assume or prove that the functor $R$ is $t$-exact, as the proof of \cref{ch3:lm:restricted-functor-on-pi-localizing} goes through regardless.  

The following argument is due to Maxime Ramzi.

\begin{lemma}
    \label{ch3:lm:realization-functor}
    Let $\C$ be a $t$-stable category and $\Der(\C^\heart)$ the derived category of its heart. In this situation there is a realization functor $R\:\Der(\C^{\heart})\to \C$ extending the inclusion $\C^\heart\hookrightarrow \C$. 
\end{lemma}
\begin{proof}
     The inclusion $\C^\heart\hookrightarrow \C$ extends to a functor $\Fun(\Delta\op, \C^\heart)\to \C$ via geometric realization, which preserves weak equivalences by \cite[1.2.4.4, 1.2.4.5]{Lurie_HA}. Via the Dold--Kan correspondence this gives a essentially unique colimit preserving functor $\Der(\C^\heart)\geqz\to \C$, which extends uniquely to a functor $\Der(\C^\heart)\to \C$ by \cite[1.4.4.5]{Lurie_HA}, as $\C$ is stable. This functor preserves both colimits and the heart $\C^\heart$. 
\end{proof}

We can then functorially relate the $\pi$-stable localizing subcategories of $\Der(\C^\heart)$ and $\C$ via the realization functor. 

\begin{corollary}
    Let $\C$ be a $t$-stable category and let $R\: \Der(\C^\heart)\to \C$
    be the realization functor from \cref{ch3:lm:realization-functor}. For any weak localizing subcategory $\A\subseteq \C^\heart$, the functor $R$ restricts to a functor
    \[R\: \A_{\Der(\C^\heart)}\to \A_\C,\]
    where the former category denotes the unique $\pi$-stable lift of $\A$ to $\Der(\C^\heart)$, and the latter the unique $\pi$-stable lift of $\A$ to $\C$. 
\end{corollary}
\begin{proof}
    This follows immediately from \cref{ch3:lm:restricted-functor-on-pi-localizing}, the $\pi$-stability of $\A_\C$ and the fact that the identity restricts to the identity functor $\A\simeq\A_{\Der(\C^\heart)}^\heart \to \A_\C^\heart\simeq \A$. 
\end{proof}

\begin{remark}
    By \cref{ch3:prop:pi-stable-then-t-stable} the $\pi$-stable localizing subcategory $\A_\C$ is also $t$-stable, with heart $\A$. Hence, there is also a realization functor $R'\: \Der(\A)\to \A_\C$, and a natural question to ask is whether this coincides with the above restricted functor $R\: \A_{\Der(\C^\heart)}\to \A_\C$. There is an inclusion $\Der(\A)\subseteq \A_{\Der(\C^\heart)}$, as the latter is a $\pi$-stable localizing subcategory of $\Der(\C^\heart)$, but we do not know if this is always an equivalence. In particular, we don't know whether $\D(\A)$, treated as a subcategory of $\Der(\C^\heart)$, is always a $\pi$-stable localizing subcategory. 
\end{remark}

\begin{addendum}
    \index{I-power torsion!Comodule}
    To add some credibility to the above remark we recall 
    the localizing subcategory $\Comodt$ of $I_n$-torsion comodules in $\ComodE$, as studied studied in \hyperref[ch1]{Paper I} as well as \cref{ch0:sssec:torsion-and-completion-for-comodules} . We know that $\Der(\A)$ comes equipped with a natural $t$-structure for any Grothendieck abelian category $\A$, making $\Der(\A)$ into a $t$-stable category. In particular, this holds for $\Der(E_*E)$. By \cref{ch3:thm:main} we know that there is a unique $\pi$-exact localizing subcategory $\L \subseteq \Der(E_*E)$ with heart $\L^\heart \simeq \Comodt$. By \cite[3.7(2)]{barthel-heard-valenzuela_2020} this localizing subcategory is precisely 
    \[\Der(E_*E)^{I_n-tors} \simeq \Der(\Comodt),\]
    as it is $\pi$-stable, hence equivalent to $\L$ by a maximality argument---see \cref{ch3:lm:pi-stable-are-the-biggest}. 

    By \cref{ch1:thm:pulling-out-torsion} this also holds for the periodic category $\Dper(\Comodt)$. 
\end{addendum}













