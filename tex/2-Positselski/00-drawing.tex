

\newpage
\tikz[remember picture,overlay]\node[opacity=1,inner sep=0pt] at (current page.center)%
{\includegraphics[%
clip,
width=1.05\paperwidth,
height=1.05\paperheight
]{chaptertitles/paper2.pdf}};

\clearpage


\subsection*{Description}

The main result of the second paper concerns a mathematical concept called a duality theory. A duality is a way to view a collection of objects ``through a mirror'' and study their reflections instead of their direct features. Studying a concept in its mirror image, and then dualizing, should be equivalent to studying the concept directly. This mathematical mirror can be a wide variety of things, but for the drawing we have chosen a classical planar mirroring to signify this effect. The flowing lines of each of the boxes are precisely mirror images of each other, along the symmetry line right in the middle. 

The colors again have no mathematical meaning, and are there only to add visual interst, and to connect to the colors of the papers. 

\newpage