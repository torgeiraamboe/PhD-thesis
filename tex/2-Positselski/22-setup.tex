
\section{General preliminaries}

The goal of this section is to introduce comodules and contramodules over an cocommutative coalgebra in some $\infty$-category $\C$. In order to do this we first review some basic facts about coalgebras, monads and comonads. 

We will for the rest of this section work in some fixed presentably symmetric monoidal $\infty$-category $\C$. In other words, $\C$ is a commutative algebra object in $\PrL$, the category of presentable $\infty$-categories and left adjoint functors. In particular, the monoidal product, which we denote by $-\otimes-\:\C\times \C\to\C$ preserves colimits in both variables. We denote the unit of the monoidal structure by $\1$. \index{Presentable $\infty$-category!Symmetric monoidal}


\subsection{Coalgebras, monads and comonads}

We denote the category of commutative algebras in $\C$ by $\CAlg(\C)$\index{Cocommutative coalgebra}. These are the coherently commutative ring objects in $\C$. By \cite[2.4.2.7]{Lurie_HA} there is a symmetric monoidal structure on $\C\op$, and we define the category of cocommutative coalgebras in $\C$ to be the category 
\[\cCAlg(\C):= \CAlg(\C\op)\op.\] 
Classical coalgebras will be referred to as \emph{discrete}\index{Cocommutative coalgebra!Discrete} in order to avoid confusion. 

\begin{proposition}
    The following properties hold for the category $\cCAlg(\C)$. 
    \begin{enumerate}
        \item The forgetful functor 
        \[U\:\cCAlg(\C)\to \C\] 
        is conservative and creates colimits. \label{ch2:prop:properties-coalgebras:item1}
        \item The categorical product of two coalgebras $C, D$ is given by the tensor product of their underlying objects $C\otimes D$. \label{ch2:prop:properties-coalgebras:item2}
        \item The $\infty$-category $\cCAlg(\C)$ is presentably symmetric monoidal when equipped with the cartesian monoidal structure. In particular, this means that the forgetful functor $U$ is symmetric monoidal. \label{ch2:prop:properties-coalgebras:item3}
        \item The forgetful functor $U$ has a lax-monoidal right adjoint 
        \[cf\:\C\to\cCAlg(\C).\] 
        The image of an object $X\in \C$ is called the cofree coalgebra on $X$. \label{ch2:prop:properties-coalgebras:item4}
    \end{enumerate}
\end{proposition}
\begin{proof}
    The presentability and creation of colimits by the forgetful functor is proven in \cite[3.1.2]{lurie_2018_ELL1} and \cite[3.1.4]{lurie_2018_ELL1}. The cartesian symmetric monoidal structure on $\cCAlg(\C)$ follows from \cite[3.2.4.7]{Lurie_HA}. The last item follows from the first three together with the adjoint functor theorem, \cite[5.5.2.9]{lurie_09}. 
\end{proof}

% Old proof 
% \cref{ch2:prop:properties-coalgebras:item1} follows from \cite[3.2.2.5, 3.2.2.6]{Lurie_HA} while \cref{ch2:prop:properties-coalgebras:item2} follows from \cite[3.2.4.7]{Lurie_HA}. \cref{ch2:prop:properties-coalgebras:item3} is \cite[3.1.4]{lurie_2018_ELL1}, while \cref{ch2:prop:properties-coalgebras:item4} follows from \cref{ch2:prop:properties-coalgebras:item1} and \cref{ch2:prop:properties-coalgebras:item3} via the adjoint functor theorem \cite[5.5.2.9]{lurie_09}. 


Given any $\infty$-category $\D$, the category of endofunctors $\Fun(\D, \D)$ can be given the structure of a monoidal category via composition of functors, see \cite[1.15]{christ_2023}. 

\begin{definition}
    \index{Monad}
    \index{Comonad}
    A monad $M$ on $\D$ is an associative algebra in $\Fun(\D, \D)$. Similarly, a comonad $C$ is a coassociative coalgebra in $\Fun(\D, \D)$. 
\end{definition}

\begin{example}
    Any adjunction of $\infty$-categories
    \begin{center}
    \begin{tikzcd}
        \D \arrow[r, yshift=2pt, "F"] & \mathcal{E} \arrow[l, yshift=-2pt, "G"]
    \end{tikzcd}
    \end{center}
    gives rise to a monad $GF\: \D\to \D$ and a comonad $FG\:\mathcal{E}\to \mathcal{E}$. We call these the \emph{adjunction monad} and \emph{adjunction comonad} of the adjunction $F\dashv G$. 
\end{example}

The category $\D$ is left tensored over $\Fun(\D, \D)$ via evaluation of functors. Hence, for any monad $M$ on $\D$ we get a category of left modules over $M$ in $\D$. 

\begin{definition}
    \index{Monad!Eilenberg--Moore category}
    \index{Monad!Module}
    Let $\D$ be an $\infty$-category and $M$ a monad on $\D$. We define the \emph{Eilenberg--Moore category} of $M$ to be the category of left modules $\LMod_M(\D)$. Objects in $\LMod_M(\D)$ are referred to as \emph{modules over M}. 
\end{definition}

\begin{remark}
    \index{Comonad!Eilenberg--Moore category}
    \index{Comonad!Comodule}
    Dually, any comonad $C$ on $\D$ gives rise to a category of left comodules over $C$ in $\D$. We also call this the Eilenberg--Moore category of $C$, and denote it by $\LComod_C(\D)$. Its objects are referred to as \emph{comodules} over $C$. 
\end{remark}

Given a monad $M$ on $\D$ we have a forgetful functor 
\[U_M\: \LMod_M(\D)\to \D.\] 
By \cite[4.2.4.8]{Lurie_HA} this functor admits a left adjoint 
\[F_M\: \D\to \LMod_M(\D)\] 
given on objects by $X\longmapsto MX$. We call this the \emph{free module} functor. The adjunction $F_M\dashv U_M$ is called the \emph{free-forgetful} adjunction of $M$. 

\begin{definition}
    \index{Monad!-ic adjunction}
    \index{Monad!-ic functor}
    An adjunction is said to be \emph{monadic} if it is equivalent to the free-forgetful adjunction $F_M\dashv U_M$ of a monad $M$. A functor $G\: \mathcal{E}\to \D$ is called \emph{monadic} if it is equivalent to the right adjoint $U_M$ for some monadic adjunction. 
\end{definition}

The existence of the free-forgetful adjunction for a monad $M$ implies that any monad is the adjunction monad of some adjunction. However, there can be more than one adjunction $F\dashv G$ such that $M$ is the adjunction monad for this adjunction.

\begin{definition}
    \index{Monad!Free module}
    \index{Monad!Kleisli category}
    Let $\D$ be an $\infty$-category and $M$ a monad on $\D$. A left $M$-module $B\in \LMod_M(\D)$ is \emph{free} if it is equivalent to an object in the image of $F_M$. The full subcategory of free modules is called the \emph{Kleisli category} of $M$, and is denoted $\LMod_M\fr(\D)$. 
\end{definition}

The free-forgetful adjunction restricts to an adjunction on the Kleisli category: 
\begin{center}
\begin{tikzcd}
    \D \arrow[r, yshift=2pt, "{F_M}"] & \LMod_M\fr(\D) \arrow[l, yshift=-2pt, "{U_M\fr}"].
\end{tikzcd}
\end{center}
By \cite[1.8]{christ_2023} this is the minimal adjunction whose adjunction monad is equivalent to $M$---in particular, the forgetful functor $U_M\fr$ is fully faithful.  

Using Lurie's $\infty$-categorical version of the Barr--Beck theorem we can also identify the free-forgetful adjunction as the maximal adjunction with adjunction monad $M$. 

\begin{theorem}[{\cite[4.7.3.5]{Lurie_HA}}]
    \label{ch2:thm:Lurie-BB}
    \index{Barr--Beck}
    A functor $G\: \mathcal{E}\to \D$ of $\infty$-categories is monadic if and only if 
    \begin{enumerate}
        \item $G$ admits a left adjoint,
        \item $G$ is conservative, and
        \item the category $\mathcal{E}$ admits colimits of $G$-split simplicial objects, and these are preserved under $G$. 
    \end{enumerate}
\end{theorem}

\begin{remark}
    By definition, if a functor $G\: \mathcal{E}\to \D$ is monadic, then there is an equivalence of categories $\mathcal{E}\simeq \LMod_{GF}(\D)$, where $F$ is the left adjoint of $G$. 
\end{remark}

\begin{definition}
    \index{Comonad!Cofree comodule}
    \index{Comonad!-ic adjunction}
    Dually, given any comonad $C$ on an $\infty$-category $\mathcal{E}$, there is a forgetful functor $U_C\: \LComod_C(\mathcal{E})\to \mathcal{E}$, which admits a right adjoint 
    \[F_C\: \mathcal{E}\to \LComod_C(\mathcal{E}).\] 
    We call this the \emph{cofree comodule functor}, and hence the adjunction $U_C\dashv F_C$ is called the \emph{cofree-forgetful} adjunction of $C$. Any adjunction 
    \begin{center}
    \begin{tikzcd}
        \D \arrow[r, yshift=2pt, "F"] & \mathcal{E} \arrow[l, yshift=-2pt, "G"]
    \end{tikzcd}
    \end{center}
    equivalent to a cofree-forgetful adjunction for some comonad $C$ on $\mathcal{E}$ is said to be \emph{comonadic}. A functor $F\: \D\to \mathcal{E}$ is said to be \emph{comonadic} if it is equivalent to the left adjoint of a comonadic adjunction. 
\end{definition}

\begin{remark}
    \index{Comonad!Kleisli category}
    The essential image of $F_C$ in $\LComod_C(\mathcal{E})$ determines the Kleisli category $\LComod_C\fr(\mathcal{E})$ of cofree coalgebras. The cofree-forgetful adjunction restricts to an adjunction on cofree comodules, 
    \begin{center}
    \begin{tikzcd}
        \Comod_C\fr(\mathcal{E}) \arrow[r, yshift=2pt, "{U_C\fr}"] & \mathcal{E} \arrow[l, yshift=-2pt, "F_C"]
    \end{tikzcd}
    \end{center}
    which is the minimal adjunction whose adjunction comonad is $C$. 
\end{remark}


\subsection{Comodules and contramodules}

Recall that we have fixed a presentably symmetric monoidal $\infty$-category $\C$. Let us now construct the monads and comonads of interest for this paper. We want to mention that the paper \cite{hristova-jones-rumynin_2023} has been of influence for this section.

\begin{example}
    \label{ch2:ex:algebra-module-monad}
    Let $A\in \CAlg(\C)$ be a commutative algebra object in $\C$. The algebra structure on $A$ induces an algebra structure on the endofunctor $A\otimes(-)\:\C\to \C$, hence it is a monad on $\C$---see \cite[1.15]{christ_2023}. By \cite[1.17]{christ_2023} the Eilenberg--Moore category of this monad is equivalent to the category of modules over $A$ in $\C$. As $A$ is commutative we denote this by $\Mod_A(\C)$. As $\C$ is presentable and the monad $A\otimes(-)$ preserves colimits, there is a right adjoint $\iHom(A,-)\:\C\to \C$. This is a comonad on $\C$. Since these form an adjoint monad-comonad pair, their Eilenberg--Moore categories are equivalent,
    \[\Mod_A(\C)\simeq \LMod_{A\otimes(-)}(\C)\simeq \LComod_{\iHom(A,-)}(\C),\]
    see \cite[V.8.2]{maclane-moerdijk_1994} in the $1$-categorical situation. The $\infty$-categorical version is exactly the same, and follows from the monadicity and comonadicity of the adjunctions. 
\end{example}

\begin{remark}
    \label{ch2:rm:internal-adjunction}
    \index{Internal adjunction}
    We also mention that the hom-tensor adjunction is an \emph{internal adjunction}, in the sense that there is an equivalence of internal hom-objects 
    \[\iHom(X\otimes Y, Z) \simeq \iHom(X, \iHom(Y, Z)).\]
    This follows from the hom-tensor adjunction together with a Yoneda argument. 
\end{remark}

% Since $\C$ is a symmetric monoidal category, we can restrict our attention to \emph{monoidal monads} and \emph{monoidal comonads}. For us these are exactly the monads (resp. comonads) that appear as the adjunction monad for a monoidal adjunctions, i.e., $F:\C\rightleftarrows \D:G$ such that $F$ (resp. $G$) is symmetric monoidal. Given a monoidal monad $M$, the corresponding Eilenberg--Moore category $\LMod_M(\C)$ inherits the structure of a presentably symmetric monoidal $\infty$-category. Since $\C$ is symmetric monoidal, the tensor product of two free $M$-modules is still free, hence this restricts to a symmetric monoidal structure on $\LMod_M\fr(\C)$. 

The above example changes in an interesting way when replacing the algebra $A$ with a coalgebra $C$. 

\begin{example}
    \label{ch2:ex:coalgebra-comonad}
    Let $C\in \cCAlg(\C)$ be a cocommutative coalgebra object in $\C$. The coalgebra structure on $C$ induces a coalgebra structure on the endofunctor $C\otimes(-)\:\C\to\C$, hence it is a comonad on $\C$. By an argument dual to \cite[1.17]{christ_2023} the Eilenberg--Moore category of this comonad is equivalent to the category of comodules over the coalgebra $C$, defined as 
    \[\Comod_C(\C):= (\Mod_{C}(\C\op))\op.\] 
    As the functor $C\otimes (-)$ is accessible, the category $\Comod_C(\C)$ is presentable by \cite[3.8]{ramzi_2024}---see also \cite[2.1.11]{peroux_2020}. 
\end{example}

\begin{example}
    Let $C\in \cCAlg(\C)$ be a cocommutative coalgebra in $\C$. The comonad $C\otimes (-)$ preserves colimits, hence it has a right adjoint $\iHom(C,-)$. The coalgebra structure on $C$, together with the internal adjunction equivalence from \cref{ch2:rm:internal-adjunction}, gives the endofunctor $\iHom(C,-)$ an algebra structure. It is an algebra structure and not a coalgebra structure, as the internal hom is contravariant in the first variable. Hence, $\iHom(C,-)$ is a monad on $\C$.
\end{example} 

Notice that the pair $C\otimes(-)\dashv \iHom(C,-)$ is not an adjoint monad-comonad pair; it is an an adjoint comonad-monad pair. The difference is subtle, but it means, in particular, that their Eilenberg--Moore categories might not be equivalent. This possible non-equivalence is the \emph{raison d'être} for contramodules, which we can then define as follows.

% \begin{lemma}
%     For any adjoint comonad-monad pair $C\dashv M$ on a presentable $\infty$-category $\C$, there is an equivalence 
%     \[\Comod\fr_C(\C)\simeq \Mod\fr_M(\C).\]
% \end{lemma}
% \begin{proof}
%     The objects in the categories are the same, given by an underlying object in $\C$ with additional structure. Hence, it is enough to show that the mapping spaces in the two categories are equivalent. Let $X, Y$ be two objects in $\C$. As the cofree comodule functor $F_C$ is right adjoint to the forgetful functor $U_C$, we get $\Hom_{\Comod_C}(F_C X, F_C Y)\simeq \Hom_\C(C (X), Y)$. By the adjunction $C\dashv M$ on $\C$ we get $\Hom_\C(C(X), Y) \simeq \Hom_\C(X, M(Y))$, which gives finally $\Hom_\C(X, M(Y))\simeq \Hom_{\Mod_M}(F_M X, F_M Y)$, as the free module functor $F_M$ is a left adjoint to the forgetful functor $U_M$.
% \end{proof}

% \begin{remark}
%     This has been noted and proved several times in the $1$-categorical literature on monads and comonads, see for example \cite[Theorem 3]{kleiner_1990}. The proof above is by \cite[2.5]{bohm-brzezinski-wisbauer_2009}. 
% \end{remark}



\begin{definition}
    \index{Contramodule}
    Let $C\in \cCAlg(\C)$ be a cocommutative coalgebra. A \emph{contramodule} over $C$ is a module over the internal hom-monad $\iHom(C,-)\:\C\to \C$. The category of contramodules over $C$ in $\C$ is the corresponding Eilenberg--Moore category, which will be denoted $\ContraC(\C)$. 
\end{definition}

\begin{notation}
    Since we are working in a fixed category $\C$ we will often simply write $\ContraC$ for the category of contramodules, and $\ComodC$ for the category of comodules. 
\end{notation}

\begin{notation}
    We denote the mapping space in $\Comod_C$ by $\Hom_C$ and the mapping space in $\Contra_C$ by $\Hom^C$. Similarly, the forgetful functors will be denoted 
    \[U_C\:\Comod_C\to \C \quad\text{and}\quad  U^C\: \Contra_C\to \C\] 
    respectively, while their adjoints---the cofree and free functors---will be denoted 
    \[C\otimes (-)\:\C\to \Comod_C \quad \text{and}\quad \iHom(C, -)\: \C\to \Contra_C,\] 
    hoping that it is clear from context whether we use them as above or as endofunctors on $\C$.
\end{notation}

The following proposition is standard for monads and comonads, see for example \cite[5.7]{riehl-verity_2015}. 

\begin{proposition}
    \label{ch2:prop:creates-colimits-and-limits}
    If $C$ is a cocommutative coalgebra in $\C$, then the forgetful functor $U_C\:\Comod_C\to \C$ creates colimits. Similarly, the forgetful functor $U^C\:\Contra_C\to \C$ creates limits. 
\end{proposition}







\subsection{The dual monoidal Barr--Beck theorem}
\label{ch2:ssec:dual-barr-beck}

Lurie's version of the Barr--Beck monadicity theorem, see \cite[Section 4.7]{Lurie_HA}, allows us to recognize monadic functors from simple criteria. Combined with a recognition theorem for when a monoidal monadic functor is equivalent to $R\otimes(-)$ for some commutative ring $R$, Mathew--Neumann--Noel extended the Barr--Beck theorem to a monoidal version. In this short section we prove a categorical dual version of their result. 

Let $F:\C\rightleftarrows \D:G$ be a pair of adjoint functors between presentably symmetric monoidal $\infty$-categories, such that the left adjoint $F$ is symmetric monoidal. This means that the right adjoint $G$ is lax-monoidal, and does in particular preserve algebra objects. There is for any two objects $X\in \C$ and $Y\in \D$, a natural map
\[F(G(Y)\otimes_\C X)\overset{\simeq}\to FG(Y)\otimes_\D F(X) \to Y\otimes_\D F(X)\]
where the first map is by the symmetric monoidality of $F$, and the second is given by the adjunction counit. By the adjunction property, there is an adjoint map 
\[G(Y)\otimes_\C X \to G(Y\otimes_\D F(X)).\]

\begin{definition}
    \index{Projection formula!Monadic}
    An adjoint pair $F\dashv G$ as above is said to satisfy the \emph{monadic projection formula} if the map 
    \[G(Y)\otimes_\C X \to G(Y\otimes_\D F(X))\]
    is an equivalence for all $X\in \C$ and $Y\in \D$. 
\end{definition}

We now state the monoidal Barr--Beck theorem of Mathew, Naumann and Noel. 

\begin{theorem}[{\cite[5.29]{mathew-naumann-noel_2017}}]
    \label{ch2:thm:monoidal-BB}
    \index{Barr--Beck!Monoidal}
    Let $F:\C\rightleftarrows \D:G$ be an adjunction of presentably symmetric monoidal $\infty$-categories, such that the left adjoint $F$ is symmetric monoidal. If, in addition
    \begin{enumerate}
        \item $G$ is conservative,
        \item $G$ preserves arbitrary colimits, and
        \item $F\dashv G$ satisfies the monadic projection formula,
    \end{enumerate}
    then the adjunction is monoidally monadic, and there is an equivalence of monads 
    \[GF \simeq G(\1_\D)\otimes_\C(-).\]
    In particular, there is an equivalence $\D\simeq \Mod_{G(\1_\D)}(\C)$ of symmetric monoidal $\infty$-categories. 
\end{theorem}

\begin{remark}
    Note that this result is stated only for stable $\infty$-categories in \cite{mathew-naumann-noel_2017}, but also holds unstably by a combination of Lurie's $\infty$-categorical Barr--Beck theorem, \cref{ch2:thm:Lurie-BB}, together with the fact that the monadic projection formula applied to the unit gives an equivalence of monads by \cite[3.6]{elmanto-kolderup_2020}. 
\end{remark}

There is also a dual version of the classical Barr--Beck theorem, see for example \cite[4.5]{brantner-mathew_2023}. We wish to extend this to a monoidal version. 

Let $F:\C\rightleftarrows \D:G$ pair of adjoint functors between symmetric monoidal categories, such that the right adjoint $G$ is symmetric monoidal. This means that the left adjoint $F$ is op-lax-monoidal, and does in particular preserve coalgebra objects. There is for any two objects $X\in \C$ and $Y\in \D$, a natural map
\[X\otimes_\C G(Y)\to GF(X)\otimes_\C G(Y) \overset{\simeq}\to G(F(X)\otimes_\D Y)\]
where the first map is given by the adjunction unit and the second by the symmetric monoidality of $G$. By the adjunction property, there is an adjoint map 
\[F(X\otimes_\C G(Y)) \to F(X)\otimes_\D Y .\]

\begin{definition}
    \index{Projection formula!Comonadic}
    An adjoint pair $F\dashv G$ as above is said to satisfy the \emph{comonadic projection formula} if the map 
    \[F(X\otimes_\C G(Y)) \to F(X)\otimes_\D Y\]
    is an equivalence for all $X\in \C$ and $Y\in \D$. 
\end{definition}

\begin{theorem}
    \index{Barr--Beck!Dual monoidal}
    \label{ch2:thm:dual-monoidal-BB}
    Let $F:\C\rightleftarrows \D:G$ be an adjunction of presentably symmetric monoidal $\infty$-categories, such that the right adjoint $G$ is symmetric monoidal. If, in addition
    \begin{enumerate}
        \item $F$ is conservative,
        \item $F$ preserves arbitrary limits, and
        \item $F\dashv G$ satisfies the comonadic projection formula,
    \end{enumerate}
    then the adjunction is comonadic, and there is an equivalence of comonads 
    \[FG\simeq F(\1_C)\otimes_\D (-)\] 
    In particular, this gives an equivalence $\C\simeq \Comod_{F(\1_\C)}(\D).$
\end{theorem}

\begin{remark}
    Before the proof, let us explain intuitively why the statement makes sense. The unit $\1_\C$ in a presentably symmetric monoidal $\infty$-category $\C$ is both a commutative algebra\index{Commutative algebra} and a cocommutative coalgebra\index{Cocommutative coalgebra}. In the above adjunction we have that the right adjoint $G$ is symmetric monoidal, hence its left adjoint $F$ is op-lax monoidal. In particular, it sends coalgebras to coalgebras, meaning that $F(\1_\C)$ is an cocommutative coalgebra in $\D$. By \cref{ch2:ex:coalgebra-comonad} tensoring with $F(\1_\C)$ is a comonad, not a monad, as for \cref{ch2:thm:monoidal-BB}. 
\end{remark}

\begin{proof}
    By \cite[4.5]{brantner-mathew_2023} the adjunction is comonadic. A dual version of \cite[3.6]{elmanto-kolderup_2020} shows that there is a map of comonads 
    \[\phi\: FG \to F(\1_\C)\otimes_\D (-),\] 
    and consequently an adjunction 
    \begin{center}
        \begin{tikzcd}
            \Comod_{FG}(\D) \arrow[r, yshift=2pt, "\phi_*"] & \Comod_{F(\1_\C)\otimes_\D (-)}(\D) \arrow[l, yshift=-2pt, "\phi^*"]
        \end{tikzcd}
    \end{center}
    By applying the projection formula to the unit $\1_\C$ we get that $\phi$ is a natural equivalence, which means that the adjunction $(\phi_*, \phi^*)$ is an adjoint equivalence. By \cref{ch2:ex:coalgebra-comonad} the Eilenberg--Moore category of the comonad $F(\1_\C)\otimes_\D (-)$ is equivalent to the category of comodules over the cocommutative coalgebra $F(\1_\C)$, finishing the proof. 
\end{proof}

\begin{remark}
    \label{ch2:rm:monoidal-structure-comodules}
    It would be nice to understand when the above result gives an equivalence of symmetric monoidal categories. For this one would first need a symmetric monoidal structure on $\Comod_C$. In the dual situation of $\Mod_R(\C)$---the category of modules over a commutative algebra $R\in \C$---this is done by Lurie's relative tensor product, see \cite[Section 4.5.2]{Lurie_HA}. But, for such a relative monoidal product to exist on $\Comod_C$ one needs the tensor product in $\C$ to commute with cosifted limits, which is rarely the case. However, as we will see in the following section, we can in fact obtain a symmetric monoidal structure and a symmetric monoidal equivalence in certain situations---for example when the comonad is idempotent. 
\end{remark}

