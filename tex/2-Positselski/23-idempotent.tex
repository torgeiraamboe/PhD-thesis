
\section{Positselski duality}
\label{ch2:sec:positselski-duality}

Classical Positselski duality, usually called the co-contra correspondence, is an adjunction between comodules and contramodules over a discrete $R$-coalgebra $C$, where $R$ is an algebra over a field $k$. In particular, the categories involved are abelian, which makes some constructions easier. For example, the monoidal structure on $\Mod_R$ induces monoidal structures on $\Comod_C$ via the relative tensor construction---given by a certain equalizers. For $\infty$-categories the relative tensor construction is more complicated, as we need the monoidal structure to behave well with all higher coherencies, as mentioned in \cref{ch2:rm:monoidal-structure-comodules}. We can, however, restrict our attention to a certain type of coalgebra, fixing these issues. This also puts us in the setting we are interested in regarding local duality---see \cref{ch2:ssec:local-duality}. 

\subsection{Coidempotent coalgebras}
\label{ch2:ssec:coidempotent-coalgebras}

We now restrict our attention to the special class of coalgebras that we will focus on for the remainder of the paper. 

\begin{definition}
    \index{Cocommutative coalgebra!Separable}
    \index{Cocommutative coalgebra!Coidempotent}
    A cocommutative coalgebra $C\in \cCAlg(\C)$ is said to be \emph{coseparable} if the comultiplication map 
    \[\Delta\: C\to C\otimes C\op\] 
    admits a $(C,C)$-bicomodule section $s\: C\otimes C\to C$. It is \emph{coidempotent} if $\Delta$ is an equivalence. 
\end{definition}

\begin{remark}
    \label{ch2:rm:coidempotent-implies-separable}
    Any coidempotent coalgebra is in particular separable, see \cite[1.6(1)]{ramzi_2023} for a formally dual statement. 
\end{remark}

The first reason for our focus on coidempotent coalgebras is that their categories of comodules inherit a symmetric monoidal structure from $\C$, which is rarely the case for general coalgebras, see \cref{ch2:rm:monoidal-structure-comodules}. 

\begin{lemma}
    \label{ch2:lm:coidempotent-then-comod-monoidal}
    Let $C$ be a coidempotent cocommutative coalgebra in $\C$. The category of $C$-comodules $\Comod_C$ inherits the structure of a presentably symmetric monoidal $\infty$-category making the cofree comodule functor a symmetric monoidal smashing colocalization.
\end{lemma}
\begin{proof}
    The category $\Comod_C$ is presentable by \cite[2.1.11]{peroux_2020}. Let $M$ and $N$ be two comodules. Their relative tensor product in $\Comod_C$ is defined by the the two sided co-bar construction,
    \[M\otimes_C N := \lim_n (M \otimes C^{\otimes n} \otimes N),\]
    but, as $C$ is coidempotent this is just the object $N\otimes C\otimes M$, which is the cofree comodule on the underlying object of $M\otimes N$. This means that the relative tensor product is defined for all comodules. The unit for the monoidal structure $-\otimes_C-$ is $C$, and the monoidal structure is symmetric monoidal as the monoidal structure in $\C$ is. 

    The endofunctor $C\otimes (-)\: \C\to \C$ is idempotent when $C$ is coidempotent. Hence, as the forgetful functor $U_C\: \Comod_C\to \C$ is fully faithful whenever $C$ is coidempotent, the cofree comodule functor $C\otimes (-)\: \C\to \Comod_C$ is a smashing colocalization of $\C$. Hence it is a symmetric monoidal functor by a dual version of \cite[2.2.1.9]{Lurie_HA}, as it is obviously compatible with the symmetric monoidal structure in $\C$, due to the coidempotency of $C$. 
\end{proof}

\begin{lemma}
    The symmetric monoidal structure on the category $\Comod_C$ is closed. 
\end{lemma}
\begin{proof}
    As the cofree-forgetful adjunction creates colimits in the category $\Comod_C$, the functor 
    \[-\otimes_C- \simeq C\otimes (-\otimes -)\: \Comod_C \times \Comod_C \to \Comod_C\] 
    preserves colimits separately in each variable. In particular, the functor $M\otimes_C (-)$ preserves colimits, hence has a right adjoint $\iHom_C(M,-)$ for any comodule $M$ by the adjoint functor theorem, \cite[5.5.2.9]{lurie_09}. This determines a functor 
    \[\iHom_C(-,-)\: \Comod_C\op\times \Comod_C\to \Comod_C\]
    making $\Comod_C$ a closed symmetric monoidal category.  
\end{proof}

\begin{remark}
    This adjunction, being a hom-tensor adjunction is also internally adjoint in the sense of \cref{ch2:rm:internal-adjunction}\index{Internal adjunction}. Hence we have an equivalence 
    \[\iHom_C(M\otimes_C N, A) \simeq \iHom_C(M, \iHom_C(N, A))\]
    for all comodules $M, N$ and $A$. 
\end{remark}

Another important reason for using coidempotent coalgebras in this paper is the following result. Recall that a comodule over a coalgebra $C$ is called cofree, if it is of the form $M\otimes C$ for some $M\in \C$. These are precisely the comodules in the image of the right adjoint to the forgetful functor $U_C\:\Comod_C\to \C$, when $C$ is coidempotent. This is a slightly weaker coalgebraic version of \cite[1.13, 1.14]{ramzi_2023}. See also \cite[3.6]{brzezinski_2010} for a related $1$-categorical version. 

\begin{lemma}
    \label{ch2:lm:comod-over-separable-are-cofree}
    \index{Comonad!Cofree comodule}
    Every comodule over a coidempotent coalgebra $C$ is a retract of a cofree comodule. In particular, there is an equivalence 
    \[\ComodC(\C)\simeq \ComodC\fr(\C)\]
    between the Eilenberg--Moore category and the Kleisli category of the comonad $C\otimes (-)$ on $\C$. 
\end{lemma}
\begin{proof}
    As coidempotent coalgebras are separable, see \cref{ch2:rm:coidempotent-implies-separable}, the result will follow from the fact that the forgetful functor $U_C\: \Comod_C\to \C$ is separable, in the sense that the adjunction unit map 
    \[\Id_{\Comod_C}\to C\otimes U(-)\]
    has a $\C$-linear section, whenever $C$ is separable. The section is given by 
    \[C\otimes M \overset{\simeq}\to (C\otimes C)\otimes_C M \to C\otimes_C M\overset{\simeq}\longleftarrow M\]
    for any comodule $M$. 
\end{proof}

\begin{remark}
    \label{ch2:rm:unique-structure}
    The fact that $C$ is coidempotent implies that any $C$-comodule $M$ has a unique comodule structure. In particular, if $M$ is a comodule, then the cofree comodule $C\otimes M$ is equivalent to $M$. 
\end{remark}

We get a similar statement for contramodules over $C$. Recall that a contramodule is said to be free if it is of the form $\iHom(C,M)$ for some $M\in \C$. \index{Contramodule!Free}

\begin{proposition}
    \label{ch2:prop:contra-over-separable-are-free}
    Let $C\in \cCAlg(\C)$ be a separable coalgebra. Then every contramodule over $C$ is a retract of a free contramodule. In particular, there is an equivalence 
    \[\ContraC(\C)\simeq \ContraC\fr(\C)\]
    between the Eilenberg--Moore and Kleisli category of the monad $\iHom(C,-)$ on $\C$. 
\end{proposition}
\begin{proof}
    We can prove this by showing that the section for the separable coalgebra $C$ gives a section of the forgetful functor $U^C\:\Contra_C\to \C$. The section is, for a contramodule $X$, given by the adjoint map $M\to \iHom(C, M)$ to the section of the forgetful functor $U_C$ on $\Comod_C$ from \cref{ch2:lm:comod-over-separable-are-cofree}.  
\end{proof}


We know from \cref{ch2:lm:coidempotent-then-comod-monoidal} that the cofree comodule functor 
\[C\otimes(-)\: \C\to \Comod_C\]
can be given the structure of a symmetric monoidal functor when the coalgebra $C$ is coidempotent. Naturally we want a similar statement for the free contramodule functor 
\[\iHom(C,-)\: \C\to \Contra_C.\]

\begin{remark}
    \label{ch2:rm:contramodule-structure-on-hom-from-comodule}
    Let $M$ be a $C$-comodule and $V$ any object in $\C$. The structure map $\rho_M\: M\to C\otimes M$ induces a $C$-contramodule structure on the internal hom-object $\iHom(M, V)$, via 
    \[\iHom(C, \iHom(M, V))\simeq \iHom(C\otimes M, V)\overset{-\circ \rho_M}\to \iHom(M, V).\]
\end{remark}

\begin{lemma}
    \label{ch2:lm:free-contra-monoidal}
    \index{Presentable $\infty$-category!Symmetric monoidal}
    Let $C$ be a coidempotent cocommutative coalgebra in $\C$. The category of $C$-contramodules $\Contra_C$ inherits the structure of a presentably symmetric monoidal $\infty$-category, making the free contramodule functor a symmetric monoidal localization.
\end{lemma}
\begin{proof}
    The functor $\iHom(C,-)\: \C \to \C$ is an idempotent functor, as we have
    \[\iHom(C, \iHom(C,-))\simeq \iHom(C\otimes C, -)\simeq \iHom(C,-)\]
    by the internal adjunction property together with the coidempotency of $C$. The forgetful functor $U^C\: \Contra_C\to \C$ is a fully faithful functor, which means that the free contramodule functor $\iHom(C,-)\:\C\to \Contra_C$ is a localization. 
    
    In order to determine that it induces a symmetric monoidal structure on $\Contra_C$ we need to check that the free functor is compatible with the monoidal structure in $\C$. By \cite[2.12(3)]{nikolaus_2016} it is enough to check that $\iHom(V,X)\in \Contra_C$ for any $X\in \Contra_C$ and $V\in \C$. By \cref{ch2:prop:contra-over-separable-are-free} we can assume that $X \simeq \iHom(C, A)$ for some $A\in \C$. By the hom-tensor adjunction we get 
    \[\iHom(A, \iHom(C, V))\simeq \iHom(C\otimes V, A).\]
    The latter is a $C$-contramodule by \cref{ch2:rm:contramodule-structure-on-hom-from-comodule}, as $C\otimes V$ is a $C$-comodule. 

    We can then apply \cite[2.2.1.9]{lurie_09}, which tells us that the free contramodule functor $\iHom(C,-)\: \C\to \Contra_C$ can be given the structure of a symmetric monoidal functor. As $\Contra_C$ is a localization of a presentably symmetric monoidal category by an accessible functor, it is also presentably symmetric monoidal.  
\end{proof}



% We now need to compare the internal hom objects in $\C$ and $\Comod_C$. By the cofree-forgetful adjunction we have $\Hom_C(M, C\otimes A) \simeq \Hom(U_C M, A)$ for any $C$-comodule $M$ and object $A\in \C$. We wish for this to be enhanced to an internal object relationship. 

% \begin{lemma}
%     Let $M$ be a $C$-comodule and $A\in \C$. There is an equivalence
%     \[U_C \iHom_C(M, C\otimes A)\simeq \iHom(U_C M, A)\]
%     as objects in $\C$. 
% \end{lemma}

% We saw in \cref{ch2:rm:cofree-comodule-smashing-colocalization} that the cofree comodule functor was a smashing colocalization. We can naturally wonder whether the corresponding free contramodule functor has the same property. It turns out not to be a localization of $\C$, but not a smashing one.

% \begin{lemma}
%     \label{ch2:lm:free-contramodule-functor-localization}
%     The free contramodule functor $\iHom(C,-)\: \C\to \Contra_C$ is a localization. 
% \end{lemma}
% \begin{proof}
%     We prove the equivalent statement that the endofunctor $\iHom(C,-)\:\C\to \C$ is an exact idempotent functor. By the adjunction with $C\otimes -$ we have
%     \[\iHom(C, \iHom(C, -))\simeq \iHom(C\otimes C, -) \simeq \iHom(C,-),\]
%     where the last equivalence is due to the idempotency of $C$. The functor is exact as $\Contra_C$ is a stable category. 
% \end{proof}

% \begin{remark}
%     We will see in \cref{ch2:cor:free-contra-monoidal-localization} that it is in fact a symmetric monoidal localization. 
% \end{remark}

We can now deduce our main result, namely that Positselski duality is a symmetric monoidal equivalence for coidempotent coalgebras. 

\begin{theorem}
    \index{Positselski duality}
    \index{Contramodule!Free}
    \index{Comonad!Cofree comodule}
    \label{ch2:thm:Positselski-duality-coidempotent}
    Let $\C$ be a presentably symmetric monoidal category and $C\in \C$ a coidempotent cocommutative coalgebra. In this situation there are mutually inverse symmetric monoidal functors
    \begin{center}
        \begin{tikzcd}
            \ComodC(\C) \arrow[rr, yshift=2pt, "{\iHom(C, -)}"] && \ContraC(\C) \arrow[ll, yshift=-2pt, "C\otimes(-)"]
        \end{tikzcd}
    \end{center}
    given on the underlying objects by the free contramodule functor and the cofree comodule functor respectively. 
\end{theorem}
\begin{proof}
    By \cref{ch2:lm:comod-over-separable-are-cofree} and \cref{ch2:prop:contra-over-separable-are-free} every $C$-comodule is a retract of a cofree comodule, and every $C$-contramodule is a retract of a free contramodule. Hence, it is enough to prove that the functors are mutually inverse equivalences between cofree and free objects.  
    
    Let $A$ be any object in $\C$. Denote by $C\otimes A$ the corresponding cofree comodule and $\iHom(C, A)$ the corresponding free contramodule. A simple adjunction argument, using both the cofree-forgetful adjunction and the hom-tensor adjunctions in $\C$ and $\Comod_C$, shows that there is an equivalence 
    \[\iHom_C(M, C\otimes A)\simeq C\otimes \iHom(U_C M, A)\]
    for any comodule $M$. In other words, the internal comodule hom is determined by the underlying internal hom in $\C$. For $M = C$ we get
    \[C\otimes \iHom(C, A)\simeq \iHom_C(C, C\otimes A)\]
    which is equivalent to $C\otimes A$ as $C$ is the unit in $\Comod_C$. 

    We wish to show that $\iHom(C, C\otimes A) \simeq \iHom(C, A)$. We do this by showing that the cofree-forgetful functor is an internal adjunction, in the sense of \cref{ch2:rm:internal-adjunction}. 

    Let $B$ be an arbitrary object in $\C$, and recall our notation $\Hom(-,-)$ for the mapping space in $\C$. By the hom-tensor adjunction in $\C$ we have 
    \[\Hom(B, \iHom(C, C\otimes A)) \simeq \Hom(C\otimes B, C\otimes A).\]
    Both of these are in the image of the forgetful functor 
    \[U_C\:\Comod_C\to \C.\]
    As $U_C$ is fully faithful whenever $C$ is coidempotent, we get 
    \[\Hom(C\otimes B, C\otimes A) \simeq \Hom_C(C\otimes B, C\otimes A),\]
    where we recall that the latter denotes maps of comodules. By the cofree-forgetful adjunction we have 
    \[\Hom_C(C\otimes B, C\otimes A) \simeq \Hom(C\otimes B, A),\]
    which by the hom-tensor adjunction in $\C$ finally gives 
    \[\Hom(C\otimes B, A) \simeq \Hom(B, \iHom(C, C\otimes A)).\]
    Summarizing the equivalences we have 
    \[\Hom(B, \iHom(C, C\otimes A))\simeq \Hom(B, \iHom(C, A)),\] 
    which by a Yoneda argument implies that there is an equivalence of internal hom-objects $\iHom(C, C\otimes A)\simeq \iHom(C, A)$. 

    % Since comodule structures are unique in $\Comod_C$, see \cref{ch2:rm:unique-structure}, there is an equivalence $C\otimes U_C M \simeq M$ for any comodule $M$. This means that we can assume our object $A\in \C$ to be in the image of the forgetful functor. There is a comodule structure on $\iHom(U_C M, C\otimes A)$, given by the equivalence  
    % \[ C\otimes \iHom(U_C M, C\otimes A) \iHom_C(M, C\otimes C\otimes A) \simeq \iHom(M, C\otimes A)\]
    % and the comodule structure on $\iHom(M, C\otimes A)$. By uniqueness, this implies that $\iHom(M, C\otimes A) \simeq \iHom_C(M, C\otimes A)$ for any comodule $M$. Which by the earlier uniqueness remark implies
    % \[\iHom(C, C\otimes A) \simeq \iHom_C(C, C\otimes A)\simeq C\otimes \iHom(C, A)\simeq \iHom(C, A),\]
    % showing that the functors form a mutually inverse pair of equivalences. 

    % Since $\Contra_C$ is equivalent to a symmetric monoidal category, it can itself be made into a symmetric monoidal category. We can do this by declaring that the contramodule tensor product $X\otimes^C Y$ of two contramodules $X$ and $Y$, is 
    % \[X\otimes^C Y := \iHom(C, (C\otimes X)\otimes_C (C\otimes Y)).\]
    % By definition this makes $\iHom(C, -) \: \Comod_C \to \Contra_C$ a symmetric monoidal functor. The functor $C\otimes - \: \Contra_C \to \Comod_C$ is symmetric monoidal because $C\otimes (-)$ is a smashing colocalization. 

    We know by \cref{ch2:lm:coidempotent-then-comod-monoidal} and \cref{ch2:lm:free-contra-monoidal} that the cofree comodule functor and the free contramodule functor are both symmetric monoidal. By the arguments above, we know that the equivalence $\Comod_C\simeq \Contra_C$ is given by the compositions 
    \begin{center}
        \begin{tikzcd}
            \Comod_C 
            \arrow[rr, yshift = 2pt, "U_C"] 
            && 
            \C 
            \arrow[ll, yshift = -2pt, "C\otimes -"] 
            \arrow[rr, yshift = 2pt, "{\iHom(C, -)}"] 
            && 
            \Contra_C 
            \arrow[ll, yshift = -2pt, "U^C"]
        \end{tikzcd}
    \end{center}
    The composition from left to right is an op-lax symmetric monoidal functor, and the composition from right to left is a lax symmetric monoidal functor. Since they are both equivalences they are necessarily also symmetric monoidal. 
\end{proof}

\begin{remark}
    \label{ch2:rm:holds-generally-for-separable}
    We do believe that the above result to hold more generally. In fact, we believe it should hold for all separable cocommutative coalgebras, as this holds in the $1$-categorical situation. However, it will in general not be a monoidal equivalence, due to the lack of monoidal structures. 
\end{remark}


% \begin{corollary}
%     \label{ch2:cor:free-contra-monoidal-localization}
%     Equipped with the symmetric monoidal structure from \cref{ch2:thm:Positselski-duality-coidempotent} the free contramodule functor $\iHom(C, -)\: \C\to \Contra_C$ is symmetric monoidal. 
% \end{corollary}
% \begin{proof}
%     By definition we have
%     \[\iHom(C, A)\otimes^C \iHom(C, B) := \iHom(C, (C\otimes \iHom(C, A))\otimes_C (C\otimes \iHom(C, B))),\]
%     which by the inverse equivalence of $C\otimes -$ and $\iHom(C, -)$ is equivalent to 
%     \[\iHom(C, (C\otimes \iHom(C, A))\otimes_C (C\otimes \iHom(C, B)) \simeq \iHom(C, (C\otimes A)\otimes_C (C\otimes B)).\]
%     By construction of the monoidal structure on $\Comod_C$ we have $(C\otimes A)\otimes_C (C\otimes B)\simeq C\otimes (A\otimes B)$, hence we have
%     \[\iHom(C, (C\otimes A) \otimes_C (C\otimes B))\simeq \iHom(C, C\otimes (A \otimes B))\simeq \iHom(C, A\otimes B),\]
%     where the last equivalence follows from the proof of \cref{ch2:thm:Positselski-duality-coidempotent}. 
% \end{proof}

% \begin{remark}
%     As the functor $\iHom(C, -)\:\C\to \Contra_C$ is an coidempotent functor, via the forgetful functor $U^C\:\Contra_C\to \C$, the category $\Contra_C$ is in fact a localization of $\C$, as the forgetful functor $U_C\: \Comod_C\to \C$ is fully faithful when $C$ is coidempotent. This means that the monoidal structure on $\Contra_C$ is the one induced from $\C$ through the localization, as expected. 
% \end{remark}