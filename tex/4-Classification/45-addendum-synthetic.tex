
\section{Addendum: Subcategories of synthetic spectra}
\label{ch3:addendum}

One of the main motivations for \cref{ch3:thm:main} was to understand localizing subcategories of Pstr\a{}gowski's category of $E$-based synthetic spectra --- the main reference is \cite{pstragowski_2022}. This section is not part of the paper \cite{aambo_2024_localizing}, but is added to flesh out this example further, and to relate this paper to \cref{ch1}. 

We will focus only on the case of synthetic spectra based on height $n$ Morava $E$ theory in this section, even though synthetic spectra, and all of the results we cover will also work over many other Adams type ring spectra. We will also not use the standard category of syntehtic spectra, but instead a ``local'' variant, that for familiar readers lie somewhere between $E_n$-based syntehtic spectra and its hypercompletion. 

\begin{definition}
    A finite pectrum $P\in \Sp^\omega$ is said to \emph{$E_n$-finite projective} if its $E_n$-homology is finitely generated and projective as an $E_{n*}$-module. The full subcategory of $E_n$-finite projectives is denoted $\Sp\fp$. 
\end{definition}

\begin{remark}
    By \cref{ch0:rm:dualizable/compact-comodules}, a finite spectrum $P$ is $E_n$-finite projective if and only if $E_{n*}P$ is a dualizable comodule. 
\end{remark}

We can equip the category $\Sp\fp$ with a Grothendieck topology, defined by covers being single maps $P\to P'$ such that the induced map on $E_n$-homology is an epimorphism. This makes $\Sp\fp$ an excellent $\infty$-site --- see \cite[Section 2.3]{pstragowski_2022} for details. 

\begin{definition}
    A hypercomplete $E_n$-based synthetic spectrum is hypercomplete additive sheaf $X\: \Sp^{\mathrm{fp}, \mathrm{op}}\to \Sp$. The category of hypercomplete $E_n$-based synthetic spectra will be denoted $\hSynE := \mathrm{P}_\Sigma(\Sp\fp;\Sp)$. 
\end{definition}

\begin{remark}
    This is not the standard notation for the category of synthetic spectra, as the subscript is usually the base ring $E$. But, as we only use synthetic spectra over $E_n$, and we thought $\hSynE$ to be a good shorthand notation for $\Syn_{E_n}$.  
\end{remark}

We will not go through the theory of synthetic spectra in detail here, but recall some fundamentals briefly. There is by \citeme a lax monoidal fully faithful functor $\nu\:\Sp_n \to \hSynE$ called the synthetic analog. The category $\hSynE$ is compactly generated by the objects $\nu L_n P$ for $P\in \Sp\fp$, which are also dualizable. The unit is also compact, which implies that $\hSynE$ is rigidly compactly generated. 

The category of hypercomplete synthetic spectra has a natural right complete $t$-structure that is compatible with filtered colimits. The heart of this $t$-structure is the category of comodules over $E_{n*}E_n$, which as before we denote by $\ComodE$. 

The functor $\nu$ induces a deformation parameter $\tau$ on any synthetic spectrum $X$, see \citeme, making $\hSynE$ act as a one-parameter deformation between $\Sp_n$ and $\Der(\ComodE)$, as in the following result. 

\begin{theorem}
    \label{ch3:add:thm:deformation-properties-of-syn}
    Inverting the deformation parameter $\tau$ gives an equivalence 
    \[\hSynE[\tau^{-1}]\simeq \Sp_n\]
    of symmetric monoidal stable $\infty$-categories. Furthermore, killing $\tau$, via tensoring with its cofiber, gives an equivalence 
    \[\Mod_{C\tau}(\hSynE)\simeq \Der(\ComodE)\]
    of symmetric monoidal stable $\infty$-categories. 
\end{theorem}

In a certain sense, this makes $\hSynE$ a categorification of the adapted homology theory $E_{n*}\: \Sp_n\to \ComodE$. The main result for this addendum is to construct a categorification of the restricted adapted homology theory $E_{n*}\: \Mn \to \ComodE^{I_n-tors}$ that we studied in \cref{ch1}, with associated natural deformation properties as above. 

% Given any $E_n$-finite projective $P$ we can localize it at $E_n$ to obtain a compact $E_n$-local spectrum $L_n P$. As $E_n$-homology is invariant under $E_n$-localization, this still has finitely generated and projective $E_n$-homology. We will denote by $\Sp_n\fp$ the full subcategory of $\Sp_n^\omega$ consisting of $E_n$-localizations of $E_n$-finite projectives. This still has a Grothendieck topology given by single $E_n$-epimorphisms, which, as $E_n$-localization is smashing, still is an excellent $\infty$-site. 

% \begin{definition}
%     An $E_n$-local synthetic spectrum is an additive sheaf $X\: \Sp_n^{\mathrm{fp}, \mathrm{op}}\to \Sp$. The category of $E_n$-local synthetic spectra will be denoted $\hSynE := \mathrm{P}_\Sigma(\Sp_n\fp;\Sp)$. 
% \end{definition}

% All of the natural constructions for $\SynE$ also go through for $\hSynE$. We will not go through these here, but refer to \cite{pstragowski_2022}. We do, however, summarize the main properties that we will use for th rest of this addendum. 







\subsection{Localizing subcategories in comodules}

One of the reason we chose to work with $E_n$ spesifically, rather than more general Adams type ring spectra, is that we have a very good understanding of localizing subcategories of the heart of the natural $t$-structure on $\SynE$. In fact, using the partial classification of localizing subcategories in $\Comod_{BP_*BP}$, due to Hovey--Strickland in \cite{hovey-strickland_2005a}, Barthel and Heard was able to classify localizing subcategories in $\ComodE$ completely. 

\begin{proposition}[{\cite[2.17]{barthel-heard_2018}}]
    Let $\T$ be a localizing subcategory in a Grothendieck abelian category $\A$, and $\Psi\:\A\to \A/\T$ the associated Gabriel quotient. If $S$ is a localizing subcategory in $\A/\T$, then there is a localizing subcategory $\overline{S}$ in $\A$ containing $\T$ such that $\Psi(\overline{S})=S$. 
\end{proposition}

The above result is precisely what gives the classification of localizing subcategories of $\ComodE$. Recall that for any $0\leq k\leq n$ we have an ideal $I_k \subseteq \pi_* E_n$, called the Landweber ideals of $E$. More precisely these are given by $I_k = (p, v_1, v_2, \ldots, v_{k-1})$. These ideals are finitely generated regular invariant ideals, hence \cref{ch0:sssec:torsion-and-completion-for-comodules} gives us for any such $k$ a localizing subcategory $\ComodE^{I_k-tors}\subseteq \ComodE$, called the category of $I_k$-torsion comodules. 

\begin{theorem}[{\cite[2.21]{barthel-heard_2018}}]
    \label{ch3:add:thm:classification-of-abelian-localizing}
    If $\T$ is a localizing subcategory in $\ComodE$, then there is an integer $0\leq k\leq n$ such that $\T \simeq \ComodE^{I_k-tors}$. 
\end{theorem}

\begin{remark}
    By the above result we do, in fact, get a chain of localizing subcategories
    \[\T_0 \subseteq \T_1\subseteq \cdots \subseteq \T_n \]
    in $\ComodE$, corresponding each to one of the generators in the Landweber ideal $I_n=(p, v_1, v_2, \ldots, v_{n-1})\subseteq \pi_* E_n$. Hence this result also classifies the localizing subcategories in the torsion categories $\ComodE^{I_k-tors}$ themselves.  
\end{remark}


\subsection{Monochromatic synthetic spectra}

For simplicity we focus here on the maximal localizing subcategory, $\Comodt$. Via the homology theory $E_*$, which the heart-valued homotopy groups in $\hSynE$ is supposed to generalize, we know that $I_n$-torsion comodules correspond to monochromatic spectra, as introduced in \cref{ch0:sssec:monochromatic-duality}. Hence, we make the following definition. 

\begin{definition}
    The unique $\pi$-exact lift of the localizing subcategory $\Comodt$ is denoted $\Mn\hSynE$. We call it the category of height $n$ \emph{monochromatic synthetic spectra}. 
\end{definition}

The justification for this name also in syntehtic spectra is due to the following result. 

\begin{lemma}
    \label{ch3:add:lm:mono-iff-syn-mono}
    For a spectrum $X\in \Spn$ we have $\nu X \in \Mn\hSynE$ if and only if $X\in \Mn$. 
\end{lemma}
\begin{proof}
    By definition we have $\nu X \in \Mn\hSynE$ if and only if $\pi_*^\heart \nu X \in \Comodt$. By \cite[4.21, 4.22]{pstragowski_2022} there is an isomorphism of $E_*E$-comodules $\pi_k^\heart \nu X \simeq E_{n*} X[-k]$, meaning that the $E_{n*}$-homology of $X$ is $I_n$-torsion. By \cref{ch1:lm:monochromatic-iff-torsion-comodules} this is the case if and only if $X\in \Mn$, finishing the proof. 
\end{proof}










\subsection{Compact generation}

By our result \cref{ch1:lm:torsion-comodules-generated-by-compacts} from \cref{ch1} we know that the category $\Comodt$ is compactly generated, with an explicit set of compact generators given by 
\[\Tors_n\fp := \{G\otimes E_*/I_n^k \mid G\in \ComodE\fp, k\geq 1\}.\]
For the moment we want to avoid any near-lying telescopes in $E_n$-based synthetic spectra, and and prove that the lift of $\Comodt$, i.e., $\Mn\hSynE$, is also a compactly generated localizing subcategory. This will require a more refined analysis compared to just lifting the localizing subcategories via \cref{ch3:thm:main}. 

Let us start with the prestable situation. The functor $\pi_0 \: \Syn_{n, \geq 0} \to \ComodE$ preserves compact objects, so instead of pulling back to the whole category $\Syn_{n, \geq 0}$ we can instead pull back to compact objects. 

\begin{definition}
    A full subcategory $\T\geqz \subseteq \Syn_{n, \geq 0}^\omega$ is \emph{thick} if it is closed under finite coproducts, cofiber sequences and subobjects. 
\end{definition}

\begin{remark}
    This is exactly the definition of a localizing subcategory of a prestable $\infty$-category, just with finite coproducts rather than all coproducts. This distinction is then similar to the distinction between localizing and Serre subcategories of abelian categories, and localizing vs. thick subcategories of stable $\infty$-categories. 
\end{remark}

We can further sharpen the analogy between Serre subcategories and thick subcategories. 

\begin{lemma}
    If $\T^\heart$ is a Serre subcategory of $\ComodE^\omega$, then the full subcategory $\T\geqz\subseteq \Syn_{n, \geq 0}^\omega$ such that $t\in \T\geqz$ if and only if $\pi^\heart_k t \in \T^\heart$ for all $k\geq 0$, is a thick subcategory of $\Syn_{n, \geq 0}^\omega$. 
\end{lemma}
\begin{proof}
    The proof is identical to \cite[C.5.2.7]{lurie_SAG}, just with finite coproducts rather than all coproducts. 
\end{proof}

After lifting to the prestable category, the next step --- as before --- is to stabilize. In order to do this in the case where we only have compact objects, we utilize another ``small'' stabilization instead of using $\Sp(-)$. 

\begin{definition}
    Let $\mathcal{E}$ be a pointed category with finite limits. The \emph{Spanier--Whitehead category} of $\mathcal{E}$ is defined to be the colimit of the diagram 
    \begin{center}
        \begin{tikzcd}
            \mathcal{E} \arrow[r, "\Sigma"] & \mathcal{E} \arrow[r, "\Sigma"] & \mathcal{E} \arrow[r, "\Sigma"] & \cdots
        \end{tikzcd}
    \end{center}
    where $\Sigma$ is the functor given by the cofiber of the map $0\rightarrow x$ for $x\in \mathcal{E}$. 
\end{definition}

The first thing we need is to compare prestable and stable compact objects. 

\begin{theorem}
    \label{ch3:add:thm:prestable-freyd-stabilizes-to-stable-Freyd}
    The is an equivalence $\SW(\Syn_{n, \geq 0}^\omega)\simeq \Syn_{n}^\omega$ of symmetric monoidal stable $\infty$-categories.  
\end{theorem}
\begin{proof}
    The category $\Syn_{n, \geq 0}^\omega$ is a prestable category closed under finite limits, hence it is the connected part of a $t$-structure on some stable $\infty$-category, which is precicely the Spanier--Whitehead category $\SW(\Syn_{n, \geq 0}^\omega)$, see \cite[C.1.1, C.1.2]{lurie_SAG}. 

    By \cite[C.1.1.6]{lurie_SAG} there is a commutative diagram of $\infty$-categories
    \begin{center}
        \begin{tikzcd}
            \Cat^{\mathrm{rex}}_\infty \arrow[r, "\SW(-)"] \arrow[d, "\Ind(-)"'] & \Cat^{\mathrm{rex}}_\infty \arrow[d, "\Ind(-)"] \\
            \PrL \arrow[r, "\Sp(-)"']                                                    & \PrL                                           
        \end{tikzcd}
    \end{center}
    meaning that there is an equivalence
    \[\Ind(\SW(\Syn_{n, \geq 0}^\omega))\simeq \Sp(\Ind(\Syn_{n, \geq 0}^\omega)).\]
    As all functors are symmetric monoidal, the equivalence is also symmetric monoidal. The category $\Ind(\Syn_{n, \geq 0}^\omega)$ is $\Syn_{n, \geq 0}$ --- as it is compactly generated --- which we know stabilizes to $\Syn_{n}$. This category we know has a collection of compact generators, $\Syn_{n}^\omega$, which is a small stable $\infty$-category, giving an equivalence $\Ind(\Syn_{n}^\omega)\simeq \Syn_{n}$ by definition. As the functor $\Ind$ is an equivalence between small stable $\infty$-categories and compactly generated $\infty$-categories, we get our wanted equivalence $\SW(\Syn_{n, \geq 0}^\omega)\simeq \Syn_{n}^\omega$. 
\end{proof}

This now allows us to finally define the lift of a Serre subcategory to the stable $\infty$-world. 

\begin{definition}
    Given a Serre subcategory $\T^\heart$, we define its \emph{stable lift} $\T$ to be the Spanier--Whitehead category of its prestable lift $\T:= \SW(\T\geqz)$. 
\end{definition}

\begin{remark}
    Intuitively one should think about this as the ``small'' version of the construction from \cref{ch3}, where one lifts an abelian localizing subcategory through the $t$-structure by first lifting to the prestable category and then stabilizing. The Spanier--Whitehead construction is the natural verison of stabilization for small categories, as is made clear by the commutative diagram in the above proof. 
\end{remark}

We have now defined our lift, and it remains to prove that it has the expected properties: it should in particular be a thick subcategory --- in the stable sense. 

\begin{lemma}
    \label{ch3:add:lm:stable-lift-is-thick}
    Let $\T^\heart\subseteq \ComodE^\omega$ be a Serre subcategory. The stable lift $\T$ is a thick subcategory of $\hSynE^\omega$. 
\end{lemma}
\begin{proof}
    We have a fully faithful inclusion $\T\geqz\hookrightarrow\Syn_{n, \geq 0}^\omega$, which gives a fully faithful inclusion 
    \[\T = \SW(\T\geqz)\hookrightarrow \SW(\Syn_{n, \geq 0}^\omega)\simeq \Syn_{n}^\omega\]
    by \cref{ch3:add:thm:prestable-freyd-stabilizes-to-stable-Freyd}. As $\T$ is a stable $\infty$-category by definiton, we need only to check that it is closed under finite colimits in $\Syn_{n}^\omega$. 
    
    Given a finite colimit in $\T$, it factors through $\T\geqz$ at some finite stage in the diagram 
    \begin{center}
        \begin{tikzcd}
            \T\geqz \arrow[r, "\Sigma"] & \T\geqz \arrow[r, "\Sigma"] & \T\geqz \arrow[r, "\Sigma"] & \cdots
        \end{tikzcd}
    \end{center}
    As $\T\geqz$ is closed under finite colimits in $\Syn_{n, \geq 0}^\omega$, together with the fact that
    \begin{center}
        \begin{tikzcd}
            \T\geqz \arrow[r, "\Sigma"] \arrow[d] & \T\geqz \arrow[d] \\
            \Syn_{n, \geq 0}^\omega \arrow[r, "\Sigma"']                  & \Syn_{n, \geq 0}^\omega                  
        \end{tikzcd}
    \end{center}
    commutes, and lastly that all of the maps $\T\geqz\to \SW(\T\geqz)\simeq \T$ and 
    \[\Syn_{n, \geq 0}^\omega \to \SW(\Syn_{n, \geq 0}^\omega)\simeq \Syn_{n}^\omega\] 
    preserve finite colimits --- see \cite[C.1.1.5]{lurie_SAG} --- this implies that also the fully faithful inclusion $\T \subseteq \Syn_{n}^\omega$ preserves finite colimits, finishing the proof. 
\end{proof}

We know that there is a unique $\pi$-exact lift of $\Loc(B)$ via \cref{ch3:thm:main}, which wenote by $\L$. We now prove that this lift $\L$ is in fact uniquely determined by $B$. 

\begin{lemma}
    \label{ch3:add:lm:lift-of-B-works-with-localizing-lift}
    For any Serre subcategory $\T^\heart$, we have $\L\cap \Syn_{n}^\omega = \T$. 
\end{lemma}
\begin{proof}
    As $\T^\heart \subseteq \Loc(\T^\heart)$ we also have $\T \subseteq \L$ by \cref{ch3:lm:pi-stable-are-the-biggest}, as the latter is $\pi$-stable. This gives the first of the inclusions:
    \[\T = \T \cap \Syn_{n}^\omega\subseteq \L \cap \Syn_{n}^\omega.\]
    
    Let $l$ be an object in $(\L\cap \Syn_{n}^\omega)\geqz$. This means that $l\in \L\geqz$ and $l \in \Syn_{n, \geq 0}^\omega$. Hence, $\pi_k l \in \Loc(\T^\heart)\cap \ComodE^\omega \simeq \T^\heart$ for all $k\geq 0$, which by definition implies $l \in \Syn_{n, \geq 0}^\omega$, giving 
    \[(\L\cap \Syn_{n}^\omega)\geqz \subseteq \T\geqz.\]
    This gives the other inclusion upon taking Spanier--Whitehead categories. 
\end{proof}


For the uniqueness of $\T$ we will need the following lemma, stating that the lift of a compactly generated abelian localizing subcategory is a compactly generated stable localizing subcategory. 

\begin{lemma}
    \label{ch3:add:lm:loc-of-lift-is-localizing-lift}
    There is an equivalence of localizing subcategories $\Loc(\T) = \L$. 
\end{lemma}
\begin{proof}
    By \cref{ch3:add:lm:lift-of-B-works-with-localizing-lift} these have the same compact objects, hence $\Loc(\T)\subseteq \L$. If we can prove that $\Loc(\T)$ is a $\pi$-stable localizing subcategory with heart $\Loc(\T^\heart)$, then we are done by the uniqueness of the lift $\L$. 

    Now, $\T$ is $\pi$-stable, hence we have $\pi_k t \in \Loc(\T^\heart)$ if and only if $t\in \Loc(\T)$ for all compact $t$. We know that $\Loc(\T)$ is generated by $\T$ under filtered colimits, which means, as $\pi_0$ preserves filtered colimits and $\Loc(\T^\heart)$ is closed under these, that also $\pi_k t\in \Loc(\T)$ if and only if $t\in \Loc(\T)$ for all (not neccessarily compact) objects $t$. It also follows from this that $\Loc(\T)^\heart = \Loc(\T^\heart)$, hence we get $\Loc(\T) = \L$ by uniqueness of the lift. 
\end{proof}

We can now finally prove that the lift $\T$ is unique. 

\begin{theorem}
    \label{ch3:add:thm:uniqueness-of-lift}
    Given a Serre subcategory $\T\subseteq \ComodE^\omega$, the lift $\T$ is unique. 
\end{theorem}
\begin{proof}
    Let $\T'$ be another stable lift of $\T^\heart$, in other words it is a $\pi$-stable thick subcategory with heart $\T^\heart$. By the same arguments as in \cref{ch3:add:lm:loc-of-lift-is-localizing-lift}, we get two $\pi$-stable localizing subcategories $\Loc(\T)$ and $\Loc(\T')$, which neccessarily must have the same heart $\Loc(\T^\heart)$. By uniqueness of the lift $\L$ we must then have $\Loc(\T) = \L = \Loc(\T')$. By \cref{ch3:add:lm:lift-of-B-works-with-localizing-lift} we conclude that 
    \[\T = \Loc(\T)\cap \Syn_{n}^\omega = \Loc(\T')\cap \Syn_{n}^\omega = \T',\]
    finishing the proof. 
\end{proof}

As a consequence we get that the category of monochromatic synthetic spectra $\Mn\hSynE$ is compactly generated, as it is equivalent to the category $\Loc(\T)$ associated to the stable lift $\T$ of the Serre subcategory of compact objects in $\Comodt$. 

\begin{corollary}
    The category of monochromatic synthetic spectra $\Mn\hSynE$ is compactly generated. 
\end{corollary}

We can also show that $\Mn\hSynE$ is a $\otimes$-ideal, which will mean that we can apply the local duality results from \cref{ch1:app:barr-beck} to compute the deformation properties of $\Mn\hSynE$. To spoil the result, it will be precicely what it should. 

First note that the maximal localizing subcategory $\Comodt$ is in fact a $\otimes$-ideal. Furthermore, the full subcategory of compact objects $\ComodE^{\omega, I_n-tors}$ is a maximal Serre $\otimes$-ideal of $\ComodE^\omega$. We will use this to our advantage. 

\begin{lemma}
    The unique separating prestable thick subcategory $\T\geqz$, lifting $\ComodE^{\omega, I_n-tors}$, is a prestable maximal thick $\otimes$-ideal. 
\end{lemma}
\begin{proof}
    The maximality of $\T\geqz$ is immediate from the fact that it is separating, and that $\T^\heart \simeq \ComodE^{\omega, I_n-tors}$ is maximal. 

    Assume there are objects $t\in \T\geqz$ and $x \in \Syn_{n, \geq 0}^\omega$ such that $t \otimes x \not\in \T\geqz$. As $\T\geqz$ is maximal this means that the subset $\T\geqz \cup \{t \otimes x\}$ generates $\Syn_{n, \geq 0}^\omega$ under finite colimits, i.e., 
    \[M:= \mathrm{Thick}(\T\geqz \cup \{t \otimes x\}) \simeq \Syn_{n, \geq 0}^\omega.\]
    But, the discrete objects in $M$ can only generate $\ComodE^{\omega, I_n-tors}$, as it is a $\otimes$-ideal, and $\pi_0$ is symmetric monoidal by \cite[A.12]{antieau_nikolaus_2020}, leading to a contradiction. Hence, no such two objects can exist, finishing the proof. 
\end{proof}

We can also quite easily see that this implies that the Spanier--Whitehead category $\T := \SW(\T\geqz)$ is a thick $\otimes$-ideal of $\hSynE^\omega$. We can do this by categorifying the property of being a $\otimes$-ideal. 

\begin{lemma}
    \label{ch3:add:lm:categorical-ideal-property}
    Let $\E$ be a small symmetric monoidal $\infty$-category, and $\L\subseteq \E$ a presentable full subcategory. Precomposing with the inclusion we get a functor 
    \[\mu\: \L\times \E \to \E\times\E \to \E.\]
    The subcategory $\L$ is an $\otimes$-ideal if and only if $\mu$ factors through the inclusion $\L\hookrightarrow \E$. 
\end{lemma}
\begin{proof}
    This it literally just a reformulation of the definition of an ideal, as the functor $\mu\: \L\times \E \to \E$ sends $(L,E) \mapsto L\otimes E$. 
\end{proof}

This will imply the ideal property for the Spanier--Whitehead category. 

\begin{lemma}
    \label{ch3:add:lm:stabilizing-ideal-gives-ideal}
    If $\T\geqz$ is a thick $\otimes$-ideal in $\Syn_{n, \geq 0}^\omega$, then its Spanier--Whitehead stabilization $\T$ is a thick $\otimes$-ideal of $\hSynE^\omega$. 
\end{lemma}
\begin{proof}
    We have already proven the thick subcategory property, hence it remains only to show that it is an ideal. The Spanier--Whitehead functor $\SW(-)\: \Cat_\infty^{\mathrm{rex}}\to \Cat_\infty^{\mathrm{rex}}$ is symmetric monoidal with respect to the cartesian product, hence it sends the composite 
    \[\T\geqz\times \Syn_{n, \geq 0}^\omega \to \Syn_{n, \geq 0}^\omega\times \Syn_{n, \geq 0}^\omega \to \Syn_{n, \geq 0}^\omega\]
    to 
    \[\T\times \Syn_{n}^\omega \to \Syn_{n}^\omega\times \Syn_{n}^\omega\to \Syn_{n}^\omega.\]
    As the symmetric monoidal structures on $\Syn_{n, \geq 0}^\omega$ and $\T\geqz$ was induced from $\Syn_{n}^\omega$, we do recover the original symmetric monoidal structure in this process. The former functor factors through $\T\geqz$ by \cref{ch3:add:lm:categorical-ideal-property}, hence the latter factors through $\T$. This means that it is an ideal, again by \cref{ch3:add:lm:categorical-ideal-property}. 
\end{proof}

\begin{remark}
    This means that there is a collection of compact objects $\K \subseteq \Mn\hSynE$ such that $\Mn\hSynE \simeq \Loc^\otimes (\K)$. By \cref{ch0:thm:local-duality} this means that the associated colocalization functor
    \[\Gamma_n\: \hSynE \to \Mn\hSynE\]
    is a smashing colocalization. 
\end{remark}

We now wish to find a good description of these compact generators. The category of monochromatic spectra $\Mn$ is compactly generated by the $E_n$-localization of any finite type $n$ spectrum $F(n)$, see \cref{ch0:def:type-n-spectrum} and \cref{ch0:prop:torsion-is-monochromatic}. One natural guess for the compact generators of monochromatic synthetic spectra could then be to lift these to the synthetic setting. 

\begin{construction}
    By \cite[4.23]{pstragowski_2022} we can lift the fiber sequence $\S\overset{p}\to\S\to\S/p$ to a fiber sequence $\nu \S \overset{\widetilde{p}}\to \nu \S\to \nu (\S/p)$ in synthetic spectra $\hSynE$, as it induces a short exact sequence 
    \[0\to E_{n*} \overset{\cdot p}\to E_{n*}\to E_{n*}/p\to 0\]
    on $E_{n*}$-homology. In particular, $\nu (\S/p)\simeq (\nu \S)/\widetilde{p}$. Similar ideas were used by Burklund to prove the existence of $\E_1$ structures on Moore spectra in \cite{burklund_2022}. 
    
    Now, as $\nu (\S/p)$ is a finite number of cones away from the synthetic sphere, it is a compact object in $\hSynE$. We can iterate this construction to lift generalized Moore spectra into the synthetic setting. These are then compact synthetic objects that behave similarily to finite type $n$ spectra. 
\end{construction}

\begin{lemma}
    There is a finite type $n$ spectrum $F(n)$, whose synthetic analog is compact. 
\end{lemma}
\begin{proof}
    Let $I_n$ be the Landweber ideal $(p,v_1, v_2, \ldots, v_{n-1})$. By \cite[4.14]{hovey-strickland_99} there is a finite type $n$ generalized Moore spectrum $\S/J$ for $J=(p^{i_0}, v_1^{i_1}, \ldots, v_{n-1}^{i_{n-1}})$ constructed by iterated fiber sequences. These fiber sequences all induce short exact sequences on $E_{n*}$-homology, hence we can lift them to fiber sequences in synthetic $\hSynE$ by \cite[4.23]{pstragowski_2022}. In particular we have $\nu (\S/J) \simeq (\nu \S)/\widetilde{J}$ for $\widetilde{J} = (\widetilde{p}^{i_0}, \widetilde{v}_1^{i_1}, \ldots, \widetilde{v}_{n-1}^{i_{n-1}})$. Since we used a finite number of shifts and fiber sequences, $\nu (\S/J)$ is a compact object in $\hSynE$. 
\end{proof}

\begin{definition}
    A synthetic spectrum $X$ is said to be of \emph{synthetic type} $n$, if it is compact, and $X\simeq \nu F(n)$ for a finite type $n$ spectrum $F(n)$. 
\end{definition}

Our goal is to show that such a synthetic type $n$ spectrum $\nu F(n)$ does indeed generate $\Mn \hSynE$.

\begin{lemma}
    \label{ch3:add:monochromatic-synthetic-is-gen-by-type-n}
    There is an equivalence $\Loc^\otimes(\nu L_n F(n))\simeq \Mn\hSynE$ of symmetric monoidal stable $\infty$-categories. 
\end{lemma}
\begin{proof}
    As the spectrum $L_n F(n)$ is monochromatic, we have by \cref{ch3:add:lm:mono-iff-syn-mono} that $\nu L_n F(n)$ lies in $\Mn\hSynE$. In particular we have 
    \[\Loc^{\otimes}(\nu L_n F(n))\subseteq \Mn\hSynE.\]
    As both categories are compactly generated $\otimes$-ideals, they are equivalent to the localizing $\otimes$-ideal generated by their respective units. As the unit in $\Loc^\otimes(\nu L_n F(n))$ is a monochromatic syntehtic spectrum, it has to be equivalent to the unit $\Gamma_n \nu \S$ in $\Mn\hSynE$, as units are unique, and the inclusion
    \[\Loc^{\otimes}(\nu L_n F(n))\subseteq \Mn\hSynE\] shows that the monoidal tructures are the same. As they are both the localizing ideal generated by the same object, they are equivalent. 
\end{proof}








\subsection{Deformation properties}

We can now round off this addendum by showing that $\Mn\hSynE$ has the desired deformation properties. We start with the generic fiber. 

\begin{lemma}
    \label{ch3:add:lm:monochromatic-synthetic-special-fiber}
    There is an equivalence
    \[\Mod_{C\tau}(\Mn\hSynE) \simeq \Der(\Comodt)\]
    of symmetric monoidal stable $\infty$-categories. 
\end{lemma}
\begin{proof}
    By \cref{ch3:add:thm:deformation-properties-of-syn} there is a monoidal Barr--Beck adjunction 
    \[\hSynE \rightleftarrows \Der(\ComodE).\]
    As $\nu X \otimes C\tau \simeq E_{n*} X$, we get a local duality adjunction 
    \[(\hSynE, \nu L_n F(n))\rightleftarrows (\Der(\ComodE), E_{n_*}F(n)).\]
    By \cref{ch1:thm:modular-bb-torsion} there is an induced monoidal Barr--Beck adjunction 
    \[\Loc^\otimes(\nu L_n F(n))\rightleftarrows \Loc^\otimes(E_{n*}F(n)).\]
    The left hand side is equivalent to $\Mn\hSynE$ by \cref{ch3:add:monochromatic-synthetic-is-gen-by-type-n}, so we need only to identify the right. The localizing ideal is only dependent on the radical of the ideal $J$ used to constructu the type $n$ synthetic spectrum $F(n)$, and the radical of $J$ is equivalent to the radical of $I_n$, meaning that the right hand side can be identified with $E_{n*}/I_n$, which gives 
    \[\Loc^\otimes(E_{n*}/I_n) \simeq \Der(\ComodE)^{I_n-tors}\simeq \Der(\Comodt),\]
    where the last equivalence is due to \cite[3.7(2)]{barthel-heard-valenzuela_2020}. Hence, as the above adjunction is Barr--Beck, we get 
    \[\Mod_{C\tau}(\Mn\hSynE) \simeq \Der(\Comodt),\]
    finishing the proof. 
\end{proof}

\begin{lemma}
    Inverting the deformation parameter $\tau$ gives an equivalence $\Mn\hSynE[\tau^{-1}] \simeq \Mn$ of symmetric monoidal $\infty$-categories. 
\end{lemma}
\begin{proof}
    The proof is similar to \cref{ch3:add:lm:monochromatic-synthetic-special-fiber}, but we include it for completion. 

    It follows from \cref{ch3:add:thm:deformation-properties-of-syn} that there is a local duality adjunction 
    \[(\hSynE, \nu L_n F(n)) \rightleftarrows (\Sp_n, \tau^{-1}\nu L_n F(n))\]
    which by \cref{ch1:thm:modular-bb-torsion} induces a Barr--Beck adjunction on the respective localizing $\otimes$-ideals. By \citeme there is an equivalence $\tau^{-1}\nu X \simeq X$, hence we get a monoidal Barr--Beck adjunction 
    \[\Loc^\otimes(\nu L_n F(n))\rightleftarrows \Loc^\otimes(L_n F(n)).\]
    The former is again $\Mn\hSynE$, and the latter is $\Mn$. As the adjunction is Barr--Beck we get 
    \[\Mn\hSynE[\tau^{-1}]\simeq \Mn,\]
    just as wanted. 
\end{proof}






