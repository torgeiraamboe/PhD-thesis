
\section{The algebraic model}

The goal of this section is to set up the necessary background material that will be used throughout the paper. We use these to construct convenient algebraic approximations of categories arising from chromatic homotopy theory. 

\subsection*{Some conventions}

We freely use the language of $\infty$-categories, as developed by Joyal \cite{joyal_02} and Lurie \cite{lurie_09, Lurie_HA}. Even though we are dealing with both classical $1$-categories and $\infty$-categories in this paper, we will sometimes refer to them both as \emph{categories}, hoping that the prefix is clear from the context. 

We denote by $\PrLs$ the $\infty$-category of presentable stable $\infty$-categories and colimit preserving functors. Together with the Lurie tensor product, it is a symmetric monoidal $\infty$-category. The category of algebras $\Alg(\PrLs)$ is then the category of presentable stable $\infty$-categories with a symmetric monoidal structure commuting with colimits separately in each variable. 

Let $\C, \D\in \Alg(\PrLs)$. A localization is a functor $f\colon \C\to \D$ with a fully faithful right adjoint $i$. We denote the composite by $L= i\circ f$. The adjoint $i$ identifies $\D$ with a full subcategory of $\C$, which we denote by $\C_L$. We then view $L$ as a functor $L\colon \C\to\C_L$, that is left adjoint to the inclusion, and by abuse of notation also call these localizations. 










%%%%%%%%%%%%%%%%%%%%%%%%%%%%%%%%%%%%%%%%%%%%%%%%%%%%%%%%%%%%%%%%%%%%%%%%%%%%%%









\subsection{Local duality}

The theory of abstract local duality, proved in \cite{hovey-palmiery-strickland_97} and generalized to the $\infty$-categorical setting in \cite{barthel-heard-valenzuela_2018} will be important for the entire paper. In particular, it is the technology that will allow us to translate \cref{ch1:thm:B} into \cref{ch1:thm:A}. 

\begin{definition}
    \label{ch1:def:local-duality-context}
    \index{Local duality context}
    A pair $(\C, \K)$, where $\C\in \Alg(\PrLs)$ is compactly generated by dualizable objects, and $\K$ is a subset of compact objects, is called a \emph{local duality context}.
\end{definition}

\begin{construction}
    Let $(\C, \K)$ be a local duality context. We define $\C^{\K-tors}$ to be the localizing tensor ideal generated by $\K$, denoted $\Loc^\otimes_\C(\K)$. Further we define $\C^{\K-loc}$ to be the left orthogonal complement $(\C^{\K-tors})^\perp$, i.e., the full subcategory consisting of objects $C\in \C$ such that $\Hom_\C(T,C)\simeq 0$ for all $T\in \C^{\K-tors}$. Similarly, define $\C^{\K-comp}$ to be the left-orthogonal complement of $\C^{\K-loc}$, i.e. $\C^{\K-comp}= (\C^{\K-loc})^\perp$. These full subcategories are respectively called the $\K$-torsion, $\K$-local and $\K$-complete objects in $\C$. We have inclusions into $\C$, denoted $i_{\K-tors}$, $i_{\K-loc}$ and $i_{\K-comp}$ respectively. 
    
    By the adjoint functor theorem, \cite[5.5.2.9]{lurie_09}, the inclusions $i_{\K-loc}$ and $i_{\K-comp}$ have left adjoints $L_\K$ and $\Lambda_\K$ respectively, while $i_{\K-tors}$ and $i_{\K-loc}$ have right adjoints $\Gamma_\K$ and $V_\K$ respectively. These are then, by definition, localizations and colocalizations. Since the torsion, local and complete objects are ideals, these localizations and colocalizations are compatible with the symmetric monoidal structure of $\C$, in the sense of \cite[2.2.1.7]{Lurie_HA}. In particular, by \cite[2.2.1.9]{Lurie_HA} we get unique induced symmetric monoidal structures such that $L_\K$, $\Lambda_\K$, $\Gamma_\K$ and $V_\K$ are symmetric monoidal functors. 

    For any $X\in \C$, these functors assemble into two cofiber sequences:
    $$\Gamma_\K X \longrightarrow X \longrightarrow L_\K X \quad \text{and}\quad V_\K X \longrightarrow X \longrightarrow \Lambda_\K X.$$
    Note also that these functors only depend on the localizing subcategory $\C^{\K-tors}$, not on the particular choice of generators $\K$. Thus, when the set $\K$ is clear from the context, we often omit it as a subscript when writing the functors. 
\end{construction}

The following theorem is a slightly restricted version of the abstract local duality theorem of \cite[3.3.5]{hovey-palmiery-strickland_97} and \cite[2.21]{barthel-heard-valenzuela_2018}.  

\begin{theorem}
    \label{ch1:thm:local-duality}
    \index{Local duality}
    Let $(\C, \K)$ be a local duality context. Then
    \begin{enumerate}
        \item the functors $\Gamma$ and $L$ are \emph{smashing}, meaning that there are natural equivalences $\Gamma X\simeq X\otimes \Gamma \1$ and $LX\simeq X\otimes L\1$,
        \item the functors $\Lambda$ and $V$ are \emph{cosmashing}, meaning there are natural equivalences $\Lambda X\simeq \iHom(\Gamma \1,X)$ and $VX\simeq \iHom(L\1, X)$, and 
        \item the functors $\Gamma\colon \C^{\K-comp}\longrightarrow \C^{\K-tors}$ and $\Lambda\colon \C^{\K-tors}\longrightarrow \C^{\K-comp}$ are mutually inverse symmetric monoidal equivalences of categories,
    \end{enumerate}
    This can be summarized by the following diagram of adjoints
    \begin{center}
        \begin{tikzcd}
                & {\C^{\K-loc}} \\
                & {\C} \\
                {\C^{\K-tors}} && {\C^{\K-comp}}
                \arrow["L", xshift=-4pt, from=2-2, to=1-2]
                \arrow[from=1-2, to=2-2]
                \arrow["V", xshift=4pt, from=2-2, to=1-2, swap]

                \arrow["\Lambda", yshift=2pt, xshift=2pt, from=2-2, to=3-3]
                \arrow[yshift=-2pt, xshift=0pt, from=3-3, to=2-2]

                \arrow["\Gamma", yshift=-2pt, xshift=0pt, from=2-2, to=3-1]
                \arrow[yshift=2pt, xshift=-2pt, from=3-1, to=2-2]
                
                \arrow[bend left=35, dashed, from=3-1, to=1-2]
                \arrow[bend left=35, dashed, from=1-2, to=3-3]

                \arrow["\simeq"', swap, from=3-1, to=3-3]
        \end{tikzcd}    
    \end{center}
\end{theorem}

\begin{remark}
    \label{ch1:rm:monoidal-structure-in-local-duality}
    \cref{ch1:thm:local-duality} implies, in particular, that the symmetric monoidal structure induced by the localization $L$ and the colocalization $\Gamma$ is just the symmetric monoidal structure on $\C$ restricted to the full subcategories. This is not the case for $\C^{\K-comp}$, where the symmetric monoidal structure is given by $\Lambda(-\otimes_\C-)$. The functor $V$ also induces a symmetric monoidal structure on $\C^{\K-loc}$, but this coincides with the one induced by $L$, due to their associated endofunctors on $\C$ defining an adjoint symmetric monoidal monad-comonad pair. Note that we will not need or focus on the functor $V$, hence it will be omitted from the local duality diagrams for the rest of the paper. 
\end{remark}

We have two main examples of interest for this paper. 

\begin{example}
    \label{ch1:ex:local-duality-comod}
    Let $(A,\Psi)$ be an Adams type Hopf algebroid, for example the Hopf algebroid $(R_*, R_*R)$ for an Adams type ring spectrum $R$---see \cite[A.1]{ravenel_86} and \cite{hovey_04} for details. Denote by $\Der(\Psi)$ the derived $\infty$-category associated to the symmetric monoidal Grothendieck abelian category $\Comod_\Psi$. This is defined using the model structure from \cite{barnes-roitzheim_2011}. If $I\subseteq A$ is a finitely generated invariant regular ideal, then $(\Der(\Psi), A/I)$ is a local duality context, with associated local duality diagram
    \begin{center}
        \begin{tikzcd}
            & {\Der(\Psi)^{I-loc}} \\
            & {\Der(\Psi)} \\
            {\Der(\Psi)^{I-tors}} && {\Der(\Psi)^{I-comp}}
            \arrow["L_I^\Psi", xshift=-2pt, from=2-2, to=1-2]
            \arrow[xshift=2pt, from=1-2, to=2-2]

            \arrow["\Delta_I^\Psi", yshift=2pt, xshift=2pt, from=2-2, to=3-3]
            \arrow[yshift=-2pt, xshift=0pt, from=3-3, to=2-2]

            \arrow["\Gamma^\Psi_I", yshift=-2pt, xshift=0pt, from=2-2, to=3-1]
            \arrow[yshift=2pt, xshift=-2pt, from=3-1, to=2-2]

            \arrow[bend left=35, dashed, from=3-1, to=1-2]
            \arrow[bend left=35, dashed, from=1-2, to=3-3]

            \arrow["\simeq"', swap, from=3-1, to=3-3]
        \end{tikzcd}    
    \end{center}
\end{example}

In \cref{ch1:ssec:the-algebraic-model} we compare $\Der(\Psi)^{I-tors}$ to a more concrete category: the derived category of $I$-power torsion comodules.  

The following example comes from chromatic homotopy theory. For a good introduction, see \cite{barthel-beaudry_19}. 

\begin{example}
    \label{ch1:ex:local-duality-chromatic}
    Let $E$ denote Morava $E$-theory at prime $p$ and height $n$. If $F(n)$ is a finite type $n$ spectrum, then the pair $(\Sp\np, L_n F(n) )$ is a local duality context. The corresponding diagram can be recognized as
    \begin{center}
        \begin{tikzcd}
                & {\Sp_{n-1,p}} \\
                & {\Sp\np} \\
                {\M\np} && {\SpK}
                \arrow["L_{n-1}", xshift=-2pt, from=2-2, to=1-2]
                \arrow[xshift=2pt, from=1-2, to=2-2]

                \arrow["L_{K_p(n)}", yshift=2pt, xshift=2pt, from=2-2, to=3-3]
                \arrow[yshift=-2pt, xshift=0pt, from=3-3, to=2-2]

                \arrow["M\np", yshift=-2pt, xshift=0pt, from=2-2, to=3-1]
                \arrow[yshift=2pt, xshift=-2pt, from=3-1, to=2-2]

                \arrow[bend left=35, dashed, from=3-1, to=1-2]
                \arrow[bend left=35, dashed, from=1-2, to=3-3]

                \arrow["\simeq"', swap, from=3-1, to=3-3]
        \end{tikzcd}    
    \end{center}
    where $\M\np$ is the height $n$ monochromatic category and $\SpK$ is the category of spectra localized at height $n$ Morava $K$-theory $K_p(n)$. The functor $L_{n-1}$ is the Bousfield localization at $E_{n-1}$, while $L_{K_p(n)}$ is the Bousfield localization at $K_p(n)$, see \cite{bousfield_1979_localization}. The local duality then exhibits the classical equivalence $\M\np\simeq \SpK$, see \cite[6.19]{hovey-strickland_99}. 
\end{example}

\begin{remark}
    \label{ch1:rm:local-duality-modules}
    There is also a version of this local duality diagram for modules over $E$, see \cite[4.2, 5.1]{greenlees-may_1995}, or alternatively \cite[3.7]{barthel-heard-valenzuela_2018} for a version more similar to the above. This gives equivalences 
    \[\M\np\Mod_E\simeq \Mod_E^{I_n-tors}\simeq \Mod_E^{I_n-comp}\simeq L_{K_p(n)}\Mod_E,\]
    where $I_n$ is the Landweber ideal $(p,v_1, \ldots, v_{n-1})\subseteq E_*$.
\end{remark}







%%%%%%%%%%%%%%%%%%%%%%%%%%%%%%%%%%%%%%%%%%%%%%%%%%%%%%%%%%%%%%%%%%%%%%%%%%%%%%










\subsection{The periodic derived torsion category}
\label{ch1:ssec:the-algebraic-model}

In this section we identify the category $\Der(\Psi)^{I-tors}$---as obtained in \cref{ch1:ex:local-duality-comod}---as the derived category of $I$-power torsion comodules. We also modify the category to exhibit some needed periodicity. 

\begin{definition}
    \label{ch1:def:I-power-torsion-comodule}
    \index{I-power torsion}
    Let $(A,\Psi)$ be an Adams Hopf algebroid and $I\subseteq A$ a regular invariant ideal. The\emph{ $I$-power torsion} of a comodule $M$ is defined as 
    $$T_I^\Psi M = \{x\in M \mid I^kx = 0 \text{ for some } k\in \N\}.$$
    We say a comodule $M$ is \emph{$I$-torsion} if the natural map $T_I^\Psi M\longrightarrow M$ is an equivalence. 
\end{definition}

\begin{remark}
    \label{ch1:rm:torsion-iff-underlying-is-torsion}
    One can similarly define $I$-power torsion $A$-modules. If $(A,\Psi)$ is an Adams Hopf algebroid, then a $\Psi$-comodule $M$ is $I$-power torsion if and only if its underlying module is $I$-power torsion, see \cite[5.7]{barthel-heard-valenzuela_2018}. 
\end{remark}

\begin{remark}
    \label{ch1:rm:torsion-comodules-grothendieck-monoidal}
    By \cite[5.10]{barthel-heard-valenzuela_2018} the full subcategory of $I$-torsion comodules, which we denote $\Comod_\Psi^{I-tors}$, is a Grothendieck abelian category. It also inherits a symmetric monoidal structure from $\Comod_\Psi$. 
\end{remark}

The following technical lemma will be needed later. 

\begin{lemma}
    \label{ch1:lm:torsion-comodules-generated-by-compacts}
    Let $(A,\Psi)$ be an Adams Hopf algebroid, where $A$ is noetherian and $I\subseteq A$ a regular invariant ideal. Then $\Comod_\Psi^{I-tors}$ is generated under filtered colimits by the compact $I$-power torsion comodules. 
\end{lemma}
\begin{proof}
    By \cite[3.4]{barthel-heard-valenzuela_2020} $\Comod_\Psi^{I-tors}$ is generated by the set 
    $$\mathrm{Tors}_\Psi^{fp}:=\{G\otimes A/I^k \mid G \in \Comod_\Psi^{fp}, k\geq 1\},$$
    where $\Comod_\Psi^{fp}$ is the full subcategory of dualizable $\Psi$-comodules. Since $I$ is finitely generated and regular, $A/I^k$ is finitely presented as an $A$-module, hence it is compact in $\Comod_\Psi$ by \cite[1.4.2]{hovey_04}, and in $\Comod_\Psi^{I-tors}$ as colimits are computed in $\Comod_\Psi$. As $A$ is noetherian, being finitely generated and finitely presented coincide. The tensor product of finitely generated modules is finitely generated, hence any element in $\mathrm{Tors}_\Psi^{fp}$ is compact. 
\end{proof}

\begin{remark}
    The assumption that the ring $A$ is noetherian can most likely be removed, but it makes no difference to the results in this paper.  
\end{remark}

\begin{notation}
    Since $\Comod_\Psi^{I-tors}$ is Grothendieck abelian we have an associated derived stable $\infty$-category $\Der(\Comod_\Psi^{I-tors})$ which we denote simply by $\Der(\Psi^{I-tors})$.
\end{notation}

We e can now compare the torsion category obtained from local duality and the derived category of $I$-power torsion comodules. 

\begin{lemma}[{\cite[3.7(2)]{barthel-heard-valenzuela_2020}}]
    \label{ch1:lm:derived-torsion-if-homology-torsion}
    Let $(A,\Psi)$ be an Adams Hopf algebroid and $I\subseteq A$ a regular invariant ideal. There is an equivalence of categories 
    $$\Der(\Psi)^{I-tors}\simeq \Der(\Psi^{I-tors}).$$ 
    Furthermore, an object $M\in \Der(\Psi)$ is $I$-torsion if and only if the homology groups $H_* M$ are $I$-power torsion $\Psi$-comodules.
\end{lemma}

In order to state both the general algebraicity machinery of \cite{patchkoria-pstragowski_2021} and our results, we need the respective derived categories to exhibit the periodic nature of the spectra we are interested in. This is done via the periodic derived category. There are several ways to constructing this, but we follow \cite{franke_96} in spirit, using periodic chain complexes. 

\begin{definition}
    \label{ch1:def:periodic-chain-complex}
    \index{Periodic chain complex}
    Let $\A$ be an abelian category with a local grading, i.e., an auto-equivalence $T\colon \A\to\A$, and denote $[1]$ the shift functor on the category of chain complexes $\Ch(\A)$ in $\A$. A chain complex $C\in \Ch(\A)$ is called \emph{periodic} if there is an isomorphism $\phi\colon C[1]\longrightarrow TC$. The full subcategory of periodic chain complexes is denoted by $\Chp(\A)$. 
\end{definition}

\begin{definition}
    \index{Periodization}
    The forgetful functor $\Chp(\A)\longrightarrow \Ch(\A)$ has a left adjoint $P$, called the \emph{periodization}. 
\end{definition}

\begin{definition}
    \label{ch1:def:periodic-derived-category}
    \index{Periodic derived category}
    Let $\A$ be a locally graded abelian category. Then the \emph{periodic derived category} of $\A$, denoted $\Dper(\A)$ is the $\infty$-category obtained by localizing $\Chp(\A)$ at the quasi-isomorphism. It is in fact stable by \cite[7.8]{patchkoria-pstragowski_2021}. 
\end{definition}

\begin{remark}
    \label{ch1:rm:periodic-derived-as-modules}
    If $\A$ is a symmetric monoidal category, then $P\1$ is a commutative ring object called the \emph{periodic unit}. By \cite[2.3]{barnes-roitzheim_2011} the category of periodic chain complexes $\Chp(\A)$ is equivalent to $\Mod_{P\1}(\Ch(\A))$. This descends also to the derived categories, giving an equivalence 
    $$\Dper(\A)\simeq \Mod_{P\1}(\Der(\A)),$$
    see for example \cite[3.7]{pstragowski_2021}. 
\end{remark}

We will also need local duality  for the periodic derived category associated to a Hopf algebroid. 

\begin{construction}
    \label{ch1:const:periodic-derived-local-duality}
    Let $(A, \Psi)$ be an Adams type (graded) Hopf algebroid. Then the shift functor $[1]\colon \Comod_\Psi\longrightarrow \Comod_\Psi$
    defined by $(TM)_k = M_{k-1}$ is a local grading on $\Comod_\Psi$. Denote the corresponding periodic derived category by $\Dper(\Psi)$. The pair $(\Dper(\Psi), P(A/I))$ is a local duality context with associated local duality diagram
    \begin{center}
    \begin{tikzcd}
        & {\Dper(\Psi)^{I-loc}} \\
        & {\Dper(\Psi)} \\
        {\Dper(\Psi)^{I-tors}} && {\Dper(\Psi)^{I-comp}}
        \arrow["L_I^\Psi", xshift=-2pt, from=2-2, to=1-2]
        \arrow[xshift=2pt, from=1-2, to=2-2]

        \arrow["\Lambda_I^\Psi", yshift=2pt, xshift=2pt, from=2-2, to=3-3]
        \arrow[yshift=-2pt, xshift=0pt, from=3-3, to=2-2]

        \arrow["\Gamma^\Psi_I", yshift=-2pt, xshift=0pt, from=2-2, to=3-1]
        \arrow[yshift=2pt, xshift=-1pt, from=3-1, to=2-2]

        \arrow[bend left=35, dashed, from=3-1, to=1-2]
        \arrow[bend left=35, dashed, from=1-2, to=3-3]

        \arrow["\simeq"', swap, from=3-1, to=3-3]
    \end{tikzcd}    
    \end{center}
    The functors in the diagram are induced by the functors from \cref{ch1:ex:local-duality-comod}. In fact, there is a diagram 
    \begin{center}
        \begin{tikzcd}
            \Der(\Psi)^{I-tors} 
            \arrow[d, xshift=-2pt, "P", swap] 
            \arrow[r, yshift=2pt]       
            & \Der(\Psi) 
            \arrow[d, xshift=-2pt, "P", swap] 
            \arrow[r, yshift=2pt, "L_I^\Psi"]
            \arrow[l, yshift=-2pt, "\Gamma_I^\Psi"]       
            & \Der(\Psi)^{I-loc} 
            \arrow[d, xshift=-2pt, "P", swap] 
            \arrow[l, yshift=-2pt]      \\
            \Dper(\Psi)^{I-tors} 
            \arrow[r, yshift=2pt]   
            \arrow[u, xshift=2pt] 
            & \Dper(\Psi) 
            \arrow[r, yshift=2pt, "L_I^\Psi"]   
            \arrow[l, yshift=-2pt, "\Gamma_I^\Psi"] 
            \arrow[u, xshift=2pt] 
            & \Dper(\Psi)^{I-loc} 
            \arrow[l, yshift=-2pt] 
            \arrow[u, xshift=2pt] 
        \end{tikzcd}    
    \end{center}
    that is commutative in all possible directions. Here the unmarked horizontal arrows are the respective fully faithful inclusions. 
\end{construction} 

\begin{remark}
    In the specific case of $(A, \Psi) = (E_0, E_0E)$ and $I\subseteq E_0$ the Landweber ideal $I_n$, then the above construction is \cite[3.12]{barthel-schlank-stapleton_2021}. 
\end{remark}

There is now some ambiguity to take care of for our category of interest $\Dper(\Psi)^{I-tors}$. In the picture above, we do mean that we take $I$-torsion objects in $\Dper(\Psi)$, i.e., $[\Dper(\Psi)]^{I-tors}$, but we could also take the periodization of the category $\Der(\Psi^{I-tors})$ as our model. Luckily, there is no choice, as they are equivalent. This can be thought of as the periodic version of \cref{ch1:lm:derived-torsion-if-homology-torsion}.

\begin{theorem}
    \label{ch1:thm:pulling-out-torsion}
    Let $(A, \Psi)$ be an Adams Hopf algebroid and $I\subseteq A$ a finitely generated invariant regular ideal. Then there is an equivalence of stable $\infty$-categories 
    $$[\Dper(\Psi)]^{I-tors}\simeq \Dper(\Psi^{I-tors}).$$
\end{theorem}

The proof of this uses the fact that Barr--Beck adjunctions commute with local duality. Proving this here disrupts the flow of the paper, so we defer it to \cref{ch1:app:barr-beck}. 

\begin{proof}
    As $\Comod_\Psi$ is symmetric monoidal we have by \cref{ch1:rm:periodic-derived-as-modules} an equivalence
    \[\Dper(\Psi)\simeq \Mod_{P\1}(\Der(\Psi)),\]
    coming from the periodicity Barr--Beck adjunction.
    By \cref{ch1:thm:modular-bb-torsion} this induces a Barr--Beck adjunction on the torsion subcategories, which gives an equivalence 
    $$[\Dper(\Psi)]^{I-tors}\simeq \Mod_{\Gamma_I^\Psi (P\1)}(\Der(\Psi)^{I-tors}).$$
    Since $\Gamma_I^\Psi$ is a smashing colocalization, and $P$ is given by tensoring with $P(\1)$, they do in fact commute. By \cref{ch1:lm:derived-torsion-if-homology-torsion} we have $\Der(\Psi)^{I-tors}\simeq \Der(\Psi^{I-tors})$, hence the above equivalence can be rewritten as
    $$[\Dper(\Psi)]^{I-tors}\simeq \Mod_{P(\Gamma_I^\Psi \1)}(\Der(\Psi^{I-tors})).$$
    Now, also $\Comod_\Psi^{I-tors}$ is symmetric monoidal, so \cref{ch1:rm:periodic-derived-as-modules} gives an equivalence 
    \[\Dper(\Psi^{I-tors})\simeq \Mod_{P(\Gamma_I^\Psi\1)}(\Der(\Psi^{I-tors})),\]which finishes the proof.
\end{proof}

\begin{addendum}
    This result, and others like it, was one of the inspirations for writing the paper \cite{aambo_2024_localizing} --- see \cref{ch3}. There we prove some uniqueness results for localizing subcategories that have the property that heart-valued homotopy groups can detect objects. For the above example, both categories have the property that an object $X\in \Dper(\Psi)$ lies in $[\Dper(\Psi)]^{I-tors}$ or $\Dper(\Psi^{I-tors})$ if and only if its homology groups $H_k X$ lies in the heart $\Comod_\Psi^{I-tors}$, which is a localizing subcategory of $\Comod_\Psi$. By \cref{ch3:thm:main} this means that the categories have to be equivalent, which gives another proof of the above result. 
\end{addendum}


