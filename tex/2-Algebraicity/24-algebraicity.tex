
\section{Algebraicity for monochromatic categories}

We are now ready to prove our main results. We start by proving \cref{ch2:thm:C}, which we will later use to prove \cref{ch2:thm:B}. The main result, \cref{ch2:thm:A} will then follow by using certain local duality arguments. 

\subsection{Monochromatic modules}
\label{ch2:ssec:algebraicity-modules}

For the rest of this section, we assume that $E$ is the height $n$ Johnson--Wilson theory $E(n)$. This is an $\E_1$-ring spectrum concentrated in degrees divisible by $2p-2$, with coefficient ring $\pi_* E(n) \cong \Z_{(p)}[v_1, v_2, \ldots, v_{n-1}, v_n^{\pm 1}]$, where $|v_i| = 2p^i-2$. The goal of this section is to prove \cref{ch2:thm:C}, which we do in three steps. First we show that the functor $\pi_*\colon \Modt \longrightarrow \modt$ is a conservative adaped homology theory. We then show that $\modt$ has finite cohomological dimension, and lastly that it admits a splitting. 

The following lemma is the $I_n$-power torsion version \cite[3.14]{barthel-frankland_15}, and the proof is similar.

\begin{lemma}
    \label{ch2:lm:monochromatic-iff-torsion-modules}
    If $M$ is an $E$-module, then $M\in \Modt$ if and only if $\pi_* M\in \modt$. 
\end{lemma}
\begin{proof}
    Let $X\in \Modt$. By \cite[3.19]{barthel-heard-valenzuela_2018} there is a strongly convergent spectral sequence of $E(n)_*$-modules with signature 
    $$E_2^{s,t} = (H_{I_n}^{-s}\pi_* X)_t \implies \pi_{s+t}M_n X,$$
    where $H_{I_n}^{-s}$ denotes local cohomology. By \cite[2.1.3(ii)]{brodmann-sharp_1998} the $E_2$-page consist of only $I_n$-power torsion modules. As $\modt$ is abelian, it is closed under quotients and subobjects, as as the higher pages are created from the $E_2$-page using quotients and subobjects, they must also consist of only $I_n$-power torsion modules. In particular, the $E_\infty$-page is all $I_n$-power torsion. By Grothendieck's vanishing theorem, see for example \cite[6.1.2]{brodmann-sharp_1998}, $H_{I_n}^s(-)\cong 0$ for $s>n$, hence the abutment of the spectral sequence $\pi_* M_n X$ is a finite filtration of $I_n$-power torsion $E_*$-modules, and is therefore itself an $I_n$-power torsion module. Since $X$ was assumed to be monochromatic, i.e. $X\in \Modt$, we have $\pi_* M_n X\cong \pi_* X$, and thus $\pi_* X\in \modt$. 

    Assume now $X\in \ModE$ such that its homotopy groups are $I_n$-power torsion. Monochromatization gives a map $\phi\colon M_n X\longrightarrow X$, and as $\pi_*M_nX$ is $I_n$-power torsion this map factors on homotopy groups as 
    $$\pi_* M_n X\longrightarrow H^0_{I_n}\pi_* X\longrightarrow \pi_* X,$$
    where the first map is the edge morphism in the above-mentioned spectral sequence. As $\pi_* X$ was assumed to be $I_n$-power torsion we have $\pi_*X\cong H^0_{I_n}\pi_* X$, and $H^s_{I_n}\pi_* X \cong 0$ for $s>0$. Hence the spectral sequence collapses to give the isomorphism $\pi_* M_n X\cong H^0_{I_n}\pi_* X$, which shows that $\pi_* \phi$ is an isomorphism. As $\pi_*$ is conservative $\phi$ was already an isomorphism, hence $X\in \Modt$. 
\end{proof}

\begin{lemma}
    \label{ch2:lm:conservative-adapted-torsion-modules}
    For any prime $p$ and non-negative integer $n$, the functor 
    $$\pi_*\colon \Modt\longrightarrow \modt$$
    is a conservative adapted homology theory. 
\end{lemma}
\begin{proof}
    We first note that the functor $\pi_*\colon \ModE\longrightarrow \modE$ is a conservative adapted homology theory. By \cref{ch2:lm:monochromatic-iff-torsion-modules} its restriction to $\Modt$ lands in $\modt$, hence autmoatically $\pi_*\colon \Modt\longrightarrow \modt$ is a conservative homology theory. 
    
    Let $J$ be an injective $I_n$-power torsion $E_*$-module. We can embed $J\longrightarrow Q$ into an injective $E_*$-module $Q$, as $\modE$ has enough injectives. After applying the torsion functor $T^{E_*}_{I_n}$ this map has a section, as $J \cong T^{E_*}_{I_n}J$ is injective. In particular, any injective $J$ is a retract of $T^{E_*}_{I_n}Q$ for some injective $E_*$-module $Q$, hence we can assume $J$ to be of that form. By \cite[2.1.4]{brodmann-sharp_1998} any such $J = T^{E_*}_{I_n}Q$ is injective as an object of $\modE$. 
    
    Now, as $\pi_*$ is adapted on $\ModE$ we can chose a faithful injective lift $\bar{J}$ of $J$ to $\ModE$, and since $\bar{J}$ was assumed to have $I_n$-torsion homotopy groups we know by \cref{ch2:lm:monochromatic-iff-torsion-modules} that $\bar{J}$ is an object of $\Modt$. In particular, we have faithful lifts for any injective in $\modt$, which means that $\pi_*\colon \Modt\longrightarrow \modt$ is adapted. 
\end{proof}

Let $C^{I_n}$ denote the $I_n$-adic completion functor on $\modE$. It is neither left nor right exact, see \cite[Appendix A.]{hovey-strickland_99}. As $E_*$ is an integral domain, the higher right derived functors vanish by \cite[5.1]{greenlees-may_92}. For $i\geq 0$ we denote the $i$'th left derived functor of $C^{I_n}$ by $L^{I_n}_i$. For any $M\in \ModE$ there is a natural map $L^{I_n}_0 M \longrightarrow C^{I_n}M$. It is always an epimorphism, but usually not an isomorphism. 

\begin{lemma}
    \label{ch2:lm:cohomological-dimension-torsion-modules}
    For any prime $p$ and non-negative integer $n$, the category $\modt$ has cohomological dimension $n$. 
\end{lemma}
\begin{proof}
    Note first that the category $\modE$ has cohomological dimension $n$, and that $\Ext$-groups in $\modt$ are computed in $\modE$. By \cite[2.1.4]{brodmann-sharp_1998}, this implies that the cohomological dimension of $\modt$ cannot be greater than $n$, so it remains to prove that it is exactly $n$. We prove this by computing an $\Ext^n_{E_*}$ group that is non-zero.
    
    By \cite[A.2(d)]{hovey-strickland_99} we have $L^{I_n}_0 M \cong \Ext_{E_*}^n(H_{I_n}^n(E_*), M)$ for any $E_*$module $M$. In other words, the derived completion of an $E_*$-module is the $n$'th derived functor of maps from the $I_n$-local cohomology of $E_*$ into $M$. Choosing $M=E_*/I_n$ we get 
    $$L^{I_n}_0 (E_*/I_n)\cong \Ext_{E_*}^n(H^n_{I_n}(E_*), E_*/I_n).$$
    As any bounded $I_n$-torsion $E_*$-module is $I_n$-adically complete we have, as remarked in \cite[1.4]{barthel-heard_16}, an isomorphism $L^{I_n}_0 (E_*/I_n)\cong E_*/I_n$. The local cohomology of $E_*$ is also $I_n$-torsion, in particular $H_{I_n}^n E_* = E_*/I_n^\infty$. Hence we have 
    $$\Ext_{E_*}^n(E_*/I_n^\infty, E_*/I_n)\cong E_*/I_n \not \cong 0,$$
    showing that there are two $I_n$-power torsion $E_*$-modules with non-trivial $n$'th $\Ext$, which concludes the proof.
\end{proof}

\begin{lemma}
    \label{ch2:lm:splitting-torsion-modules}
    For any prime $p$ and non-negative integer $n$, the category $\modt$ has a splitting of order $2p-2$. 
\end{lemma}
\begin{proof}
    By \cite[8.1]{patchkoria-pstragowski_2021} the category $\modE$ has a splitting of order $2p-2$. We define the pure weight $\phi$ component of $\modt$, denoted $\Mod_{E_*, \phi}^{I_n-tors}$, to be the essential image of $T_{I_n}^{E_*}\colon \modE\longrightarrow \modt$ restricted to the pure weight $\phi$ component $\Mod_{E_*, \phi}$. We claim that this defines a splitting of order $2p-2$ on $\modt$. 

    As $\Mod_{E_*, \phi}$ is a Serre subcategory, and being $I_n$-power torsion is a property closed under sub-objects, quotients, and extensions, also $\Mod_{E_*, \phi}^{I_n-tors}$ is a Serre subcategory. As $E_*$ is concentrated in degrees divisible by $2p-2$ every $I_n$-power torsion module decomposes into its pure weight components. This also gives a decomposition of $\modt$. The shift functor on $I_n$-power torsion modules simply shifts the underlying module, hence shift-invariance follows from the shift-invariance on $\modE$. 

    % For $(2)$, we note that we have a diagram of adjoint functors 
    % \begin{center}
    %     \begin{tikzcd}
    %         \modE \arrow[r, "{[1]}", yshift=2] \arrow[d, "T_{I_n}^{E_*}", xshift=2] & \modE \arrow[l, yshift=-2] \arrow[d, "T_{I_n}^{E_*}", xshift=2] \\
    %         \modt \arrow[r, "{[1]}", yshift=2] \arrow[u, "i", xshift=-2] & \modt \arrow[l, yshift=-2] \arrow[u, "i", xshift=-2]
    %     \end{tikzcd}
    % \end{center}
    % which is commutative from bottom left to top right. Here $[1]$ denotes the local grading on $\modt$. We want the diagram to commute from top left to bottom right, which can be obtained by the dual Beck-Chevalley condition. This reduces to checking $[-1]\circ i \simeq i\circ [-1]$, which is true due to the commutativity and the fact that $[1]$ and $[-1]$ are autoequivalences. Hence we have $[1]\circ T_{I_n}^{E_*} \simeq T_{I-n}^{E_*}\circ [1]$. In fact, the diagram is commutative in all possible directions. This means that for any $I_n$-power torsion $E_*$-module $M$ of pure weight $\phi$, we have 
    % $$[k]M \cong [k]T_{I_n}^{E_*}M \cong T_{I_n}^{E_*}[k]N \in \Mod_{E_*, \phi + k \mod 2p-2}^{I_n-tors}$$
    % as $[k]M\in \Mod_{E_*, \phi+k \mod 2p-2}$. 
    
    % For the final point $(3)$, note that any subcategory of a product category is a product of subcategories. Hence, the $\modt$ splits as a product of the pure weight components. In particular, the functor 
    % $$\prod_{\phi \in \Z/(2p-2)}\Mod_{E_*, \phi}^{I_n-tors}\longrightarrow \modt$$
    % defined by $(M_\phi) \longmapsto \bigoplus_\phi M_\phi$ is an equivalence of categories. 
\end{proof}

% \begin{remark}
%     This is the part where it was important we chose a version of Morava $E$-theory that is concentrated in degrees divisible by $2p-2$. If we instead chose a $2$-periodic $E$-theory, for example $E_n$, then neither $\modE$ nor $\modt$ would have a splitting of the above degree. 
% \end{remark}

We can now summarize the above discussion with the first of our main results. 

\begin{theorem}[\cref{ch2:thm:C}]
    \label{ch2:thm:main-modules}
    Let $p$ be a prime and $n$ a non-negative integer. If $k=2p-2-n>0$, then the functor 
    $$\pi_*\colon \Modt\longrightarrow\modt$$
    is a $k$-exotic homology theory, giving an equivalence 
    $$h_k \Modt \simeq h_k \Dper(\modt).$$
    In particular, monochromatic $E$-modules are exotically algebraic at large primes. 
\end{theorem}
\begin{proof}
    By \cref{ch2:lm:cohomological-dimension-torsion-modules} the cohomological dimension of the category $\modt$ is $n$, and by \cref{ch2:lm:splitting-torsion-modules} we have a splitting on $\modt$ of order $2p-2$. Hence, by \cref{ch2:lm:conservative-adapted-torsion-modules} the functor $$\pi_*\colon \Modt \longrightarrow \modt$$
    is a $k$-exotic homology theory for $k=2p-2-n>0$, which gives an equivalence 
    $$h_k \Modt \simeq h_k \Dper(\modt)$$
    by \cref{ch2:thm:franke-algebraicity}.
\end{proof}

We can also phrase this dually in terms of $K_p(n)$-local $E$-modules. 

\begin{corollary}
    \label{ch2:cor:main-modules-dual}
    Let $p$ be a prime, $n$ a positive integer and $K_p(n)$ be height $n$ Morava $K$-theory at the prime $p$. If $k=2p-2-n>0$, then we have a $k$-exotic algebraic equivalence 
    $$h_k L_{K_p(n)}\ModE \simeq h_k \Dper(\modE)^{I_n-comp}.$$ 
    In particular, $K_p(n)$-local $E$-modules are exotically algebraic at large primes. 
\end{corollary}
\begin{proof}
    The equivalence is constructed from the equivalences obtained from \cref{ch2:rm:local-duality-modules}, \cref{ch2:thm:main-modules}, \cref{ch2:thm:pulling-out-torsion} and \cref{ch2:const:periodic-derived-local-duality}. In particular, we have
    \begin{align*}
        h_k \Modc
        &\hspace{2pt}\overset{\ref{ch2:rm:local-duality-modules}}\simeq 
        h_k \hspace{2pt} \Modt \\
        &\hspace{2pt}\overset{\ref{ch2:thm:main-modules}}\simeq 
        \hspace{2pt} h_k \Dper(\modt) \\
        &\overset{\ref{ch2:thm:pulling-out-torsion}}\simeq
        h_k \Dper(\modE)^{I_n-tors} \\
        &\overset{\ref{ch2:const:periodic-derived-local-duality}}\simeq 
        h_k \Dper(\modE)^{I_n-comp},
    \end{align*}
    where we have used that an equivalence of $\infty$-categories induces an equivalence on homotopy $k$-categories.
\end{proof}

Now, let $HE_*$ be the Eilenberg--MacLane spectrum of $E_*$. By Schwede--Shipleys's derived Morita theory, see \cite[7.1.1.16]{Lurie_HA}, there is a symmetric monoidal equivalence of categories $\Der(E_*)\simeq \Mod_{HE_*}$, and we can form a local duality diagram for $\Mod_{HE_*}$ corresponding to \cref{ch2:ex:local-duality-comod} for the discrete Hopf algebroid $(E_*, E_*)$. By arguments similar to \cref{ch2:lm:monochromatic-iff-torsion-modules} and \cref{ch2:lm:conservative-adapted-torsion-modules} one can show that the homotopy groups functor $\pi_*\colon \Mod_{HE_*}\longrightarrow \modE$ restricts to a conservative adapted homology theory 
$$\pi_* \colon \Mod_{HE_*}^{I_n-tors}\longrightarrow \modt.$$
In the same range as \cref{ch2:thm:main-modules} this is also $k$-exotic. We can then combine the algebraicity for $\Modt$ and $\Mod_{HE_*}$ to get the following statement. 

\begin{corollary}
    Let $p$ be a prime and $n$ a non-negative integer. If $k=2p-2-n>0$, then there is an exotic equivalence $h_k \Modt \simeq h_k \Mod_{HE_*}^{I_n-tors}.$
\end{corollary}



\subsection{Monochromatic spectra}
\label{ch2:ssec:algebraicity-spectra}

Having proven that monochromatic $E$-modules are algebraic at large primes, we now turn to the larger category of all monochromatic spectra $\M\np$ with the same goal. The strategy is exactly the same as in \cref{ch2:ssec:algebraicity-modules}: we first prove that the conservative adapted homology theory $E_*\colon \Sp\np\longrightarrow \comod\EE$ restricts to a conservative adapted homology theory on $\M\np$, before proving that $\Comod\EE^{I_n-tors}$ has a splitting and finite cohomological dimension. This will prove \cref{ch2:thm:B}, which we then convert into a proof of \cref{ch2:thm:A}, as in \cref{ch2:cor:main-modules-dual}. 

In this section the choice of $v_n$-periodic Landweber exact ring spectrum $E$ does not matter, as the categories $\Sp\np$ and $\comod\EE$ are equivalent for all such spectra---see \cite[1.12]{hovey_95} and \cite[4.2]{hovey-strickland_2005a} respectively. However, to make the interaction with \cref{ch2:ssec:algebraicity-modules} as simple as possible we will continue to use the height $n$ Johnson--Wilson spectrum $E(n)$.

\begin{lemma}
    \label{ch2:lm:monochromatic-iff-torsion-comodules}
    If $X$ is a $E$-local spectrum, then $X\in \M\np$ if and only if $E_*X\in \Comodt$. 
\end{lemma}
\begin{proof}
    Assume first that $X\in \M\np$. We have $E\otimes X\in \ModE^{I_n-tors}$ as
    $$E\otimes X\simeq E\otimes M_n X\simeq M_n (E\otimes X),$$
    where the last equivalence follows from $M_n$ being smashing. In particular, the restricted functor $E_*\colon \M\np\longrightarrow \comod\EE$ factors through $\ModE^{I_n-tors}$. By \cref{ch2:lm:monochromatic-iff-torsion-modules} and \cref{ch2:rm:torsion-iff-underlying-is-torsion} this means that $E_*X$ is an $I_n$-power torsion $E_*E$-comodule. 

    For the converse, assume that we have $X\in \Sp\np$ such that $E_*X\in \Comodt$. Using the monochromatization functor we obtain a comparison map $M_n X\longrightarrow X$, which induces a map on $E$-modules $E\otimes M_n X\longrightarrow E\otimes X$. This map is an isomorphism on homotopy groups, as $E_*X$ was assumed to be $I_n$-power torsion. As $E_*$ is conservative on $\Sp\np$, the original comparison map $M_n X\longrightarrow X$ was an isomorphism, meaning that $X\in \M\np$. 
\end{proof}

\begin{lemma}
    \label{ch2:lm:conservative-adapted-torsion-comodules}
    For any prime $p$ and non-negative integer $n$, the functor 
    \[E_*\colon \M\np\longrightarrow \Comodt\]
    is a conservative adapted homology theory. 
\end{lemma}
\begin{proof}
    First note that the image of the functor $E_*\colon \Sp\np\longrightarrow \comod\EE$ restricted to $\M\np$ is contained in $\Comodt$ by \cref{ch2:lm:monochromatic-iff-torsion-comodules}. The functor $E_*\colon \M\np\longrightarrow \Comodt$ is then automatically a conservative homology theory. The category $\Comodt$ has enough injectives as it is Grothendieck by \cref{ch2:rm:torsion-comodules-grothendieck-monoidal}. Hence, it only remains to prove that we have faithful lifts for all injective objects. 

    Let $J$ be an injective in $\Comodt$. As in the proof of \cref{ch2:lm:conservative-adapted-torsion-modules} we can assume that $J$ has the form $J = T^{E_*E}_{I_n} P$ for some injective $E_*E$-comodule $P$, as being torsion is a property of the underlying module. By \cite[2.1(c)]{hovey-strickland_2005b} any injective $E_*E$-comodule is a retract of $E_*E\otimes_{E_*} Q$ for some injective $E_*$-module $Q$. Hence, we can further assume that $J$ has the form $J = T^{E_*E}_{I_n}(E_*E\otimes_{E_*}Q)$.

    From \cite[5.7]{barthel-heard-valenzuela_2018} it follows that there is a commutative diagram of adjoint functors 
    \begin{center}
        \begin{tikzcd}
            \comod\EE 
            \arrow[r, yshift=2pt, "\epsilon_*"] 
            \arrow[d, xshift=2pt, "T_{I_n}^{E_*E}"] 
            & \modE 
            \arrow[l, yshift=-2pt, "\epsilon^*"] 
            \arrow[d, xshift=2pt, "T^{E_*}_{I_n}"] \\
            \Comodt 
            \arrow[r, yshift=2pt, "\epsilon_*"] 
            \arrow[u, xshift=-2pt] 
            & \modt 
            \arrow[l, yshift=-2pt, "\epsilon^*"] 
            \arrow[u, xshift=-2pt]  
        \end{tikzcd}
    \end{center}
    where $\epsilon_* \dashv \epsilon^*$ is the forgetful-cofree adjunction. In particular, the functor $\epsilon^*$ is given by $E_*E\otimes_{E_*}(-)$. To justify the notation in the bottom row, let us prove that the cofree functor on $I_n$-power torsion modules is also given by $E_*E\otimes_{E_*}(-)$. In order to do this we prove that for an $I_n$-power torsion $E_*$-module $M$, that $T^{E_*E}_{I_n}(E_*E\otimes_{E_*}M) \cong E_*E\otimes_{E_*}M$. 

    By \cite[5.5]{barthel-heard-valenzuela_2018} there is an isomorphism
    \[T^{E_*E}_{I_n} (E_*E\otimes_{E_*}M) \cong \colim_k \iHom_{E_*E}(E_*/I_n^k, E_*E\otimes_{E_*}M),\]
    which by \cite[4.4]{barthel-heard-valenzuela_2018} gives
    \[\colim_k \iHom_{E_*E}(E_*/I_n^k, E_*E\otimes_{E_*}M) \cong \colim_k (E_*E \otimes_{E_*} \Hom_{E_*}(E_*/I_n^k, M)).\]
    As the tensor product $-\otimes_{E_*}-$ commutes with filtered colimits separately in each variable, and $M$ was assumed to be $I_n$-power torsion, the right hand side is $E_*E\otimes_{E_*}M$.    
    
    Now, choosing the injective $Q$ in the top right corner and going through the square gives an isomorphism $T^{E_*E}_{I_n}(E_*E\otimes_{E_*}Q)\cong E_*E\otimes_{E_*}T_{I_n}^{E_*}Q$. By \cite[2.1.4]{brodmann-sharp_1998} we know that $T^{E_*}_{I_n}Q$ is an injective $E_*$-module, and by \cite[2.1(a)]{hovey-strickland_2005b} the cofree comodule $E_*E\otimes_{E_*} T^{E_*}_{I_n}Q$ is an injective $E_*E$-comodule. Hence, $J=T^{E_*E}_{I_n}(E_*E\otimes_{E_*}Q)$ is injective also as an object in $\comod\EE$.

    Finally, as $E_*$ has faithful injective lifts from $\comod\EE$ to $\Sp\np$, there is a lift $\bar{J}$ such that $[X,\bar{J}]\simeq \Hom_{E_*E}(E_*X, J)$ and $E_*\bar{J}\simeq J$. By \cref{ch2:lm:monochromatic-iff-torsion-comodules} we know that $\bar{J}\in \M\np$, as $J$ was assumed to be $I_n$-power torsion, hence we have found our faithful injective lift. 
\end{proof}

\begin{lemma}
    \label{ch2:lm:cohomological-dimension-torsion-comodules}
    Let $p$ be a prime and $n$ a non-negative integer. If $p-1\nmid n$, then the category $\Comodt$ has cohomological dimension $n^2+n$. 
\end{lemma}
\begin{proof}
    The proof follows \cite[2.5]{pstragowski_2021} closely, which is itself a modern reformulation of \cite[3.4.3.9]{franke_96}. As in \cref{ch2:lm:cohomological-dimension-torsion-modules} we note that also $\Ext$-groups in $\Comodt$ are computed in $\comod\EE$. We start by defining \emph{good targets} to be $I_n$-power torsion comodules $N$ such that $\Ext_{E_*E}^{s,t}(E_*/I_n, N)=0$ for all $s>n^2+n$ and \emph{good sources} to be $I_n$-power torsion comodules $M$ such that $\Ext_{E_*E}^{s,t}(M,N)=0$ for all $s>n^2+n$ and $I_n$-torsion comodules $N$. 

    By the Landweber filtration theorem, see for example \cite[5.7]{hovey-strickland_2005a}, we know that any finitely presented comodule $M$ has a finite filtration 
    $$0=M_0 \subseteq M_1 \subseteq \cdots \subseteq M_{s-1}\subseteq M_s=M,$$
    where $M_r/M_{r-1} \cong E_*/I_{j_r}[t_r]$ and $j_r\leq n$. When $M$ is $I_n$-power torsion we get $j_r=n$ for all $r$, as noted in \cite[4.3]{hovey-strickland_2005a}. For primes $p$ not dividing $n+1$ Morava's vanishing theorem, see for example \cite[6.2.10]{ravenel_86}, gives us that $\Ext_{E_*E}^{s,t}(E_*, E_*/I_n) = 0$ for all $s>n^2$. As the generators for the ideal $I_i$ form a regular sequence, we get short exact sequences of the form 
    \[0\longrightarrow E_*/I_{i-1} \overset{v_i}\longrightarrow E_*/I_{i-1} \longrightarrow E_*/I_i\longrightarrow 0\]
    for $0\leq i\leq n$. By the induced long exact sequence in $\Ext$-groups, we get that 
    \[\Ext_{E_*E}^{s,t}(E_*/I_n, E_*/I_n) = 0\] 
    for $s>n^2+n$, which by the Landweber filtration implies that any finitely presented $I_n$-power torsion comodule is a good target.
    
    The comodule $E_*/I_n$ has a finite resolution of $E_*E$-comodules that are projective as modules over $E_*$. The $\Ext$-functor out of these projectives can be computed using the cobar complex, see \cite[A1.2.12]{ravenel_86}, implying that the functor $\Ext_{E_*E}^{s,t}(E_*/I_n, -)$ commutes with filtered colimits. By \cref{ch2:lm:torsion-comodules-generated-by-compacts} any $I_n$-power torsion comodule is a filtered colimit of finitely presented ones, hence any $I_n$-power torsion comodule is a good target. 

    Note that the above argument also proves that $E_*/I_n$ is a good source, which by the Landweber filtration argument implies that any finitely presented $I_n$-torsion comodule is a good source. Again, by \cref{ch2:lm:torsion-comodules-generated-by-compacts}, the category $\Comodt$ is generated under filtered colimits by finitely presented comodules. Hence, we can apply \cite[2.4]{pstragowski_2021} to any injective resolution 
    $$0\longrightarrow M \longrightarrow J_0 \longrightarrow J_1\longrightarrow \cdots$$
    to get that the map $J_{n^2+n}\longrightarrow \Ima(J_{n^2+n}\longrightarrow J_{n^2+n+1})$ is a split surjection, and that the object $\Ima(J_{n^2+n}\longrightarrow J_{n^2+n+1})$ is injective. Hence, any injective resolution can be modified to have length $n^2+n$, which concludes the proof. 
\end{proof}

\begin{remark}
    In a previous version of this paper, we claimed that the cohomological dimension was $n^2$. We want to thank Piotr Pstr\a{}gowski for pointing out the gap in the proof. This means that $\Comod\EE^{I_n-tors}$ has the same cohomological dimension as the non-torsion category $\comod\EE$, as seen in \cref{ch2:ex:cohomological-dimension-comodEE}. However, we do obtain something slightly stronger, as our result holds for all $p-1\nmid n$, while the analogue in $\Comod\EE$ only holds when $p-1>n$. In fact, $\Comod\EE$ does not have finite cohomological dimension when $p-1\leq n$, as noted in \cite[2.6]{pstragowski_2021}. This difference happens because we only need the $\Ext^s$-groups out of $E_*/I_n$ to vanish for large $s$, which is given to us by Morava's vanishing theorem whenever $p-1\nmid n$. For non-torsion comodules one has to have stronger vanishing results. These can be obtained by using the chromatic spectral sequence, which only gives the vanishing results for $p-1>n$ instead of for $p-1\nmid n$. 
\end{remark}

\begin{lemma}
    \label{ch2:lm:splitting-torsion-comodules}
    For any prime $p$ and non-negative integer $n$, the category $\Comodt$ has a splitting of order $2p-2$. 
\end{lemma}
\begin{proof}
    As $E$ is concentrated in degrees divisible by $2p-2$, \cite[8.13]{patchkoria-pstragowski_2021} shows that $\comod\EE$ has a splitting of order $2p-2$. The proof of the induced splitting on the $I_n$-torsion category is then identical to \cref{ch2:lm:splitting-torsion-modules}. 
\end{proof}

We can now summarize the above results with our second main result, which is the monochromatic analogue of \cref{ch2:ex:chromatic-algebraicity}. 

\begin{theorem}[\cref{ch2:thm:B}]
    \label{ch2:thm:main-spectra}
    Let $p$ be a prime and $n$ a non-negative integer. If we have $k=2p-2-n^2-n>0$, then the restricted functor $E_*\colon \M\np\longrightarrow \Comodt$ is $k$-exotic. In particular, there is an equivalence 
    \begin{equation*}
        h_k \M\np \simeq h_k\Dper(E_*E^{I_n-tors}),
    \end{equation*}
    meaning that monochromatic homotopy theory is exotically algebraic at large primes. 
\end{theorem}
\begin{proof}
    By \cref{ch2:lm:cohomological-dimension-torsion-comodules}, the cohomological dimension of $\Comodt$ is $n^2+n$ and by \cref{ch2:lm:splitting-torsion-comodules} we have a splitting of order $2p-2$. The restricted functor $E_*$ is then by \cref{ch2:lm:conservative-adapted-torsion-comodules} $k$-exotic whenever $k=2p-2-n^2-n>0$, which by \cref{ch2:thm:franke-algebraicity} finishes the proof. 
\end{proof}

\begin{remark}
    \label{ch2:rm:monochromatic-corresponds-to-torsion}
    By \cref{ch2:thm:pulling-out-torsion} there is an equivalence $\Dper(E_*E^{I_n-tors})\simeq \Fr\np^{I_n-tors}$ and by \cref{ch2:ex:local-duality-chromatic} there is an equivalence $\M\np\simeq \Sp\np^{I_n-tors}$. This means that we can write the equivalence in \cref{ch2:thm:main-spectra} as 
    $$h_k\Sp\np^{I_n-tors}\simeq h_k\Fr\np^{I_n-tors}$$
    for $k=2p-2-n^2-n>0$. This is more in line with thinking about \cref{ch2:thm:main-spectra} as ``coming from'' the chromatic algebraicity of \cref{ch2:ex:chromatic-algebraicity} on localizing ideals. This formulation is perhaps also easier to connect to the limiting case $p\to \infty$ as described using ultra-products in \cite{barthel-schlank-stapleton_2021}, which can be stated informally as 
    $$\lim_{p\to \infty} \Sp\np^{I_n-tors}\simeq \lim_{p\to \infty} \Fr\np^{I_n-tors}.$$
\end{remark}


Via \cref{ch2:thm:local-duality} we can now obtain the associated exotic algebraicity statement for the category of $K_p(n)$-local spectra.

\begin{theorem}[\cref{ch2:thm:A}]
    \label{ch2:thm:main-spectra-dual}
    Let $p$ be a prime, $n$ a non-negative integer and $K_p(n)$ be height $n$ Morava $K$-theory at the prime $p$. If $k=2p-2-n^2>0$, then we have a $k$-exotic algebraic equivalence 
    $$h_k \SpK \simeq h_k \Fr\np^{I_n-comp}.$$ 
    In other words, $K_p(n)$-local homotopy theory is algebraic at large primes. 
\end{theorem}
\begin{proof}
    As we did in \cref{ch2:cor:main-modules-dual}, we construct the equivalence from a sequence of equivalences coming from \cref{ch2:thm:local-duality} and \cref{ch2:thm:main-spectra}. More precisely we use equivalences coming from \cref{ch2:ex:local-duality-chromatic}, \cref{ch2:thm:main-spectra}, \cref{ch2:thm:pulling-out-torsion} and \cref{ch2:const:periodic-derived-local-duality}, which give
    \begin{align*}
        h_k \SpK
        &\overset{\hspace{2pt}\ref{ch2:ex:local-duality-chromatic}\hspace{2pt}}
        \simeq 
        h_k \M\np \\
        &\overset{\ref{ch2:thm:main-spectra}}
        \simeq 
        h_k \Dper(\Comodt) \\
        &\overset{\ref{ch2:thm:pulling-out-torsion}}
        \simeq 
        h_k \Fr\np^{I_n-tors} \\
        &\overset{\ref{ch2:const:periodic-derived-local-duality}}
        \simeq 
        h_k \Fr\np^{I_n-comp},
    \end{align*}
    where we again have used that an equivalence of $\infty$-categories induces an equivalence on homotopy $k$-categories.
\end{proof}

\begin{remark}
    As in \cref{ch2:rm:monochromatic-corresponds-to-torsion} we can phrase \cref{ch2:thm:main-spectra-dual} as $h_k\Sp\np^{I_n-comp}\simeq h_k\Fr\np^{I_n-comp}$. 
\end{remark}






%%%%%%%%%%%%%%%%%%%%%%%%%%%%%%%%%%%%%%%%%%%%%%%%%%%%%%%%%%%%%%%%%%%%%%%
%%%%%%%%%%%%%%%%%%%%%%%%%%%%%%%%%%%%%%%%%%%%%%%%%%%%%%%%%%%%%%%%%%%%%%%




\subsection*{Some remarks on future work}

The reason why \cref{ch2:thm:franke-algebraicity} works so well, is that there is a deformation of stable $\infty$-categories lurking behind the scenes. One does not need this in order to apply the theorem, but it is there regardless. In the case of some $v_n$-periodic Landweber exact ring spectrum $E$, the deformation associated with the adapted homology theory $E_*\colon \Sp\np \longrightarrow \comod\EE$ is equivalent to the category of hypercomplete $E$-based synthetic spectra, $\hSynE$, introduced in \cite{pstragowski_2022}. As both $\Sp\np$ and $\comod\EE$ are invariant under the choice of such $E$, we conjecture that this is true also for $\hSynE$. 

Our restricted homology theory $E_*\colon \M_{n,p}\longrightarrow \Comodt$ should then be associated to a deformation $\hSynE^{I_n-tors}$ coming from a local duality theory for $\hSynE$, in the sense that there is a diagram of stable $\infty$-categories 
\begin{center}
    \begin{tikzcd}
        \M_{n,p}\simeq\Sp\np^{I_n-tors} & \hSynE^{I_n-tors} \arrow[l, "\tau^{-1}"'] \arrow[r, "\tau\sim 0"] & \Fr\np^{I_n-tors}.
    \end{tikzcd}    
\end{center}
Since $E_*$ is adapted on $\M\np$, we abstractly know that there is a deformation $D^\omega(\M\np)$ arising out of the work of Patchkoria-Pstr{\k a}gowski in \cite{patchkoria-pstragowski_2021}, called the perfect derived category. This should give an equivalent ``internal'' approach to $I_n$-torsion synthetic spectra, much akin to how we have equivalences $\M\np\simeq \Sp\np^{I_n-tors}$ and $D(E_*E)^{I_n-tors}\simeq D(E_*E^{I_n-tors})$. 

In \cite{barkan_2023}, Barkan provides a monoidal version of \cref{ch2:thm:franke-algebraicity} by using filtered spectra. His deformation $\mathscr{E}\np$ is equivalent to $\hSynE$, which by the above remarks hints towards a monoidal version of \cref{ch2:thm:main-spectra} as well. We originally intended to incorporate such a result into this paper but decided against it in order to keep it free from deformation theory. We do, however, state the conjectured monoidal result, which we hope to pursue in future work.

\begin{conjecture}
    Let $p$ be a prime and $n$ a natural number. If $k$ is a positive natural number such that $2p-2>n^2+(k+3)n+k-1$, then we have a symmetric monoidal equivalence 
    \[h_k \M\np\simeq h_k \Fr\np^{I_n-tors}\]
    of $k$-categories. 
\end{conjecture}

As \cref{ch2:thm:local-duality} is monoidal, this would give a similar statement for the $K_p(n)$-local category, i.e., a symmetric monoidal equivalence 
\[h_k \SpK\simeq h_k \Fr\np^{I_n-comp}.\]

Since $E$-based synthetic spectra are categorifications of the $E$-Adams spectral sequence, one should expect the above-mentioned local duality for $\hSynE$ to give a category $\hSynE^{I_n-comp}$, which categorifies the $K_p(n)$-local $E$-Adams spectral sequence. We plan to study such categorifications of the $K_p(n)$-local $E$-Adams spectral sequence in future work joint with Marius Nielsen. 