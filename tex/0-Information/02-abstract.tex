

\section*{Abstract}
\addcontentsline{toc}{section}{Abstract}

This thesis consists of three papers, all focused on understanding certain features of localizing subcategories, and all motivated by understanding features of chromatic homotopy theory. 

Our first paper uses Patchkoria--Pstr\a{}gowski's version of Franke's algebraicity theorem to prove that monochromatic homotopy theory is complately algebraic when the prime $p$ is large compared to the chromatic height $n$. In particular, we prove that the category of $K_p(n)$-local spectra --- $K_p(n)$ being Morava $K$-theory --- is equivalent to the derived $I_n$-complete objects in Franke's category of periodic comodules over the Hopf algebroid $E_*E$ associated to height $n$ Morava $E$-theory. 

In the second paper we introduce the notion of contramodules over cocommutative coalgebras in presentably symmetric monoidal $\infty$-categories. When the coalgebra $C$ is coidempotent, we prove that there is a symetric monoidal duality between the category of comodules and contramodules over $C$, which we call Positselski duality. When the ambient category is stable and compactly generated by dualizable objects, this duality recovers local duality in the sense of Hovey--Palmieri--Strickland, which allows us to describe the category of $K_p(n)$-local spectra as contramodules over the monochromatization of the sphere specrum. 

The third paper studies how certain localizing subcategories compatible with a given $t$-structure on a stable $\infty$-category, can be classified by using the associated Grothendieck prestable $\infty$-category and the associated Grothendieck abelian heart. In particular, we prove that there is a bijection between the $t$-structure compactible localizing subcategories and the prestable localizing subcategories of the connected component, extending a result of Lurie to the stable setting. 


\newpage 
\section*{Sammendrag}
\addcontentsline{toc}{section}{Sammendrag}

Denne avhandlingen består av tre artikler, der alle prøver å forstå visse egenskaper av lokaliserende underkategorier, og alle er motivert av å forstå egenskaper innboende i kromatisk homotopiteori. 

Vår første artikkel bruker Patchkoria--Pstr\a{}gowskis versjon av Frankes algebraisitetsteorem til å bevise at monokromatisk homotopiteori er fullstendig algebraisk når primtallet $p$ er mye større enn den kromatiske høyden $n$. Mer presist viser vi at kategorien av $K_p(n)$-lokale spectra, der $K_p(n)$ er Morava $K$-teori, er ekvivalent til $I_n$-komplette objekter i Frankes kategori av periodiske komodules over Hopf algebroiden $E_*E$ assosiert til Morava $E$-teori med høyde $n$. 

I den andre artikkelen introduserer vi kontramoduler over kokommutative koalgebraer i presenterbare symmetrisk monoidale $\infty$-kategorier. Når koalgebraen $C$ er koidempotent viser vi at det er en symmetrisk monoidal dualitet mellom komoduler og kontramoduler over $C$, som vi kaller Positselski-dualitet. Når bakgrunnskategorien er stabil og kompakt-generert av dualiserbare objekter gjennskaper dette Hovey--Palmieri--Stricklands teori om lokal dualitet, som lar oss beskrive kategorien av $K_p(n)$-lokale som kontramoduler over monokromatiseringen av sfærespektumet. 

Den tredje artikkelen studerer hvordan lokaliserende underkategorier som er kompatible med en gitt $t$-struktur på en stabil $\infty$-kategori, kan klassifiseres ved å bruke den assosierte Grothendieck-prestabile $\infty$-kategorien og det Grothendieck-abelske hjertet. Vi viser at det er en bijeksjon mellom disse $t$-struktur-kompatible lokaliserende underkategoriene, og prestabile lokaliserende underkategorier av den sammenhengende delen, som utvider et resultat av Lurie til den stabile situasjonen. 


