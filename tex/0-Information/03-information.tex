
\section*{Information}
\addcontentsline{toc}{section}{Information}

The contents of this thesis consist mainly of material from the papers \cite{aambo_2024_algebraicity}, \cite{aambo_2024_positselski} and \cite{aambo_2024_localizing}, where the candidate is the only author. In addition there are some added remarks, some further results not yet presented in any papers, some more historical background, as well as more in-depth introductions to the central ideas of the thesis. 

This material is structured into four chapters. \hyperref[ch0]{Chapter 0} consists of mathematical preliminaries and background, as well as a short summary of each paper. There is also an introduction for the layperson, mostly aimed at family and friends, for situating the topics of this thesis amidst the broad world of mathematics. 

The three remaining chapters each consists of one of the above mentioned paper: \hyperref[ch1]{chapter 1} presents the paper ``Algebraicity in monochromatic homotopy theory'' (\cite{aambo_2024_algebraicity}), \hyperref[ch2]{chapter 2} presents the paper ``Positselski duslity in stable $\infty$-categories'' (\cite{aambo_2024_positselski}), while \hyperref[ch3]{chapter 3} presents the paper ``Classification of localizing subcategories along $t$-structures'' (\cite{aambo_2024_localizing}). 

Before each of the papers there is a titlepage, a poem and a drawing, each representing the contents of the paper. These are all made by the author. The poems are in the form of limericks, and each describe the main result from the associated paper. The drawings have two functions: enumerate the papers and give a visual feel for what the paper is about. A description of the drawing can be found on the subsequent page. 
